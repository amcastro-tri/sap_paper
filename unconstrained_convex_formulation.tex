\section{An Unconstrained Convex Formulation}
\label{sec:unconstrained_convex_formulation}
Remarkably,  the dual optimal impulses of~\eqref{eq:dual_regularized} can be constructed from 
the primal optimal velocities of~\eqref{eq:primal_regularized}
using a simple projection operation. 
 Following~\cite{bib:todorov2014}, we call this construction  \textit{inverse dynamics}. 
 Moreover, this projection
decomposes into a set of individual projections for each
contact impulse $\bgamma_i$ given the separable structure of the constraints.
Letting $\vf{y}_i(\vf{v}_{c,i}) = -\vf{R}_i^{-1}(\vf{v}_{c,i}-\hat{\vf{v}}_{c,i})$,
these projections take the form
\begin{equation}
  \begin{aligned}
	\bgamma_i(\vf{v}_{c,i})&= P_{\mathcal{F}_i}(\vf{y}_i(\vf{v}_{c,i}))\\
	&= \argmin_{\bgamma\in\mathcal{F}_i} \quad 
		\frac{1}{2}(\bgamma-\vf{y}_i)^T\vf{R}_i(\bgamma-\vf{y}_i),\\
	\end{aligned}
	\label{eq:y_projection}
\end{equation}
where $\vf{R}_i\in\mathbb{R}^{3\times3}$ is the $i\text{-th}$ diagonal block of
the regularization matrix $\mf{R}$. That is, $\bgamma_i$ is the projection
$P_{\mathcal{F}_i}$ of $\vf{y}_i(\vf{v}_{c,i})$ onto the friction cone
$\mathcal{F}_i$ using the norm defined by $\vf{R}_i$.
The projection $P_{\mathcal{F}}(\mf{y})$ onto 
the full cone $\mathcal{F} :=
\mathcal{F}_1 \times F_2 \times \cdots \times \mathcal{F}_{n_c}$ 
is obtained by simply stacking together the individual projections
$P_{\mathcal{F}_i}(\vf{y}_i)$ from Eq. (\ref{eq:y_projection}),
where we form $\mf{y}$ by stacking together each
$\vf{y}_i$ from all contact pairs.  In this notation,
the optimal impulse $\bgamma$ of~\eqref{eq:dual_regularized} 
and the optimal velocities $\mf{v}$ of~\eqref{eq:primal_regularized} 
satisfy  $\bgamma = P_{\mathcal{F}}(y(\mf{v}))$.


The inverse dynamics formula $\bgamma = P_{\mathcal{F}}(y(\mf{v}))$  and the optimality
condition $\sigma = \bgamma$ from Theorem~\ref{th:primal_dual}
allow us to eliminate both $\sigma$ and the constraints from the primal
formulation. In total, we obtain the following unconstrained problem
in velocities only
\begin{eqnarray}
	\min_{\mf{v}} \ell_p(\mf{v}) = \frac{1}{2}\Vert\mf{v}-\mf{v}^*\Vert_{A}^2 +
	\frac{1}{2}\Vert P_\mathcal{F}(\mf{y}(\mf{v}))\Vert_R^2.
	\label{eq:primal_unconstrained}
\end{eqnarray}
Correctness of this reformulation is asserted by the following theorem, 
which we prove in Appendix \ref{app:unconstrained_formulation_equivalance}.
\begin{theorem}
  If  $\mf{v}$ solves the unconstrained formulation (\ref{eq:primal_unconstrained}),
    then $(\mf{v}, \bsigma)$,
    with $\bsigma =P_\mathcal{F}(\mf{y}(\mf{v}))$,
    solves the primal
    formulation (\ref{eq:primal_regularized}).
    \label{th:unconstrained_formulation_equivalance}
\end{theorem}
Lemma \ref{lem:PropertiesOfObj} in Appendix \ref{app:sap_converge}
shows that the unconstrained cost $\ell_p(\mf{v})$ is strongly convex and
differentiable with Lipschitz continuous gradients. Therefore, the unconstrained
formulation (\ref{eq:primal_unconstrained}) has a unique solution, and can be
efficiently solved.  
Section \ref{sec:sap_solver} presents our novel SAP solver specifically designed
for its solution.
