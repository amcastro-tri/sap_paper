\section{An Unconstrained Convex Formulation}
\label{sec:unconstrained_convex_formulation}

A remarkable property of the dual optimal solution, the impulses, is that they
can be can be constructed analytically from the optimal velocities of the primal
formulation in \eqref{eq:primal_regularized}. This is referred to as the
\textit{inverse dynamics} solution \cite{bib:todorov2014}. Moreover, given the
separable structure of the constraints, the impulse $\bgamma_i$ at the
$i\text{-th}$ contact point is the solution to the following convex optimization
problem
\begin{eqnarray}
	\bgamma_i(\vf{v}_{c,i})&=& P_{\mathcal{F}_i}(\vf{y}_i(\vf{v}_{c,i}))
	\nonumber\\
	&=&\begin{aligned} \argmin_{\bgamma\in\mathcal{F}_i} \quad &
		\frac{1}{2}(\bgamma-\vf{y}_i)^T\vf{R}_i(\bgamma-\vf{y}_i) \end{aligned}
	\label{eq:y_projection}\\
	\vf{y}_i(\vf{v}_{c,i}) &=& -\vf{R}_i^{-1}(\vf{v}_{c,i}-\hat{\vf{v}}_{c,i})	
\end{eqnarray}
where $\vf{R}_i\in\mathbb{R}^{3\times3}$ is the $k\text{-th}$ diagonal block of
the regularization matrix $\mf{R}$. That is, $\bgamma_i$ is the projection
$P_{\mathcal{F}_i}$ of $\vf{y}_i(\vf{v}_{c,i})$ onto the friction cone
$\mathcal{F}_i$ using the norm defined by $\vf{R}_i$.

We use the analytical inverse dynamics to write an unconstrained convex
formulation in terms of velocities only
\begin{eqnarray}
	\min_{\mf{v}} \ell_p(\mf{v}) = \frac{1}{2}\Vert\mf{v}-\mf{v}^*\Vert_{A}^2 +
	\frac{1}{2}\Vert P_\mathcal{F}(\mf{y}(\mf{v}))\Vert_R^2
	\label{eq:primal_unconstrained}
\end{eqnarray}
where, as with other quantities, we form $\mf{y}$ by stacking each
$\vf{y}_i$ from all contact pairs together. Given the separable structure of the
constraints, projection $P_{\mathcal{F}}(\mf{y})$ onto $\mathcal{F} :=
\mathcal{F}_1 \times F_2 \times \cdots \times \mathcal{F}_{n_c}$ in the norm
defined by $\mf{R}$ is formed by stacking together the individual projections
$P_{\mathcal{F}_i}(\vf{y}_i)$ from Eq. (\ref{eq:y_projection}).

The unconstrained formulation in Eq. (\ref{eq:primal_unconstrained}) includes
the same cost in velocities as the primal formulation from Eq.
(\ref{eq:primal_regularized}). However, the conic constraint $\mf{g}\in
\mathcal{F}^*$ in Eq. (\ref{eq:primal_regularized}) has been replaced by a cost
term using the analytical inverse dynamics. Lemma
\RedHighlight{Add proper reference} in Appendix \RedHighlight{Add proper reference} proofs
that the unconstrained cost $\ell_p(\mf{v})$ is strongly convex and
differentiable with Lipschitz continuous gradients. Therefore the problem
stated in Eq. (\ref{eq:primal_unconstrained}) has a unique solution.

We state the main result of this section in the following result, proved in
Appendix \ref{app:unconstrained_formulation_equivalance}.
\begin{theorem}
    The pair $\{\mf{v},\bsigma\}$, with velocities $\mf{v}$ solution to the
    unconstrained formulation in Eq. (\ref{eq:primal_unconstrained}) and
    $\bsigma=P_\mathcal{F}(\mf{y}(\mf{v}))$, is primal optimal of the primal
    formulation in Eq. (\ref{eq:primal_regularized}).
    \label{th:unconstrained_formulation_equivalance}
\end{theorem}

Section \ref{sec:sap_solver} presents our novel SAP solver specifically designed
to solve the unconstrained formulation in Eq. (\ref{eq:primal_unconstrained}).
