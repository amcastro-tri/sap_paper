The Taylor expansion of $\mf{m}(\mf{v})$ at $\mf{v}=\mf{v}^*$ reads
\begin{align}
	\mf{m}(\mf{v}) &= \mf{m}^* + \frac{\partial \mf{m}}{\partial \mf{v}} (\mf{v}-\mf{v}^*) +
	\mathcal{O}_m(\Vert\mf{v}-\mf{v}^*\Vert^2)\nonumber\\
	&=\frac{\partial \mf{m}}{\partial \mf{v}}(\mf{v}-\mf{v}^*) +
	\mathcal{O}_m(\Vert\mf{v}-\mf{v}^*\Vert^2),
	\label{eq:m_taylor_expansion}
\end{align}
where we use the fact that by definition $\mf{m}^*=\mf{m}(\mf{v}^*)=\mf{0}$. All
derivatives are evaluated at $\mf{v} = \mf{v}^*$ unless otherwise noted. We
first evaluate the Jacobian of the mass matrix term in Eq.
(\ref{eq:m_definition})
\begin{align*}
	\frac{\partial \left( \mf{M}(\mf{q}^{\theta}(\mf{v}))(\mf{v}-\mf{v}_0) \right)}{\partial \mf{v}}
	= \mf{M}(\mf{q}^{\theta}(\mf{v}^*)) + \mf{E},
\end{align*}
where we defined
\begin{align*}
	\mf{E} = \frac{\partial \mf{M}(\mf{q}^{\theta})}{\partial\mf{v}} (\mf{v}^*-\mf{v}_0).
\end{align*}
Note that by combining Eqs. (\ref{eq:theta_method}) and (\ref{eq:scheme_q_update}), the
mid-step configuration $\mf{q}^{\theta}$ can be written as
\begin{align*}
	\mf{q}^{\theta}(\mf{v}) &= \mf{q}_0 + \delta t \theta \dot{\mf{q}}^{\theta_{vq}} \\
	                          &= \mf{q}_0 + \delta t \theta \mf{N}(\mf{q}^{\theta})\mf{v}^{\theta_{vq}}(\mf{v}).
\end{align*}
Hence by the chain rule, $\mf{E}$ can be further calculated as
\begin{align*}
	\mf{E} = \delta t\theta\frac{\partial \mf{M}(\mf{q}^{\theta}) }{\partial\mf{q}}
             \frac{\partial\dot{\mf{q}}^{\theta_{vq}}}{\partial\mf{v}}
			 (\mf{v}^*-\mf{v}_0).
\end{align*}
Notice that 
\begin{align*}
		\Vert\mf{E}\Vert 
		&\le \delta t\theta \left\| \frac{\partial\mf{M}(\mf{q}^{\theta})}{\partial\mf{q}}  \right\|
			\left\| \frac{\partial\dot{\mf{q}}^{\theta_{vq}}}{\partial\mf{v}}  \right\|
		    \left\| \mf{v}^*-\mf{v}_0 \right\| \\
		&= \mathcal{O}(\delta t^2),
\end{align*}
since $\Vert\mf{v}^*-\mf{v}_0\Vert = \mathcal{O}(\delta t)$.

We proceed similarly to expand the Jacobian of
$\mf{F}_1(\mf{v})=\mf{F}_1(\mf{q}^{\theta}(\mf{v}), \mf{v}^{\theta}(\mf{v}))$
as
\begin{align*}
	\frac{\partial\mf{F}_1}{\partial \mf{v}} = -\delta t\,\theta\theta_{vq}\mf{K}(\mf{q}^{\theta},
	\mf{v}^{\theta})-\theta\mf{D}(\mf{q}^{\theta}, \mf{v}^{\theta}),
\end{align*}
with $\mf{K}$ and $\mf{D}$ the stiffness and damping matrices defined by Eqs.
(\ref{eq:stiffness_matrix})-(\ref{eq:dissipation_matrix}).

We can now write the Jacobian of $\mf{m}(\mf{v})$ in Eq.
(\ref{eq:m_taylor_expansion}) as
\begin{align*}
	\frac{\partial \mf{m}}{\partial \mf{v}} = \mf{A} + \mf{E} - \delta t\frac{\partial \mf{F}_2}{\partial \mf{v}},
\end{align*}
where we defined
\begin{align*}
	\mf{A}=\mf{M}+ \delta t^2\theta\theta_{qv}\mf{K}+\delta t\theta\mf{D}.
\end{align*}
With these definitions the Taylor expansion in Eq. (\ref{eq:m_taylor_expansion})
becomes
\begin{align*}
	\frac{\partial\mf{m}}{\partial\mf{v}}(\mf{v}-\mf{v}^*) &= \mf{A}(\mf{v}-\mf{v}^*) + \mf{E}(\mf{v}-\mf{v}^*) \\
	&- \delta t\frac{\partial\mf{F}_2}{\partial\mf{v}}(\mf{v}-\mf{v}^*) + \mathcal{O}_m(\Vert\mf{v}-\mf{v}^*\Vert^2).
\end{align*}

Since contact is compliant, forces are finite within the finite interval $\delta
t$ and therefore $\Vert\mf{v}-\mf{v}^*\Vert=\mathcal{O}(\delta t)$. Thus
\begin{align*}
	\mf{E}(\mf{v}-\mf{v}^*)=\mathcal{O}_E(\delta t^3), \\
    \delta t\frac{\partial \mf{F}_2}{\partial \mf{v}}(\mf{v}-\mf{v}^*)=\mathcal{O}_G(\delta t^2), \\ 
    \mathcal{O}_m(\Vert\mf{v}-\mf{v}^*\Vert^2) = \mathcal{O}_m(\delta t^2).
\end{align*}
Therefore, the positive definite linearization
\begin{align*}
	\mf{A}(\mf{v}-\mf{v}^*) + \mathcal{O}_E(\delta t^3) + \mathcal{O}_G(\delta t^2) +
	\mathcal{O}_m(\delta t^2) = \mf{J}^T\mf{\bgamma},
\end{align*}
agrees with the original momentum balance in Eq. (\ref{eq:scheme_momentum}) to second
order.

Finally, notice that $\mf{A}$ is a linear combination of positive definite
matrices with non-negative scalars in the linear combination, and therefore
$\mf{A}\succ 0$.\hfill\IEEEQED
