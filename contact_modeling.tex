% Dummy comment for Reviewable.

\subsection{Contact Modeling}
We model the normal component of the impulse $\gamma_{n}$ during a time
interval of size $\delta t$ with the compliant law
\begin{equation}
    \gamma_{n}/\delta t = (-k\phi - \tau_{d}\,k\,v_{n})_+
    \label{eq:compliant_model}
\end{equation}
where $k$ is the stiffness parameter, $\tau_{d}$ is a \textit{dissipation time
scale} and $(x)_+=\max(0, x)$ is the \textit{positive part} operator. This model
of compliance can be written as the equivalent complementarity condition
\begin{equation}
    0 \le \phi + \tau_{d}\,v_{n} + \frac{c}{\delta t} \gamma_{n}\perp \gamma_{n} \ge 0
\end{equation}
where $c=k^{-1}$ is the compliance parameter and $0 \le a\perp b \ge 0$ denotes
complementarity, i.e. $a \ge 0$, $b \ge 0$ and $a\,b=0$.

The tangential component $\bgamma_{t}\in\mathbb{R}^2$ of the contact impulse is
modeled with Coulomb's law of dry friction as
\begin{equation}
    \bgamma_{t}=\argmin_{\Vert\bgamma_{t}\Vert\leq\mu\gamma_{n}}\vf{v}_{t}\cdot\bgamma_{t}
    \label{eq:maximum_dissipation_principle}
\end{equation}
where $\mu > 0$ is the coefficient of friction. Equation
(\ref{eq:maximum_dissipation_principle}) describes the \emph{maximum dissipation
principle}, which states that friction impulses maximize the rate of energy
dissipation. In other words, friction impulses oppose the sliding velocity
direction. Moreover, Eq. (\ref{eq:maximum_dissipation_principle}) states that
contact impulses $\bgamma$ are constrained to belong to the friction cone
$\mathcal{F}=\{[\vf{x}_t, x_n] \in\mathbb{R}^3 \,|\, \Vert\vf{x}_t\Vert\le \mu
x_n\}$.

The optimality conditions for Eq. (\ref{eq:maximum_dissipation_principle}) are
\cite{bib:stewart2000rigid, bib:tasora2011}
\begin{flalign}
    &\mu\gamma_{n}\vf{v}_{t} + \lambda \bgamma_{t} = \vf{0}\nonumber\\
    &0\le \lambda \perp \mu\gamma_{n}-\Vert\bgamma_{t}\Vert \ge 0
    \label{eq:mpd_optimality_conditions}
\end{flalign}
where $\lambda$ is the multiplier needed to enforce Coulomb's law condition
$\Vert\bgamma_{t}\Vert \le \mu\gamma_{n}$. Notice that in the form we
wrote Eq. (\ref{eq:mpd_optimality_conditions}), multiplier $\lambda$
has units of velocity and it is zero during stiction and takes the value
$\lambda=\Vert\vf{v}_{t}\Vert$ during sliding. Finally, the total contact
impulse $\bgamma\in\mathbb{R}^3$ expressed in the contact frame $C$ is given by
$\bgamma=[\bgamma_{t}\,\gamma_{n}]$.
