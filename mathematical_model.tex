\section{Discrete Time Formulation}
\label{sec:discrete_formulation}

We introduce a variant of the $\theta\text{-method}$ to implicitly include force
elements such as springs and dampers and even internal forces as those arising
when modeling soft body deformations. Moreover, our variant includes the
symplectic midpoint rule, allowing to attain second order accuracy. Our
discrete formulation is described by the update
\begin{eqnarray}
	&&\mf{M}(\mf{q}^{\theta_{q}})(\mf{v}-\mf{v}_0) + 
	dt\,\mf{F}(\mf{q}^{\theta_{q}}, \mf{v}^{\theta_v}) = \mf{J}(\mf{q}^{\theta_{q}})^T\mf{\bgamma},
	\label{eq:v_update}\\
	&&\mf{q} = \mf{q}_0 + dt\mf{N}(\mf{q}^{\theta_{q}})\mf{v}^{\theta_{vq}},
	\label{eq:q_update}
\end{eqnarray}
where from now on \textit{nought} variables refer to quantities evaluated at the
previous time step. $\mf{F}(\mf{q}, \mf{v})$ includes Coriolis and gyroscopic
terms in multibody dynamics and modeling elements such as springs, dampers and
even internal forces when modeling soft body dynamics \CyanHighlight{Xuchen: 
Gravity and actuation? Perhaps also mention there are negative signs associated
with some of these terms because it's common to see Mdv = Fdt.}
Mid-step quantities are defined in accordance to the standard
$\theta\text{-method}$ method as
\begin{eqnarray}
	\mf{q}^{\theta} &=& \theta\mf{q} + (1-\theta)\mf{q}_0,\\
	\mf{v}^{\theta} &=& \theta\mf{v} + (1-\theta)\mf{v}_0.
\end{eqnarray}

In this form, parameters $\theta$ allow us to obtain some of the most popular
schemes for forward dynamics:
\begin{itemize}
	\item Explicit Euler with $\theta_q=\theta_{v}=\theta_{vq} = 0$,
	\item Symplectic Euler with $\theta_{q} = 0$ and $\theta_v = \theta_{vq}=1$,
	\item Implicit Euler with $\theta_{q} = \theta_v = \theta_{vq}= 1$,
	\item Symplectic midpoint rule, which is second order, with $\theta_{q} =
	\theta_v = \theta_{vq}= 1/2$,
\end{itemize}

We solve Eqs. (\ref{eq:v_update}) and (\ref{eq:q_update}) in two stages. In the
first stage we solve for the \textit{free motion} velocities $\mf{v}^*$ in the
absence of constraint impulses 
\begin{eqnarray}
	&&\mf{M}^*(\mf{v}^*-\mf{v}_0) + dt\,\mf{F}(\mf{q}^{\theta_q}(\mf{v}^*), \mf{v}^{\theta_v}(\mf{v}^*)) = \mf{0},
	\label{eq:vstar_definition}\\
	&&\mf{q}^{\theta_q}(\mf{v}^*) = \mf{q}_0 + dt\theta_q\mf{N}(\mf{q}^{\theta_{q}})\mf{v}^{\theta_{vq}}(\mf{v}^*),
	\label{eq:qstar_definition}
\end{eqnarray}
where $\mf{M}^* = \mf{M}(\mf{q}^{\theta_q}(\mf{v}^*))$.
\CyanHighlight{Xuchen: In equation \eqref{eq:vstar_definition} and \eqref{eq:qstar_definition},
it's not immediately clear what $\mf{v}^{\theta_v}(\mf{v}^*)$ and $\mf{v}^{\theta_{vq}}(\mf{v}^*)$
are referring to. It may be more clear if when introducing the $\theta$-method we make the
dependency explicit by saying
\begin{eqnarray*}
	\mf{v}^{\theta}(\mf{v}) &=& \theta\mf{v} + (1-\theta)\mf{v}_0.
\end{eqnarray*}}
For implicit schemes,
we need an iterative method to solver for $\mf{v}^*$. Notice that, in the
absence of constraint impulses, the next step velocity is $\mf{v}=\mf{v}^*$ and
it is computed with the order of accuracy of the $\theta\text{-method}$. We also
expect to recover the order of accuracy of the $\theta\text{-method}$ when
constraints do not produce work.
\CyanHighlight{Xuchen: Is there a reference for this?}
This is the case of equality constraints and
rolling friction for instance. That is, we'd expect very good energy
conservation for a rolling wheel when using the second order, symplectic,
midpoint rule, with $\theta=1/2$.

For the second stage we linearize Eq. (\ref{eq:v_update})
around $\mf{v}^*$ to obtain a linearized balance of momentum
\begin{eqnarray}
	\mf{A}(\mf{v}-\mf{v}^*)= \mf{J}^T\mf{\gamma}
	\label{eq:momentum_linearized}
\end{eqnarray}
with
\begin{eqnarray}
	\mf{A}&=&\mf{M}+\theta_q\theta_{qv} dt^2\,\mf{K}+\theta_v dt\,\mf{D},
	\label{eq:expression_for_A}\\
	\mf{K}&=&\frac{\partial \mf{F}}{\partial \mf{q}}\Bigr|_{(\mf{q}^{\theta_q}, \mf{v}^{\theta_v})},\\
	\mf{D}&=&\frac{\partial \mf{F}}{\partial \mf{v}}\Bigr|_{(\mf{q}^{\theta_q},
	\mf{v}^{\theta_v})}.
	\label{eq:dissipation_matrix}
\end{eqnarray}
where the only assumption in Eq. (\ref{eq:expression_for_A}) is that the map
$\mf{N}(\mf{q})$ is the identity, which is commonly the case for degrees of
freedom used to model compliant elements.
\CyanHighlight{Xuchen: This assumption isn't really necessary right? It feels
a little expected to suddenly restrict our attention to compliant elements.}
When $\mf{F}(\mf{q}, \mf{v})$ is
linear the stiffness matrix $\mf{K}$ and the damping matrix $\mf{D}$ are
constant and the two stage process described by Eqs. (\ref{eq:vstar_definition})
to (\ref{eq:momentum_linearized}) is exact and solves the original problem
stated by Eq. (\ref{eq:v_update}).


A typical example of constant stiffness and damping is that of a spring-damper
model at a joint. In this case these matrices only contribute positive diagonal
elements to matrix $\mf{A}$, increasing the stability of the scheme.

An example of complex stiffness and damping matrices is when using a Finite
Element Model (FEM) of large soft body deformations. In this case $\mf{K}$ and
$\mf{D}$ are large sparse, positive definite, matrices.

Our convex approximation of contact requires $\mf{A}\succ 0$. Therefore only
compliant and dissipation elements that lead to positive definite matrices can
be included in Eq. (\ref{eq:expression_for_A}). This is the case for joint
spring-damper models and deformable models. It is not the case for Coriolis and
gyroscopic terms. These terms can still be treated implicitly in Eqs.
(\ref{eq:vstar_definition}) and (\ref{eq:qstar_definition}), but their
derivative is assumed to be zero in Eq. (\ref{eq:dissipation_matrix}), an
approximation only accurate to first order.

