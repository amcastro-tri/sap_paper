\section{Formulation}

In this section we describe our formulation that extends the work in
\cite{bib:anitescu2010}, \cite{bib:todorov2014} to implicitly include force
elements such as springs and dampers and internal forces for soft body
simulation. In addition we show how to achive higher order with the symplectic
second order implicit midpoint rule. Then Section
\ref{sec:constraints_based_modeling_framework} extends the work in
\cite{bib:todorov2014} by using regularization to model true physical compliance
instead of a means to introduce a Baumgarte-style stabilization.

To be concrete, we will consider the $\theta\text{-method}$ that defines the
state at time $t+\theta dt$ as
\begin{eqnarray}
	\mf{v}^\theta(\mf{v}) &=& \theta\mf{v}+(1-\theta)\mf{v}_0\\
	\dot{\mf{q}}^\theta(\mf{v}) &=& \mf{N}(\mf{q}_0)\mf{v}^\theta\\
	\mf{q}^\theta(\mf{v}) &=& \mf{q}_0 + dt\dot{\mf{q}}^\theta
\end{eqnarray}
where $\mf{q}_0$ and $\mf{v}_0$ are the positions and velocities at time $t$,
$\mf{N}(\mf{q})$ is the kinematic mapping between generalized velocities and
position rates and $\theta$ is a scalar in the range $[0, 1]$. Typical choices
of $\theta$ are $\theta = 0$ for explicit Euler, $\theta=1$ for implicit Euler
and $\theta=1/2$ for the second order implicit midpoint rule.

After discretizing the conservation of momentum in time using the
$\theta\text{-method}$, the equations describing the update step from an initial
velocity $\mf{v}_0$ to the next velocity $\mf{v}$ in general looks like
\begin{equation}
	\mf{M}(\mf{v}-\mf{v}_0) + dt\,\mf{F}(\mf{q}^\theta(\mf{v}), \mf{v}^\theta(\mf{v})) = \mf{J}^T\mf{\gamma}
	\label{eq:general_momentum_equations}
\end{equation}
where $\mf{M}\succ 0$ is the mass matrix, in general a function of the
configuration $\mf{q}$ and $\mf{F}(\mf{q}, \mf{v})$ encodes modeling elements
such as springs and dampers and even internal forces when modeling soft body
dynamics.

We will solve Eq. (\ref{eq:general_momentum_equations}) in two stages. In the
first stage we solve for a velocity $\mf{v}^*$ in the absence of constraint
impulses 
\begin{equation}
	\mf{M}(\mf{v}^*-\mf{v}_0) + dt\,\mf{F}(\mf{q}^\theta(\mf{v}^*), \mf{v}^\theta(\mf{v}^*)) = \mf{0}
	\label{eq:vstar_definition}
\end{equation}

This stage might require a Newton-Raphson iteration to account for
nonlinearities in $\mf{F}(\mf{q}, \mf{v})$. Notice that, in the absence of
constraint impulses, the next step velocity is $\mf{v}=\mf{v}^*$ and it is
computed with the order of accuracy of the  $\theta\text{-method}$. We also
expect to recover the order of accuracy of the $\theta\text{-method}$ when
constraints do not produce work. This is the case of equality constraints and
rolling friction for instance. That is, we'd expect very good energy
conservation for a rolling wheel when using the second order, symplectic,
implicit midpoint rule, with $\theta=1/2$.

For the second stage we linearize Eq. (\ref{eq:general_momentum_equations})
around $\mf{v}^*$ to obtain
\begin{eqnarray}
	\mf{A}(\mf{v}-\mf{v}^*)= \mf{J}^T\mf{\gamma}
	\label{eq:momentum_linearized}
\end{eqnarray}
where we have defined
\begin{eqnarray}
	\mf{K}&=&\frac{\partial \mf{F}(\mf{q}, \mf{v})}{\partial \mf{q}}\\
	\mf{D}&=&\frac{\partial \mf{F}(\mf{q}, \mf{v})}{\partial \mf{v}}\\
	\mf{A}&=&\mf{M}+\theta dt^2\,\mf{K}+\theta dt\,\mf{D}
\end{eqnarray}

When $\mf{F}(\mf{q}, \mf{v})$ is linear the stiffness matrix $\mf{K}$ and the
damping matrix $\mf{D}$ are constant and the two stage process described by Eqs.
(\ref{eq:vstar_definition}) and (\ref{eq:momentum_linearized}) is exact and
solves the original problem stated by Eq. (\ref{eq:general_momentum_equations}).


A typical example of constant stiffness and damping is that of a spring-damper
model at a joint. In this case these matrices only contribute positive diagonal
elements to matrix $\mf{A}$, increasing the stability of the scheme.

An example of complex stiffness and damping matrices is when using a Finite
Element Model (FEM) of large soft body deformations. In this case $\mf{K}$ and
$\mf{D}$ are large sparse, positive definite, matrices.

\section{Constraints Based Modeling Framework}
\label{sec:constraints_based_modeling_framework}

This extends the work in \cite{bib:todorov2014} by using regularization to model
true physical compliance instead of a means to introduce a Baumgarte-style
stabilization. As a result, we not only obtain a model that enables the modeling
of true compliant elements, but we gain a very simple to understand insight on
the unphysical artifacts introduced by the convex approximation of contact.

We consider holonomic constraints $\mf{p}(\mf{q};t)=\mf{0}$ as well as
non-holonomic constraints $\mf{u}(\mf{v}; \mf{q},t)=\mf{J}_u\,\mf{v}+\mf{b}_u=
\mf{0}$. We treat both sets of constraints at the velocity level to write
\begin{equation}
	\mf{v}_c = \mf{J}\mf{v}+\mf{b}=\mf{0}
	\label{eq:velocity_level_constraints}
\end{equation}
where $\mf{v}_c$ is the constraints velocity, $\mf{J}$ the constraints Jacobian
and $\mf{b}$ is a velocity bias.

In our compliant formulation of constraints we relax Eq.
(\ref{eq:velocity_level_constraints}) so that when impulses $\bgamma$ are in the
interior of $\mathcal{C}$, they behave as the linear spring and damper law
\begin{equation}
	\bgamma = -dt(\mf{k}\,\mf{p} + \mf{c}\,\mf{v}_c)
	\label{eq:compliant_constraints}
\end{equation}
where stiffness $\mf{k}$ and damping $\mf{c}$ matrices are diagonal. For
Non-holonomic constraints stiffness is zero and damping \textit{regularizes} the
problem.

In the discrete time setting we again use the $\theta\text{-method}$ with
parameter $\theta_c$, different from $\theta$ in Eq.
(\ref{eq:general_momentum_equations}) to gain more control on the stability and
accuracy of the method. E.g.: we typically use $\theta_c=1$ for contact
constraints for additional stability while we use $\theta_c=1/2$ for holonomic
constraints for better energy conservation. Using $\mf{v}_c^{\theta_c}$ and
$\mf{p}^{\theta_c}=\mf{p}_0+dt\mf{v}_c^{\theta_c}$ in Eq.
(\ref{eq:compliant_constraints}) and grouping terms we can write

\begin{eqnarray}
	\gamma_i &=& -R_i^{-1}(v_{c,i}-\hat{v}_i)\nonumber\\
	R_i^{-1} &=& \theta_c dt (dt\,k_i+c_i)\nonumber\\
	\hat{v}_i &=& -\frac{k_i}{\theta_c(dt\,k_i+c_i)}p_{0,i}-
	              \frac{1-\theta_c}{\theta_c}v_{c0,i}
\end{eqnarray}
\todo{contrast this to Torodorv's stabilization and show how his is unstable.}

We can then define the regularization matrix $\mf{R}=\text{diag}(\{R_i\})$ and
fit the compliant laws within the framework developed in \cite{bib:todorov2014}
\begin{eqnarray}
	\mf{y} &=& -\vf{R}^{-1}(\mf{v}_c-\hat{\mf{v}}) \label{eq:y_definition}\\
	\bgamma &=& P_\mathcal{C}(\mf{y})
	\label{eq:projection_definition}
\end{eqnarray}
where $P_\mathcal{C}(\mf{y})$ is the projection onto $\mathcal{C}$ of $\mf{y}$
using the norm defined by $\vf{R}$. That is
\begin{equation}
	\begin{aligned}
		P_\mathcal{C}(\mf{y})=\argmin_{\bgamma\in\mathcal{C}} \quad & \frac{1}{2}(\bgamma-\mf{y})^T\mf{R}(\bgamma-\mf{y})
	\end{aligned}
\end{equation}

That is, while the impulse is strictly inside the convex set $\mathcal{C}$, Eq.
(\ref{eq:y_definition}) essentially enforces a penalty on the constraint
violation untilt $\mf{y}$ comes out of $\mathcal{C}$ and then Eq.
(\ref{eq:projection_definition}) projects it back to its boundary.

Before jumping into the details provided in Section
\ref{app:analytical_inverse_dynamics_derivations} here we'll summarize how, with
the proper definition of the velocity bias $\hat{\mf{v}}$, convex set
$\mathcal{C}$ and regularization $\mf{R}$, we can model a variety of physical
effects such as
\begin{itemize}
	\item Joint limits.
	\item Joint dry friction.
	\item PD control with force limits.
	\item Frictional contact.
\end{itemize}
\todo{continue propagating $\theta_c$ to the rest of the constraints.}

\subsection{Joint Limits}

Given a one-dof joint modeled with (scalar) minimal coordinates $q$ and $v$, we
can model the limit $q_l < q < q_u $ using two constraints (one for each, lower
and upper). In this case we have
\begin{eqnarray}
	\mf{g} &=&
	% J*v
	\begin{bmatrix}
		v\\
		v\\
	\end{bmatrix} -
	% vhat
	\begin{bmatrix}
		\hat{v}_l\\
		\hat{v}_u\\
	\end{bmatrix}\\
	\mf{R} &=& R\,\mf{I}_2
\end{eqnarray}
with
\begin{eqnarray}
	\hat{v}_l&=&-\frac{q_0-q_l}{\theta_c(dt+\tau)}-\frac{1-\theta_c}{\theta_c}v_0\\
	\hat{v}_u&=&-\frac{q_0-q_u}{\theta_c(dt+\tau)}-\frac{1-\theta_c}{\theta_c}v_0\\
	R^{-1}&=&\theta_c dt^2 k(1+\tau/dt)
\end{eqnarray}
where the dissipation rate $\tau$ is defined such that $c=\tau\,k$. The convex
set is $\mathcal{C}=\mathbb{R}^+$ with the projection
\begin{eqnarray}
	\gamma = y^+= \max(0, y)
\end{eqnarray}


\subsection{Joint Dry Friction}

In this case we'd like the joint impulse to be limited within an interval
$\mathcal{C} = [-dt\tau_M, dt\tau_M]$, where $\tau_M$ is a specified load.
Within that interval, given our regularized model, we'd like the joint force to
penalize motion. We can achieve this with
\begin{equation}
	y = -dt\frac{\tau_M}{v_s}v
\end{equation}

where $v_s$ is a \textit{stiction tolerance} with units of m/s for prismatic
joints and with units of rad/s for revolute joints. Therefore we have
\begin{eqnarray}
	\hat{v} &=& 0\nonumber\\
	R^{-1} &=& dt\frac{\tau_M}{v_s}
\end{eqnarray}

\subsection{PD Control with Force Limits}
In this case stiffness and dissipation are replaced by proportional and
derivative PD gains, respectively. To make the analogy with a spring-mass model
closer, the proportional gain is written as $k_p = k$ and the derivative gain as
$k_d = c = \tau k_p$. For this case $\mathcal{C} = [\gamma_l, \gamma_u] =
[dt\,u_l, dt\,u_u]$, where $u_l < u_u$ are lower and upper actuation limits.  In
this case, when inside the set $\mathcal{C}$, we want the impulse to be
\begin{eqnarray}
	y/dt = -k_p(q-q_d)-k_d(v-v_d)
\end{eqnarray}
where $q_d$ and $v_d$ are desired position and velocity respectively. We can
accomplish this with
\begin{eqnarray}
	\hat{v} &=& -\frac{q_0-q_d}{dt+\tau}+\frac{\tau}{dt}v_d\nonumber\\
	R^{-1}  &=& dt^2k(1+\tau/dt)
\end{eqnarray}

The convex set is $\mathcal{C} = [dt\,u_l, dt\,u_u]$ with projection
\begin{equation}
	\gamma = \min(\gamma_u, \max(\gamma_l, y))
\end{equation}


\subsection{Frictional Contact}
In this case we have
\begin{equation}
	\vf{g} = \vf{v}_c - \hat{\vf{v}}_c = \mf{J_c}\mf{v} - \hat{\vf{v}}_c
\end{equation}
where $\mf{J}_c$ is the contact Jacobian and 
\begin{eqnarray}
	\hat{\vf{v}}_c &=&
	\begin{bmatrix}
		0\\
		0\\
		\hat{v}_n \end{bmatrix}\nonumber\\
	\hat{v}_n &=& -\frac{\phi_0}{dt+\tau}\nonumber\\
	\mf{R} &=& \text{diag}([R_t, R_t, R_n]) = 
	\begin{bmatrix}
		R_t &   0 & 0\\
		  0 & R_t & 0\\
		  0 &   0 & R_n
	\end{bmatrix}
\end{eqnarray}
with $R_n^{-1} = dt^2k(1+\tau/dt)$ and $R_t=\sigma_t R_n$, where the
dimensionless parameter $\sigma_t$ allow us to control the amount of
regularization in the tangential direction.

In this case the convex set is the friction cone $\mathcal{C} = \mathcal{F}$ and
the projection can be computed in closed form as

\begin{equation}
	\bgamma = P_\mathcal{F}(\vf{y}) = 
\begin{dcases}
	% Region I, stiction
	\vf{y} 
	% When we  have:
	& \text{stiction, } y_r < \mu y_n\\
	%
	%
	% Region II, sliding.
	\begin{bmatrix}
		\mu\gamma_n\hat{\vf{t}}\\
		\frac{1}{1+\tilde\mu^2}\left(y_n +
	\mu\frac{R_t}{R_n}y_r\right)
	\end{bmatrix}
	% When we  have:
	& \text{sliding, } -\mu \frac{R_t}{R_n} y_r < y_n \leq \frac{y_r}{\mu}\\
	%
	%
	% Region III, no contact.
    \vf{0} & \text{no contact, } y_n \leq -\mu \frac{R_t}{R_n} y_r
\end{dcases}	  
	\label{eq:contact_projection}
\end{equation}
where $\vf{y}_t$ and $y_n$ are the tangential and normal components of $\vf{y}$,
the radial component is defined as $y_r=\Vert\vf{y}_t\Vert$ and the tangent
vector as $\hat{\vf{t}}=\vf{y}_t/y_r$. We also defined the common dimensionless
factors $\tilde\mu=\mu\,(R_t/R_n)^{1/2}$ and $\hat\mu=\mu\,R_t/R_n$.


\section{Physical Intuition, Principle of Maximum Dissipation and Artifacts}
\label{sec:physical_intuition}

\todo{you might probably start a subsection here on the physical insight for these equations?}

To gain physical insight on what exaclty these forces are modeling, we
substitute $\vf{y}=\mf{R}^{-1}(\vf{v}_c - \hat{\vf{v}}_c)$ into Eq.
(\ref{eq:contact_projection}) to obtain an expression of the impulse as a
function of signed distance $\phi$ and normal and tangential velocities, $v_n$
and $\vf{v}_t$

\begin{equation}
	\bgamma = P_\mathcal{F}(\vf{y}) = 
\begin{dcases}
	% Region I, stiction
	\begin{bmatrix}
		-\vf{v}_t/R_t\\
		-dt(k\phi + c v_n)
	\end{bmatrix}
	% When we  have:
	& \text{stiction, } y_r < \mu y_n\\
	%
	%
	% Region II, sliding.
	\begin{bmatrix}
		\mu\gamma_n\hat{\vf{t}}\\
		-\frac{dt}{1+\tilde\mu^2}\left(k(\phi-(dt+\tau)\mu\Vert\vf{v}_t\Vert) + cv_n \right)
	\end{bmatrix}
	% When we  have:
	& \text{sliding, } -\mu \frac{R_t}{R_n} y_r < y_n \leq \frac{y_r}{\mu}\\
	%
	%
	% Region III, no contact.
    \vf{0} & \text{no contact, } y_n \leq -\mu \frac{R_t}{R_n} y_r
\end{dcases}	  
	\label{eq:gamma_piecewise}
\end{equation}
where $\phi= \phi_0 + dt\,v_n$.

From this expanded solution we see that friction forces behave exactly as a
model of regulirized friction
\begin{equation}
	\bgamma_t = -\min\left(\frac{\Vert\vf{v}_t\Vert}{R_t}, \mu\gamma_n\right)\hat{\vf{t}}
\end{equation}
That is, in the \textit{stiction} regime, regularized friction behaves as a high
viscosity fluid and it satisfies the maximum dissipation principle during
\textit{sliding}. This expression show us the convenience of having a separate
regularization coefficient $R_t$ for the tangential direction. It allows us to
control the stiction approximation separately from the compliance in the normal
direction introduced by $R_n$. To better model stiction we make the choice
$R_t=\sigma_t R_n$, with $\sigma_t \ll 1$, typically $\sigma_t=10^{-3}$ in our
simulations.

In the stiction region, we see that the normal forces model compliant contact
with linear stiffness $k$ and linear dissipation $c$. 

In the sliding region we see however that the convex approximation introduces
unphysical artifacts. Recall we are interested in the limit $R_t \ll R_n$ to
model better stiction and therefore $\tilde\mu \rightarrow 0$. This makes the
factor $1+\tilde{\mu}^2$ close to one and we can ignore it in our analysis.
While we'd like to recover $\gamma_n = -dt(k\phi + c v_n)$ as in stiction, we
instead see that the slip velocity $\Vert\vf{v}_t\Vert$ unphysically couples
into the normal forces as $\gamma_n=-dt(k(\phi-(dt+\tau)\mu\Vert\vf{v}_t\Vert) +
cv_n)$. 

This is consistent with the formulation in \cite{bib:anitescu2010} for rigid
contact when $k\rightarrow \infty$ leading to an unphysical \textit{gliding
effect} at a positive distance $\phi=dt\mu\Vert\vf{v}_t\Vert$. Notice that
\textit{gliding} goes away as $dt\rightarrow 0$ since the formulation converges
to the original contact problem \cite{bib:anitescu2006}. The effect of
compliance is to \textit{soften} this effect. 

In the limit to $dt\rightarrow 0$ we find out that the convex approximation
models, when sliding, the force law
$\gamma_n=-dt(k(\phi-\tau\mu\Vert\vf{v}_t\Vert)$. This tells us that, unlike the
rigid case, the \textit{gliding} effect unfortunately does not go away as
$dt\rightarrow 0$ but it persists with a finite value that now depends on the
dissipation rate, $\phi\approx\tau\mu\Vert\vf{v}_t\Vert$.

We close this discussion by making the following remarks relevant to robotics
applications:
\begin{enumerate}
	\item In robotics applications we are mostly interested in the stiction
	regime, typically for grasping, locomotion or rolling contact for mobile
	bases with wheels. This regime is precisely where the convex approximation
	does not introduce artifacts.
	\item Sliding is usually avoided or if importnat, it is usually at low
	velocities and therefore the term $dt\mu\Vert\vf{v}_t\Vert$ is negligible.
	\item Even though the effect of sliding cannot be neglected in the
	dissipation term during stiction, we point out that for robotics
	applications most often dissipation is high to model a zero restitution
	coefficient. Therefore the high dissipative nature of this term makes the
	effect of $\Vert\vf{v}_t\Vert$ negligible for most cases with low slip
	velocities.
	\item For robotics, we are definitely interested on the onset to sliding.
	This is captured by the approximation which properly models the Colulomb
	friction law.
\end{enumerate}

\section{Rigid Contact}
\todo{explain how Todorov's choice leads to an unstable time stepping schem while this scheme is unconditionally stable.}

Todorov in \cite{bib:todorov2014} sets the regularization parameters as
$R_i=\varepsilon N_{ii}$, where $\varepsilon$ is a small dimensionless
coefficient that controls the amount of regularization. Higher regularization
will make the problem better conditioned though at the expense of large
compliance and a poor stiction approximation. Lower regularization will converge
to the \textit{hard constraints} limit, however leading to a poorly conditioned
formulation. 

In this work we choose $\varepsilon$ from an analysis of the time scales
introduced by this regularization. The basic idea is that if the time scales
introduced by this numerical compliance cannot be resolved by a given time step
size $dt$, the system is effectively rigid. It essentially makes no difference
on the results if regularization is decreased beyond this point. 

For each contact point we define $g_i=\Vert\mathbf{W}_{ii}\Vert/3$ where
$\mathbf{W}_{ii}$ is the $3\times 3$ diagonal block of the Delassus operator
$\mathbf{W}$. The factor $3$ in our definition is so that $g_i$ is the RMS value
of the entries of $\mathbf{W}_{ii}$. This definition ensures that $g_i > 0$.
$g_i$ has unit of $\text{kg}^{-1}$ and represents the inverse of an effective
mass $m_i=g_i^{-1}$ for the $i\text{-th}$ contact. For instance, for the contact
between a point mass $m$ and the ground we have $g_i=(3m)^{-1}$ and $m_i=3m$,
see Section \ref{sec:conveyor_belt}. 

Regularization in the normal direction introduces a numerical compliance of
stiffness $k$, see Section \ref{sec:physical_intuition}, and this will therefore
induce a dynamics with a natural frequency in the order of $\omega_n^2=k/m_i$.
This relates to the period $T_n$ of this dynamics by $\omega_n=2\pi/T_n$.

Since the time stepping scheme will not resolve time scales in the order or
below the time step $dt$, we want to choose our regularization parameters so
that the numerical dynamics introduced cannot be resolved. We then make $T_n =
\alpha dt$, with alpha $\alpha \approx 1.0$. Therefore we need for the stiffness
$k=4\pi^2m_i/(\alpha^2 dt^2)$. We learned in Section
\ref{sec:physical_intuition} that $R_n=(dt^2\,k)^{-1}$ and therefore we arrive
to the final expression for our regularization in the normal direction
\begin{equation}
	R_n = \frac{\alpha^2}{4\pi^2}g_i = \frac{\alpha^2}{4\pi^2}\Vert\mathbf{W}_{ii}\Vert
\end{equation}

This is the same as Todorov's regularization taking $\varepsilon =
\alpha^2/(4\pi^2)\approx 0.025\,\alpha^2$. This analysis shows us that:
\begin{itemize}
	\item We can use a single parameter $\varepsilon$, independent of the time
	step, that leads to essentially the same numerically introduced artificial
	dynamics. 
	\item When $\alpha=1.0$ we have $\varepsilon=0.025$ which introduces a fair
	amount of regularization into the system. However the numerical time scales
	are only within $dt$ and are underresolved, as desired.
\end{itemize}

It is useful to estimate the amount of penetration for a point mass resting on
the ground. In this case we have
\begin{eqnarray}
	\phi &=& \frac{m\,g}{k} \nonumber\\
	&=& \frac{\alpha^2}{4\pi^2}\frac{m}{m_i}\,g\,dt^2\nonumber\\
	&=& \frac{\alpha^2}{12\pi^2}\,g\,dt^2
\end{eqnarray}

Taking $\alpha=1.0$ and on Earth's gravity, a typical simulation time step of
$dt=10^{-3}~\text{s}$ leads to $\phi\approx 8.3\times 10^{-8}~\text{m}$ and
using a very large simulation time tep of $dt=10^{-2}~\text{s}$ leads to
$\phi\approx 8.3\times 10^{-6}~\text{m}$, well within acceptable bounds for
typical robotics applications.

There is another reason to choose $\alpha\approx 1.0$. As we will see in the
example of Section \ref{sec:conveyor_belt}, the numerical dynamics is close to
being critically damped and numerical oscillations due to regularization are
damped out within a few time steps, independent of step size (see Fig.
\ref{fig:normal_velocity}). This is an expected result if we consider the
stability analysis on an implicit Euler scheme, which resembles this formulation
very closely when we think of it as an implicit scheme on a multibody system
with the compliant forces  in Section \ref{sec:physical_intuition}. This is a
very desired effect for us being interested on robotics applications with
perfectly inelastic contact.

