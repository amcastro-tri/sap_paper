\section{Convex Approximation of Contact Dynamics}
\label{sec:previous_work}

Previous work from Anitescu introduces a \textit{convex approximation} of the
contact problem in \cite{bib:anitescu2006}. In this approximation, the contact
impulses are solutions to the following convex optimization problem
\begin{eqnarray}
	\min_{\gamma\in \mathcal{F}} \ell(\bgamma) =
	\frac{1}{2}\bgamma^T\mathbf{W}\bgamma + {\bm r}^T \bgamma,
	\label{eq:dual_cost}
\end{eqnarray}
where $\mathbf{W} =
\mathbf{J}\mathbf{M}^{-1}\mathbf{J}^T\in\mathbb{R}^{3n_c\times 3n_c}$ is the
\emph{Delassus operator}, $\mathcal{F} := \mathcal{F}_1 \times \mathcal{F}_2
\times \cdots \times \mathcal{F}_{n_c}$ is a Cartesian product of friction
cones, and ${\bm r}$ defines a linear cost that encodes external force
contributions and stabilization terms used to impose non-penetration at the
position level. This convex approximation implies the modified non-penetration
condition
\begin{equation}
	0 \le \phi_i - \delta t \Vert {\bm v}_{t,i} \Vert \perp \gamma_{n,i} \ge 0.
	\label{eq:convex_approximation_complementarity_condition}
\end{equation}
which leads to sliding objects gliding at a distance $\phi_i = \delta
t\Vert\vf{v}_{t,t}\Vert$, an artifact of the convex approximation. 

Though not strictly applicable to contact problems, Todorov
\cite{bib:todorov2011, bib:todorov2014} uses the \emph{Gauss's principle of
least constraint} to obtain the following regularized form of Anitescu's
formulation
\begin{eqnarray}
	\min_{\gamma\in \mathcal{F}} \ell(\bgamma) =
	\frac{1}{2}\bgamma^T(\mathbf{W}+\mathbf{R})\bgamma + {\bm r}^T \bgamma,
	\label{eq:dual_regularized}
\end{eqnarray}
where $\mathbf{R}$ is a diagonal positive matrix introduced to make
$\mathbf{W}+\mathbf{R}\succ 0$ since in general we only have $\mathbf{W} \succeq
0$. This makes the problem strictly convex and thus a unique solution exists.
Regularization is used as a means to add constraint stabilization with a set of
global parameters that control the amount of numerical compliance introduced by
the formulation. This is different from our approach described in Section
\ref{sec:physical_intuition}, where we show how to use regularization to model
physical compliance, with well defined physical parameters.
