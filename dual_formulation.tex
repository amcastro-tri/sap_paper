\section{Convex Approximation of Contact Dynamics}
\label{sec:previous_work}

To improve computational tractability and ensure existence of a solution,
Anitescu introduces a \textit{convex approximation} of the contact problem
\cite{bib:anitescu2006}. In this approximation, the contact impulses are
solutions to the following convex optimization problem
\begin{eqnarray}
	\min_{\gamma\in \mathcal{F}} \ell(\bgamma) =
	\frac{1}{2}\bgamma^T\mathbf{W}\bgamma + {\bm r}^T \bgamma,
	\label{eq:dual_cost}
\end{eqnarray}
where $\mathbf{W} =
\mathbf{J}\mathbf{M}^{-1}\mathbf{J}^T\in\mathbb{R}^{3n_c\times 3n_c}$ is the
\emph{Delassus operator}, $\mathcal{F} := \mathcal{F}_1 \times F_2 \times \cdots
\times \mathcal{F}_{n_c}$ is a Cartesian product of friction cones, and ${\bm
r}$ defines a linear cost that encodes external force contributions and
stabilization terms used to impose non-penetration at the position level.

Further, \cite{bib:anitescu2006} uses a polyhedral approximation to linearize
the friction cone constraint $\bgamma_i\in\mathcal{F}_i$. In this work, we do
not linearize the cone constraints, but work directly with the second order cone
constraints. This approach is preferred given that the polyhedral approximation
is known to introduce non-physical anisotropy \cite{bib:li2018implicit}. In
addition, the linearization of the friction cone results in a far larger problem
due to the additional constraints needed to represent the polyhedral cone.

Anitescu shows in \cite{bib:anitescu2006} that the optimality conditions for the
problem in Eq. (\ref{eq:dual_cost}) imply the conservation of momentum in Eq.
(\ref{eq:momentum_balance}), the maximum dissipation principle, and the modified
non-penetration condition
\begin{equation}
	0 \le \phi_i - \delta t \Vert {\bm v}_{t,i} \Vert \perp \gamma_{n,i} \ge 0,
	\label{eq:convex_approximation_complementarity_condition}
\end{equation}
where $\vf{v}_{t,i}$ is the tangential component of the contact velocity
$\vf{v}_{c,i} = [\vf{v}_{t,i}\,v_{n,i}]$. Notice that the modified
non-penetration condition in Eq.
(\ref{eq:convex_approximation_complementarity_condition}) introduces coupling
with the sliding velocity, an artifact of the convex approximation. We provide a
detailed discussion on the physical validity of this approximation in Section
\ref{sec:physical_intuition}, along with guidelines for determining its
applicability to robotic simulation.

In \cite{bib:todorov2011, bib:todorov2014} Todorov introduces regularization to
the formulation in Eq. (\ref{eq:dual_cost}). Though not strictly applicable to
contact problems, the \emph{Gauss's principle of least constraint} is used to
obtain the following regularized form of Anitescu's formulation
\begin{eqnarray}
	\min_{\gamma\in \mathcal{F}} \ell(\bgamma) =
	\frac{1}{2}\bgamma^T(\mathbf{W}+\mathbf{R})\bgamma + {\bm r}^T \bgamma,
	\label{eq:dual_regularized}
\end{eqnarray}
where $\mathbf{R}$ is a diagonal positive matrix introduced to make
$\mathbf{W}+\mathbf{R}\succ 0$ since in general we only have $\mathbf{W} \succeq
0$. This makes the problem strictly convex and thus a unique solution exists.
Regularization is used as a means to add constraint stabilization with a set of
global parameters that control the amount of numerical compliance introduced by
the formulation. This is different from our approach described in Section
\ref{sec:physical_intuition}, where we show how to use regularization to model
physical compliance, with well defined physical parameters.
