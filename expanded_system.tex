\RedHighlight{See where to put these. Super important. For now I'll place them
here cause I don't wanna forget.}

The Euclidean projection $P_{\mathcal{F}_{\tilde{\mu}}}$ can be written as
	\begin{equation}
		\tilde{\bgamma} = P_{\mathcal{F}_{\tilde{\mu}}}(\tilde{\vf{y}}) = 
	\begin{dcases}
		% Region I, stiction
		\tilde{\vf{y}}
		% When we  have:
		& \text{stiction, } \tilde{y}_r < \tilde{\mu} \tilde{y}_n\\
		%
		%
		% Region II, sliding.
		\begin{bmatrix}
			\tilde{\mu}\tilde{\gamma}_n\hat{\vf{t}}\\
			\frac{1}{1+\tilde\mu^2}\left(\tilde{y}_n +
		\tilde\mu\tilde{y}_r\right)
		\end{bmatrix}
		% When we  have:
		& \text{sliding, } -\tilde\mu \tilde{y}_r < \tilde{y}_n \leq \tilde{y}_r/\tilde\mu\\
		%
		%
		% Region III, no contact.
		\vf{0} & \text{no contact, } \tilde{y}_n \leq -\tilde\mu\tilde{y}_r
	\end{dcases}	  
		\label{eq:projection_tilde}
	\end{equation}
	
With the help of the Eqs.
(\ref{eq:yt_derivatives})-(\ref{eq:tangential_projections}) the gradient can be
computed directly from Eq. (\ref{eq:projection_tilde})

\begin{equation}
	\sGtilde = 
	\begin{dcases}
		% Region I, stiction
		\mf{I}_3
		% When we  have:
		& \text{stiction, } \tilde{y}_r < \tilde{\mu} \tilde{y}_n\\
		%
		%
		% Region II, sliding.
		\frac{1}{1+\tilde\mu^2}
		\begin{bmatrix}
			\tilde\mu^2\mf{P} & \tilde\mu\sthat\\
			\tilde\mu\sthat^T & 1
		\end{bmatrix} +
		\frac{\tilde\mu\tilde\gamma_n}{\tilde{y}_r}
		\begin{bmatrix}
			\mf{P}^\perp & \mf{0}\\
			\mf{0} & 0
		\end{bmatrix}
		% When we  have:
		& \text{sliding, } -\tilde\mu \tilde{y}_r < \tilde{y}_n \leq \tilde{y}_r/\tilde\mu\\
		%
		%
		% Region III, no contact.
		\mf{0}_3 & \text{no contact, } \tilde{y}_n \leq -\tilde\mu\tilde{y}_r
	\end{dcases}
	\label{eq:Gtilde}
\end{equation}
We will show below that $\sGtilde\succeq 0$ even in the sliding regime.

\begin{lemma}
In the sliding region, the gradient $\sGtilde$ can be decomposed as the linear
combination of two projections.
\end{lemma}
\begin{proof}
We write Eq. (\ref{eq:Gtilde}) as
	\begin{equation}
		\sGtilde = \mf{P}_1 + \lambda_3 \mf{P}_3
	\end{equation}	
	where
	\begin{eqnarray}
		\mf{P}_1 &=& 
		\frac{1}{1+\tilde\mu^2}
			\begin{bmatrix}
				\tilde\mu^2\mf{P} & \tilde\mu\sthat\\
				\tilde\mu\sthat^T & 1 \end{bmatrix}\\
		\mf{P}_3 &=& 		
			\begin{bmatrix}
				\mf{P}^\perp & \mf{0}\\
				\mf{0} & 0
			\end{bmatrix}
	\end{eqnarray}
are orthogonal projections along directions $\vf{v}_1$ and $\vf{v}_3$
respectively with
\begin{eqnarray}
	% First
	\vf{v}_1 =& \frac{1}{(1+\tilde\mu^2)^{1/2}}
	\begin{bmatrix}
		\tilde\mu\sthat\\
		1 \end{bmatrix},\\
	%
	% Third
	\vf{v}_3 =& 
	\begin{bmatrix}
		\sthat^\perp\\
		0 \end{bmatrix},
\end{eqnarray}
and $\lambda_3 = \tilde\mu\tilde\gamma_n/\tilde{y}_r$.
\end{proof}
		
%%%%%%%%%%%%%%%%%%%%%%%%%%%%%%%%%%%%%%%%%%%%%%%%%%%%%%%%%%%%%%%%%%%%%%%%%%%%%%%%
%%%%%%%%%%%%%%%%%%%%%%%%%%%%%%%%%%%%%%%%%%%%%%%%%%%%%%%%%%%%%%%%%%%%%%%%%%%%%%%%

\begin{theorem}
During sliding, the eigenvectors $\vf{v}_i$ and eigenvalues $\lambda_i$ of
$\sGtilde$ are
\begin{eqnarray}
	% First
	\vf{v}_1 =& \frac{1}{(1+\tilde\mu^2)^{1/2}}
	\begin{bmatrix}
		\tilde\mu\sthat\\
		1 \end{bmatrix},&\quad\lambda_1 = 1\nonumber\\
	%
	% Second
	\vf{v}_2 =& \frac{1}{(1+\tilde\mu^2)^{1/2}}
	\begin{bmatrix}
		\sthat\\
		-\tilde\mu \end{bmatrix},&\quad\lambda_2 = 0\nonumber\\
	%
	% Third
	\vf{v}_3 =& 
	\begin{bmatrix}
		\sthat^\perp\\
		0 \end{bmatrix},&\quad\lambda_3 =
	\frac{\tilde\mu\tilde\gamma_n}{\tilde{y}_r}
	\label{eq:Gtilde_eigenvectors}
\end{eqnarray}
\end{theorem}
\begin{proof}
We can prove this theorem geometrically. Notice that $\vf{v}_1$ is a vector
along the wall of the cone. Geometrically, projections of vectors lying on the
wall of the cone should give the same vector, and thus $\lambda_1=1$. On the
ohter hand $\vf{v}_2$ is a vector perpendicular to the wall of the cone.
Therefore its proejction is zero, and thus $\lambda_2=0$. Finally, notice
$\vf{v}_3 = \vf{v}_1\times\vf{v}_2$ is a vector tangent to the circumference of
the cone for a fixed $\tilde\gamma_n$. Infinitesimal variations along this
direction at a radial distance $\tilde{y}_r$ project onto the wall of the cone
with radius $\tilde\mu\tilde\gamma_n$, therefore
$\lambda_3=\tilde\mu\tilde\gamma_n/\tilde{y}_r$.

We can verify that indeed $\sGtilde\vf{v}_i=\lambda_i\vf{v}_i$ algebraically.
\end{proof}
\begin{corollary}
$\sGtilde$ is positive semi-definite.
\end{corollary}
\begin{proof}
In the stiction and no-contact regions this is trivially true. In the sliding
region we use the fact that eigenvalues $\lambda_i$ are always positive.
\end{proof}

%%%%%%%%%%%%%%%%%%%%%%%%%%%%%%%%%%%%%%%%%%%%%%%%%%%%%%%%%%%%%%%%%%%%%%%%%%%%%%%%
%%%%%%%%%%%%%%%%%%%%%%%%%%%%%%%%%%%%%%%%%%%%%%%%%%%%%%%%%%%%%%%%%%%%%%%%%%%%%%%%

\begin{theorem}
	The gradient of the regularizer can be written as
	\begin{equation}
		\nabla_\mf{v}\ell_R = -\mf{J}^T\bgamma
		\label{eq:ellR_gradient}
	\end{equation}
\end{theorem}
\begin{proof}
	I believe you can simply use the optimality conditions of the primal for a
	much simpler and cleaner proof. No need to go through cumbersome
	algebra?\todo{do this.}
\end{proof}

\begin{lemma}
	Using the chain rule we can prove that $\mf{G} = \nabla_{\vf{y}}\bgamma =
	\ssqrtRinv\sGtilde\ssqrtR$.
	\label{lemma:PPtilde}
\end{lemma}

%%%%%%%%%%%%%%%%%%%%%%%%%%%%%%%%%%%%%%%%%%%%%%%%%%%%%%%%%%%%%%%%%%%%%%%%%%%%%%%%
%%%%%%%%%%%%%%%%%%%%%%%%%%%%%%%%%%%%%%%%%%%%%%%%%%%%%%%%%%%%%%%%%%%%%%%%%%%%%%%%

Lemma \ref{lemma:PPtilde} will help us proof the following theorem.
\begin{theorem}
	The Hessian of the regularizer can be written as
	\begin{equation}
		\nabla_\mf{v}^2\ell_R = \sJT\ssqrtRinv\,\sGtilde\,\ssqrtRinv\sJ
		\label{eq:ellR_hessian_from_Ptilde}
	\end{equation}
\end{theorem}
\begin{corollary}
	$\nabla_\mf{v}^2\ell_R\succeq 0 $ since $\sGtilde \succeq 0$.
\end{corollary}
\begin{proof}
This simply entails using the chain rule in Eq. (\ref{eq:ellR_gradient}) 
\begin{eqnarray}
	\mf{H}_R = \nabla_\mf{v}^2\ell_R &=& -\sJT
	\nabla_{\vf{v}_c}\bgamma\,\sJ\nonumber\\
	&=&-\sJT \nabla_{\vf{y}}\bgamma\nabla_{\vf{v}_c}\vf{y}\,\sJ\nonumber\\
	&=&\sJT \nabla_{\vf{y}}\bgamma\sRinv\,\sJ
\end{eqnarray}
and substitution of the result in Lemma \ref{lemma:PPtilde} allows us to obtain
 Eq. (\ref{eq:ellR_hessian_from_Ptilde}).
\end{proof} 

%%%%%%%%%%%%%%%%%%%%%%%%%%%%%%%%%%%%%%%%%%%%%%%%%%%%%%%%%%%%%%%%%%%%%%%%%%%%%%%%
%%%%%%%%%%%%%%%%%%%%%%%%%%%%%%%%%%%%%%%%%%%%%%%%%%%%%%%%%%%%%%%%%%%%%%%%%%%%%%%%

\begin{theorem}
The root square of gradient $\sGtilde$ is 
\begin{equation}
	\sGtilde^{1/2} = \mf{P}_1 + \lambda_3^{1/2}\mf{P}_3
\end{equation}
\end{theorem}
\begin{proof}
The eigenvectors of $\sGtilde$ form an orthonormal basis of $\mathbb{R}^3$ and
therefore we can diagonalize $\sGtilde$ via an orthogonal transformation. In
this case, we can write the root square as
\begin{equation}
	\sGtilde = \sum_{i=1}^{3} \lambda_i\vf{v}_i\otimes\vf{v}_i
	\label{eq:toot_square_from_eigenvalues}
\end{equation}
Direct substitution of Eq. (\ref{eq:Gtilde_eigenvectors}) into Eq.
(\ref{eq:toot_square_from_eigenvalues}) completes the proof.
\end{proof}


%%%%%%%%%%%%%%%%%%%%%%%%%%%%%%%%%%%%%%%%%%%%%%%%%%%%%%%%%%%%%%%%%%%%%%%%%%%%%%%%
%%%%%%%%%%%%%%%%%%%%%%%%%%%%%%%%%%%%%%%%%%%%%%%%%%%%%%%%%%%%%%%%%%%%%%%%%%%%%%%%


\begin{theorem}
We can write the Hessian of the regularizer as
\begin{equation}
	\mf{H}_R=\sJbarT\sJbar
\end{equation}
where $\sJbar=\sB\,\sJ$ and $\sB = \sGtilde^{1/2}\,\ssqrtRinv$.
\end{theorem} 
\begin{proof}
We use substitute $\sGtilde=\sGtilde^{1/2}\sGtilde^{1/2}$ in Eq.
(\ref{eq:ellR_hessian_from_Ptilde}) to write
\begin{eqnarray}
	\nabla_\mf{v}^2\ell_R &=&
	\sJT\ssqrtRinv\,\sGtilde^{1/2}\,\sGtilde^{1/2}\,\ssqrtRinv\sJ\nonumber\\
	&=& (\sGtilde^{1/2}\,\ssqrtRinv\sJ)^T\,(\sGtilde^{1/2}\,\ssqrtRinv\sJ)
\end{eqnarray}
where we used the fact that both $\sGtilde^{1/2}$ and $\ssqrtRinv$ are both
symmetric.
\end{proof}

\subsection{The Expanded System}
If we define $\Delta\mf{z}=-\sJbar\Delta\mf{dv}$ we can write the original
Newton iteration from Eq. (\ref{eq:Newton_iteration}) in expanded form as
\begin{eqnarray}
	\begin{bmatrix}
		\sM & -\sJbarT\\
		\sJbar & \mf{I}
	\end{bmatrix}
	\begin{bmatrix}
		\Delta\mf{v}\\
		\Delta\mf{z} \end{bmatrix}=
	\begin{bmatrix}
		-\nabla_\mf{v}\ell_p\\
		\mf{0}
	\end{bmatrix}	
\end{eqnarray}
