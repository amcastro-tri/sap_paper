% Dummy comment for Reviewable.

\subsection{Conditioning of the Problem}
\label{sec:understanding_model_parameters}

Regularization parameters not only determine the physical model, but also affect
the robustness and performance of the SAP solver. Modeling near-rigid objects
and avoiding viscous drift during stiction require very small values of $R_t$
and $R_n$ that can lead to badly ill-conditioned problems. Under these
conditions, the Hessian of the system exhibits a large condition number, and
round-off errors can render the search direction of Newton iterations useless.
We show in this section how a judicious choice of the regularization parameters
leads to much better conditioned system of equations, without sacrificing
accuracy. This is demonstrated in Section \ref{sec:test_cases} with a variety of
tests cases.

\textbf{Near-Rigid Contact}. With the formulation presented in this work, all
bodies are modeled as compliant. Therefore, rigid objects must be modeled as
\emph{near-rigid} bodies with large stiffness. However, as mentioned above,
blindly choosing large values of stiffness can lead to ill-conditioned systems
of equations. Here, we propose a principled way to choose the stiffness
parameter when modeling near-rigid contact.

Consider the dynamics of a mass particle $m$ laying on the ground, with contact
stiffness $k$ and dissipation time scale $\tau_d$. When in contact, the dynamics
of this particle is described by the equations of a harmonic oscillator with
natural frequency $\omega_n^2 = k/m$, or period $T_n = 2\pi/\omega_n$, and
damping ratio $\zeta=\tau_d\omega_n/2$. We say the contact is \emph{near-rigid}
when $T_n \lesssim \delta t$ and the time step $\delta t$ cannot temporally
resolve the contact dynamics. 

In this \emph{near-rigid} regime, we use compliance as a means to add a
Baumgarte-like \emph{stabilization} to avoid constraint drift, as similarly done
in \cite{bib:todorov2011}. Choosing the time scale of the contact to be $T_n =
\beta \delta t$ with $\beta \le 1$, we model inelastic contact with a
dissipation that leads to a critically damped oscillator, or $\zeta=1$. This
dissipation is $\tau_d=2\zeta/\omega_n$, or in terms of the time step,
\begin{equation*}
    \tau_d=\frac{\beta}{\pi}\delta t.
\end{equation*}

Using the harmonic oscillator equations, we can estimate the value of stiffness
from the frequency $\omega_n$ as $k=4\pi^2 m/(\beta^2 \delta t^2)$. Since
$\tau_d\approx\delta t$, $R_n^{-1} = \delta t k(\delta t+\tau_d) \approx \delta
t^2k$, and we estimate the regularization parameter as
\begin{equation*}
	R_n = \frac{\beta^2}{4\pi^2}\text{w},
\end{equation*}
where we defined $\text{w}=1/m$.

It is useful to estimate the amount of penetration for a point mass resting on
the ground. In this case we have
\begin{align*}
	\phi &= \frac{mg}{k}, \\
	&= \frac{\beta^2}{4\pi^2}m\text{w}g\delta t^2,\\
	&= \frac{\beta^2}{4\pi^2}g\delta t^2,
\end{align*}
independent of mass. Taking $\beta=1.0$ and Earth's gravitational constant, a
typical simulation time step of $\delta t=10^{-3}~\text{s}$ leads to
$\phi\approx 2.5\times 10^{-7}~\text{m}$, and a large simulation time step of
$\delta t=10^{-2}~\text{s}$ leads to $\phi\approx 2.5\times 10^{-5}~\text{m}$,
well within acceptable bounds to consider a body rigid for typical robotics
applications.

For a general multibody system, we define the per-contact effective mass as
$\text{w}_i=\Vert\mathbf{W}_{ii}\Vert_\text{rms}=\Vert\mathbf{W}_{ii}\Vert/3$
where $\mathbf{W}_{ii}$ is the $3\times 3$ diagonal block of the Delassus
operator $\mathbf{W}=\mf{J}\mf{M}^{-1}\mf{J}^T$ for the $i$-th contact.
Explicitly forming the Delassus operator is an expensive operation. Instead we
use an $\mathcal{O}(n)$ approximation. Given contact $i$ involving trees $t_1$
and $t_2$, we form the approximation $\mathbf{W}_{ii}\approx\mf{J}_{i
t_1}\mf{M}_{t_1}^{-1}\mf{J}_{i t_1}^T+\mf{J}_{i t_2}\mf{M}_{t_2}^{-1}\mf{J}_{i
t_2}^T$. Using this approximation, we estimate the frequency of the contact
dynamics as $\omega_n=\sqrt{k\text{w}_i}$.

Finally, we compute the regularization parameter in the normal direction as
\begin{eqnarray}
    R_n = \max\left(\frac{\beta^2}{4\pi^2}\Vert\mathbf{W}_{ii}\Vert_\text{rms},
    \frac{1}{\delta t k(\delta t+\tau_d)}\right)
    \label{eq:normal_regularization}.
\end{eqnarray}
With this strategy, our model automatically switches between modeling compliant
contact with stiffness $k$ when the time step $\delta t$ can resolve the
temporal dynamics of the contact, and using stabilization to model near-rigid
contact with the amount of stabilization controlled by parameter $\beta$. In all
of our simulations, we use $\beta=1.0$.

\textbf{Stiction}. Given that our model regularizes friction, we are interested
in estimating a bound on the slip velocity at stiction. We propose the following
regularization for friction
\begin{equation}
    R_t = \sigma \text{w},
    \label{eq:tangential_regularization}
\end{equation}
where $\sigma$ is a dimensionless parameter.

To understand the effect of $\sigma$ in the approximation of stiction, we
consider once again a point of mass $m$ in contact with the ground under
gravity, for which $\text{w}\approx 1/m$. We push the particle with a horizontal
force of magnitude $F=\mu\gamma_n$ so that friction is right at the boundary of
the friction cone and the slip velocity due to regularization, $v_s$, is
maximized. Then in stiction, we have
\begin{equation*}
    \|\bgamma_t\| = \frac{v_s}{R_t} = \mu m g \delta t.
\end{equation*}
Using our proposed regularization in Eq. (\ref{eq:tangential_regularization}),
we find the maximum slip velocity
\begin{equation}
    v_s \approx \mu\sigma g \delta t,
    \label{eq:slip_estimation}
\end{equation}
independent of the mass and linear with the time step size. Even though the
friction coefficient $\mu$ can take any non-negative value, most often in
practical applications $\mu < 1$. Values in the order of 1 are in fact
considered as large friction values. Therefore, for this analysis we consider
$\mu\approx 1$. In all of our simulations, we use $\sigma=10^{-3}$. With Earth's
gravitational constant, a typical simulation with time step of $\delta
t=10^{-3}~\text{s}$ leads to a stiction velocity of $v_s\approx
10^{-5}\text{m}/\text{s}$, and with a large step of $\delta t=10^{-2}~\text{s}$,
$v_s\approx 10^{-4}\text{ m}/\text{s}$. Smaller friction coefficients lead to
even tighter bounds. These values are well within acceptable bounds even for
simulation of grasping tasks, which are significantly more demanding than
simulation for other robotic applications, see Section \ref{sec:test_cases}.

\textbf{Sliding Soft Contact}. As we discussed in Section
\ref{sec:physical_intuition}, we require $R_t/R_n\ll 1$ so that we model
compliance accurately during sliding. Now, in the \emph{near-rigid} contact
regime, the condition $R_t/R_n\ll 1$ is no longer required since in this regime
regularization is only used to apply stabilization and avoid constraint drift.
Therefore, we only need to verify this condition in the \emph{soft contact}
regime, when time step $\delta t$ can properly resolve the contact dynamics,
i.e. according to our criteria, when $\delta t < T_n$. In this regime,
$R_n^{-1}\approx \delta t^2k$, and using Eq.
(\ref{eq:tangential_regularization}) we have
\begin{equation*}
    \frac{R_t}{R_n}\approx \sigma \delta t^2 \omega_n^2=4\pi^2\sigma\left(\frac{\delta t}{T_n}\right)^2
    \lesssim 4\pi^2\sigma
\end{equation*}
where in the last inequality we used the assumption that we are in the soft
regime where $\delta t < T_n$. Since $\sigma \ll 1$ and in particular we use
$\sigma=10^{-3}$ in all of our simulations, we see that $R_t/R_n \ll 1$.
Moreover, $R_t/R_n$ goes to zero quadratically with $\delta t/T_n$ as the time
step is reduced and the dynamics of the compliance is better resolved in time.

Summarizing, we have shown that our choice of regularization parameters enjoys
the following properties
\begin{enumerate}
    \item Users only provide physical parameters; contact stiffness $k$,
    dissipation time scale $\tau_d$, and friction coefficient $\mu$. There is no
    need for users to tweak solver parameters.
    \item In the \emph{near-rigid} limit, our regularization in Eq.
    (\ref{eq:normal_regularization}) automatically switches the method to model
    rigid contact with constraint stabilization to avoid excessively large
    stiffness parameters and the consequent ill-conditioning of the system.
    \item Frictional regularization is parameterized by a single dimensionless
    parameter $\sigma$. We estimate a bound for the slip velocity during
    stiction to be $v_s \approx \mu \sigma \delta t g$. For $\sigma=10^{-3}$,
    the slip during stiction is well within acceptable bounds for robotics
    applications.
    \item We show that $R_t/R_n \ll 1$ when $\delta t$ can resolve the dynamics
    of the compliant contact, as required to accurately model compliance during
    sliding.
\end{enumerate}
