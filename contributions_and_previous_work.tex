\section{Novel Contributions}

Since we build on previous work \cite{bib:anitescu2006,
bib:anitescu2010,bib:todorov2014}, it is worth summarizing our novel
contributions:
\begin{itemize}
	\item Clear algebraic expressions for the \textit{inverse dynamics} of
	contact are provided.
	\item We show how to use these expressions do model true compliant contact.
	This is different from \cite{bib:todorov2014} where regularization is used
	as a Baumgarte-like stabilization.
	\item The analasis of the analytical inverse dynamics expressions here
	provided allows to make a clear analogy with compliant contact.
	\item Our clear analogy with compliant contact allow us to succinctly
	describe the artifacts introduced by the convex approximation of contact.
	\item We write a primal formulation of compliant contact written in terms of
	velocities, in contrast to its dual in terms of impulses.
	\item We show how the primal formulation when combined with the analytical
	 inverse dynamics leads to an unconstrained convex optimization problem.
	\item We develop a custom solver for this unconstrained formulation and
	provide thorough details for a practical implementation.
	\item We conclude with a number of demonstrations illustrating the
	effectiveness of our methodology and provide future research directions.
\end{itemize}

\section{Previous Work}
\label{sec:previous_work}

The momentum equations for the dynamics of contact with friction read
\begin{eqnarray}
	\mathbf{M}\mathbf{v} = \mathbf{M}\mathbf{v}^* + \mathbf{J}^T\mathbf{\gamma}
	\label{eq:momentum_balance}
\end{eqnarray}
where $\mathbf{v}^*$ are the generalized velocities of the system in the absence
of contact forces and $\bgamma$ are the impulses in the contact space. For rigid
body dynamics the formulation is completed with the following constraints
\begin{enumerate}
	\item non-penetration constraint, $0\le\phi_i\perp\gamma_n\ge0$
	\item friction cone constraint and, $\Vert\bgamma_{t,i}\Vert\le
	\mu_i\gamma_{n,i}$
	\item principle of maximum dissipation
\end{enumerate}

Anitescu introduces a \textit{convex relaxation} of the contact problem in
\cite{bib:anitescu2006}. This relaxation is introduced as a modification to the
complementarity constraint between the signed distance $\phi$ and normal impulse
$\gamma_n$ as
\begin{equation}
	0 < \phi - h \Vert {\bm v}_t \Vert \perp \gamma_n > 0
	\label{eq:convex_relaxation_complementarity_condition}
\end{equation}
where $h$ is the discrete time step.

\RedHighlight{Sherm: Somehow mention how to jump from the "KKT" to the conic
optimization problem below.}

This relaxation leads to a convex quadratic program with conic constraints as
\begin{eqnarray}
	\min_{\gamma\in \mathcal{F}} \ell(\bgamma) =
	\frac{1}{2}\bgamma^T\mathbf{W}\bgamma + {\bm r}^T
	\bgamma
	\label{eq:dual_cost}
\end{eqnarray}
where $\bgamma$ are the contact point impulses (i.e. with units of
$\text{N}\cdot\text{s}$), $\mathbf{W} = \mathbf{J}^T\mathbf{M}^{-1}\mathbf{J}$
is the Delassus operator (the inverse of the inertia matrix projected on the
contact space), $\mathcal{F}$ is the feasible set of all contact impulses
satisfying the cone constraint and, ${\bm r} = {\bm v}^* - \hat{\mf{v}}$, with
$\hat{\mf{v}}$ a \textit{stabilization velocity}. Refer to Section
\ref{sec:constraints_based_modeling_framework} for details on how $\hat{\mf{v}}$
is used to model physical compliance.

In \cite{bib:todorov2011, bib:todorov2014} Todorov introduces regulariztion to
the formulation in Eq. (\ref{eq:dual_cost}), though the derivation is inspired
on the Gauss's principle of least constraint. The regularized formulation reads

\begin{eqnarray}
	\min_{\gamma\in \mathcal{F}} \ell(\bgamma) =
	\frac{1}{2}\bgamma^T(\mathbf{W}+\mathbf{R})\bgamma + {\bm r}^T
	\bgamma
	\label{eq:dual_regularized}
\end{eqnarray}
where $\mathbf{R}$ is a diagonal positive definite matrix that has the effect of
making $\mathbf{W}+\mathbf{R}\succ 0$ since in general we only have $\mathbf{W}
\succeq 0$. This makes the problem strictly convex, with a unique solution. 

For this regularized formulation it is shown in \cite{bib:todorov2014} that if
we do know the velocities at the next time step, we can write analytical
algebraic expressions for the impulses, i.e. $\bgamma = \bgamma(\mf{v})$. This
remarkable result is a consequence of the separable structure of the constraints
in the dual formulation. Thanks to this structure we can solve for the impulses
for each $k\text{-th}$ constraint \textit{separately} from
\begin{eqnarray}
	\bgamma_k&=&
	\begin{aligned}
		\argmin_{\bgamma\in\mathcal{C}} \quad &
	\frac{1}{2}(\bgamma-\mf{y}_k)^T\mf{R}_k(\bgamma-\mf{y}_k) \end{aligned}\\
	\mf{y}_k &=& -\vf{R}_k^{-1}(\mf{v}_{c,k}-\hat{\mf{v}}_k)
\end{eqnarray}
where $\mf{v}_{c,k}$ is the $k\text{-th}$ constraint velocity and
$\hat{\mf{v}}_k$ is a stabilization term we use to model compliance. We show a
detailed calculation of this \textit{inverse dynamics} and its gradients in
Appendix \ref{app:analytical_inverse_dynamics_derivations}.

In this work we will show in Section \ref{sec:unconstrained_convex_formulation}
how to exploit this algebraic inverse dynamics to write an unconstrained convex
formulation that can be solved with Newton's method. The resulting strategy is
accurate, robust and can be warm-started from the previous time step solution.
