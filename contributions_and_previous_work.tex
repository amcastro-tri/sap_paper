\section{Novel Contributions}
\RedHighlight{TODO: Rewrite for paper.}

Since we build on previous work \cite{bib:anitescu2006,
bib:anitescu2010,bib:todorov2014}, it is worth summarizing our novel
contributions:
\begin{itemize}
	\item Clear algebraic expressions for the \textit{inverse dynamics} of
	contact are provided.
	\item We show how to use these expressions do model true compliant contact.
	This is different from \cite{bib:todorov2014} where regularization is used
	as a Baumgarte-like stabilization.
	\item The analasis of the analytical inverse dynamics expressions here
	provided allows to make a clear analogy with compliant contact.
	\item Our clear analogy with compliant contact allow us to succinctly
	describe the artifacts introduced by the convex approximation of contact.
	\item We write a primal formulation of compliant contact written in terms of
	velocities, in contrast to its dual in terms of impulses.
	\item We show how the primal formulation when combined with the analytical
	 inverse dynamics leads to an unconstrained convex optimization problem.
	\item We develop a custom solver for this unconstrained formulation and
	provide thorough details for a practical implementation.
	\item We conclude with a number of demonstrations illustrating the
	effectiveness of our methodology and provide future research directions.
\end{itemize}

\section{Multibody Dynamics with Contact}
\label{sec:multibody_dynamics_with_contact}

We model our multibody system using generalized coordinates. Therefore the state
is fully described by the generalized positions $\mf{q}\in\mathbb{R}^{n_q}$ and
the generalized velocities $\mf{v}\in\mathbb{R}^{n_v}$. We describe the
$k\text{-th}$ contact between two bodies by the relative velocity
$\vf{v}_{c,k}\in\mathbb{R}^3$ between these two bodies at this point, expressed
in a contact frame $C_k$ for which we arbitrarily choose the $z\text{-axis}$ to
coincide with the surface normal $\hat{\vf{n}}_k$ reported by the geometry
engine. We form the vector $\mf{v}_{c}\in\mathbb{R}^{3n_c}$ of contact
velocities by stacking velocites $\vf{v}_{c,k}$ for all contacts together.
Similarly, we denote with $\bgamma_k\in\mathbb{R}^3$ the contact impulse at a
specific contact point and with $\bgamma\in\mathbb{R}^{3n_c}$ the vector of all
contact impulses. In general, unless otherwise specified, we use bold italics
for vectors in $\mathbb{R}^3$ and non-italics bold for their stacked
counterpart.

The contact Jacobian $\mf{J}(\mf{q})\in\mathbb{R}^{3n_c\times n_v}$ relates
generalized velocities to contact velocity by $\mf{v}_c=\mf{J}\,\mf{v}$.

We use a discrete time stepping approach to advance the dynamics of the system
forward in time and we write the balance of momentum as
\begin{eqnarray}
	\mathbf{M}\mathbf{v} = \mathbf{M}\mathbf{v}^* + \mathbf{J}^T\mathbf{\gamma}
	\label{eq:momentum_balance}
\end{eqnarray}
where $\mf{M}(\mf{q})\in\mathbb{R}^{n_v\times n_v}$ is the mass matrix and
$\mf{v}^*\in\mathbb{R}^{n_v}$ are the \textit{free motion} generalized
velocities when there is no contact.

Denoting with $\phi_k$ the \textit{gap function} or \textit{signed distance} at
a contact point and with $\gamma_{n,k}$ and with $\bgamma_{t,k}$ the normal and
tangential components of the impulse $\bgamma_k=[\bgamma_{t,k}\,\gamma_{n,k}]$
at that point, the formulation is most often completed with the following set of
standard constraints to be satisfied at each contact
\begin{enumerate}
	\item non-penetration constraint $0\le\phi_k\perp\gamma_{n,k}\ge0$,
	\item\label{it:cone_constraint} friction cone constraint
	$\Vert\bgamma_{t,k}\Vert\le \mu_k\gamma_{n,k}$ and,
	\item the principle of maximum dissipation.
\end{enumerate}
where $\mu_k$ is the coefficient of friction for the $k\text{-th}$ contact.
Condition \ref{it:cone_constraint} states that contact impulses at point $k$ are
constrained to belong to the friction cone
$\mathcal{F}_k=\{\vf{x}\in\mathbb{R}^3 \,|\, \Vert\vf{x}_t\Vert\le \mu_k x_n\}$.

While these constraints model the widely accepted model of Colulomb friction,
the resulting formulation leads to a very difficult to solve, non-convex,
Non-linear Complimentarity Problem. 

\RedHighlight{Alejandro: say something how other author's impose the principle
of maximum dissipation and more or less what it means.}

\RedHighlight{Alejandro: say something about
the existence of solutions for this formulation. Reference other people's work.}

Plausible good references on this are
\cite{bib:stewart1996implicit,bib:stewart2000implicit,bib:chakraborty2007implicit,bib:acary2018solving,bib:pang1999unified,bib:alart2018inconsistency}.

\section{Convex Approximation of Contact Dynamics}
\label{sec:previous_work}


In pursuit of a formulation for which the existence of a solution is guaranteed
and with the hope that such a formulation leads to more robust and efficient
solvers, Anitescu introduces a \textit{convex relaxation} of the contact problem in
\cite{bib:anitescu2006}. This relaxation is introduced as a modification to the
complementarity constraint between the signed distance $\phi$ and normal impulse
$\gamma_n$ as
\begin{equation}
	0 < \phi - dt \Vert {\bm v}_t \Vert \perp \gamma_n > 0
	\label{eq:convex_relaxation_complementarity_condition}
\end{equation}
where $dt$ is the discrete time step and $\vf{v}_t$ is the tangential component
of the contact velocity $\vf{v}_c = [\vf{v}_t\,v_n]$. Together with a set of
complementarity conditions to impose maximum dissipation on a linearized
friction cone, Anitescu shows in \cite{bib:anitescu2006} that the new set of
constraints corresponds to the optimality conditions of the following quadratic
program with conic constraints as
\begin{eqnarray}
	\min_{\gamma\in \mathcal{F}} \ell(\bgamma) =
	\frac{1}{2}\bgamma^T\mathbf{W}\bgamma + {\bm r}^T
	\bgamma
	\label{eq:dual_cost}
\end{eqnarray}
where $\mathbf{W} =
\mathbf{J}^T\mathbf{M}^{-1}\mathbf{J}\in\mathbb{R}^{3n_c\times 3n_c}$ is the
Delassus operator, $\mathcal{F}=\prod_{k=1}^{n_c}\mathcal{F}_k$ is the feasible
set of all contact impulses satisfying the friction cone constraints and, the
linear cost ${\bm r}$ encodes the free motions $\mf{v}^*$ and stabilization
terms to impose non-interpenetration at the positions level.

In \cite{bib:todorov2011, bib:todorov2014} Todorov introduces regularization to
the formulation in Eq. (\ref{eq:dual_cost}). Though not strictly applicable to
contact problems, Todorov uses the \textit{Gauss's principle of least
constraint} to obtain this convex approximation of the contact problem. The
regularized formulation reads

\begin{eqnarray}
	\min_{\gamma\in \mathcal{F}} \ell(\bgamma) =
	\frac{1}{2}\bgamma^T(\mathbf{W}+\mathbf{R})\bgamma + {\bm r}^T
	\bgamma
	\label{eq:dual_regularized}
\end{eqnarray}
where $\mathbf{R}$ is a diagonal positive definite matrix introduced to
make $\mathbf{W}+\mathbf{R}\succ 0$ since in general we only have $\mathbf{W}
\succeq 0$. This makes the problem strictly convex, with a unique solution. 

One fo the most remarkable results from \cite{bib:todorov2014} is that if we
 happen to know the velocities of the system, we can write analytical algebraic
 expressions for the impulses. Moreover, the separable structure of the
 constraints allows us to write the impulse $\bgamma_k$ at contact $k$ solely as
 a function of the relative contact velocity $\vf{v}_{c,k}$ at that point, i.e.
 $\bgamma_k = \bgamma_k(\vf{v}_{c,k})$.

In this work we will show in Section \ref{sec:unconstrained_convex_formulation}
how to exploit this \textit{analytical inverse dynamics} to write an
unconstrained convex formulation that can be solved using Newton's method. The
resulting strategy is accurate, robust and can be effectively warm-started from
the previous time step solution.
