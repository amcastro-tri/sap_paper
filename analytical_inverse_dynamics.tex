\section{Analytical Inverse Dynamics}
\label{sec:analytical_inverse_dynamics}

\sout{Given a set of $n_k$ constraints defined by their constraint velocity
$\vf{v}_{c,k}$ and impulse $\bgamma_k\in\mathcal{C}_k$, we form a global vector
$\vf{v}_c$ that concatenates all constraint velocities and define the Cartesian
product $\mathcal{C}=\prod_{k=1}^{n_k}\mathcal{C}_k$. Since each $\mathcal{C}_k$
is convex, so is $\mathcal{C}$. We proceed in a similar way with $\bgamma_k$,
$\hat{\vf{v}}_k$, $\mf{R}_k$, $\mf{J}_k$ to obtain the global quantities
$\bgamma$, $\hat{\mf{v}}$, $\mf{R}$, $\mf{J}$. With these definitions we can
compactly write the constraint $\bgamma\in\mathcal{C}$ for all impulses in the
problem.}

\textit{Inverse dynamics} refers to the computation of the impulses given we
know the velocities of the system. It is shown in \cite{bib:todorov2014} that
this problem can be solved analytically given the separable structure of the
constraints. Then the impulse for each $k\text{-th}$ constraint can be solved
from
\begin{eqnarray}
	\bgamma_k&=&
	\begin{aligned}
		\argmin_{\bgamma\in\mathcal{C}} \quad &
	\frac{1}{2}(\bgamma-\mf{y}_k)^T\mf{R}_k(\bgamma-\mf{y}_k) \end{aligned}\\
	\mf{y}_k &=& -\vf{R}_k^{-1}(\mf{v}_{c,k}-\hat{\mf{v}}_k)
\end{eqnarray}

From now on we will omit the index $k$. 