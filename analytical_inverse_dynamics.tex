\textit{Inverse dynamics} refers to the computation of the impulses given known
velocities of the system. It is shown in \cite{bib:todorov2014} that contact
impulses are solution to the convex program stated in Eq.
(\ref{eq:y_projection}). This problem can be solved analytically using simple
geometry. By noticing that the change of variables
$\tilde\bgamma=\vf{R}^{1/2}\bgamma$ (and respectively
$\tilde{\vf{y}}=\vf{R}^{1/2}\vf{y}$) leads to a projection with Euclidean norm
on a cone $\tilde{\mathcal{F}}$ with friction coefficient
$\tilde\mu=\mu\,(R_t/R_n)^{1/2}$
\begin{eqnarray}
	P_\mathcal{F_{\tilde\mu}}(\tilde{\vf{y}})&=&\argmin_{\tilde\bgamma\in\mathcal{F_{\tilde\mu}}}
		\quad \frac{1}{2}\Vert\tilde\bgamma-\tilde{\vf{y}}\Vert_2^2\nonumber\\
	P_\mathcal{F}(\vf{y}) &=&
	\vf{R}^{-1/2}P_\mathcal{F_{\tilde\mu}}(\tilde{\vf{y}})
	\label{eq:local_optimization_problem_tilde}
\end{eqnarray}

The optimization problem in the Euclidean norm given by Eq.
(\ref{eq:local_optimization_problem_tilde}) can be solved by inspection. If
$\tilde{\vf{y}}\in\tilde{\mathcal{F}}$, then we simply have $\tilde{\bgamma}
= \tilde{\vf{y}}$, we call this \textit{Region I}. If however $\tilde{\vf{y}}$
is inside the polar cone $\tilde{\mathcal{F}}^\circ$, the closest point to
$\tilde{\vf{y}}_i$ within the (tilde) friction cone is zero, i.e.
$\tilde{\bgamma} =\vf{0}$. We call this \textit{Region III}. Finally, if
$\tilde{\vf{y}}$ is in the region outside both $\tilde{\mathcal{F}}$ and its
polar $\tilde{\mathcal{F}}^\circ$, then the closest point is it's Euclidean
projection on the boundary of $\tilde{\mathcal{F}}$. We call this
\textit{Region II}. To be more precise, we define the Region I as the closed set
including the interior of the friction cone and its boundary. We define Region
III as the open set only including the interior of the polar cone. Finally
Region II is the open set that consists all points in $\mathbb{R}^3$ except
those in Regions I and III.

Figure \ref{fig:cone_regions} shows a schematic of
$\tilde{\mathcal{F}}$, $\tilde{\mathcal{F}}^\circ$ and labels the three
different regions. From Fig. \ref{fig:cone_regions}, for a cone forming an angle
$\theta$ with the z axis, we have $\tan(\theta)=\tilde\mu$ and
$\cos(\theta)=1/(1+\tilde\mu^2)$. Then the projection of a point
$\tilde{\vf{y}}$ in Region II can be written as
\begin{eqnarray}
	\tilde{\bgamma}_t &=& \tilde{\mu}\tilde{\gamma}_n\hat{\vf{t}}\nonumber\\
	\tilde{\gamma}_n &=& \frac{1}{1+\tilde{\mu}^2}\left(\tilde{y}_n +
	\tilde{\mu}\tilde{y}_r\right)\nonumber		
\end{eqnarray}
where the tangent vector is defined as
$\hat{\vf{t}}=\vf{y}_t/\Vert\vf{y}_t\Vert=-\vf{v}_t/\Vert\vf{v}_t\Vert$. 
\begin{figure}[!h]
    \centering
    %\vspace{6pt}
    \includegraphics[width=0.45\columnwidth]{figures/cone_regions.png}
    \caption{Geometry of $\tilde{\mathcal{F}}$ and regions in the
    $\tilde{\vf{y}}$ space.}
    \label{fig:cone_regions}
\end{figure}

Finally, we obtain Eq. (\ref{eq:analytical_y_projection}) by applying the transformation
$\bgamma=\mf{R}^{-1/2}\tilde\bgamma$ to recover the impulses projected onto the
friction cone $\mathcal{F}$ using the norm in $\vf{R}$.

Though the tangent vector $\hat{\vf{t}}$ is not defined for $\vf{y}=\vf{0}$, the
projection still is, $P_\mathcal{F}(\vf{0})=\vf{0}$. In practice however, we
introduce a smooth \emph{soft-norm} defined as
$\|\vf{x}\|_s=\sqrt{\|\vf{x}\|^2+\varepsilon_s^2}$ and we define the tangent
vector as $\hat{\vf{t}}=\vf{y}_t/\Vert\vf{y}_t\Vert_s$. This newly defined tangent vector is smooth and has the desired property that it leads to the right
projection result in the limit to $\vf{y}\rightarrow 0$. In addition, not only
the projection is well defined, but also its gradients in Appendix
\ref{app:gradients_derivation}. In practice we use $\varepsilon_s=10^{-8}$.

