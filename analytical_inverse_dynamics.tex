% Dummy comment for Reviewable.

\subsection{Analytical Inverse Dynamics}
\label{sec:analytical_inverse_dynamics}

The dual optimal impulses of~\eqref{eq:dual_regularized} can be
constructed from the primal optimal velocities of~\eqref{eq:primal_regularized}
using a simple projection operation. Following~\cite{bib:todorov2014}, we call
this construction  \textit{analytical inverse dynamics}. Moreover, this
projection decomposes into a set of individual projections for each contact
impulse $\bgamma_i$ given the separable structure of the constraints. Letting
$\vf{y}_i(\vf{v}_{c,i}) = -\vf{R}_i^{-1}(\vf{v}_{c,i}-\hat{\vf{v}}_{c,i})$,
these projections take the form
\begin{equation}
  \begin{aligned}
	\bgamma_i(\vf{v}_{c,i})&= P_{\mathcal{F}_i}(\vf{y}_i(\vf{v}_{c,i}))\\
	&= \argmin_{\bgamma\in\mathcal{F}_i} \quad 
		\frac{1}{2}(\bgamma-\vf{y}_i)^T\vf{R}_i(\bgamma-\vf{y}_i),\\
	\end{aligned}
	\label{eq:y_projection}
\end{equation}
where $\vf{R}_i\in\mathbb{R}^{3\times3}$ is the $i\text{-th}$ diagonal block of
the regularization matrix $\mf{R}$. That is, $\bgamma_i$ is the projection
$P_{\mathcal{F}_i}$ of $\vf{y}_i(\vf{v}_{c,i})$ onto the friction cone
$\mathcal{F}_i$ using the norm defined by $\vf{R}_i$.
\reviewquestion{R1-Q7}{Remarkably, the projection map $P_{\mathcal{F}_i}$ can be
evaluated \emph{analytically}. We provide algebraic expressions for it in
Section~\ref{sec:physical_intuition} and derivations in
Appendix~\ref{app:analytical_inverse_dynamics_derivations}.} The projection
$P_{\mathcal{F}}(\mf{y})$ onto the full cone $\mathcal{F} := \mathcal{F}_1
\times \mathcal{F}_2 \times \cdots \times \mathcal{F}_{n_c}$ is obtained by
simply stacking together the individual projections
$P_{\mathcal{F}_i}(\vf{y}_i)$ from Eq. (\ref{eq:y_projection}), where we form
$\mf{y}$ by stacking together each $\vf{y}_i$ from all contact pairs.  In this
notation, the optimal impulse $\bgamma$ of~\eqref{eq:dual_regularized} and the
optimal velocities $\mf{v}$ of~\eqref{eq:primal_regularized} satisfy  $\bgamma =
P_{\mathcal{F}}(\mf{y}(\mf{v}))$.
