\section{Analytical Inverse Dynamics}
\label{sec:analytical_inverse_dynamics}

The work in \cite{bib:todorov2014} shows that the dual in Eq.
(\ref{eq:dual_regularized}) can be solved analytically if we know the velocities
of the system. This is referred to as the \textit{inverse dynamics} solution.
Moreover, given the separable structure of the constraints, the impulse
$\bgamma_k$ at the $k\text{-th}$ contact point is solution to the following
convex optimization problem
\begin{eqnarray}
	\bgamma_k(\vf{v}_{c,k})&=& P_{\mathcal{F}_k}(\vf{y}_k(\vf{v}_{c,k})) \nonumber\\
	&=&\begin{aligned}
		\argmin_{\bgamma\in\mathcal{F}_k} \quad &
	\frac{1}{2}(\bgamma-\vf{y}_k)^T\vf{R}_k(\bgamma-\vf{y}_k) \end{aligned}
    \label{eq:y_projection}\\
	\vf{y}_k(\vf{v}_{c,k}) &=& -\vf{R}_k^{-1}(\vf{v}_{c,k}-\hat{\vf{v}}_k)    
\end{eqnarray}
where $\vf{R}_k\in\mathbb{R}^{3\times3}$ is the $k\text{-th}$ diagonal block of
the regularization matrix $\mf{R}$. That is, $\bgamma_k$ is the projection onto
the cone $\mathcal{F}_k$ using the norm defined by $\vf{R}_k$.

Dropping subscript $k$ for simplicity, we solve Eq. (\ref{eq:y_projection})
analytically in Appendix \ref{app:analytical_inverse_dynamics_derivations} for a
regularization matrix of the form
\begin{eqnarray}	
	\vf{R} &=& \text{diag}([R_t, R_t, R_n]) = 
	\begin{bmatrix}
		R_t &   0 & 0\\
		  0 & R_t & 0\\
		  0 &   0 & R_n
	\end{bmatrix}
    \label{eq:Rk}
\end{eqnarray} 

The solution can be written as
\begin{eqnarray}
	\bgamma &=& P_\mathcal{F}(\vf{y})
    \label{eq:analytical_y_projection}\\
    &=&\begin{dcases}
	% Region I, stiction
	\vf{y} 
	% When we  have:
	& \text{stiction, } y_r < \mu y_n\\
	%
	%
	% Region II, sliding.
	\begin{bmatrix}
		\mu\gamma_n\hat{\vf{t}}\\
		\frac{1}{1+\tilde\mu^2}\left(y_n +
        \hat\mu y_r\right)
	\end{bmatrix}
	% When we  have:
	& \text{sliding, } -\hat\mu y_r < y_n \leq \frac{y_r}{\mu}\\
	%
	%
	% Region III, no contact.
    \vf{0} & \text{no contact, } y_n \leq -\hat\mu y_r
\end{dcases}\nonumber	
\end{eqnarray}
where $\vf{y}_t$ and $y_n$ are the tangential and normal components of $\vf{y}$,
the radial component is defined as $y_r=\Vert\vf{y}_t\Vert$ and the tangent
vector as $\hat{\vf{t}}=\vf{y}_t/y_r$. We also defined the
coefficients $\tilde\mu=\mu\,(R_t/R_n)^{1/2}$ and $\hat\mu=\mu\,R_t/R_n$, result
of the \textit{warping} introduced by the metric $\vf{R}$.
