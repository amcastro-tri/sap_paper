\section{Conclusion}
\label{sec:future_directions}

We presented a formulation of compliant contact with a novel physics-based
parameterization. We showed that forces can be succinctly described by
analytical expressions with a clear physical intuition. This allowed us to
incorporate not only point contact but also more complex models of surface
patches. We then showed that when these forces are used in the momentum
equations, we obtain the optimality conditions for an unconstrained convex
formulation. We made a rigorous presentation of the numerical approximations and
a novel characterization of the artifacts introduced by the convex approximation
of contact; the approximation is exact for sticking contact and introduces an
$\mathcal{O}(\delta t\|\vf{v}_t\|)$ \emph{gliding} effect for sliding contact.

We developed a two-stage time stepping approach based on the
$\theta\text{-method}$ and we showed that with the midpoint rule it can achieve
second order accuracy even in problems with frictional contact. Our formulation
does not linearize the friction cone but it works with the second order cone
constraints directly.

We presented SAP, a robust and performant open source solver for this
formulation that warm-starts very effectively in practice. SAP globally
converges at least at a linear-rate and exhibits quadratic convergence when
additional smoothness conditions are satisfied. We showed that SAP exhibits
these two convergence regimes in simulations of practical relevance. We provided
thorough details for implementation, including analytical formulae for gradients
and Hessian, sparsity analysis and custom line search.

We compared the performance of SAP against commercial and open source
optimization solvers. Using quantitative accuracy metrics we showed that SAP
outperforms the alternatives not only without sacrificing accuracy, but even at
higher accuracy and added robustness. SAP can be up to 50 times faster than
Gurobi in small problems with up to a dozen objects and up to 10 times faster in
medium sized problems with about 100 objects. Even though SAP uses the
supernodal algebra implemented for Geodesic IPM, it performs at least two times
faster due to its effective warm-starts from the previous time-step solution.
Moreover, SAP is significantly more robust in practice given that it guarantees
a hard bound on the error in momentum, effectively providing a certificate of
accuracy.

We have incorporated SAP into the open source robotics toolkit Drake
\cite{bib:drake}, and hope that the simulation and robotics communities can
benefit from our contribution.
