\section{Conclusion}
\label{sec:future_directions}

We presented a novel unconstrained convex formulation of compliant contact. In
this formulation constraints are eliminated using analytic formulae that we
developed. Our scheme incorporates the midpoint rule into a two-stage scheme,
with demonstrated second order accuracy. We rigorously characterized our
numerical approximations and the artifacts introduced by the convex
approximation of contact. We reported limitations of our method and discussed
extensions and areas of further research.

We showed that regularization maps to physical compliance, allowing us to
eliminate algorithmic parameters and to incorporate complex models of continuous
contact patches. Moreover, we studied the trade off between regularization and
numerical conditioning for the simulation of \emph{near-rigid} bodies and
the accurate resolution of stiction.

We presented SAP, a robust and performant solver that warm-starts very
effectively in practice. SAP globally converges at least at a linear-rate and
exhibits quadratic convergence when additional smoothness conditions are
satisfied. SAP can be up to 50 times faster than Gurobi in small problems with
up to a dozen objects and up to 10 times faster in medium sized problems with
about 100 objects. Even though SAP uses the supernodal algebra implemented for
Geodesic IPM, it performs at least two times faster due to its effective
warm-starts from the previous time-step solution. Moreover, SAP is significantly
more robust in practice given that it guarantees a hard bound on the error in
momentum, effectively providing a certificate of accuracy.

We have incorporated SAP into the open source robotics toolkit Drake
\cite{bib:drake}, and hope that the simulation and robotics communities can
benefit from our contribution.
