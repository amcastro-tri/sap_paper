\section{Future Research Directions}

I summarize here a few ideas
\begin{itemize}
	\item \textbf{Exploiting sparsity}. This would entail investigating custom
	linear solvers that exploit the structure of the problem by using
	Featherstone's $\mathcal{O}(n)$ operators.
	\item \textbf{Other solvers}. The current strategy might not scale well for
	simulations with a large amount of objects. It is worth investigating other
	alternatives that exploit sparsity and can be parallelized. In particular,
	we'll explore the Conex solver developed by Frank Permenter
	\cite{bib:permenter2020} which exhibits promising warm-starting properties.	
	\item \textbf{SoftSim}. Simulation of large sparse dynamical systems with
	much higher number of dofs than constraints could be solved efficiently
	performing a Schur complement of matrix $\mf{A}$ to split dofs directly
	involved in contact from those that are not. Then the contact problem
	reduces to a much smaller system that only includes those dofs directly
	involved in contact.
	\item \textbf{Hydroelastics}. We can incorporate discrete Hydroelastics
	within this framework. We need to solve the \textit{gliding artifact}. For
	this it is proposed to use the concept of \textit{margins} as introduced by
	the Bullet physics engine. That is, our hydroelastic meshes would extend
	outwards from the boundary of the object to represent a \textit{negative}
	pressure field that'd allow us to detect contact before it happens, much
	like using SDFs for point contact.
\end{itemize}
