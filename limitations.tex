\section{Limitations}
\label{sec:limitations}

All models are an approximation of reality, while numerical methods can only
approximate our models. We list here the limitations we identified for our
method.

\textbf{Convex Approximation:} The convex approximation amounts to a
\emph{gliding effect} during sliding at a distance $\phi\sim \delta
t\mu\Vert\vf{v}_t\Vert$. Regularization amounts to \emph{compliance} in the
normal direction, though in this work we see this as a property rather than as a
limitation. In the tangential direction, regularization leads to a model of
regularized friction, Eq. \eqref{eq:regularized_friction}. Details are provided
in Section \ref{sec:contact_modeling_parameters}.

\textbf{Stiffness and Dissipation:} \reviewquestion{R4-Q3}{Our method requires
stiffness $\mf{K}$ and damping $\mf{D}$ matrices to be SPD or an SPD
approximation must be used, Section \ref{sec:two_stage_scheme}. This is exact
for joint level spring-dampers though only an approximation for other forces
such as a spatial arrangement of springs (in an unstable configuration $\mf{K}$
might not be SPD).}

\textbf{Linear Approximations:} \reviewquestion{AE/R1-Q1/R1-Q2}{Algorithm
\ref{alg:sap_time_stepping} evaluates the SPD gradient $\mf{A}$ once at
$\mf{v}^*$ each time step. In other words, our method replaces the original
balance of momentum \eqref{eq:scheme_momentum} with its linear approximation
\eqref{eq:momentum_linearized}. This is exact for many important cases and
second order in the general case, refer to Section
\ref{sec:unconstrained_convex_formulation} for details.}

\textbf{Delassus Operator Estimation:} \reviewquestion{R1-Q6}{In Section
\ref{sec:conditioning} we use a diagonal approximation of the Delassus operator
to estimate stiffness in the \emph{near-rigid} regime. Corner cases exist.
Consider a pile of books. While the inertia of a contact at the bottom of the
pile is estimated solely on the mass of one book, this contact is supporting the
weight of the entire stack. Stiffness is underestimated and user intervention is
needed to set proper parameters.}

\textbf{Scalability:} \reviewquestion{R4-Q7}{We see no reason SAP with the
direct supernodal solver explained in Section \ref{sec:problem_sparsity} could
not scale to thousands of bodies. However, scalability needs to be studied
further along with the usage of iterative solvers with pre-conditioning.}

\RedHighlight{TODO: Inlude references provided by
Frank that support the usage of direct solvers. ADMM paper from Boyd?.}
