\section{Limitations}
\label{sec:limitations}

All models are approximations of reality, while numerical methods can only
approximate our models. We list here the limitations we identify for our
method.

\textbf{Convex Approximation:} The convex approximation amounts to a
\emph{gliding effect} during sliding at a distance $\phi\sim \delta
t\mu\Vert\vf{v}_t\Vert$. Regularization leads to a model of
regularized friction, Eq. \eqref{eq:regularized_friction}. Details are provided
in Section \ref{sec:contact_modeling_parameters}.

\textbf{Stiffness and Dissipation:} \reviewquestion{R4-Q3}{Our method requires
stiffness $\mf{K}$ and damping $\mf{D}$ matrices to be SPD or SPD
approximations (see Section \ref{sec:two_stage_scheme}). 
For joint level spring-dampers, the exact  $\mf{K}$ and $\mf{D}$ can be used, but for
other forces such as those arising from a spatial arrangement of springs, an SPD approximation
must be made as $\mf{K}$ might not be SPD in certain configurations.}

\textbf{Linear Approximations:} \reviewquestion{AE/R1-Q1/R1-Q2}{Algorithm
\ref{alg:sap_time_stepping} evaluates the SPD gradient $\mf{A}$ once at
$\mf{v}^*$ at each time step. In other words, our method replaces the original
balance of momentum \eqref{eq:scheme_momentum} with its linear approximation
\eqref{eq:momentum_linearized}. This is exact for many important cases and
accurate to second-order in the general case. See Section
\ref{sec:unconstrained_convex_formulation} for details.}

\textbf{Delassus Operator Estimation:} \reviewquestion{R4-Q6}{In Section
\ref{sec:conditioning} we use a diagonal approximation of the Delassus operator
to estimate stiffness in the \emph{near-rigid} regime. Corner cases exist.
Consider a pile of books. While the inertia of a contact at the bottom of the
pile is estimated solely on the mass of one book, this contact is supporting the
weight of the entire stack. Stiffness is underestimated and user intervention is
needed to set proper parameters.}

\textbf{Scalability:} \reviewquestion{R4-Q7}{We see no reason SAP with the
direct supernodal algebra (Section \ref{sec:problem_sparsity}) could not scale
to thousands of bodies if there is structured sparsity. However, scalability needs to be studied further, along
with the usage of iterative solvers such as Conjugate Gradient (CG), widely
used in optimization.}
