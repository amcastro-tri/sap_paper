We will start by taking derivatives of the regularizer term $\ell_R$. First we
notice that we can write this term as
\begin{equation*}
	\ell_R = \frac{1}{2}\Vert\bgamma\Vert_{R}^2 = 
	\sum_i \ell_{R_i},
\end{equation*}
with $\ell_{R_i}=1/2\Vert\bgamma_i\Vert_{R_i}^2$. Since
$\nabla_{\vf{y}_j}\ell_{R_i}=\vf{0}$ for $i\neq j$, we only need to compute the
gradients of $\ell_{R_i}(\mf{y})$ with respect to the contact point variable
$\vf{y}_i\in\mathbb{R}^3$. Dropping contact subindex $i$ for simplicity, we
write the regularization as
\begin{eqnarray*}
	\ell_R = \frac{1}{2}\Vert\bgamma\Vert_R^2=\frac{1}{2}(R_t\Vert\bgamma_t\Vert^2+R_n\gamma_n^2).
\end{eqnarray*}

We use Eq.~(\ref{eq:analytical_y_projection}) to write the cost in terms of $\vf{y}$ as
\begin{align}
	&\ell_R(\vf{y}) = 
	\label{eq:ell_R_piecewise}\\	
&\begin{dcases}
	% Region I, stiction
	\frac{1}{2}(R_t y_r^2+R_n y_n^2) & \text{stiction, } y_r \le \mu y_n\\
	% Region II, sliding.
	\frac{R_n}{2(1+\tilde\mu^2)}\left(y_n + \hat\mu y_r\right)^2 & \text{sliding, } -\hat\mu y_r < y_n \leq \frac{y_r}{\mu}\\
	% Region II, no contact.
    \vf{0} & \text{no contact, } y_n < -\hat\mu y_r
\end{dcases}\nonumber	
\end{align}

\subsection{Gradients per Contact Point}
We use the following identities to simplify expressions
\begin{equation*}
	\frac{\partial y_r}{\partial\vf{y}_t} = \hat{\vf{t}}\nonumber,
	\quad
	\frac{\partial \hat{\vf{t}}}{\partial\vf{y}_t} =
	\frac{\vf{P}^\perp(\hat{\vf{t}})}{y_r},
\end{equation*}
where the $2\times 2$ projection matrix is
\begin{equation*}
	\vf{P}^\perp(\hat{\vf{t}})=\vf{I}_2 - \vf{P}(\hat{\vf{t}}),
	\quad\text{with }
	\vf{P}(\hat{\vf{t}}) = \hat{\vf{t}}\otimes\hat{\vf{t}}\nonumber.
\end{equation*}

Taking the gradient of Eq. (\ref{eq:ell_R_piecewise}) results in
\begin{align}
	&\nabla_\vf{y}\ell_R(\vf{y}) = 
	\label{eq:gradient_ell_R_piecewise}\\
&\begin{dcases}
	%%%%%%%%%%%%%%%%%%%%
	% Region I, stiction
	\vf{R}\,\vf{y} & 
	% when,
	\text{stiction, } y_r \le \mu y_n\\
	%
	%%%%%%%%%%%%%%%%%%%%
	% Region II, sliding.
	\frac{1}{1+\tilde\mu^2}\hat{s}^\circ(\vf{y})\begin{bmatrix}
		\mu R_t\hat{\vf{t}}\\
		R_n\\
	\end{bmatrix} &
	% when,
	\text{sliding, } -\hat\mu y_r < y_n \leq \frac{y_r}{\mu}\\
	% Region II, no contact.
    \vf{0} & \text{no contact, } y_n < -\hat\mu y_r
\end{dcases}\nonumber
\end{align}
with $\hat{s}^\circ(\vf{y}) = \hat{\mu}y_r+y_n$ positive in the sliding region. We note that $\nabla_\vf{y}\ell_R(\vf{y})$ is not differentiable at the
boundaries of $\mathcal{F}$ and $\mathcal{F}^\circ$. At points of
differentiability, the Hessian $\nabla_\vf{y}^2\ell_R(\vf{y})$ is computed by
taking derivatives of Eq. (\ref{eq:gradient_ell_R_piecewise})
\begin{align}
	&\nabla_\vf{y}^2\ell_R(\vf{y}) = 
	\label{eq:hessian_ell_R_piecewise}\\
&\begin{dcases}
	%%%%%%%%%%%%%%%%%%%%
	% Region I, stiction
	\vf{R} & 
	% when,
	\text{stiction,}\\
	%
	%%%%%%%%%%%%%%%%%%%%
	% Region II, sliding.	
	\frac{R_n}{1+\tilde\mu^2}
	\begin{bmatrix}
		% ∂²ℓ/∂yₜ²:
		\hat{\mu}\left(\hat{\mu}\vf{P}(\hat{\vf{t}})+\hat{s}^\circ(\vf{y})\vf{P}^\perp(\hat{\vf{t}})/y_r\right) & 
		% ∂²ℓ/∂yₙ∂yₜ:
		\hat{\mu}\vf{t}\\
		% ∂²ℓ/∂yₜ∂yₙ:
		\hat{\mu}\vf{t}^T & 
		% ∂²ℓ/∂yₙ²:
		1\\
	\end{bmatrix} &
	% when,
	\text{sliding,}\\
	% Region II, no contact.
    \vf{0} & \text{no contact.}
\end{dcases}\nonumber
\end{align}

Clearly in the stiction region we have $\nabla_\vf{y}^2\ell_R(\vf{y})\succ 0$.
Since in the stiction region we have $\hat{s}^\circ(\vf{y})>0$, the linear
combination of $\vf{P}(\hat{\vf{t}})$ and $\vf{P}(\hat{\vf{t}})^\perp$ in Eq.
(\ref{eq:hessian_ell_R_piecewise}) is PSD (since both projection matrices are
PSD). Therefore $\nabla_\vf{y}^2\ell_R(\vf{y})\succeq 0$.

\subsection{Gradients with Respect to Velocities}
Recall we use bold italics for vectors in $\mathbb{R}^3$ and non-italics bold
for their stacked counterpart. With $\mf{y}=-\mf{R}^{-1}(\mf{J}\mf{v} -
\hat{\mf{v}}_c)$ we use the chain rule to compute the gradient in terms of
velocities 
\begin{equation}
	\nabla_\mf{v}\ell_R = -\mf{J}^T\mf{R}^{-1}\nabla_\mf{y}\ell_R,
	\label{eq:ell_velocity_gradient}
\end{equation}
which, using Eq. (\ref{eq:gradient_ell_R_piecewise}), can be shown to equal
\begin{equation}
	\nabla_\mf{v}\ell_R = -\mf{J}^T\bgamma,
	\label{eq:ell_velocity_gradient_simplified}
\end{equation}

At points of differentiability, we obtain the Hessian of the regularizer
$\ell_R(\mf{v})$ from the gradient of $\bgamma(\mf{v})$ in Eq.
\eqref{eq:ell_velocity_gradient_simplified}
\begin{align*}
	\nabla_\mf{v}^2\ell_R(\mf{v}) &= -\mf{J}^T \nabla_{\mf{v}_c}\bgamma\,\mf{J}\nonumber,\\
	\nabla_{\mf{v}_c}\bgamma &= -\nabla_\mf{y}\bgamma \mf{R}^{-1},
\end{align*}
where $\nabla_{\mf{v}_c}\!\bgamma$ is a block diagonal matrix where each
diagonal block is the $3\times 3$ matrix $\nabla_{\mf{v}_{c,i}}\!\bgamma_i$
for the $i\text{-th}$ contact. Alternatively, taking the gradient of Eq. \eqref{eq:ell_velocity_gradient} leads to the equivalent result
\begin{equation*}
	\nabla_\mf{v}^2\ell_R = \mf{J}^T\mf{R}^{-1}\nabla_\mf{y}^2\ell_R\mf{R}^{-1}\mf{J},
\end{equation*}
where we can verify indeed that
$-\nabla_{\mf{v}_c}\bgamma=\mf{R}^{-1}\nabla_\mf{y}^2\ell_R\mf{R}^{-1}$. Since
$\nabla_\mf{y}^2\ell_R\succeq 0$, it follows that
$-\nabla_{\mf{v}_c}\bgamma\succeq 0$.

We define $\vf{G}_i\in\mathbb{R}^{3\times 3}$ the matrix that evaluates to
$-\nabla_{\mf{v}_{c,i}}\!\bgamma_i$ within regions $\mathcal{R}_I$,
$\mathcal{R}_{II}$ and $\mathcal{R}_{III}$ where the projection is
differentiable. At the boundary of $\mathcal{F}$ we use the analytical
expression from $\mathcal{R}_I$. At the boundary of $\mathcal{F}^\circ$ we use
the analytical expression from $\mathcal{R}_{II}$. This extension fully
specifies $\vf{G}_i$ for all $\vf{y}_i\in\mathbb{R}^3$. Finally, we define the
${3n_c\times 3n_c}$ matrix $\mf{G}=\text{diag}(\vf{G}_i)\succeq 0$.

\subsection{Gradients of the Primal Cost}
With these results, we can now write the gradient $\nabla_\vf{v}\ell_p$ and weighting matrix $\mf{H}$ in Algorithm \ref{alg:sap}. For the gradient we
have
\begin{equation*}
	\nabla_\mf{v}\ell_p(\mf{v}) = \mf{A}(\mf{v}-\mf{v}^*) + \nabla_\mf{v}\ell_R.
\end{equation*}
which using Eq. (\ref{eq:ell_velocity_gradient_simplified}) can be written as
\begin{equation*}
	\nabla_\mf{v}\ell_p(\mf{v}) = \mf{A}(\mf{v}-\mf{v}^*) - \mf{J}^T\bgamma,
\end{equation*}
and since the unconstrained minimization seeks to satisfy the optimality
condition $\nabla_\mf{v}\ell_p=\mf{0}$, we recover the balance of momentum.

Finally, we define the weighting matrix $\mf{H}$ as
\begin{equation*}
	\mf{H} = \mf{A} + \mf{J}^T\mf{G}\,\mf{J},
\end{equation*}
which, given the definition of $\mf{G}$, returns the Hessian of $\ell_p(\vf{v})$
when the gradient is differentiable and extends the analytical expressions at
points of non-differentiability. Since $\mf{A}\succ 0$ and $\mf{G} \succeq 0$,
we have $\mf{H}\succ 0$.
