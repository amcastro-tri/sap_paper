We first compute the regularization term as
\begin{eqnarray}
	\ell_R = \frac{1}{2}(R_t\Vert\bgamma_t\Vert^2+R_n\gamma_n^2)
\end{eqnarray}

During \textbf{stiction} the cost is written as
\begin{eqnarray}
	\ell_R(\vf{y}) = \frac{1}{2}(R_t y_r^2+R_n y_n^2)
\end{eqnarray}

During \textbf{sliding} Eq. (\ref{eq:inverse_dynamics_projection}) applies and
the cost can be written as
\begin{eqnarray}
	\ell_R(\vf{y}) =
	\frac{1}{2}\gamma_n^2(R_t\mu^2+R_n)=\frac{R_n}{2}(1+\tilde\mu^2)\gamma_n^2=\frac{R_n}{2(1+\tilde\mu^2)}\left(\mu\frac{R_t}{R_n}y_r+y_n\right)^2
\end{eqnarray}

and finally when there is \textbf{no contact}
\begin{eqnarray}
	\ell_R(\vf{y}) = 0
\end{eqnarray}

Therefore $\ell_R$ is a piecewise function that we can summarize as
\begin{equation}
	\ell_R(\vf{y}) = 
\begin{dcases}
	% Region I, stiction
	\frac{1}{2}(R_t y_r^2+R_n y_n^2) & \text{stiction, } y_r < \mu y_n\\
	% Region II, sliding.
	\frac{R_n}{2(1+\tilde\mu^2)}\left(\mu\frac{R_t}{R_n}y_r+y_n\right)^2 & \text{sliding, } -\mu \frac{R_t}{R_n} y_r < y_n \leq \frac{y_r}{\mu}\\
	% Region II, no contact.
    \vf{0} & \text{no contact, } y_n \leq -\mu \frac{R_t}{R_n} y_r
\end{dcases}	  
	\label{eq:ell_R_piecewise}
\end{equation}

\subsection{Gradients per Contact Point}
In this section we compute the gradients of $\ell_R(\mf{y})$ with respect to the
$i\text{-th}$ contact point variable $\vf{y}_i\in\mathbb{R}^3$. Unless otherwise
stated, in this section we'll drop the subindex $i$ for the $i\text{-th}$
contact point. Therore in this section $\nabla_\vf{y}\ell_R\in\mathbb{R}^3$ and
$\nabla_\vf{y}^2\ell_R\in\mathbb{R}^{3\times 3}$ (notice that as per our
notation, we consistently use a bold italic font for 3D vectors and only bold,
not italic, for $n\text{-dimensional}$ vectors.)

The full gradient $\nabla_\mf{y}\ell_R$ concatenates the three-dimensional
gradients $\nabla_\vf{y}\ell_R$ while the full Hessian matrix is block-diagonal
with each $3\times 3$ block containing the local $i\text{-th}$ point Hessian.
This is particularly useful to exploit sparsity in the computations.

We use the following identities to simplify the expressions
\begin{eqnarray}
	\frac{\partial y_r}{\partial\vf{y}_t} &=& \hat{\vf{t}}\nonumber\\
	\frac{\partial \hat{\vf{t}}}{\partial\vf{y}_t} &=&
	\frac{\vf{P}^\perp(\hat{\vf{t}})}{y_r}
	\label{eq:yt_derivatives}
\end{eqnarray}
where the $2\times 2$ projection matrices along and perpendicular to
$\hat{\vf{t}}$ are defined as
\begin{eqnarray}
	\vf{P}(\hat{\vf{t}}) &=& \hat{\vf{t}}\otimes\hat{\vf{t}}\nonumber\\
	\vf{P}^\perp(\hat{\vf{t}})&=&\vf{I}_2 - \vf{P}(\hat{\vf{t}})
	\label{eq:tangential_projections}
\end{eqnarray}

Taking the gradient of Eq. (\ref{eq:ell_R_piecewise}) results in
\begin{equation}
	\nabla_\vf{y}\ell_R(\vf{y}) = 
\begin{dcases}
	%%%%%%%%%%%%%%%%%%%%
	% Region I, stiction
	\vf{R}\,\vf{y} & 
	% when,
	\text{stiction, } y_r < \mu y_n\\
	%
	%%%%%%%%%%%%%%%%%%%%
	% Region II, sliding.
	\frac{1}{1+\tilde\mu^2}\hat{s}^\circ(\vf{y})\begin{bmatrix}
		\mu R_t\hat{\vf{t}}\\
		R_n\\
	\end{bmatrix} &
	% when,
	\text{sliding, } -\mu \frac{R_t}{R_n} y_r < y_n \leq \frac{y_r}{\mu}\\
	% Region II, no contact.
    \vf{0} & \text{no contact, } y_n \leq -\mu \frac{R_t}{R_n} y_r
\end{dcases}	  
	\label{eq:gradient_ell_R_piecewise}
\end{equation}
where we defined $\hat{\mu}=\mu R_t/R_n$ and $\hat{s}^\circ(\vf{y}) =
\hat{\mu}y_r+y_n$. $\hat{s}^\circ(\vf{y})$ is a measure how close the solution
is to the polar cone $\mathcal{F}^\circ$. $\hat{s}^\circ(\vf{y}) > 0$ in the
sliding region and $\hat{s}^\circ<0$ when there is no contact.

Similarly, we can compute the Hessian of $\ell_R$ by taking the gradient in Eq.
(\ref{eq:gradient_ell_R_piecewise})
\begin{equation}
	\nabla_\vf{y}^2\ell_R(\vf{y}) = 
\begin{dcases}
	%%%%%%%%%%%%%%%%%%%%
	% Region I, stiction
	\vf{R} & 
	% when,
	\text{stiction, } y_r < \mu y_n\\
	%
	%%%%%%%%%%%%%%%%%%%%
	% Region II, sliding.
	\frac{R_n}{1+\tilde\mu^2}
	\begin{bmatrix}
		% ∂²ℓ/∂yₜ²:
		\hat{\mu}\left(\hat{\mu}\vf{P}(\hat{\vf{t}})+\hat{s}^\circ(\vf{y})\vf{P}^\perp(\hat{\vf{t}})/y_r\right) & 
		% ∂²ℓ/∂yₙ∂yₜ:
		\hat{\mu}\vf{t}\\
		% ∂²ℓ/∂yₜ∂yₙ:
		\hat{\mu}\vf{t}^T & 
		% ∂²ℓ/∂yₙ²:
		1\\
	\end{bmatrix} &
	% when,
	\text{sliding, } -\mu \frac{R_t}{R_n} y_r < y_n \leq \frac{y_r}{\mu}\\
	% Region II, no contact.
    \vf{0} & \text{no contact, } y_n \leq -\mu \frac{R_t}{R_n} y_r
\end{dcases}	  
	\label{eq:hessian_ell_R_piecewise}
\end{equation}

Clearly in the stiction region we have $\nabla_\vf{y}^2\ell_R(\vf{y})\succ 0$.
Since in the stiction region we have $\hat{s}^\circ(\vf{y})>0$, the linear
combination of $\vf{P}(\hat{\vf{t}})$ and $\vf{P}(\hat{\vf{t}})^\perp$ in Eq.
(\ref{eq:hessian_ell_R_piecewise}) is PSD (since both projection matrices are
PSD). Therefore, in the sliding region we find out that
$\nabla_\vf{y}^2\ell_R(\vf{y})\succeq 0$.
\todo{Show that the jump condition across the cone's boundary is satisfied.}

\subsection{Gradients with Respect to Velocities}
These gradients are computed using the chain rule since we know that
\begin{equation}
	\mf{y}=\mf{Jv+b}
\end{equation}

The gradient is
\begin{equation}
	\nabla_\mf{v}\ell_R = -\mf{J}^T\mf{R}^{-1}\nabla_\mf{y}\ell_R
	\label{eq:ell_velocity_gradient}
\end{equation}
which, using Eq. (\ref{eq:gradient_ell_R_piecewise}), can be written as
\begin{equation}
	\nabla_\mf{v}\ell_R = -\mf{J}^T\bgamma
	\label{eq:ell_velocity_gradient_simplified}
\end{equation}

And the Hessian is
\begin{equation}
	\nabla_\mf{v}^2\ell_R = \mf{J}^T\mf{R}^{-1}\nabla_\mf{y}^2\ell_R\mf{R}^{-1}\mf{J}
	\label{eq:ell_velocity_hessian}
\end{equation}

Since we already showed $\nabla_\mf{y}^2\ell_R\succeq 0$, it follows that
$\nabla_\mf{v}^2\ell_R\succeq 0$. Therefore the cost $\ell_R(\mf{v})$ is convex.

Recall that $\nabla_\mf{y}^2\ell_R$ is block diagonal and $\mf{R}$ is diagonal.
Therefore the Hessian $\nabla_\mf{v}^2\ell_R$ inherits the same sparsity
structure as the product $\mf{J}^T\mf{J}$.

\subsection{Gradients of the Primal Cost}
With these results, we can now write the gradient and Hessian of the primal cost
$\ell_p(\mf{v})$ in Eq. (\ref{eq:primal_unconstrained}). For the gradient we
have
\begin{equation}
	\nabla_\mf{v}\ell_p(\mf{v}) = \mf{M}(\mf{v}-\mf{v}^*) + \nabla_\mf{v}\ell_R
\end{equation}

Notice that, using Eq. (\ref{eq:ell_velocity_gradient_simplified}), the gradient
can be written as
\begin{equation}
	\nabla_\mf{v}\ell_p(\mf{v}) = \mf{M}(\mf{v}-\mf{v}^*) - \mf{J}^T\bgamma
\end{equation}
and since the unconstrained minimization looks for $\nabla_\mf{v}\ell_p=\mf{0}$,
we essentially recover the balance of momentum, as expected.

Similarly, we can write the Hessian as
\begin{equation}
	\mf{H} = \nabla_\mf{v}^2\ell_p(\mf{v}) = \mf{M} + \nabla_\mf{v}^2\ell_R
\end{equation}
and since we already proved that $\nabla_\mf{v}^2\ell_R\succeq 0$, we find that
the $\nabla_\mf{v}^2\ell_p\succ 0$. Therefore $\ell_p(\mf{v})$ is strictly
convex and there is a unique solution to the minimization problems in the
velocities $\mf{v}$.
