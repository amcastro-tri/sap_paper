\documentclass[journal]{./IEEEtran/IEEEtran}

\usepackage{amsmath} 
\usepackage{amsfonts}
\usepackage{mathtools} 
\usepackage{amssymb}
\usepackage{graphicx}
\usepackage{bm}  % Bold math
\usepackage[colorinlistoftodos, textsize=footnotesize]{todonotes}
\usepackage[colorlinks=true, allcolors=blue]{hyperref}
\usepackage{setspace}
\usepackage{algorithm}      % http://ctan.org/pkg/algorithm
\usepackage{algpseudocode}  % http://ctan.org/pkg/algorithmicx
\usepackage{tikz}
\usepackage{verbatim}
\usepackage{xcolor}
\usepackage{subcaption}

% Package adjustbox: Introduces \adjincludegraphics to include figures allowing
% trimming.
\usepackage{adjustbox}

% Package ulem: The package provides an \ul (underline) command which will break
% over line ends; this technique may be used to replace \em (both in that form
% and as the \emph command), so as to make output look as if it comes from a
% typewriter. The package also offers double and wavy underlining, and striking
% out (line through words) and crossing out (/// over words).
\usepackage[normalem]{ulem}

\usepackage[thinc]{esdiff}

% Package amsthm: Provides the proof environment. N.B. this MUST go before the
% \newtheorem below. Do not change the order.
\usepackage{amsthm}
% Theorems numbered on a per section basis.
%\newtheorem{theorem}{Theorem}[section]
%\newtheorem{corollary}{Corollary}[theorem] \newtheorem{lemma}{Lemma}[section]

\usepackage{mathrsfs}

% Theorems numbered as explained in the IEEEtran HOWTO guide.
\newtheorem{theorem}{Theorem}
\newtheorem{corollary}{Corollary}[theorem]
\newtheorem{lemma}{Lemma}
\newtheorem{prop}{Proposition}

\newif\ifcompile
%\compiletrue % uncomment out to compile

\newtheorem{remark}{Remark}

% Place this declaration AFTER amsmath is included.
\DeclareMathOperator*{\argmin}{arg\,min}
\DeclareMathOperator*{\argmax}{arg\,max}

\newcommand{\RedHighlight}[1]{{\color{red}\textbf{#1}}}
\newcommand{\CyanHighlight}[1]{{\color{cyan}\textbf{#1}}}
\newcommand{\GreenHighlight}[1]{{\color{green}\textbf{#1}}}
\newcommand{\redcolor}[1]{{\color{red}#1}}

% We use this command to mark text in reply to reviewers's comments/questions.
% Use the no-op definition below to get rid of all the marks.
\newcommand{\reviewquestion}[2]{{\color{red}\textbf{#1: }#2}}
%\newcommand{\reviewquestion}[2]{#2}

% Super short aliases to color indexes in matrices.
\newcommand{\rr}[1]{{\color{red}#1}} \newcommand{\cc}[1]{{\color{cyan}#1}}

% TikZ commands for drawing schematics.
\newcommand{\tikzmark}[1]{\tikz[overlay, remember picture] \coordinate (#1);}

%For making diagonal entries in a matrix.
\newcommand{\diagentry}[1]{\mathmakebox[1.8em]{#1}}

% Helper to create entries of the form Jₚₜ₁ᵀGₚJₚₜ₂ p = #1 t1 = #2 t2 = #3
\newcommand{\JTGJ}[3]{\mf{J}_{\rr{#1}\cc{#2}}^T\mf{G}_\rr{#1}\mf{J}_{\rr{#1}\cc{#3}}}

% Math Shortcuts: Vector Font: For 3D vectors.
\newcommand{\vf}[1]{{\bm{#1}}}
% Matrix Font: for matrices, arrays and concatenation of 3D vectors.
\newcommand{\mf}[1]{{\mathbf{#1}}}
%
% Bold greek symbols:
\newcommand{\bgamma}{{\bm\gamma}} \newcommand{\btgamma}{{\bm{\tilde\gamma}}}
\newcommand{\bsigma}{{\bm\sigma}}
% Matrices:
\newcommand{\sM}{\mf{M}} \newcommand{\sJ}{\mf{J}} \newcommand{\sJT}{\mf{J}^T}
\newcommand{\sJbar}{\bar{\mf{J}}} \newcommand{\sJbarT}{\bar{\mf{J}}^T}
\newcommand{\sB}{\mf{B}} \newcommand{\sBT}{\mf{B}^T} \newcommand{\sG}{\mf{G}}
\newcommand{\sGT}{\mf{G}^T} \newcommand{\sR}{\mf{R}}
\newcommand{\sRinv}{\mf{R}^{-1}} \newcommand{\ssqrtR}{\mf{R}^{1/2}}
\newcommand{\ssqrtRinv}{\mf{R}^{-1/2}}
% Vectors
\newcommand{\sthat}{\hat{\vf{t}}}
% Projections: gradient of gamma with respect to y.
\newcommand{\sP}{\mf{P}}
% gradient of gamma_tilde with respect to y_tilde.
\newcommand{\sGtilde}{\tilde{\mf{G}}} \newcommand{\sPt}{\mf{P}_t}

% Definition symbol. Section 3 (page 15) of "The Comprehensive LATEX Symbol
% List".
\newcommand\defeq{\stackrel{\text{\tiny def}}{=}}


% correct bad hyphenation here
\hyphenation{op-tical net-works semi-conduc-tor}

\begin{document}

% Paper Title: Titles are generally capitalized except for words such as a, an,
% and, as, at, but, by, for, in, nor, of, on, or, the, to and up, which are
% usually not capitalized unless they are the first or last word of the title.
% Linebreaks \\ can be used within to get better formatting as desired. Do not
% put math or special symbols in the title.
\title{An Unconstrained Convex Formulation of Compliant Contact}


% Author Names and IEEE memberships: Note positions of commas and nonbreaking
% spaces ( ~ ) LaTeX will not break a structure at a ~ so this keeps an author's
% name from being broken across two lines. Use \thanks{} to gain access to the
% first footnote area a separate \thanks must be used for each paragraph as
% LaTeX2e's \thanks was not built to handle multiple paragraphs.
\author{Alejandro Castro, Frank Permenter, Xuchen Han% <-this % stops a space
\thanks{All the authors are with Toyota Research Institute, Cambridge, MA, USA,
{\tt\small firstname.lastname@tri.global}.}%
}

% The paper headers
\markboth{Journal of \LaTeX\ Class Files,~Vol.~14, No.~8, August~2015}%
{Shell \MakeLowercase{\textit{et al.}}: Bare Demo of IEEEtran.cls for IEEE
Journals}

% make the title area
\maketitle

% Abstract:
\begin{abstract}
We present a convex formulation of compliant contact along with a robust
iterative method to solve it in practice. Our formulation eliminates contact
constraints analytically to write an unconstrained convex problem of
frictional contact on velocities only. Our method has proven convergence and
warms-starts effectively, enabling the simulation of complex multibody systems
with frictional contact at interactive rates. We show our model can be described
in clear physical terms, rather than in optimization specific terminology that
might be elusive to non optimization experts. This allows us to incorporate not
only point contact models of compliance, but also more sophisticated models of
compliant surface patches. Accurate physical modeling of contact is critical for
meaningful sim-to-real transfer. Moreover, we provide explicit analytical
expressions for the contact forces as a function of state, which allow us to
characterize artifacts introduced by the convex approximation in a novel and
transparent way. Based on the $\theta\text{-method}$, our time stepping scheme
includes not only the popular symplectic Euler method but also the second order
midpoint rule. We demonstrate that the midpoint rule can achieve second order
accuracy even in simulation through frictional contact. We introduce a number of
accuracy metrics and show that our method outperforms existing commercial and
open source alternatives without sacrificing accuracy. Finally, with the
simulation of relevant robotic systems for manipulation, we demonstrate our
approach can accurately resolve stiction, is robust and performs at interactive
rates. Our method is implemented in the open source Drake robotics toolkit.

\end{abstract}

% Keywords:
\begin{IEEEkeywords}
Contact Modeling, Simulation and Animation, Dexterous Manipulation, Dynamics.
\end{IEEEkeywords}


% For peer review papers, you can put extra information on the cover page as
% needed: \ifCLASSOPTIONpeerreview \begin{center} \bfseries EDICS Category:
% 3-BBND \end{center} \fi
%
% For peerreview papers, this IEEEtran command inserts a page break and creates
% the second title. It will be ignored for other modes.
\IEEEpeerreviewmaketitle

% We place each section in its own file:
% The very first letter is a 2 line initial drop letter followed
% by the rest of the first word in caps.
% 
% form to use if the first word consists of a single letter:
% \IEEEPARstart{A}{demo} file is ....
% 
% form to use if you need the single drop letter followed by
% normal text (unknown if ever used by the IEEE):
% \IEEEPARstart{A}{}demo file is ....
% 
% Some journals put the first two words in caps:
% \IEEEPARstart{T}{his demo} file is ....

\section{Introduction}
\RedHighlight{TODO: Rewrite for paper.}

\IEEEPARstart{A}{Anitescu}'s convex relaxation of contact
\cite{bib:anitescu2006} has been in existence for 14 years. However to my
knowledge there are no simulation/trajectory optimization codes out there that
can exploit this formulation. In other words, the formulation is theoretically
beautiful, but there are no practical solvers out there for it. The best
reported solver for this formulation, later on presented by Anitescu in
\cite{bib:anitescu2010, bib:tasora2011}, consists of a Projected Gauss-Seidel
(PGS) iteration which is very well known not to converge in practice and it is
usually stoped at a fixed number of iterations with an unknown amount of
numerical error. Therefore, it is unclear even today whether Anitescu's
formulation presents any advantage over other non-convex solution strategies. In
\cite{bib:todorov2014} Todorov mentions a so called \textit{generalized}
projected Gauss-Seidel (though the details were never published), which
generalizes PGS to deal with the true quadratic friction cone, anisotropy,
torsional and rolling friction.

The objectives of this reasearch are:
\begin{enumerate}
	\item To develop a \textit{practical} solution strategy that is robust and
	has good scalability to number of bodies and contacts.
	\item To extend the formulation to enable the simulation of soft deformable
	bodies with contact.
	\item To properly quantify the \textit{unphysical artifacts} introduced by
	the convex relaxation of the original physics.
	\item To answer the questions ``does the Convex formulation provide any
	significant advantage over other methods?'', ``are the unphysical artifacts
	introduced worth the price?''
	\item To provide an open source implementation of this research that
	outperforms existing methods.
\end{enumerate}

\section{Other Papers Worth Citing}
\RedHighlight{Notes to help writing the intro and suplement other sections
better.}

\begin{enumerate}
	\item Acary \cite{bib:acary2011formulation} also uses the same convex
	formulation. It might be worth mentioning differences, contributions and
	specific field of engineering.
	\item Kaufman's thesis on staggered projections
	\cite{bib:kaufman2009coupled} states this interesting observation on LCP's:
	``It has been recently noted, however, that direct LCP solvers do not, in
	practice, scale. They are not, in fact, currently able to return solutions
	for non-trivial frictionally contacting problems beyond relatively
	small-scale examples [Anitescu and Hart 2004; Erleben 2007]. In Chapter 5 we
	will discuss these difficulties further and investigate the reasons why
	these methods often fail.'' This is backed up on the works of Anitescu and
	Hart 2004; Erleben 2007 \cite{bib:anitescu2004fixed,bib:erleben2007velocity}.
\end{enumerate}

\section{Multibody Dynamics with Contact}
\label{sec:multibody_dynamics_with_contact}

We use generalized coordinates to describe our multibody system. Therefore the
state is fully described by the generalized positions
$\mf{q}\in\mathbb{R}^{n_q}$ and the generalized velocities
$\mf{v}\in\mathbb{R}^{n_v}$. We describe the $k\text{-th}$ contact between two
bodies by the relative velocity $\vf{v}_{c,k}\in\mathbb{R}^3$ between these two
bodies at this point, expressed in a contact frame $C_k$ for which we
arbitrarily choose the $z\text{-axis}$ to coincide with the surface normal
$\hat{\vf{n}}_k$ reported by the geometry engine. We form the vector
$\mf{v}_{c}\in\mathbb{R}^{3n_c}$ of contact velocities by stacking velocites
$\vf{v}_{c,k}$ for all contacts together. Similarly, we denote with
$\bgamma_k\in\mathbb{R}^3$ the contact impulse at a specific contact point and
with $\bgamma\in\mathbb{R}^{3n_c}$ the vector of all contact impulses. In
general, unless otherwise specified, we use bold italics for vectors in
$\mathbb{R}^3$ and non-italics bold for their stacked counterpart.

The contact Jacobian $\mf{J}(\mf{q})\in\mathbb{R}^{3n_c\times n_v}$ relates
generalized velocities to contact velocity by $\mf{v}_c=\mf{J}\,\mf{v}$.

We use a discrete time stepping approach to advance the dynamics of the system
forward in time and we write the balance of momentum as
\begin{eqnarray}
	\mathbf{M}\mathbf{v} = \mathbf{M}\mathbf{v}^* + \mathbf{J}^T\mathbf{\gamma}
	\label{eq:momentum_balance}
\end{eqnarray}
where $\mf{M}(\mf{q})\in\mathbb{R}^{n_v\times n_v}$ is the mass matrix and
$\mf{v}^*\in\mathbb{R}^{n_v}$ are the \textit{free motion} generalized
velocities when there is no contact.

Denoting with $\phi_k$ the \textit{gap function} or \textit{signed distance} at
a contact point and with $\gamma_{n,k}$ and with $\bgamma_{t,k}$ the normal and
tangential components of the impulse $\bgamma_k=[\bgamma_{t,k}\,\gamma_{n,k}]$
at that point, the formulation is most often completed with the following set of
standard constraints to be satisfied at each contact
\begin{enumerate}
	\item non-penetration constraint $0\le\phi_k\perp\gamma_{n,k}\ge0$,
	\item\label{it:cone_constraint} friction cone constraint
	$\Vert\bgamma_{t,k}\Vert\le \mu_k\gamma_{n,k}$ and,
	\item the principle of maximum dissipation.
\end{enumerate}
where $\mu_k$ is the coefficient of friction for the $k\text{-th}$ contact.
Condition \ref{it:cone_constraint} states that contact impulses at point $k$ are
constrained to belong to the friction cone
$\mathcal{F}_k=\{\vf{x}\in\mathbb{R}^3 \,|\, \Vert\vf{x}_t\Vert\le \mu_k x_n\}$.

While these constraints model the widely accepted model of Colulomb friction,
the resulting formulation leads to a very difficult to solve, non-convex,
Non-linear Complementarity Problem (NCP). 

\RedHighlight{Alejandro: say something how other author's impose the principle
of maximum dissipation and more or less what it means.}

\RedHighlight{Alejandro: say something about
the existence of solutions for this formulation. Reference other people's work.}

Plausible good references on this are
\cite{bib:stewart1996implicit,bib:stewart2000implicit,bib:chakraborty2007implicit,bib:acary2018solving,bib:pang1999unified,bib:alart2018inconsistency}.

\section{Convex Approximation of Contact Dynamics}
\label{sec:previous_work}

In pursuit of a formulation for which the existence of a solution is guaranteed
and with the hope that such a formulation leads to more robust and efficient
solvers, Anitescu introduces a \textit{convex approximation} of the contact
problem in \cite{bib:anitescu2006}. We discuss physical implications and
artifacts introduced by this approximation in Section
\ref{sec:physical_intuition} as well as we provide specific guidelines to
determine its applicability to simulation for robotics.

This approximation is introduced as a
modification to the complementarity constraint between the signed distance
$\phi$ and normal impulse
$\gamma_n$ as
\begin{equation}
	0 < \phi - dt \Vert {\bm v}_t \Vert \perp \gamma_n > 0
	\label{eq:convex_approximation_complementarity_condition}
\end{equation}
where $dt$ is the discrete time step and $\vf{v}_t$ is the tangential component
of the contact velocity $\vf{v}_c = [\vf{v}_t\,v_n]$. Together with a set of
complementarity conditions to impose maximum dissipation on a linearized
friction cone, Anitescu shows in \cite{bib:anitescu2006} that the new set of
constraints corresponds to the optimality conditions of the following quadratic
program with conic constraints as
\begin{eqnarray}
	\min_{\gamma\in \mathcal{F}} \ell(\bgamma) =
	\frac{1}{2}\bgamma^T\mathbf{W}\bgamma + {\bm r}^T
	\bgamma
	\label{eq:dual_cost}
\end{eqnarray}
where $\mathbf{W} =
\mathbf{J}^T\mathbf{M}^{-1}\mathbf{J}\in\mathbb{R}^{3n_c\times 3n_c}$ is the
Delassus operator, $\mathcal{F}=\prod_{k=1}^{n_c}\mathcal{F}_k$ is the feasible
set of all contact impulses satisfying the friction cone constraints and, the
linear cost ${\bm r}$ encodes the free motions $\mf{v}^*$ and stabilization
terms to impose non-interpenetration at the positions level.

In \cite{bib:todorov2011, bib:todorov2014} Todorov introduces regularization to
the formulation in Eq. (\ref{eq:dual_cost}). Though not strictly applicable to
contact problems, Todorov uses the \textit{Gauss's principle of least
constraint} to obtain this convex approximation of the contact problem. The
regularized formulation reads

\begin{eqnarray}
	\min_{\gamma\in \mathcal{F}} \ell(\bgamma) =
	\frac{1}{2}\bgamma^T(\mathbf{W}+\mathbf{R})\bgamma + {\bm r}^T
	\bgamma
	\label{eq:dual_regularized}
\end{eqnarray}
where $\mathbf{R}$ is a diagonal positive definite matrix introduced to
make $\mathbf{W}+\mathbf{R}\succ 0$ since in general we only have $\mathbf{W}
\succeq 0$. This makes the problem strictly convex, with a unique solution. 

One of the most remarkable results from \cite{bib:todorov2014} is that if we
 happen to know the velocities of the system, we can write analytical algebraic
 expressions for the impulses. Moreover, the separable structure of the
 constraints allows us to write the impulse $\bgamma_k$ at contact $k$ solely as
 a function of the relative contact velocity $\vf{v}_{c,k}$ at that point, i.e.
 $\bgamma_k = \bgamma_k(\vf{v}_{c,k})$, see details in Section
 \ref{sec:analytical_inverse_dynamics}.

In this work we will show in Section \ref{sec:unconstrained_convex_formulation}
how to exploit this \textit{analytical inverse dynamics} to write an
unconstrained convex formulation that can be solved using Newton's method. The
resulting strategy is accurate, robust and can be effectively warm-started from
the previous time step solution.

\section{Discrete Time Formulation}
\label{sec:discrete_time_formulation}

Our discrete-time model is based on the $\theta\text{-method}$ \cite[\S
II.7]{bib:hairer2008solving} and incorporates the symplectic midpoint rule to
attain second order accuracy and energy conservation. While most of the work in
the literature uses first order time-stepping schemes, the extension to the
second-order midpoint rule is analyzed in \cite{bib:potra2006linearly}. While
the work in \cite{bib:potra2006linearly} uses a polyhedral approximation of the
friction cone that leads to an LCP formulation, our approach does not
approximate the friction cone but introduces the convex approximation of contact
from \cite{bib:anitescu2006} \RedHighlight{TODO: Introduce definition of LCP in
the introduction}. We remark that combining the $\theta\text{-method}$ with the
convex approximation of contact is novel to our work. 

Time is discretized into intervals of fixed size $dt$ and we seek to advance the
state of the system from time $t^n$ to the next step at $t^{n+1} = t^n + dt$. In
the $\theta\text{-method}$ variables are evaluated at intermediate time steps
$t^\theta = \theta t^{n+1}+(1-\theta)t^{n}$, with $\theta \in [0, 1]$. We define
\emph{mid-step quantities} $\mf{q}^{\theta_{q}}$, $\mf{v}^{\theta_{v}}$, and
$\mf{v}^{\theta_{vq}}$ in accordance with the standard $\theta\text{-method}$
using scalar parameters $\theta_q$ $\theta_v$ and $\theta_{vq}$
\begin{eqnarray}
	\mf{q}^{\theta_q} &=& \theta_q\mf{q} + (1-\theta_{q})\mf{q}_0,\nonumber\\
	\mf{v}^{\theta_v} &=& \theta_v\mf{v} + (1-\theta_v)\mf{v}_0,\nonumber\\
	\mf{v}^{\theta_{vq}} &=& \theta_{vq}\mf{v} + (1-\theta_{vq})\mf{v}_0.
	\label{eq:theta_method}
\end{eqnarray}

Using the above definitions, we write our discrete update in the following form
\begin{eqnarray}
	\mf{M}(\mf{q}^{\theta_{q}})(\mf{v}-\mf{v}_0)  &=& \nonumber\\
	dt\,\mf{F}(\mf{q}^{\theta_{q}}, \mf{v}^{\theta_v}) &+&
	dt\,\mf{G}(\mf{q}^{\theta_{q}}, \mf{v}^{\theta_v}) +
	\mf{J}(\mf{q}^{\theta_{q}})^T\mf{\bgamma}, \label{eq:v_update}\\
	\mf{q} = \mf{q}_0 + dt \dot{\mf{q}}^{\theta_{vq}} &=& \mf{q}_0 + dt\mf{N}(\mf{q}^{\theta_{q}})\mf{v}^{\theta_{vq}},
	\label{eq:q_update}
\end{eqnarray}
where to simplify notation $(\mf{q}_0, \mf{v}_0)$ denotes the state at $t^n$ and
$(\mf{q}, \mf{v})$ the state at $t^{n+1}$. $\mf{N}(\mf{q})$ is the kinematic map
defined in Eq. (\ref{eq:kinematic_map}). Notice that in the update for $\mf{q}$
we allow for the evaluation of $\mf{N}$ at $t^{\theta_{q}}$ and velocities
at $t^{\theta_{vq}}$. This will allow us to write fully implicit as well as
semi-implicit symplectic schemes within the same framework. For the common case
when $\mf{N}$ is the identity matrix, $\dot{\mf{q}}$ is evaluated at
$t^{\theta_{vq}}$, justifying the notation. Note that using Eq.
(\ref{eq:q_update}) the mid-step configuration $\mf{q}^{\theta_q}$ can be
written as
\begin{equation}
	\mf{q}^{\theta_q}(\mf{v}) = \mf{q}_0 + dt\theta_q\mf{N}(\mf{q}^{\theta_{q}})\mf{v}^{\theta_{vq}}(\mf{v})
	\label{eq:position_update}
\end{equation}


We split forces into two contributions so that $\partial \mf{F}/\partial\mf{q}$ and $\partial \mf{F}/\partial\mf{v}$ are positive definite matrices while the same is generally not true for the gradients of $\mf{G}$. Therefore in $\mf{F}(\mf{q}, \mf{v})$ we will include modeling elements such as spring and dampers and even internal forces
for the modeling of soft-body deformation. $\mf{G}(\mf{q}, \mf{v})$ will include
all other contributions that cannot guarantee positive definiteness of the
gradients such as Coriolis and gyroscopic forces arising in multibody dynamics with generalized coordinates.

The discrete update in Eqs. (\ref{eq:v_update})-(\ref{eq:q_update}) generalizes
some of the most popular schemes for forward dynamics:
\begin{itemize}
	\item Explicit Euler with $\theta_q=\theta_{v}=\theta_{vq} = 0$,
	\item Symplectic Euler with $\theta_{q} = 0$ and $\theta_v = \theta_{vq}=1$,
	\item Implicit Euler with $\theta_{q} = \theta_v = \theta_{vq}= 1$,
	\item Symplectic midpoint rule, which is second order, with $\theta_{q} =
	\theta_v = \theta_{vq}= 1/2$,
\end{itemize}

Similar to the work in \cite{bib:duriez2005realistic} for the simulation of
deformable objects and to projection methods used in fluid mechanics
\cite{bib::bell1991efficient}, we solve Eqs. (\ref{eq:v_update}) and
(\ref{eq:q_update}) in two stages; first we solve for the \emph{free motion
velocities} the system would have in the absence of contact and second, we
update these free motion velocities so that they satisfy conservation of
momentum along with contact constraints.

We first define the momentum quantity
\begin{multline}
	\mf{m}(\mf{v}) =
	\mf{M}(\mf{q}^{\theta_{q}}(\mf{v}))(\mf{v}-\mf{v}_0) -\\
	dt\,\mf{F}(\mf{q}^{\theta_{q}}(\mf{v}), \mf{v}^{\theta_v}(\mf{v}))-
	dt\,\mf{G}(\mf{q}^{\theta_{q}}(\mf{v}), \mf{v}^{\theta_v}(\mf{v}))
	\label{eq:m_definition}
\end{multline}

In the first stage we solve for the
\textit{free motion} velocities $\mf{v}^*$ in the absence of constraint impulses
\begin{eqnarray}
	\mf{m}(\mf{v}^*) &=& \mf{0},
	\label{eq:vstar_definition}
\end{eqnarray}

For the implicit Euler scheme and the midpoint rule, Eq.
(\ref{eq:vstar_definition}) is implicit in $\mf{v}^*$ and we solve them using
Newton's method. The explicit and symplectic Euler schemes require inversion of
the mass matrix $\mf{M}$, which can be accomplished efficiently using the
$\mathcal{O}(n)$
\emph{Articulated Body Algorithm}
\cite{bib:featherstone2008_rigid_body_dynamics_algorithms}.

For the second stage we linearize $\mf{m}(\mf{v})$ around $\mf{v} = \mf{v}^*$ to write an approximation of Eq. (\ref{eq:v_update}) as
\begin{eqnarray}
	\mf{A}(\mf{v}-\mf{v}^*)= \mf{J}^T\mf{\gamma}
	\label{eq:momentum_linearized}
\end{eqnarray}
with
\begin{eqnarray}
	\mf{A}&=&\mf{M}+dt^2\,\theta_q\theta_{qv}\mf{K}+dt\,\theta_v\mf{D},
	\label{eq:expression_for_A}\\
	\mf{K}(\mf{q}, \mf{v})&=&-\frac{\partial \mf{F}(\mf{q}, \mf{v})}{\partial
	\mf{q}}\frac{\partial\dot{\mf{q}}^{\theta_{vq}}}{\partial\mf{v}},
	\label{eq:stiffness_matrix}\\
	\mf{D}(\mf{q}, \mf{v})&=&-\frac{\partial \mf{F}(\mf{q}, \mf{v})}{\partial
	\mf{v}},
	\label{eq:dissipation_matrix}
\end{eqnarray}
where $\mf{K}$ and $\mf{D}$ are the stiffness and damping matrices of the
system, respectively. As an example, consider modeling spring-dampers at the
joint level. In this case $\mf{K}$ and $\mf{D}$ are constant and diagonal. A
more complex example arises when using a Finite Element Model (FEM) of soft body
deformations. In this case $\mf{K}$ and $\mf{D}$ are large sparse, positive
definite, matrices.

Equation (\ref{eq:momentum_linearized}) is a first order approximation of Eq.
(\ref{eq:v_update}) as stated in the following
\begin{theorem}	
	In the absence of impacts Eq. (\ref{eq:momentum_linearized}) is a first
	order approximation of Eq. (\ref{eq:v_update}). That is
	\begin{eqnarray}
		\mf{m}(\mf{v}) = \mf{A}(\mf{v}-\mf{v}^*) + \mathcal{O}(dt^2) =
		\mf{J}^T\mf{\bgamma}.
	\end{eqnarray}
Moreover, $\mf{A} \succ 0$.
\end{theorem}

\begin{IEEEproof}
The Taylor expansion of $\mf{m}(\mf{v})$ at $\mf{v}=\mf{v}^*$ reads
\begin{eqnarray}
	\mf{m}(\mf{v}) &=& \mf{m}^* +
	\nabla\mf{m}^*(\mf{v}-\mf{v}^*) + \mathcal{O}_m(\Vert\mf{v}-\mf{v}^*\Vert^2)\nonumber\\
	&=&\nabla\mf{m}^*(\mf{v}-\mf{v}^*) + \mathcal{O}_m(\Vert\mf{v}-\mf{v}^*\Vert^2)
	\label{eq:m_taylor_expansion}
\end{eqnarray}
where the superscript $^*$ denotes quantities evaluated at $\mf{v}^*$ and we use
the fact that by definition $\mf{m}^*=\mf{m}(\mf{v}^*)=\mf{0}$. The operator
$\nabla$ denotes the gradient with respect to $\mf{v}$.

We evaluate first the gradient of the mass matrix term in Eq.
(\ref{eq:m_definition}) using the chain rule through Eq.
(\ref{eq:position_update})
\begin{eqnarray}
	\nabla[\mf{M}(\mf{q}^{\theta_{q}}(\mf{v}))(\mf{v}-\mf{v}_0)]\Bigr|_{\mf{v}^*}
	= \mf{M}(\mf{q}^{\theta_{q}}(\mf{v}^*)) + \mf{E}^*,
\end{eqnarray}
where we defined
\begin{eqnarray}
	\mf{E}^* =
	dt\theta_q\frac{\partial[\mf{M}(\mf{q}^{\theta_{q}})(\mf{v}^*-\mf{v}_0)]}{\partial\mf{q}}\frac{\partial\dot{\mf{q}}^{\theta_{vq}}}{\partial\mf{v}}.
	\label{eq:E_definition}
\end{eqnarray}

Equation (\ref{eq:E_definition}) involves cumbersome derivatives of the mass matrix with respect to configurations. However, we can still see that $\mf{E}^*=\mathcal{O}(dt\Vert\mf{v}^*-\mf{v}_0\Vert)$ where we are using the big O notation to describe its limiting behavior as $dt\rightarrow 0$. Since the free motion dynamics that defines $\mf{v}^*$ is continuous, we know that $\Vert\mf{v}^*-\mf{v}_0\Vert = \mathcal{O}(dt)$ and therefore $\mf{E}^* = \mathcal{O}(dt^2)$.

We proceed similarly to expand the gradient of $\mf{F}(\mf{v})=\mf{F}(\mf{q}^{\theta_{q}}(\mf{v}), \mf{v}^{\theta_v}(\mf{v}))$ as
\begin{eqnarray}
	\nabla\mf{F} =
	-dt\,\theta_q\theta_{vq}\mf{K}(\mf{q}^{\theta_q}, \mf{v}^{\theta_v})-\theta_v\mf{D}(\mf{q}^{\theta_q}, \mf{v}^{\theta_v})
\end{eqnarray}
with $\mf{K}$ and $\mf{D}$ the stiffness and damping matrices defined by Eqs.
(\ref{eq:stiffness_matrix})-(\ref{eq:dissipation_matrix}).

We can now write the gradient of $\mf{m}(\mf{v})$ in Eq. (\ref{eq:m_taylor_expansion}) as
\begin{eqnarray}
	\nabla\mf{m}^* = \mf{A} + \mf{E}^* - dt\nabla\mf{G}
\end{eqnarray}
where we defined
\begin{eqnarray}
	\mf{A}=\mf{M}+ dt^2\theta_q\theta_{qv}\mf{K}+dt\theta_v\mf{D},
\end{eqnarray}

With these definitions the Taylor expansion in Eq. (\ref{eq:m_taylor_expansion})
becomes
\begin{multline}
	\nabla\mf{m}^*(\mf{v}-\mf{v}^*) =\\
	\mf{A}(\mf{v}-\mf{v}^*) + \mf{E}^*(\mf{v}-\mf{v}^*) - dt\,\nabla\mf{G}\,(\mf{v}-\mf{v}^*).
\end{multline}

In the absence of impacts $\Vert\mf{v}-\mf{v}^*\Vert=\mathcal{O}(dt)$. Therefore
$\mf{E}^*(\mf{v}-\mf{v}^*)=\mathcal{O}_E(dt^3)$ and
$dt\nabla\mf{G}(\mf{v}-\mf{v}^*)=\mathcal{O}_G(dt^2)$. Then the Taylor expansion
\begin{eqnarray}
	\mf{A}(\mf{v}-\mf{v}^*) + \mathcal{O}_E(dt^3) + \mathcal{O}_G(dt^2) = \mf{J}^T\mf{\bgamma}
\end{eqnarray}
is a first order approximation of the original momentum balance in Eq.
(\ref{eq:v_update}).

Finally, notice that $\mf{A}$ is a linear combination of positive definite
matrices with positive scalars in the linear combination, and therefore
$\mf{A}\succ 0$.

\end{IEEEproof}

We summarize our discrete time stepping strategy as follows
\begin{enumerate}
	\item Solve for the \emph{free motion} velocities $\mf{v}^*$, from Eq.
	(\ref{eq:vstar_definition}).
	\item Compute the SPD approximation of the gradient $\mf{A}$, as defined in
	Eq. (\ref{eq:expression_for_A}).
	\item\label{it:solve_for_contact} Solve for the constraint impulses $\bgamma$ that satisfy the
	linearized momentum Eq. (\ref{eq:momentum_linearized}) and the contact
	constraints. 
	\item Update the positions according to Eq. (\ref{eq:q_update}).
\end{enumerate}

Impulses in item \ref{it:solve_for_contact} are the solution to our convex
approximation of compliant contact described in the next section. Section
\ref{sec:sap_solver} presents the SAP solver to solve it in practice.


Notice that, in the absence of constraint impulses, the next step velocity is
$\mf{v}=\mf{v}^*$ and it is computed with the order of accuracy of the
$\theta\text{-method}$. We also expect to recover the properties of the
$\theta\text{-method}$ when contact constraints are not active. This is the case
of rolling friction for which the contact constraints behave as a bi-lateral
constraints to impose zero slip velocity. We demonstrate this in Section XXX
with an example simulation of billiard balls.  \RedHighlight{Alejandro: Add
proper reference to the billiard balls example in the numerical examples
section.}


\section{A Primal Formulation of Compliant Contact}

% I took a stab at rewriting this intro to address some 
% of the comments I had (see below). 

% START REWRITE
%In this section we give a quadratic program that augments the balance of momentum stated in Eq.
%(\ref{eq:momentum_linearized}) so that contact impulses model Coulomb friction
%and satisfy the principle of maximum dissipation when sliding. 
%We call this QP our \emph{primal formulation}, as we will see that it's dual
%is precisely the QP~(\ref{eq:dual_regularized}).


%Recall that this QP~(\ref{eq:dual_regularized}) has cost matrix $W =J^T M^{-1} J^{-1} + R$
%where $R$ is a regularization term. 
%Our primal formulation will avoid construction of $J^T M^{-1} J^{-1}$, 
%which improves practical robustness. In addition, it always has a unique solution, even for $R = 0$.
%Finally, we note that for $R =0$, our formulation reduces to \cite{bib:mazhar2014}.
%\ref{sec:unconstrained_convex_formulation} and \ref{sec:solver_details} we
%describe a methodology to solve it in practice.

% END REWRITE


%START ORIGINAL
In this section we augment the balance of momentum stated in Eq.
(\ref{eq:momentum_linearized}) so that contact impulses model Colomb friction
and satisfy the principle of maximum dissipation when sliding. 

\RedHighlight{Havent we already justified regularization when we introduced
\ref{eq:dual_regularized}?
This seems repetitive.  I would focus only on how our formulation differs-compares with \ref{eq:dual_regularized}.}
An alternative is to use the convex approximation as stated in Eq.
\RedHighlight{It is unclear why this is an alternative, e.g., how does the principle
of max disp. appear in this equation?}
(\ref{eq:dual_cost}). However, this formulation is severely ill conditioned due
to the fact that contact forces for rigid body dynamics problems are most often
underdetermined. Even if compliance is added in the normal direction, friction
\RedHighlight{Unclear what you mean by compliance. Compliance is added to the primal constraints, right?}
forces for the simplest problem configurations will be underdetermined.
Regularization in Eq. (\ref{eq:dual_regularized}) helps to solve this problem in
theory but it leads to very ill conditioned systems of equations in practice. 

We make the following observation for the convex approximation in Eq.
(\ref{eq:dual_cost}); even if the set of contact forces is not unique (when no
regularization is added), velocities are. This fact inspired the search
for an equivalent formulation but in velocities instead of impulses. Such a
\RedHighlight{The name "primal formulation" what make much sense unless you such duality first.}
\textit{primal} formulation is presented in \cite{bib:mazhar2014} for rigid
contact, though to the knowledge of the authors a practical solver based on this
formulation has never been presented.

In this section we extend the formulation in \cite{bib:mazhar2014} to include
the modeling of compliance and in Sections
\ref{sec:unconstrained_convex_formulation} and \ref{sec:solver_details} we
describe a methodology to solve it in practice.
% END ORIGINAL 



We write our primal formulation of compliant contact by introducing a new
decision variable $\vf{\sigma}\in\mathbb{R}^{3n_c}$ as
\begin{equation}
	\begin{aligned}
	\min_{\mf{v},\bsigma} \quad & \ell_p(\mf{v},\bsigma) = \frac{1}{2}(\mf{v}-\mf{v}^*)^T\mf{A}(\mf{v}-\mf{v}^*) + \frac{1}{2} \Vert\bsigma\Vert_{R}^2\\
	\textrm{s.t.} \quad & \mf{g} = (\mf{J}\mf{v}-\hat{\mf{v}} + \mf{R}\bsigma) \in \mathcal{F}^*\\
	\end{aligned}
	\label{eq:primal_regularized}
\end{equation}
where $\mathcal{F^*}= \mathcal{F}^*_1 \times \mathcal{F}^*_2 \times \cdots \times \mathcal{F}^*_{n_k}$ is the \emph{dual cone} of the convex
cone $\mathcal{F}$. The positive diagonal matrix $\mf{R}\in\mathbb{R}^{3n_c\times
3n_c}$ and the vector of stabilization velocities $\hat{\mf{v}}$ encode the
problem data needed to model compliant contact. We will establish a clear
physical meaning for these terms when we provide analytical expressions for the
impulses in Section \ref{sec:analytical_inverse_dynamics}. Finally,
$\Vert\bsigma\Vert_R^2=\bsigma^T\mf{R}\bsigma$.

\begin{theorem}
The dual of Eq. (\ref{eq:primal_regularized}) is given by Eq.
(\ref{eq:dual_regularized}). The pair $\{\mf{v},\bsigma\}$ is primal optimal and
$\bgamma$ is dual optimal. Moreover, $\bsigma = \bgamma$.
\end{theorem}

\begin{proof}
The Lagrangian of the primal formulation in Eq. (\ref{eq:primal_regularized}) is
\begin{equation}
	\mathcal{L}(\mf{v},\bsigma,\vf{\gamma}) = \frac{1}{2}(\mf{v}-\mf{v}^*)^T\mf{A}(\mf{v}-\mf{v}^*) + \frac{1}{2} \Vert\bsigma\Vert_{R}^2 - \vf{\gamma}^T\mf{g}
	\label{eq:primal_lagrangian}
\end{equation}
with $\vf{\gamma}\in\mathcal{F}$ the dual variable to enforce the constraint
$\vf{g}\in \mathcal{F}^*$. We obtain the dual of Eq.
(\ref{eq:primal_regularized}) by minimizing the Lagrangian jointly in the
variables $\mf{v}$ and $\bsigma$ and replacing the result back to obtain the
dual cost $\ell_d(\vf{\gamma})$. Minimizing jointly in the variables $\mf{v}$
and $\bsigma$ leads to the conditions
\begin{eqnarray}
	\mf{A}(\mf{v}-\mf{v}^*) &=& \mf{J}^T\vf{\gamma}\\
	\vf{\sigma} &=& \vf{\gamma}
\end{eqnarray}
where with the first equation we find out that multipliers $\bgamma$ are indeed
impulses and we recover the balance of momentum, and the second equation allows
us to eliminate $\vf{\sigma}$. When we replace these results back into the
Lagrangian in Eq. (\ref{eq:primal_lagrangian}) we obtain the dual
\begin{eqnarray}
	\min_{\bgamma\in \mathcal{F}} \ell_d(\bgamma) =
	\frac{1}{2}\bgamma^T(\mathbf{W}+\mathbf{R})\bgamma + {\bm r}^T
	\bgamma
\end{eqnarray}
where, in contrast to previous work, our Delassus operator
$\mf{W}=\mf{J}\mf{A}\mf{J}^T$ now also contains the contribution of internal
force elements (through Eq. (\ref{eq:expression_for_A})) and
$\mf{r}=\mf{v}_c^*-\hat{\mf{v}}$ with $\mf{v}_c^*=\mf{J}\mf{v}^*$.
\end{proof}

\RedHighlight{This seems repetitive if we also mention mazhar in the intro of this section}
We note that our formulation is an extension of~\cite{bib:mazhar2014} that
incorporates  compliance through the term $R$. Sections
\ref{sec:unconstrained_convex_formulation} and \ref{sec:solver_details} 
describe a methodology to solve our formulation efficiently practice, whereas,
to our knowledge, no efficient algorithm for~\cite{bib:mazhar2014} has been given. 



% formulation in \cite{bib:mazhar2014} to include
%the modeling of compliance and in Sections
%\ref{sec:unconstrained_convex_formulation} and \ref{sec:solver_details} we



\section{An Unconstrained Convex Formulation}
\label{sec:unconstrained_convex_formulation}

From Theorem \ref{th:primal_dual_equivalence} we know that $\bsigma = \bgamma$.
Therefore we can use the analytical inverse dynamics from Section
\ref{sec:analytical_inverse_dynamics} to eliminate $\bsigma$ from the primal
formulation and obtain an unconstrained convex formulation in terms of
generalized velocities $\mf{v}$ only.

We then substitute $\bsigma=\bgamma=P_\mathcal{F}(\mf{y})$ from Eq.
\ref{eq:analytical_y_projection} into Eq. (\ref{eq:primal_regularized}) to
obtain our unconstrained convex formulation in velocities only
\begin{eqnarray}
	\min_{\mf{v}} \ell_p(\mf{v}) =
	\frac{1}{2}(\mf{v}-\mf{v}^*)^T\mf{A}(\mf{v}-\mf{v}^*) +
	\ell_R(\mf{v})
	\label{eq:primal_unconstrained}
\end{eqnarray}
with the regularizer
\begin{equation}
	\ell_R(\mf{v}) = \frac{1}{2}\Vert P_\mathcal{F}(\mf{y}(\mf{v}))\Vert_R^2
\end{equation}

In order for the velocities solution of Eq. (\ref{eq:primal_unconstrained}) to
be a solution of the original problem stated in Eq.
(\ref{eq:primal_regularized}), they must also satisfy the conic constraint
$\mf{g}\in\mathcal{F}^*$. We stablish this result in the following
\begin{lemma}
    velocites solution to Eq. (\ref{eq:primal_unconstrained}) satisfy the conic
    constraint $\mf{g}\in\mathcal{F}^*$ when the impulses are given by
    $\bsigma=\bgamma=P_\mathcal{F}(\mf{y})$.
    \label{lemma:conic_constraints_are_satisfied_analytically}
\end{lemma}
\begin{IEEEproof}
    The proof follows immediately from Moreau's decomposition theorem. That is,
    with $\mf{v}_c - \hat{\mf{v}}\in\mathbb{R}^{3n_c}$ and $\bsigma$ the result
    from the projection into $\mathcal{F}$ using the $\mf{R}$ norm, the linear
    combination $\mf{v}_c - \hat{\mf{v}} + \mf{R}\bsigma$ must be in
    $\mathcal{F}^*$.
\end{IEEEproof}

Therefore we can state the following
\begin{theorem}
    Velocities $\mf{v}$ solution to Eq. (\ref{eq:primal_unconstrained}) and
    $\bsigma=P_\mathcal{F}(\mf{y})$ from the analytical solution in Eq.
    (\ref{eq:analytical_y_projection}) are solution to the original primal
    formulation stated in Eq. (\ref{eq:primal_regularized}).
\end{theorem}
\begin{IEEEproof}
    Theorem \ref{th:primal_dual_equivalence} states that $\bsigma=\bgamma$.
    Given known primal-optimal velocities, $\bgamma$ is given by the same
    analytical inverse dynamics used in the unconstrained formulation in Eq.
    (\ref{eq:primal_unconstrained}).
    In addition, these velocities and the analytical impulses satisfy the cone
    constraints by Lemma \ref{lemma:conic_constraints_are_satisfied_analytically}.
\end{IEEEproof}

This new formulation forms the foundation of the novel SAP solver developed in
this work, Section \ref{sec:sap_solver}.


% Dummy comment for Reviewable.

\subsection{Analytical Inverse Dynamics}
\label{sec:analytical_inverse_dynamics}

Remarkably,  the dual optimal impulses of~\eqref{eq:dual_regularized} can be
constructed from the primal optimal velocities of~\eqref{eq:primal_regularized}
using a simple projection operation. Following~\cite{bib:todorov2014}, we call
this construction  \textit{inverse dynamics}. Moreover, this projection
decomposes into a set of individual projections for each contact impulse
$\bgamma_i$ given the separable structure of the constraints. Letting
$\vf{y}_i(\vf{v}_{c,i}) = -\vf{R}_i^{-1}(\vf{v}_{c,i}-\hat{\vf{v}}_{c,i})$,
these projections take the form
\begin{equation}
  \begin{aligned}
	\bgamma_i(\vf{v}_{c,i})&= P_{\mathcal{F}_i}(\vf{y}_i(\vf{v}_{c,i}))\\
	&= \argmin_{\bgamma\in\mathcal{F}_i} \quad 
		\frac{1}{2}(\bgamma-\vf{y}_i)^T\vf{R}_i(\bgamma-\vf{y}_i),\\
	\end{aligned}
	\label{eq:y_projection}
\end{equation}
where $\vf{R}_i\in\mathbb{R}^{3\times3}$ is the $i\text{-th}$ diagonal block of
the regularization matrix $\mf{R}$. That is, $\bgamma_i$ is the projection
$P_{\mathcal{F}_i}$ of $\vf{y}_i(\vf{v}_{c,i})$ onto the friction cone
$\mathcal{F}_i$ using the norm defined by $\vf{R}_i$. The projection
$P_{\mathcal{F}}(\mf{y})$ onto the full cone $\mathcal{F} := \mathcal{F}_1
\times \mathcal{F}_2 \times \cdots \times \mathcal{F}_{n_c}$ is obtained by simply
stacking together the individual projections $P_{\mathcal{F}_i}(\vf{y}_i)$ from
Eq. (\ref{eq:y_projection}), where we form $\mf{y}$ by stacking together each
$\vf{y}_i$ from all contact pairs.  In this notation, the optimal impulse
$\bgamma$ of~\eqref{eq:dual_regularized} and the optimal velocities $\mf{v}$
of~\eqref{eq:primal_regularized} satisfy  $\bgamma =
P_{\mathcal{F}}(\mf{y}(\mf{v}))$.

\subsection{Compliant Contact, Principle of Maximum Dissipation and Artifacts}
\label{sec:physical_intuition}

Thus far, $\vf{R}_i$ and $\hat{\vf{v}}_{c,i}$ have been treated as known problem
data. This section makes an explicit connection of these quantities with
physical parameters to model compliant contact with regularized friction. Notice
this approach is different from the one in \cite{bib:todorov2014}, where
regularization is not used to model physical compliance but rather to introduce
a user tunable Baumgarte-style stabilization to avoid constraint drift. We want
to model compliant contact as in Eq. (\ref{eq:compliant_model}) with stiffness
$k$ (in N/m) and linear dissipation $d = \tau_d k$ where $\tau_d$ (in seconds)
is the \textit{dissipation time scale}. 

Dropping subscript $i$ for simplicity, we solve the projection problem in Eq.
(\ref{eq:y_projection}) analytically in Appendix
\ref{app:analytical_inverse_dynamics_derivations} for a regularization matrix of
the form $\vf{R} = \text{diag}([R_t, R_t, R_n])$
\begin{eqnarray}
	\bgamma &=& P_\mathcal{F}(\vf{y}) \label{eq:analytical_y_projection}\\
    &=&\begin{dcases}
	% Region I, stiction
	\vf{y} 
	% When we  have:
	& \text{stiction, } y_r \le \mu y_n\\
	%
	%
	% Region II, sliding.
	\begin{bmatrix}
		\mu\gamma_n\hat{\vf{t}}\\
		\frac{1}{1+\tilde\mu^2}\left(y_n + \hat\mu y_r\right)
	\end{bmatrix}
	% When we  have:
	& \text{sliding, } -\hat\mu y_r < y_n \leq \frac{y_r}{\mu}\\
	%
	%
	% Region III, no contact.
    \vf{0} & \text{no contact, } y_n < -\hat\mu y_r \end{dcases}\nonumber	
\end{eqnarray}
where $\vf{y}_t$ and $y_n$ are the tangential and normal components of $\vf{y}$,
$y_r=\Vert\vf{y}_t\Vert$ is the radial component, and
$\hat{\vf{t}}=\vf{y}_t/y_r$ is the unit tangent vector. We also define the
coefficients $\tilde\mu=\mu\,(R_t/R_n)^{1/2}$ and $\hat\mu=\mu\,R_t/R_n$ that
result from the \textit{warping} introduced by the metric $\vf{R}$.

Our compliant model of contact is defined by
\begin{eqnarray*}
	\hat{\vf{v}}_c &=&
	\begin{bmatrix}
		0\\
		0\\
		\hat{v}_n \end{bmatrix}\nonumber,\\
	\hat{v}_n &=& -\frac{\phi_0}{\delta t+\tau_d},
\end{eqnarray*}
where $\phi_0$ is the previous step signed distance. The normal direction
regularization parameters is taken as $R_n^{-1} = \delta t k(\delta t+\tau_d)$.
To gain physical insight into our model, we substitute
$\vf{y}=-\vf{R}^{-1}(\vf{v}_c - \hat{\vf{v}}_c)$ into Eq.
(\ref{eq:analytical_y_projection}) to obtain
\begin{align*}
	&\bgamma(\vf{v}_c) = P_\mathcal{F}(\vf{y}(\vf{v}_c))\\
&=\begin{dcases}
	% Region I, stiction
	\begin{bmatrix}
		-\vf{v}_t/R_t\\
		-\delta t(k\,\phi + d\,v_n)
	\end{bmatrix}
	% When we  have: y_r < \mu y_n
	& \text{stiction, } \\
	%
	%
	% Region II, sliding.
	\begin{bmatrix}
		\mu\gamma_n\hat{\vf{t}}\\
		-\frac{\delta t}{1+\tilde\mu^2}\left(k(\phi-(\delta
		t+\tau_d)\mu\Vert\vf{v}_t\Vert) + d\,v_n \right)
	\end{bmatrix}
	% When we  have:  -\mu \frac{R_t}{R_n} y_r < y_n \leq \frac{y_r}{\mu}
	& \text{sliding, }\\
	%
	%
	% Region III, no contact.  y_n \leq -\mu \frac{R_t}{R_n} y_r
    \vf{0} & \text{no contact, } \end{dcases}\nonumber	
\end{align*}
where $\phi= \phi_0 + \delta t\,v_n$ approximates the signed distance function
at the next time step. Let us now analyze the resulting forces from this model.

\textbf{Friction Forces}. We see that friction forces behave exactly as a model
of regularized friction
\begin{equation*}
	\bgamma_t = \min\left(\frac{\Vert\vf{v}_t\Vert}{R_t}, \mu\gamma_n\right)\hat{\vf{t}},
\end{equation*}
with $\bgamma_t$ linear with the (very small) slip velocity during stiction and
with a maximum value given by $\mu\gamma_n$, effectively modeling Coulomb's
friction. Notice that to better model stiction, we are interested in small
values of $R_t$. We discuss our parameterization of $R_t$ in Section
\ref{sec:understanding_model_parameters}. Moreover, since $\hat{\vf{t}} =
\vf{y}_t/\Vert\vf{y}_t\Vert = -\vf{v}_t/\Vert\vf{v}_t\Vert$, friction forces
oppose sliding and therefore satisfy the principle of maximum dissipation.

\textbf{Normal Forces}. We observe that in stiction, we recover the compliant
model given by Eq. (\ref{eq:compliant_model}), as desired. In the sliding
region, however, we see that the convex approximation introduces unphysical
artifacts. 

Firstly, the factor $1+\tilde{\mu}^2$ models an effective stiffness
$k_\text{eff}=k/(1+\tilde{\mu}^2)$ and dissipation
$d_\text{eff}=d/(1+\tilde{\mu}^2)$, different from the physical values. This
tell us that in order to accurately model compliance during sliding we must
satisfy the condition $\tilde\mu=\mu\,(R_t/R_n)^{1/2} \approx 0$ or,
equivalently, $R_t \ll R_n$. Section \ref{sec:understanding_model_parameters}
introduces a parameterization of $R_t$ that satisfies this condition.

Secondly, while we'd like to recover $\gamma_n = -\delta t(k\,\phi + d\,v_n)$ as
in stiction, we instead see that the slip velocity $\vf{v}_t$ unphysically
couples into the normal forces as $\gamma_n=-\delta t(k(\phi-(\delta
t+\tau_d)\mu\Vert\vf{v}_t\Vert) + d\,v_n)$. We can write this as
\begin{equation*}
  \gamma_n/\delta t=-(k\,\phi_\text{eff} + d\,v_n),
\end{equation*}
with an \textit{effective} signed distance $\phi_\text{eff} = \phi-(\delta
t+\tau_d)\mu\Vert\vf{v}_t\Vert$. That is, we recover the dynamics of compliant
contact but with a spurious drift of magnitude $(\delta
t+\tau_d)\mu\Vert\vf{v}_t\Vert$.

This is consistent with the formulation in \cite{bib:anitescu2010} for rigid
contact when $k\rightarrow \infty$ and $\tau_d=0$, leading to an unphysical
\textit{gliding effect} at a positive distance $\phi=\delta
t\mu\Vert\vf{v}_t\Vert$. Notice that the \textit{gliding} goes away as $\delta
t\rightarrow 0$ since the formulation converges to the original contact problem
\cite{bib:anitescu2006}. The effect of compliance is to \textit{soften} this
gliding effect. 

With finite compliance,  the normal force when sliding goes to
$-k(\phi-\tau_d\mu\Vert\vf{v}_t\Vert)$ in the limit to $\delta t\rightarrow 0$.
This tells us that, unlike the rigid case, the \textit{gliding} effect
unfortunately does not go away as $\delta t\rightarrow 0$. It persists with a
finite value that now depends on the dissipation rate,
$\phi\approx\tau_d\mu\Vert\vf{v}_t\Vert$.

We close this discussion by making the following remarks relevant to robotics
applications:
\begin{enumerate}
	\item We are mostly interested in the stiction regime, typically for
	grasping, locomotion, or rolling contact for mobile bases with wheels. This
	regime is precisely where the convex approximation does not introduce
	artifacts.
	\item Sliding usually happens with low velocities and therefore the term
	$\delta t\mu\Vert\vf{v}_t\Vert$ is negligible.
	\item For robotics applications, we are mostly interested in inelastic
	contact. We will see that this can be effectively modeled with
	$\tau_d\approx\delta t$ in Section \ref{sec:understanding_model_parameters}.
	Therefore, in this regime, the term $\tau_d\mu\Vert\vf{v}_t\Vert$ also goes
	to zero as $\delta t\rightarrow 0$.
	\item We are definitely interested in the onset of sliding. This is captured
	by the approximation which properly models the Colulomb friction law.
\end{enumerate}

\algblockdefx{RepeatUntil}{EndRepeatUntil}{\textbf{repeat until}}{}
\algnotext{EndRepeatUntil}

\section{Semi-Analytical Primal Solver}
\label{sec:sap_solver}

Our Semi-Analytical Primal Solver, or SAP for short, essentially is a Newton
iteration used to find the minimum of the unconstrained formulation stated in
Eq. (\ref{eq:primal_unconstrained}), where constraints have been eliminated
using our analytical inverse dynamics. At each $m\text{-th}$ Newton iteration,
SAP uses analytical expressions of both the gradient and Hessian of
$\ell_p(\mf{v})$, with line search to find the unique optimal solution
\begin{algorithm}[H]
	\caption{SAP Newton Iteration}	
	\begin{algorithmic}
	\State Initialize $\mf{v}^0 \gets \mf{v}_0$
	\RepeatUntil $~\Vert\tilde{\nabla}\ell_p\Vert < \varepsilon_a + \varepsilon_r\max(\Vert\tilde{\mf{p}}\Vert,\Vert\tilde{\mf{j}_c}\Vert)$, Eq. \eqref{eq:stopping_criteria}
	\State $\displaystyle \Delta\mf{v}^{m} = -\mf{H}^{-1}(\mf{v}^m)\nabla_\mf{v}\ell_p(\mf{v}^m)\nonumber$
	\State $\displaystyle \alpha^m = \argmin_{t\in\mathbb{R}^{++}} \ell_p(\mf{v}^m + t \Delta\mf{v}^{m})$
	\State $\displaystyle \mf{v}^{m+1} = \vf{v}^m + \alpha^{m}\Delta\mf{v}^{m}$
	\EndRepeatUntil
	\State\Return $\mf{v}$, $\bgamma=P_\mathcal{F}(\vf{y}(\mf{v}))$
\end{algorithmic}
\end{algorithm}
where we use the previous time step velocity $\mf{v}_0$ to initialize the
iteration. The stopping criteria is discussed in below in Section
\ref{sec:stopping_criteria}.

Specifics of the line search algorithm are critical to the success of the SAP
solver given that the cost can undergo large changes as $\mf{v}^m$ explores
states corresponding to different contact modes. We explored two line-search
algorithms: an approximate backtracking line-search with Armijo's stopping
criteria and an exact (to machine epsilon) line-search. We show how a careful
pre-computation of commonly occurring terms enables the exact line-search step
for a small fraction of the total cost. While backtracking line-search with
Armijo's rule guarantees the convergence of SAP, we found out that exact
line-search can lead to a performance improvement between 15\% to 35\%. In the
next subsections we describe each component of the solver in detail.

\subsection{Gradients}
\label{sec:gradients}

%Appendix \ref{app:gradients_derivation}

We summarize the main results required for implementation. The gradient of the
primal cost $\ell_p$ is
\begin{equation}
	\nabla_\mf{v}\ell_p(\mf{v}) = \mf{A}(\mf{v}-\mf{v}^*) + \nabla_\mf{v}\ell_R,
\end{equation}
where we defined the regularizer cost as $\ell_R(\mf{v})=\frac{1}{2}\Vert
P_\mathcal{F}(\mf{y}(\mf{v}))\Vert_R^2$. We show in Appendix
\ref{app:gradients_derivation} that this gradient can be computed analytically,
with the result
\begin{equation}
	\nabla_\mf{v}\ell_p(\mf{v}) = \mf{A}(\mf{v}-\mf{v}^*) - \mf{J}^T\bgamma(\mf{v})
	\label{eq:primal_gradient}
\end{equation}
with $\bgamma(\mf{v})=P_\mathcal{F}(\vf{y})$ from the analytical inverse
dynamics in Eq. \eqref{eq:analytical_y_projection}. This recovers the original
momentum balance in Eq. (\ref{eq:momentum_linearized}) since Newton's method
solves for $\nabla_\mf{v}\ell_p=\mf{0}$.

We obtain the Hessian $\mf{H}=\nabla_\mf{v}^2\ell_R$ by taking the gradient of
Eq. (\ref{eq:primal_gradient}). The result is
\begin{eqnarray}
	\nabla_\mf{v}^2\ell_R(\mf{v}) &=& \mf{J}^T\mf{G}\,\mf{J}\nonumber\\
	\mf{G} &=&-\nabla_{\mf{v}_c}\bgamma = \nabla_\mf{y}\bgamma \mf{R}^{-1}
	\label{eq:ellR_hessian}
\end{eqnarray}
where $\nabla_{\mf{v}_c}\!\bgamma$ is a block diagonal matrix where each
diagonal elements is the $3\times 3$ matrix $\nabla_{\mf{v}_{c,i}}\!\bgamma_i$
for the $i\text{-th}$ contact. As shown in Appendix
\ref{app:gradients_derivation}, $\nabla_{\mf{v}_c}\bgamma\succeq 0$ and thus
$\nabla_\mf{v}^2\ell_R(\mf{v})\succeq 0$.

Finally, the Hessian needed in Newtons's method is
\begin{equation}
	\mf{H}= \mf{A} + \mf{J}^T\mf{G}\,\mf{J}
	\label{eq:ell_hessian}
\end{equation}
which, since $\mf{A}\succ 0$, $\mf{H}$ is strictly positive definite. Notice
that to compute the requried gradient and Hessian we need analytical expressions
for the projection $P_\mathcal{F}(\vf{y})$ and its gradient
$\nabla_\mf{y}P_\mathcal{F}(\vf{y})$, both provided in Appendices
\ref{app:analytical_inverse_dynamics_derivations} and
\ref{app:gradients_derivation} respectively.

\subsection{Line Search}
At each $m\text{-th}$ Newton iteration, our implementation of the backtracking
line-search tests candidate steps lengths until Armijo's criteria \cite[\S
3.1]{bib:nocedal2006numerical} is satisfied
\begin{algorithm}[H]
	\caption{Backtracking line-search}	
	\begin{algorithmic}
	\State With $\rho \in (0, 1)$, $c \in (0, 1)$ and $\alpha_\text{Max}>0$
	\State $\alpha\gets \alpha_\text{Max}$ 
	\RepeatUntil $~\ell_p(\alpha) <
	\ell_p(\mf{v}^m) + c\,\alpha \frac{d\ell_p}{d\alpha}(\mf{v}^m)$ \State
	$\alpha\gets \rho\alpha$
	\EndRepeatUntil
	\State\Return $\alpha$
\end{algorithmic}
\end{algorithm}	
and we typically use $\rho=0.8$, $c=10^{-4}$ and $\alpha_\text{Max}=1.5$.

For our exact line-search, we make the following observations
\begin{enumerate}
	\item $\ell_p(\alpha)$ is strictly convex, i.e. there is a unique minimum.
	\item Newton steps are well formed since $d^2\ell_p/d\alpha^2>0$.
	\item $\ell_p(\alpha)$ is a piecewise $C^1$ function. In other words,
	$d\ell_p/d\alpha$ is continuous but $d^2\ell_p/d\alpha^2$ might not be.	
	\item Given the regularization used to model friction and stiff compliance,
	gradients of $\ell_p(\alpha)$ can undergo large changes, even within a
	region where $\ell_p(\alpha)$ is continuous.
\end{enumerate}

This led us to choose a one-dimensional strategy that is robust under these
conditions. We found the method \verb;rtsafe; in \cite[\S
9.4]{bib:numerical_recipes} to work the best. \verb;rtsafe; is one-dimensional
root finder that uses the Newton-Raphson method and switches to bisection
whenever Newton's method leads to an iterate outside a search bracket or
whenever its convergence is slow. Using analytical first and second derivatives,
our line search simply reduces to finding the unique root of $d\ell/d\alpha$
using the \verb;rtsafe; algorithm. We found this method to perform so well, that
we iterate $\alpha$ to a machine precision at a negligible impact on the
computational cost. In practice this is our preferred algorithm since it allow
us to use very tight regularization parameters without having to tune tolerances
in the line search. Moreover, we observed 15\%-35\% performance improvement when
compared to the backtracking line-search.

\subsection{Efficient Analytical Derivatives For Line Search}

\verb;rtsafe; requires the first and second derivatives of $\ell_p$, while the
backtracking method only requires the first derivative to verify Armijo's
stopping criteria. We show how to compute these gradients efficiently in
$\mathcal{O}(n)$ operations.

Defining $\ell(\alpha) = \ell_p(\mf{v}+\alpha\Delta\mf{v})$ we can compute first
and second derivatives with respect to $\alpha$ using the gradient and Hessian
\begin{eqnarray}
	\frac{d\ell}{d\alpha}&=&\Delta\mf{v}^T\nabla_\mf{v}\ell(\alpha)\nonumber\\
	\frac{d^2\ell}{d\alpha^2}&=&\Delta\mf{v}^T\nabla_\mf{v}^2\ell(\alpha)\Delta\mf{v}\nonumber
\end{eqnarray}

These are expensive to compute derivatives for general non-linear functions and
most line search variations in practice are approximations that avoid their
computation altogether. We show that first and second derivatives can be
computed efficiently given the structure of the problem.

Using the gradients from Section \ref{sec:gradients} we can write
\begin{eqnarray}
	\frac{d\ell_M}{d\alpha}(\alpha)&=&\Delta\mf{v}^T\mf{A}(\mf{v}(\alpha)-\mf{v}^*)\\
	\frac{d\ell_R}{d\alpha}(\alpha)&=&-\Delta\mf{v}^T\mf{J}^T\bgamma
\end{eqnarray}
and defining the change of constraint velocity
$\Delta\mf{v}_c=\mf{J}\Delta\mf{v}$ and change of momentum $\Delta\mf{p} =
\mf{A}\Delta\mf{v}$ we obtain the much simpler and faster to compute versions
\begin{eqnarray}
	\frac{d\ell_M}{d\alpha}(\alpha)&=&\Delta\mf{p}^T(\mf{v}(\alpha)-\mf{v}^*)\\
	\frac{d\ell_R}{d\alpha}(\alpha)&=&-\Delta\mf{v}_c^T\bgamma(\alpha)
\end{eqnarray}
which only require dot products that can be computed in $\mathcal{O}(n_v)$ and
$\mathcal{O}(n_c)$ respectively.

Using the same definitions we can write simple expressions for the second
derivatives as well
\begin{eqnarray}
	\frac{d^2\ell_M}{d\alpha^2}(\alpha)&=&\Delta\mf{v}^T\mf{A}\Delta\mf{v}=\Delta\mf{v}^T\Delta\mf{p}\\
	\frac{d^2\ell_R}{d\alpha^2}(\alpha)&=&-\Delta\mf{v}_c^T
	\nabla_{\mf{v}_c}\bgamma\Delta\mf{v}_c
\end{eqnarray}
where notice that $\frac{d^2\ell_M}{d\alpha^2}$ is independent of $\alpha$ and
can be precomputed before proceeding into the line search and
$\frac{d^2\ell_R}{d\alpha^2}$ only involves $\mathcal{O}(n_c)$ operations given
the structure of $\nabla_{\mf{v}_c}\bgamma$, a block diagonal matrix.

\subsection{Problem Sparsity}
\label{sec:problem_sparsity}

The Hessian in Eq. (\ref{eq:ell_hessian}) inherits a block sparse structure from
the specific contact configuration of the problem. We want to exploit this
structure when solving Eq. (\ref{eq:Newton_iteration}) at each Newton iteration.
To that end we use a supernodal Cholesky factorization  \cite[\S
9]{bib:davis2016survey}. Implementing this factorization requires construction
of a \emph{junction tree}.  For this we apply the algorithm
\cite{bib:smail2017junction}, using cliques of $\mf{J}\,\mf{J}^T$ as input. We
use the implementation from the Conex solver \cite{bib:permenter2020}.

The block sparsity of the Hessian is best described with an example. We organize
our multibody systems as a collection of articulated \emph{tree structures}, or
a \emph{forest}. Consider the system in Fig. (\ref{fig:sparsity_example}).
\begin{figure}[!h]
	\centering
	\includegraphics[width=0.7\columnwidth]{figures/sparsity_example.png}
	\caption{\label{fig:sparsity_example} 
	Example used to describe sparsity patterns commonly encountered in the
	simulation of robotic mechanical systems. The graph on the right puts
	\textit{trees} as nodes and contact \textit{patches} as edges. Notice how
	this graph exactly describes the sparsity pattern of the matrix
	$\mf{J}^T\mf{G}\mf{J}$ in Fig. \ref{fig:JTGJ_schematic}.}
\end{figure}
In this example a robot arm mounted on a mobile base constitutes its own tree,
here labeled $t_1$. The number of degrees of freedom of the $t\text{-th}$ tree
will be denoted with $n_t$. A free body is a special case of a tree where
$n_t=6$, this is a very common case. In general the mass matrix will have a
block diagonal structure where the $t\text{-th}$ diagonal block corresponds to
the mass matrix of the $t\text{-th}$ tree. For the example in Fig.
(\ref{fig:sparsity_example}) the mass matrix will look like\\\\
\begin{equation}
	\mf{M}=\quad
	\begin{bmatrix}
		\tikzmark{M_topleft}
		\diagentry{\mf{M}_{\cc{11}}}&&&\tikzmark{M_topright}\\
		&\diagentry{\mf{M}_{\cc{22}}}\\
		&&\diagentry{\mf{M}_{\cc{33}}}\\		
		\tikzmark{M_bottomleft}&&&\diagentry{\mf{M}_{\cc{44}}}
	\end{bmatrix}
% Draw lil arrows on top and to the left.
\tikz[overlay,remember picture] {
	\draw[->,thick,color=cyan]
  ([yshift=3ex]M_topleft) -- ([yshift=3ex]M_topright) node[midway,above]
  {\scriptsize $t$}; 
  \draw[->,thick,color=cyan]
  ([yshift=1.5ex,xshift=-2ex]M_topleft) -- ([xshift=-4ex]M_bottomleft)
  node[near end,left] {\scriptsize $t$};}	
\end{equation}
\RedHighlight{Make this equation a Figure instead.}

We define as \textit{patches} a collection of contact pairs between the same two
trees. We label in red all contact patches in Fig. (\ref{fig:sparsity_example}).
Each $i\text{-th}$ pair will correspond to a single cone constraint in our
formulation. The set of constraint indexes $i$ that belong to patch $p$ is
denoted with $\mathcal{I}_p$ of size (cardinality) $|\mathcal{I}_p| = r_p$.

Notice that our definition of \textit{patches} as used here to describe sparsity
has nothing to do with the actual geometrical topology of the contact surface
between two trees. That is, the \textit{patches} as defined here, could in
general correspond to a simple connected surface or even a complex contact area
formed by a set of disconnected surfaces. Figure (\ref{fig:sparsity_example})
labels trees and patches and shows the corresponding graph where the nodes are
the trees and the edges are the contact patches.

Recall from Eq. (\ref{eq:ellR_hessian}) that $\mf{G} =
-\nabla_{\mf{v}_c}\bgamma$ is a block diagonal matrix, with $\mf{G}_i =
\nabla_{\mf{v}_{c,i}}\bgamma_i \in \mathbb{R}^{3\times 3}$ at the $i\text{-th}$
diagonal block. We can write this also as $\mf{G} = \text{diag}(\mf{G}_p)$ if we
group contact pairs by patches to define $\mf{G}_p=\text{diag}(\mf{G}_i),
\,\forall i\in\mathcal{I}_p$.

The contact Jacobian will in general be sparse since the relative velocity at a
contact pair $i$ will only involve the generalized velocities of the two trees
in contact. For the case in Fig. (\ref{fig:sparsity_example}) the Jacobian will
look like\\\\
\begin{equation}
	\mf{J}=\quad
	\begin{bmatrix}
		\tikzmark{J_topleft}\mf{0} & 
		\mf{0} & \mf{0} & \mf{J}_{\rr{1}\cc{4}}\tikzmark{J_topright}\\		
		\mf{0} & \mf{J}_{\rr{2}\cc{2}} & \mf{0} & \mf{0}\\
		\mf{J}_{\rr{3}\cc{1}} & \mf{0} & \mf{0} & \mf{0}\\
		\mf{J}_{\rr{4}\cc{1}} & \mf{0} & \mf{J}_{\rr{4}\cc{3}} & \mf{0}\\
		\tikzmark{J_bottomleft}
		\mf{0} & \mf{J}_{\rr{5}\cc{2}} & \mf{J}_{\rr{5}\cc{3}} & \mf{0}		
	\end{bmatrix}
% Draw lil arrows on top and to the left.
\tikz[overlay,remember picture] {
	\draw[->,thick,color=cyan]
  ([yshift=3ex]J_topleft) -- ([yshift=3ex]J_topright) node[midway,above]
  {\scriptsize $t$}; 
  \draw[->,thick,color=red]
  ([yshift=1.5ex,xshift=-3ex]J_topleft) -- ([xshift=-3ex]J_bottomleft)
  node[near end,left] {\scriptsize $p$};}	
\end{equation}
\RedHighlight{Make this equation a Figure instead.}\\
where each non-zero block is the Jacobian $\mf{J}_{\rr{p}\cc{t}}$ of size
$3r_p\times n_t$.

We have now the elements to describe the sparsity of the product
$\mf{J}^T\mf{G}\mf{J}$. For the example in Fig. \ref{fig:sparsity_example}, the
block sparsity of $\mf{J}^T\mf{G}\mf{J}$ is illustrated in Fig.
\ref{fig:JTGJ_schematic}. Notice how the sparsity pattern of
$\mf{J}^T\mf{G}\mf{J}$ exactly matches the graph from Fig.
(\ref{fig:sparsity_example}). Finally the Hessian will have the sparsity
structure of $\mf{A} + \mf{J}^T\mf{G}\mf{J}$.
\begin{figure*}[!h]
	\centering
	\includegraphics[width=0.8\textwidth]{figures/JTGJ_schematic.png}
	\caption{\label{fig:JTGJ_schematic} 
	Block sparsity of the $\mf{J}^T\mf{G}\mf{J}$ for the example illustrated in
	Fig. \ref{fig:sparsity_example}.}
\end{figure*}

Our implementation organizes the blocks of the Jacobian $\mf{J}$ as described in
this section, i.e. we condense rows by patches and columns by trees, so that the
supernodal Cholesky factorization can exploit this rich structure. Using this
block structure, the supernodal solver can take full advantage of specific
optimizations for dense algebra.



\subsection{Stopping Criteria}
\label{sec:stopping_criteria}

To assess convergence, we monitor the optimality condition for the unconstrained
problem in Eq. (\ref{eq:primal_unconstrained}) by evaluating the norm of the
momentum balance in Eq. (\ref{eq:primal_gradient})
\begin{equation}
	\nabla\ell_p(\mf{v}) = \mf{A}(\mf{v}-\mf{v}^*) - \mf{J}^T\bgamma.
\end{equation}

Notice that the components of $\nabla\ell_p$ have units of generalized momentum
$\mf{p}=\mf{M}\mf{v}$. Depending on the choice of generalized coordinates, the
generalized momentum components will have different units. In order to weigh all
components equally, we define the diagonal matrix $\mf{D} =
\text{diag}(\mf{M})^{-1/2}$ and perform the following change of variables
\begin{align}
	\tilde{\nabla}\ell_p &= \mf{D}\nabla\ell_p, \nonumber\\
	\tilde{\mf{p}} &= \mf{D}\mf{p}, \nonumber \\
	\tilde{\mf{j}}_c &= \mf{D}\mf{j}_c,
	\label{eq:scaled_momentum_quantities}
\end{align}
where we defined the generalized contact impulse $\mf{j}_c=\mf{J}^T\bgamma$.
With this scaling, all the new \emph{tilde} variables have the same units,
$\text{J}^{1/2}$. Using these definitions, we write our stopping criteria as
\begin{equation}
	\Vert\tilde{\nabla}\ell_p\Vert < \varepsilon_a + \varepsilon_r\max(\Vert\tilde{\mf{p}}\Vert,\Vert\tilde{\mf{j}_c}\Vert).
	\label{eq:stopping_criteria}
\end{equation}
where $\varepsilon_r$ is a dimensionless relative tolerance that we usually set
in the range $10^{-6}$ to $10^{-3}$. For all of our simulations in this paper,
we use $\varepsilon_r = 10^{-5}$. The absolute tolerance $\varepsilon_a$ is used
to detect rare cases where the solution leads to no contact and no motion,
typically due to externally applied forces. We always set this tolerance to a
small number, $\varepsilon_a=10^{-16}$.

\section{Understanding Model Parameters}
\label{sec:understanding_model_parameters}

We have provided explicit algebraic expressions for the impulses as a function of
contact velocities in Eq. \eqref{eq:analytical_y_projection} via the inverse
dynamics. While these expressions are given in terms of regularization $\mf{R}$,
Section \ref{sec:physical_intuition} describes in detail the resulting physical
model in terms of contact stiffness $k$, dissipation time constant $\tau_d$ and
friction coefficient $\mu$. Therefore, users of this model only need to provide
these physical parameters and regularization is computed from them.

Regularization parameters not only determine the physical model, but
also affect the robustness and performance of the SAP solver. Modeling near-rigid
objects and avoiding viscous drift during stiction require very small values of
$R_t$ and $R_n$ that can lead to badly ill-conditioned problems. Under these
conditions, the Hessian of the system exhibits a large condition number, and
round-off errors can render the search direction of Newton iterations
useless. We show in this section how a judicious choice of the regularization
parameters leads to much better conditioned system of equations, without
sacrificing accuracy.

\subsection{Near-Rigid Contact}

With the formulation presented in this work, all bodies are modeled as compliant.
Therefore, rigid objects must be modeled as \emph{near-rigid} bodies with large
stiffness. However, as mentioned above, blindly choosing large values of stiffness can
lead to ill-conditioned systems of equations. Here, we propose a principled way
to choose the stiffness parameter when modeling near-rigid contact.

Consider the dynamics of a mass particle $m$ laying on the ground, with contact
stiffness $k$ and dissipation time scale $\tau_d$. When in contact, the dynamics
of this particle is described by the equations of a harmonic oscillator with
natural frequency $\omega_n^2 = k/m$, or period $T_n = 2\pi/\omega_n$, and
damping ratio $\zeta=\tau_d\omega_n/2$. We say the contact is \emph{near-rigid}
when $T_n \lesssim \delta t$ and the time step $\delta t$ cannot temporally
resolve the contact dynamics. 

In this \emph{near-rigid} regime, we use compliance as a means to add a
Baumgarte-like \emph{stabilization} to avoid constraint drift, as similarly
done in \cite{bib:todorov2011}. Choosing the time scale of the contact to be
$T_n = \beta \delta t$ with $\beta \le 1$, we model inelastic contact with a dissipation
that leads to a critically damped oscillator, or $\zeta=1$. This dissipation is
$\tau_d=2\zeta/\omega_n$, or in terms of the time step,
\begin{equation}
    \tau_d=\frac{\beta}{\pi}\delta t.
\end{equation}

Using the harmonic oscillator equations, we can estimate the value of stiffness
from the frequency $\omega_n$ as $k=4\pi^2 m/(\beta^2 \delta t^2)$. Since
$\tau_d\approx\delta t$, $R_n^{-1} = \delta t k(\delta t+\tau_d) \approx \delta
t^2k$, and we estimate the regularization parameter as
\begin{equation}
	R_n = \frac{\beta^2}{4\pi^2}\text{w},
\end{equation}
where we defined $\text{w}=1/m$.

It is useful to estimate the amount of penetration for a point mass resting on
the ground. In this case we have
\begin{align*}
	\phi &= \frac{mg}{k} \\
	&= \frac{\beta^2}{4\pi^2}m\text{w}g\delta t^2\\
	&= \frac{\beta^2}{4\pi^2}g\delta t^2.
\end{align*}
Taking $\beta=1.0$ and Earth's gravitational constant, a typical simulation
time step of $\delta t=10^{-3}~\text{s}$ leads to $\phi\approx 2.5\times
10^{-7}~\text{m}$, and a large simulation time step of $\delta
t=10^{-2}~\text{s}$ leads to $\phi\approx 2.5\times 10^{-5}~\text{m}$, well
within acceptable bounds to consider a body rigid for typical robotics
applications.

For a general multibody system, we define the per-contact effective mass as
$\text{w}_i=\Vert\mathbf{W}_{ii}\Vert_\text{rms}=\Vert\mathbf{W}_{ii}\Vert/3$ where
$\mathbf{W}_{ii}$ is the $3\times 3$ diagonal block of the Delassus operator
$\mathbf{W}=\mf{J}\mf{M}^{-1}\mf{J}^T$ for the $i$-th contact. Explicitly forming the Delassus
operator is an expensive operation. Refer to Appendix \RedHighlight{[O(n)
Approximation of W]} for a diagonal approximation of $\mathbf{W}$ that can be
computed in $\mathcal{O}(n)$ operations. Using this approximation, we estimate the
frequency of the contact dynamics as $\omega_n=\sqrt{k\text{w}_i}$.

Finally, we compute the regularization parameter in the normal direction as
\begin{eqnarray}
    R_n = \max\left(\frac{\beta^2}{4\pi^2}\Vert\mathbf{W}_{ii}\Vert_\text{rms}, 
    \frac{1}{\delta t k(\delta t+\tau_d)}\right)
    \label{eq:normal_regularization}.
\end{eqnarray}
With this strategy, our model automatically switches between modeling compliant
contact with stiffness $k$ when the time step $\delta t$ can resolve the
temporal dynamics of the contact, and using stabilization to model near-rigid
contact with the amount of stabilization controlled by parameter
$\beta$. In all of our simulations, we use $\beta=1.0$.

\subsection{Stiction}
Given that our model regularizes friction, we are interested in estimating a bound on
the slip velocity at stiction. We propose the following regularization for friction
\begin{equation}
    R_t = \sigma \text{w}
    \label{eq:slip_time_scale}
\end{equation}
where $\sigma$ is a dimensionless parameter.

To understand the effect of $\sigma$ in the approximation of stiction, we
consider once again a point of mass $m$ in contact with the ground under gravity,
for which $\text{w}\approx 1/m$. We push the particle with a horizontal force of magnitude
$F=\mu\gamma_n$ so that friction is right at the boundary of the friction cone and
the slip velocity due to regularization, $v_s$, is maximized. Then in stiction, we have
\begin{equation}
    \|\bgamma_t\| = \frac{v_s}{R_t} = \mu m g \delta t
\end{equation}
Using our proposed regularization in Eq. (\ref{eq:slip_time_scale}), we find
the maximum slip velocity
\begin{equation}
    v_s \approx \mu\sigma g \delta t.
    \label{eq:slip_estimation}
\end{equation}

We see that this maximum slip velocity is independent of the mass and linear
with the time step size. Even though the friction coefficient $\mu$ can take any
non-negative value, most often in practical applications $\mu < 1$. Values in the
order of 1 are in fact considered as large friction values. Therefore, for this
analysis we consider $\mu\approx 1$.
In all of our simulations, we use $\sigma=10^{-3}$. With
Earth's gravitational constant, a typical simulation with time step of $\delta
t=10^{-3}~\text{s}$ leads to a stiction velocity of $v_s\approx
10^{-5}\text{m}/\text{s}$, and with a large step of $\delta t=10^{-2}~\text{s}$,
$v_s\approx 10^{-4}\text{ m}/\text{s}$. Smaller friction coefficients lead to
even tighter bounds. These values are well within acceptable bounds even for
simulation of grasping tasks, which are significantly more demanding than
simulation for other robotic applications, see Section \ref{sec:test_cases}.

\subsection{Sliding Soft Contact}

As we discussed in Section \ref{sec:physical_intuition}, we require $R_t/R_n\ll
1$ so that we model compliance accurately during sliding. Now, in the
\emph{near-rigid} contact regime, the condition $R_t/R_n\ll 1$ is no longer
required since in this regime regularization is only used to apply stabilization
and avoid constraint drift. Therefore, we only need to verify this condition in
the \emph{soft contact} regime, when time step $\delta t$ can properly resolve
the contact dynamics, i.e. according to our criteria, when $\delta t < T_n$. In
this regime, $R_n^{-1}\approx \delta t^2k$, and using Eq.
(\ref{eq:slip_time_scale}) we have
\begin{equation}
    \frac{R_t}{R_n}\approx \sigma \delta t^2 \omega_n^2=4\pi^2\sigma\left(\frac{\delta t}{T_n}\right)^2
    \lesssim 4\pi^2\sigma
    \label{eq:rtrn_ratio}
\end{equation}
where in the last inequality we used the assumption that we are in the soft
regime where $\delta t < T_n$. Since $\sigma \ll 1$ and in particular we
use $\sigma=10^{-3}$ in all of our simulations, we see that $R_t/R_n \ll
1$. Moreover, $R_t/R_n$ goes to zero quadratically with $\delta t/T_n$ as
the time step is reduced and the dynamics of the compliance is better resolved
in time.

Therefore, we have shown that our choice of regularization parameters enjoys the
following properties
\begin{enumerate}
    \item Users only provide physical parameters; contact stiffness $k$, dissipation time scale
    $\tau_d$, and friction coefficient $\mu$. There is no need for users to tweak
    solver parameters.
    \item In the \emph{near-rigid} limit, our regularization in
    Eq. (\ref{eq:normal_regularization}) automatically switches the method to
    model rigid contact with constraint stabilization to avoid excessively large
    stiffness parameters and the consequent ill-conditioning of the system.
    \item Frictional regularization is parameterized by a single dimensionless
    parameter $\sigma$. We estimate a bound for the slip velocity during
    stiction to be $v_s \approx \mu \sigma \delta t g$. For
    $\sigma=10^{-3}$, the slip during stiction is well within acceptable
    bounds for robotics applications.
    \item We show that $R_t/R_n \ll 1$ when $\delta t$ can resolve the dynamics
    of the compliant contact, as required to accurately model compliance during
    sliding.
\end{enumerate}

In addition, our tests cases in Section \ref{sec:test_cases} show that this
choice of regularization parameters keeps the condition number of the Hessian at
each Newton iteration under control.

\section{Test Cases}
\label{sec:test_cases}

We evaluate the robustness, accuracy, and performance of our method in a number
of simulation tests. All simulations are carried out in a system with 24 2.2 GHz
Intel Xeon cores (E5-2650 v4) and 128 GB of RAM, running Linux. However, all of
our tests are run in a single thread.

For all of our simulations, unless otherwise specified, our model uses
$\beta=1.0$ and $\sigma=10^{-3}$ for the regularization parameters in Eq.
(\ref{eq:normal_regularization}) and Eq. (\ref{eq:tangential_regularization}),
respectively.

\subsection{Performance Comparisons Against Other Solvers}
\label{sec:about_solvers}

We evaluate commercial software Gurobi, considered an industry standard, to
solve our primal formulation \eqref{eq:primal_regularized}. As an open source option, we evaluate the Geodesic interior-point method (IPM)
from \cite{bib:permenter2020}. Geodesic IPMs, in contrast with primal-dual
IPMs, do not apply Newton's method to the central-path conditions directly.
Instead, they use geodesic curves that satisfy the complementarity slackness
condition. Since the Geodesic IPM and SAP use the same supernodal linear algebra
code described in Section \ref{sec:problem_sparsity}, it is natural to compare
their performance.

For performance comparisons, we use the steady clock from the STL
\verb+std::chrono+ library to measure wall-clock time for SAP and Geodesic IPM.
For Gurobi we access the \verb+Runtime+ property reported by Gurobi. Notice this
is somewhat unfair to SAP and Geodesic IPM since Gurobi's reported time does not
include the cost of the initial setup.

\subsection{Spring-Cylinder}
\label{sec:spring_cylinder}
We are interested in evaluating the properties of the schemes presented in
Section \ref{sec:discrete_time_formulation} when used to simulate a mechanical
system with frictional contact.

To this end, we use the setup shown in Fig. \ref{fig:spring_cylinder}, consisting
of a cylinder of radius $R=0.05\text{ m}$ and mass $m=0.5\text{ kg}$ connected
to a wall to its left by a spring of stiffness $k_s=100\text{ N}/\text{m}$.
While the cylinder is free to rotate and translate in the plane, the contact
force between the cylinder and the ground constrains the cylinder's motion in
the vertical direction. The contact stiffness is $k=10^{4}\text{ N}/\text{m}$
and the dissipation time scale is $\tau_d=0.02\text{ s}$. The cylinder is
initially placed with zero velocity at $x_0=0.1\text{ m}$ to the right of the
spring's resting position, and it is then set free.
\begin{figure}[!h]
	\centering
	\includegraphics[width=0.6\columnwidth]{figures/spring_cylinder/hand_drawn_schematic.png}
	\caption{\label{fig:spring_cylinder} 
	Spring-Cylinder system. The cylinder can translate horizontally and rotate.
	Friction with the ground establishes a non-dissipative rolling contact and
	therefore the total mechanical energy is conserved.}
\end{figure}

For reference, we first simulate this setup with frictionless contact, i.e.
$\mu=0$. Without friction, the cylinder does not rotate and we effectively have a
spring-mass system with natural frequency $\omega_n=\sqrt{k_s/m}$. We use a
rather coarse time step of $\delta t=0.02\text{ s}$, discretizing each period of
oscillation with about $22$ steps. Figure
\ref{fig:frictionless_spring_cylinder_energy} shows the total mechanical energy
as a function of time computed using three different schemes; symplectic Euler,
midpoint rule, and implicit Euler. The amount of numerical dissipation
introduced by the implicit Euler scheme dissipates the initial energy in just a
few periods of oscillation. For the symplectic Euler scheme, we observe in Fig.
\ref{fig:frictionless_spring_cylinder_energy} that, while the energy is not
conserved, it stays bounded, within a band 28\% peak-to-peak wide. The figure
also confirms that the second order midpoint scheme conserves energy exactly for
the spring-mass system. These are well known theoretical properties of these
integration schemes when applied to the spring-mass system.
\begin{figure}[!h]
    \centering
    %trim={<left> <lower> <right> <upper>}
    \adjincludegraphics[width=0.49\columnwidth,trim={0 0 {0.05\width} 0},clip]{figures/spring_cylinder/frictionless_total_energy.png}
    \adjincludegraphics[width=0.49\columnwidth,trim={0 0 {0.05\width} 0},clip]{figures/spring_cylinder/frictionless_total_energy_long_term.png}    
    \caption{\label{fig:frictionless_spring_cylinder_energy} 
    Total mechanical energy for the frictionless spring-cylinder system in the
    first few periods of oscillation (left) and long term (right).}
\end{figure}

We now focus our attention to a case with frictional contact using $\mu=1$. As
we release the cylinder from its initial position at $x_0=0.1\text{ m}$,
friction with the ground establishes a rolling contact, and the system sets into
a periodic motion. Since now kinetic energy is split into translational and
rotational components, the rolling cylinder behaves as a spring-mass system with
an effective mass $m_\text{eff}=m+I_o/R^2$, with $I_o$ the rotational inertia of
the cylinder about its center. Therefore the frequency of oscillation is slower
and the same time step $\delta t=0.02\text{ s}$ as before now discretizes one
period of oscillation with about 27 steps.

Total energy is shown in Fig. \ref{fig:spring_cylinder_energy}. As before, the
implicit Euler scheme quickly dissipates the energy of the system. The
symplectic Euler scheme now exhibits a smaller a peak-to-peak variation of 24\%.
This is expected since now one one period of oscillation is resolved with 27
steps instead of 22. With friction, the midpoint rule does not conserve energy
exactly, but it does significantly better with a peak-to-peak variation of only
0.16\%. While the ideal rolling contact does not dissipate energy, the
regularized model of friction does dissipate energy given the slip velocity is
never exactly zero, though small in the order of $\sim\sigma\mu\delta t g$ as shown
in Section \ref{sec:physical_intuition}. The symplectic Euler scheme and the
midpoint rule take $10$ minutes of simulated time and about $1000$ oscillations
to dissipate 10\% of the total energy (see Fig. \ref{fig:spring_cylinder_energy}
right). This level of numerical dissipation is remarkably low when compared to a
real mechanical system with friction, considering that energy is also dissipated
through other means in real life.
\begin{figure}[!h]
    \centering
    %trim={<left> <lower> <right> <upper>}
    \adjincludegraphics[width=0.49\columnwidth,trim={0 0 {0.05\width} 0},clip]{figures/spring_cylinder/total_energy.png}
    \adjincludegraphics[width=0.49\columnwidth,trim={0 0 {0.05\width} 0},clip]{figures/spring_cylinder/total_energy_long_term.png}    
    \caption{\label{fig:spring_cylinder_energy} 
    Total mechanical energy for the spring-cylinder system with friction $\mu=1$
    in the first few periods of oscillation (left) and long term (right).}
\end{figure}

We also study the order of accuracy of our approach. We define the position error as the $L^2$-norm
\begin{equation*}
    e_q = \left(\frac{1}{T}\int_0^T dt(x(t)-x_e(t))^2\right)^{1/2}
\end{equation*}
where $x_e(t)$ is the known exact solution. We simulate for $T=5\text{
s}$, about 10 periods of oscillation. Figure \ref{fig:spring_cylinder_position_error} shows the position error as
a function of the time step. We see that even with frictional contact, the two-stage approach
with the midpoint rule achieves second order of accuracy. Both the implicit Euler and the symplectic Euler schemes are first order,
though the error is significantly smaller when using the symplectic Euler
scheme.
\begin{figure}[!h]
	\centering
	\includegraphics[width=0.6\columnwidth]{figures/spring_cylinder/position_error.png}
	\caption{\label{fig:spring_cylinder_position_error} 
	Position error as a function of time step for the spring-cylinder system
	with friction. First and second order references are shown with dashed
	lines.}
\end{figure}

\subsection{Clutter}
\label{sec:clutter}
The purpose of this case is to evaluate robustness, convergence properties, and
scalability of our SAP solver when compared against other existing methods.

The setup consists of an open container with a square base of size
$80~\text{cm}\times80~\text{cm}$ and $80~\text{cm}$ in height. Bodies are
dropped into this container in four different columns with the same number of
bodies each (see Fig. \ref{fig:clutter_snapshots}). Each column consists of an
arbitrary assortment of spheres of radius $5~\text{cm}$ and boxes with sides of
$10~\text{cm}$ in length. All bodies have density $1000\text{
kg}/\text{m}^3$ and therefore each sphere has a mass of approximately
$0.524\text{ kg}$ and each box has a mass of $1.0\text{ kg}$. We set a very high
contact stiffness of $k=10^{12}\text{ N}/\text{m}$ so that the model is in the
\emph{near-rigid} regime. The dissipation time scale is set to equal the time
step and the friction coefficient of all surfaces is $\mu=1.0$.
\begin{figure}[t]
    \centering
    %trim={<left> <lower> <right> <upper>}
    \adjincludegraphics[width=0.49\columnwidth,trim={0 {0.05\height} 0 0},clip]{figures/clutter/clutter_w_walls_t0.png}
    \adjincludegraphics[width=0.49\columnwidth,trim={0 {0.05\height} 0 0},clip]{figures/clutter/clutter_no_walls_t0.png}\\
    \vspace{0.1cm}
    \adjincludegraphics[width=0.49\columnwidth,trim={0 {0.05\height} 0 0},clip]{figures/clutter/clutter_w_walls_t2_zoom.png}
    \adjincludegraphics[width=0.49\columnwidth,trim={0 {0.05\height} 0 0},clip]{figures/clutter/clutter_no_walls_t2_zoom.png}
    \caption{Initial conditions (top) and an intermediate configuration after $2$ seconds of
    simulated time (bottom) for the clutter setup with (left) and without
    (right) walls. Many of the spheres in the configuration with no walls
	roll outside the frame in the intermediate configuration.}
    \label{fig:clutter_snapshots}
\end{figure}

We first run our simulations with 10 bodies per column for a total of 40 bodies.
We simulate 10 seconds using time steps of size $\delta t = 10\text{ ms}$.
Number of solver iterations and wall-clock time per time step are reported in
Fig. \ref{fig:clutter_w_walls_nb40} for the three solvers. We observe that SAP
needs to perform a larger number of iterations during the very energetic initial
transient. As the system reaches a steady state, however, SAP warm starts very
effectively, performing only about $3$ iterations per time step. Even though
SAP necessities a larger number of iterations to converge than Geodesic IPM during this initial transient, the wall-clock time per time step is very
similar. This tells us that SAP's cost per iteration is lower than that of
Geodesic IPM, even when they both use the same supernodal algebra. Unlike SAP and Geodesic IPM that benefit from warm start, we see that Gurobi performs
about $9$ iterations per time step in both the initial transient and the steady state.
\begin{figure}[!h]
	\centering
    %trim={<left> <lower> <right> <upper>}
    \adjincludegraphics[width=0.49\columnwidth,trim={0 0 0 0},clip]{figures/clutter/iterations_nb40.png}
    \adjincludegraphics[width=0.49\columnwidth,trim={0 0 0 0},clip]{figures/clutter/wall_clock_nb40.png}    
	\caption{\label{fig:clutter_w_walls_nb40} 
	Iterations and wall-clock time per time step for SAP, Geodesic IPM, and
	Gurobi for the clutter case with 40 bodies and with walls. Most of the
	energy is lost during the first $\sim300$ time steps as the objects pile up
	at the bottom of the box.}
\end{figure}

Figure \ref{fig:clutter_line_search} shows two examples of convergence history
for the case with walls. We denote with $\ell^0$ to the cost evaluated at the
initial guess, the previous time step velocity. With $\ell_*$ we denote the
optimal cost, which we approximate with its value from the last iteration. At
step 60 during the initial transient for which SAP requires 21 iterations to
converge, we observe that the algorithm reaches quadratic convergence after an
initial linear convergence transient. At step 520, past the initial energetic
transient, SAP exhibits linear convergence and satisfies the convergence
criteria within 5 iterations. We observe this convergence behavior for other
time steps as well. For time steps requiring about 10 iterations or more, SAP
exhibits linear convergence followed by quadratic convergence. For time steps
requiring less than about 10 iterations, the linear convergence regime is enough
to reach convergence.
\begin{figure}[!h]
	\centering
    \includegraphics[height=0.34\columnwidth]{figures/clutter/normalized_cost_step60_21its_wwalls_latex_labels.png}
	\includegraphics[height=0.34\columnwidth]{figures/clutter/normalized_cost_step520_5its_wwalls_latex_labels.png}    
	\caption{\label{fig:clutter_line_search} 
	Normalized cost as a function of Newton iterations for step 60 (left) and for step 520 (right). SAP ensures that the cost decreases with each newt iteration. Reference lines are shown for linear convergence (dotted) and quadratic convergence (dashed).}
\end{figure}

\subsubsection{Scalability}

We evaluate the performance of SAP with different problem sizes by varying the
number of objects in the clutter with all other parameters held constant. We
study scalability of the test case both with and without walls (see
Fig. \ref{fig:clutter_snapshots}) as the variation in the setup leads to very
different sequences of contact configuration. The size of the problems can be
appreciated in Fig. \ref{fig:clutter_num_contats} showing the number of contact
constraints at the end of the simulation when objects are in steady state
against the number of objects. We observe a larger number of contacts for the
configuration without walls since in this configuration many of the boxes spread
over the ground and lay flat on one of their faces, leading to a multi-contact
configuration (see Fig. \ref{fig:clutter_snapshots} for instance). Notice that
each body contributes 6 DOFs and each contact constraint contributes 3 unknowns.
Therefore, in the case with 200 bodies, the problem involves 1200 DOFs and about
2700 contact unknowns for a total of about 3000 unknowns.
\begin{figure}[!h]
	\centering
	\includegraphics[width=0.6\columnwidth]{figures/clutter/number_of_contacts.png}
	\caption{\label{fig:clutter_num_contats} 
	Total number of contacts with objects in steady state at the end of the
	simulation for setup with and without walls.}
\end{figure}

We measure the total time spent by each solver and define the \emph{speedup}
against Gurobi. Figure \ref{fig:clutter_speedup} shows the speedup for both
SAP and Geodesic IPM in the configuration with and without walls.
The setup with walls is particularly difficult given that objects are
constrained to pile up, leading to a configuration in which almost all objects
are coupled with every other object by frictional contact
(see Fig. \ref{fig:clutter_snapshots}). That is, the motion of an object at the
bottom of the pile can lead to motion of another object far on top of the pile.
In contrast, the simulation with no walls leads to \emph{islands} of objects
that do not interact with each other.

In general, we observe two regimes. For problems with less than about 40 bodies,
SAP outperforms Gurobi significantly by up to a factor of 25 in the case with
walls and up to a factor of 50 with no walls. Beyond 80 bodies, Gurobi
outperforms both SAP and Geodesic IPM in the case with walls, but SAP is about
10 times faster for the case with no walls. Though SAP shows to be about twice
as fast as Geodesic IPM for most problem sizes, it can be five times faster for
small problems with 8 bodies or less.
\begin{figure}[!h]
	\centering
    %trim={<left> <lower> <right> <upper>}
    \adjincludegraphics[width=0.49\columnwidth,trim={{0.02\width} 0 {0.05\width} 0},clip]{figures/clutter/speedup_w_walls.png}
    \adjincludegraphics[width=0.49\columnwidth,trim={{0.02\width} 0 {0.05\width} 0},clip]{figures/clutter/speedup_no_walls.png}
	\caption{\label{fig:clutter_speedup} 
	Speedup against Gurobi for both SAP and Geodesic IPM for the configuration with walls (left) and without walls (right).}
\end{figure}

It could be argued that these speedup results depend on the accuracy settings of
each solver. For a fair comparison, we define the dimensionless momentum error
as
\begin{equation}
	e_m = \frac{\Vert\tilde{\nabla}\ell_p\Vert}{\max(\Vert\tilde{\mf{p}}\Vert,\Vert\tilde{\mf{j}_c}\Vert)},\nonumber
\end{equation}
using the scaled generalized momentum quantities in Eq.
(\ref{eq:scaled_momentum_quantities}). We assess the accuracy of the
complementarity slackness with the dimensionless quantity
\begin{eqnarray*}
	e_\mu = \frac{1/n_c\sum_i|\bm{g}_i\cdot\bgamma_i|}{\ell_p}.
\end{eqnarray*}

Figure \ref{fig:clutter_errors_w_wall} shows average values of $e_m$ and $e_\mu$
over all time steps. Since SAP satisfies the complementarity slackness exactly,
$e_\mu$ is not shown. We have verified this to be true within machine precision
for all simulated cases.

SAP's momentum error is below $10^{-5}$ as expected since this is the value used
for the termination condition. Similarly, the complementarity slackness is below
$10^{-5}$ for Geodesic IPM, since this is the value used for its own termination
condition. Gurobi does a good job at satisfying the complementarity slackness.
However, it is the solver with the largest error in the momentum equations, even
though both SAP and Geodesic IPM outperform Gurobi in most of the test cases.
These metrics demonstrate that when SAP and Geodesic IPM outperform Gurobi, it
is not at the cost of losing accuracy.
\begin{figure}[!h]
	\centering
    %trim={<left> <lower> <right> <upper>}
    \adjincludegraphics[height=0.40\columnwidth,trim={0 0 {0.05\width} 0},clip]{figures/clutter/momentum_error_w_walls.png}
	\adjincludegraphics[height=0.40\columnwidth,trim={{0.05\width} 0 {0.05\width} 0},clip]{figures/clutter/momentum_error_no_walls.png}\\
    \adjincludegraphics[height=0.40\columnwidth,trim={0 0 {0.05\width} 0},clip]{figures/clutter/optimality_condition_error_w_walls.png}
    \adjincludegraphics[height=0.40\columnwidth,trim={{0.05\width} 0 {0.05\width} 0},clip]{figures/clutter/optimality_condition_error_no_walls.png}
	\caption{\label{fig:clutter_errors_w_wall} 
	Momentum balance error $e_m$ (top) and complementarity condition error
	$e_\mu$ (bottom) for the clutter case with walls (left) and without walls
	(right).}
\end{figure}

\subsubsection{Slip Parameter}

We study the effect of the slip parameter $\sigma$ in Eq.
(\ref{eq:slip_time_scale}). As before, we use $\delta t = 10\text{ ms}$ and
simulate 40 objects for 10 seconds to a steady state configuration. At this
steady state at the end of the simulation, we compute the mean slip velocity
among all contacts. Figure \ref{fig:clutter_sigma_vt} shows this mean slip
velocity along with the estimated slip in Eq. (\ref{eq:slip_estimation}), $v_s
\approx\sigma\mu\delta t g$, shown in dashed lines. We see that the mean slip
velocity remains below the estimated slip as expected in a static configuration
with objects in stiction. In the case with walls where stiction helps to hold
the steady state static configuration, we see that the mean slip velocity
closely follows the slope of the slip estimate. Without the walls, objects do
not pile up in a complex static structure but simply lie on the ground, and
therefore, the resulting slip velocities are significantly smaller. The sudden
drop in the slip velocity for $\sigma>10^{-3}$ is caused by the sensitivity of
the final state on the value of $\sigma$. As $\sigma$ increases, so does the
slip velocity bound $v_s$ and objects in the configuration without walls can
slowly drift into a configuration leading to more contacts. In particular, boxes
are more likely to slowly drift until one of their faces lies flat
on the ground, a configuration with zero slip once steady state is reached.
Finally, we observe that when using $\sigma=10^{-3}$, the amount of slip is
negligible for robotic applications.
\begin{figure}[!h]
	\centering
	\includegraphics[width=0.6\columnwidth]{figures/clutter/sigma_vt.png}
	\caption{\label{fig:clutter_sigma_vt} 
	Mean slip velocity at the end of the simulation with objects at rest as a
	function of the slip parameter. The estimated slip $v_s = \sigma\mu\delta t g$ is shown in dashed lines.}
\end{figure}

We conclude by examining the effect of $\sigma$ on the conditioning of the
system. Figure \ref{fig:clutter_sigma} shows the condition number of the Hessian
in the final configuration and the mean number of Newton iterations throughout
the simulation. We see that the condition number scales as $\sigma^{-1}$ while
the mean number of Newton iterations is roughly proportional to  $\ln(\sigma)$.
Notice that our choice $\sigma=10^{-3}$ for this paper is placed right in the
middle, in a log scale, of the range of values examined in this study.
\begin{figure}[!h]
	\centering
    %trim={<left> <lower> <right> <upper>}
    \adjincludegraphics[width=0.49\columnwidth,trim={0 0 {0.05\width} 0},clip]{figures/clutter/sigma_iterations.png}
    \adjincludegraphics[width=0.49\columnwidth,trim={0 0 {0.05\width} 0},clip]{figures/clutter/sigma_condition_number.png}
	\caption{\label{fig:clutter_sigma} 
	Effect of the slip parameter on the conditioning of the system. Mean Newton iterations per step (left) and mean condition number (right).}
\end{figure}


\subsection{Slip Control}
\label{sec:slip_control}
Sim results for the Bubble griper.

\subsection{Dual Arm Manipulation}
\label{sec:dual_arm}
\begin{figure}[!h]
	\centering
    \includegraphics[width=0.95\columnwidth]{figures/dual_arm/dual_arm_contact.png}
    \caption{\label{fig:dual_arm_contacts} Number of contact vs. time.}
\end{figure}



\section{Variations and Extensions}
\label{sec:variations_and_extensions}

The method presented in this paper can be extended in several ways:

\textbf{Expand the family of constraints:} No doubt contact constraints are the
most challenging. However, our method can be extended to include bilateral
constraints, PD controllers with force limits and even joint dry friction
\cite{bib:todorov2014}.

\textbf{Branch induced sparsity:} In this work we exploit sparsity only at the
tree level. However, branch sparsity can lead to additional performance.
Consider for instance a standing humanoid robot with a floating hip. Since arms and the upper
torso are not in contact with the ground, they can be eliminated from the
computation in terms of degrees of freedom in the legs. Additional performance
gains could be attained using specialized algebra for multibody dynamics
\cite{bib:carpentier2021}.

\textbf{Parallelization:} This work focuses on accuracy, robustness, and
convergence properties of the algorithm executed in a single thread. The sparse
algebra can be parallelized and, in particular, disjoints \emph{islands} of
bodies can be solved separately in different threads.

\reviewquestion{R4-Q3}{\textbf{Deformable FEM models:} Using the SAP solver for the modeling of deformable objects with
contact and friction is the topic of current research efforts by the authors.
FEM models lead to state dependent stiffness \eqref{eq:stiffness_matrix} and
damping \eqref{eq:dissipation_matrix} matrices with a complex structure that
requires specialized handling of sparsity. Moreover, modeling assumptions must
be carefully analyzed in order to ensure the positive definiteness of these
matrices used in our convex approximation of contact.}

\textbf{Differentiation:} Since forces are a continuous function of state, the
model is well suited for applications requiring gradients such as trajectory
optimization, machine learning, parameter estimation, and control.
Factorizations computed during forward dynamics can be reused when computing
gradients for a performant implementation.

\section{Conclusion}
\label{sec:future_directions}

We presented a formulation of compliant contact with a novel physics-based
parameterization. We show that forces can be succinctly described by analytical
expressions with a clear physical intuition. This allowed us to incorporate not
only point contact but also more complex models of surface patches. We then show
that when these forces are used in the momentum equations, we obtain the
optimality conditions for an unconstrained convex formulation. We make a
rigorous presentation of the numerical approximations and a novel
characterization of the artifacts introduced by the convex approximation of
contact; the approximation is exact for sticking contact and introduces an
$\mathcal{O}(\delta t\|\vf{v}_t\|)$ \emph{gliding} effect for sliding contact.

We developed a two-stages time stepping approach based on the
$\theta\text{-method}$ and we show that with the midpoint rule it can achieve
second order accuracy even in problems with frictional contact. Our formulation
does not linearize the friction cone but it works with the second order cone
constraints directly.

We presented SAP, a robust and performant solver for this formulation that
warm-starts very effectively in practice. We provide global converge guarantees
from any initial guess and show: worst case linear convergence, and quadratic
convergence when additional smoothness conditions are satisfied. We show SAP
exhibits these two converge regimes in simulations of practical relevance. Our
work includes thorough details for implementation, including analytical
formulae for gradients and Hessian, sparsity analysis and custom line-search.

We compare the performance of SAP against commercial and open source
optimization solvers. Using quantitative accuracy metrics we show that SAP
outperforms the alternatives not only without sacrificing accuracy, but even at
a higher accuracy and added robustness. SAP can be up to 50 times faster than
Gurobi in small problems with up to a dozen objects and up to 10 times faster in
medium sized problems with about 100 objects. Even though SAP uses the
supernodal algebra implemented for Geodesic IPM, it performs at least two times
faster given how effectively it warms-starts from the previous time-step
solution. Moreover, SAP exhibits significantly more robustness in practice given
that it guarantees a hard bound on the error in momentum, effectively providing
a certificate of accuracy.


% Appendices:
\appendices
\section{Proof of Proposition \ref{prop:gradient_of_m_approximation}}
\label{app:gradient_of_m_approximation}
The Taylor expansion of $\mf{m}(\mf{v})$ at $\mf{v}=\mf{v}^*$ reads
\begin{align}
	\mf{m}(\mf{v}) &= \mf{m}^* + \frac{\partial \mf{m}}{\partial \mf{v}} (\mf{v}-\mf{v}^*) +
	\mathcal{O}_m(\Vert\mf{v}-\mf{v}^*\Vert^2)\nonumber\\
	&=\frac{\partial \mf{m}}{\partial \mf{v}}(\mf{v}-\mf{v}^*) +
	\mathcal{O}_m(\Vert\mf{v}-\mf{v}^*\Vert^2),
	\label{eq:m_taylor_expansion}
\end{align}
where we use the fact that by definition $\mf{m}^*=\mf{m}(\mf{v}^*)=\mf{0}$. All
derivatives are evaluated at $\mf{v} = \mf{v}^*$ unless otherwise noted. We
first evaluate the Jacobian of the mass matrix term in Eq.
(\ref{eq:m_definition})
\begin{align*}
	\frac{\partial \left( \mf{M}(\mf{q}^{\theta}(\mf{v}))(\mf{v}-\mf{v}_0) \right)}{\partial \mf{v}}
	= \mf{M}(\mf{q}^{\theta}(\mf{v}^*)) + \mf{E},
\end{align*}
where we defined
\begin{align*}
	\mf{E} = \frac{\partial \mf{M}(\mf{q}^{\theta})}{\partial\mf{v}} (\mf{v}^*-\mf{v}_0).
\end{align*}
Note that by combining Eqs. (\ref{eq:theta_method}) and (\ref{eq:scheme_q_update}), the
mid-step configuration $\mf{q}^{\theta}$ can be written as
\begin{align*}
	\mf{q}^{\theta}(\mf{v}) &= \mf{q}_0 + \delta t \theta \dot{\mf{q}}^{\theta_{vq}} \\
	                          &= \mf{q}_0 + \delta t \theta \mf{N}(\mf{q}^{\theta})\mf{v}^{\theta_{vq}}(\mf{v}).
\end{align*}
Hence by the chain rule, $\mf{E}$ can be further calculated as
\begin{align*}
	\mf{E} = \delta t\theta\frac{\partial \mf{M}(\mf{q}^{\theta}) }{\partial\mf{q}}
             \frac{\partial\dot{\mf{q}}^{\theta_{vq}}}{\partial\mf{v}}
			 (\mf{v}^*-\mf{v}_0).
\end{align*}
Notice that 
\begin{align*}
		\Vert\mf{E}\Vert 
		&\le \delta t\theta \left\| \frac{\partial\mf{M}(\mf{q}^{\theta})}{\partial\mf{q}}  \right\|
			\left\| \frac{\partial\dot{\mf{q}}^{\theta_{vq}}}{\partial\mf{v}}  \right\|
		    \left\| \mf{v}^*-\mf{v}_0 \right\| \\
		&= \mathcal{O}(\delta t^2),
\end{align*}
since $\Vert\mf{v}^*-\mf{v}_0\Vert = \mathcal{O}(\delta t)$.

We proceed similarly to expand the Jacobian of
$\mf{F}_1(\mf{v})=\mf{F}_1(\mf{q}^{\theta}(\mf{v}), \mf{v}^{\theta}(\mf{v}))$
as
\begin{align*}
	\frac{\partial\mf{F}_1}{\partial \mf{v}} = -\delta t\,\theta\theta_{vq}\mf{K}(\mf{q}^{\theta},
	\mf{v}^{\theta})-\theta\mf{D}(\mf{q}^{\theta}, \mf{v}^{\theta}),
\end{align*}
with $\mf{K}$ and $\mf{D}$ the stiffness and damping matrices defined by Eqs.
(\ref{eq:stiffness_matrix})-(\ref{eq:dissipation_matrix}).

We can now write the Jacobian of $\mf{m}(\mf{v})$ in Eq.
(\ref{eq:m_taylor_expansion}) as
\begin{align*}
	\frac{\partial \mf{m}}{\partial \mf{v}} = \mf{A} + \mf{E} - \delta t\frac{\partial \mf{F}_2}{\partial \mf{v}},
\end{align*}
where we defined
\begin{align*}
	\mf{A}=\mf{M}+ \delta t^2\theta\theta_{qv}\mf{K}+\delta t\theta\mf{D}.
\end{align*}
With these definitions the Taylor expansion in Eq. (\ref{eq:m_taylor_expansion})
becomes
\begin{align*}
	\frac{\partial\mf{m}}{\partial\mf{v}}(\mf{v}-\mf{v}^*) &= \mf{A}(\mf{v}-\mf{v}^*) + \mf{E}(\mf{v}-\mf{v}^*) \\
	&- \delta t\frac{\partial\mf{F}_2}{\partial\mf{v}}(\mf{v}-\mf{v}^*) + \mathcal{O}_m(\Vert\mf{v}-\mf{v}^*\Vert^2).
\end{align*}

Since contact is compliant, forces are finite within the finite interval $\delta
t$ and therefore $\Vert\mf{v}-\mf{v}^*\Vert=\mathcal{O}(\delta t)$. Thus
\begin{align*}
	\mf{E}(\mf{v}-\mf{v}^*)=\mathcal{O}_E(\delta t^3), \\
    \delta t\frac{\partial \mf{F}_2}{\partial \mf{v}}(\mf{v}-\mf{v}^*)=\mathcal{O}_G(\delta t^2), \\ 
    \mathcal{O}_m(\Vert\mf{v}-\mf{v}^*\Vert^2) = \mathcal{O}_m(\delta t^2).
\end{align*}
Therefore, the positive definite linearization
\begin{align*}
	\mf{A}(\mf{v}-\mf{v}^*) + \mathcal{O}_E(\delta t^3) + \mathcal{O}_G(\delta t^2) +
	\mathcal{O}_m(\delta t^2) = \mf{J}^T\mf{\bgamma},
\end{align*}
agrees with the original momentum balance in Eq. (\ref{eq:scheme_momentum}) to second
order.

Finally, notice that $\mf{A}$ is a linear combination of positive definite
matrices with non-negative scalars in the linear combination, and therefore
$\mf{A}\succ 0$.\hfill\IEEEQED


\section{Proof of Theorem \ref{th:primal_dual}}
\label{app:primal_dual_proof}
The Lagrangian of the primal formulation in Eq. (\ref{eq:primal_regularized}) is
\begin{equation}
    \mathcal{L}(\mf{v},\bsigma,\vf{\gamma}) = 
\frac{1}{2}\Vert\mf{v}-\mf{v}^*\Vert_A^2 + \frac{1}{2} \Vert\bsigma\Vert_{R}^2 - \vf{\gamma}^T\mf{g},
\end{equation}
with $\vf{\gamma}\in\mathcal{F}$ the dual variable to enforce the constraint
$\vf{g}\in \mathcal{F}^*$. Minimizing the Lagrangian jointly in variables $\mf{v}$ and $\bsigma$ leads
to the optimality conditions
\begin{subequations}\label{eq:primal_optimality_conditions}
\begin{align}
    \mf{A}(\mf{v}-\mf{v}^*) &= \mf{J}^T\vf{\gamma} \label{eq:momentum_optimality}\\
    \vf{\sigma} &= \vf{\gamma}.  \label{eq:sigma_equal_gamma}
\end{align}
\end{subequations}
The optimality condition \eqref{eq:momentum_optimality} reveals that the
multipliers $\bgamma$ are indeed the contact impulses, and we recover the
balance of momentum. The optimality condition \eqref{eq:sigma_equal_gamma}
allows us to eliminate $\vf{\sigma}$. We then substitute these results back into
the Lagrangian to recover the dual in
\eqref{eq:dual_regularized}
\begin{align*}
    \min_{\bgamma\in \mathcal{F}} \ell_d(\bgamma) =
    \frac{1}{2}\bgamma^T(\mathbf{W}+\mathbf{R})\bgamma + {\bm r}^T
    \bgamma,
\end{align*}
where, in contrast to previous work, our Delassus operator
$\mf{W}=\mf{J}\mf{A}^{-1}\mf{J}^T$ now also contains the contribution of
internal force elements (through Eq. (\ref{eq:expression_for_A})) and
$\mf{r}=\mf{v}_c^*-\hat{\mf{v}}_c$ with
$\mf{v}_c^*=\mf{J}\mf{v}^*$.\hfill\IEEEQED


\section{Proof of Theorem \ref{th:unconstrained_formulation_equivalance}}
\label{app:unconstrained_formulation_equivalance}
Before proving this theorem, we need the following result.
\begin{lemma}
    The conic constraint $\mf{g}(\mf{v}, \bsigma)\in\mathcal{F}^*$ is satisfied
    if $\bsigma$ is given by $P_\mathcal{F}(\mf{y(\mf{v})})$.
    \label{lemma:conic_constraints_are_satisfied_analytically}
\end{lemma}
\begin{IEEEproof}
    Since $\bsigma$ is the projection of $\mf{y}(\mf{v})$ to the cone
    $\mathcal{F}$ with the $\mf{R}$ norm, by Moreau's decomposition theorem, we
    know that $\mf{y}(\mf{v}) - \bsigma$ is in the polar cone of $\mathcal{F}$
    with the $\mf{R}$ norm. That is,
    \begin{align*}
        \langle \mf{y}(\mf{v}) - \bsigma, \mf{x} \rangle_\mf{R} \le 0 
    \end{align*}
    for all $\mf{x} \in \mathcal{F}$, with the inner product
    $\langle\mf{v},\mf{w}\rangle_\mf{R}=\mf{v}^T\mf{R}\mf{w}$. Reorganizing
    terms, we get
    \begin{align*}
        \mf{x}^T \mf{R}(\mf{y}(\mf{v}) - \bsigma) &\le 0 \\
        \mf{x}^T (-\mf{R}\bsigma - \mf{v}_c + \hat{\mf{v}}_c) &\le 0 \\
        \langle \mf{x}, -\mf{R}\bsigma - \mf{v}_c + \hat{\mf{v}}_c \rangle &\le 0
    \end{align*}
    for all $\mf{x} \in \mathcal{F}$. Therefore, it follows that
    $-\mf{g}=-(\mf{v}_c - \hat{\mf{v}}_c + \mf{R}\bsigma)$ is in the polar cone
    of $\mathcal{F}$ and thus $\mf{g}$ is in the dual cone of $\mathcal{F}$.
\end{IEEEproof}

The optimality condition for the unconstrained formulation in
\eqref{eq:primal_unconstrained} is $\nabla\ell_p(\mf{v})=\mf{0}$. It is shown in
Appendix \ref{app:gradients_derivation} that
\begin{equation*}
    \nabla\ell_p(\mf{v})=\mf{A}(\mf{v}-\mf{v}^*) - \mf{J}^T\bgamma(\mf{v})
\end{equation*}
with impulses given
by $\bgamma(\mf{v})=P_\mathcal{F}(\mf{y}(\mf{v}))$, the dual optimal. Therefore,
$\nabla\ell_p(\mf{v})=\mf{0}$ implies \eqref{eq:momentum_optimality}, the first
optimality condition for \eqref{eq:primal_regularized}.
   
The analytical inverse dynamics solution shows that $\bgamma =
P_\mathcal{F}(\mf{y(\mf{v})})$ with the primal optimal $\mf{v}$. Hence, choosing
$\bsigma = P_\mathcal{F}(\mf{y(\mf{v})})$ with the primal optimal $\mf{v}$
satisfies \eqref{eq:sigma_equal_gamma}, the second optimality condition for
\eqref{eq:primal_regularized}.

Finally, by Lemma \ref{lemma:conic_constraints_are_satisfied_analytically}, the
cone constraint $\mf{g}(\mf{v}, \bsigma)\in\mathcal{F}^*$ is satisfied.
\hfill\IEEEQED


\section{Analytical Inverse Dynamics Derivations}
\label{app:analytical_inverse_dynamics_derivations}
% Dummy comment for Reviewable.

\subsection{Analytical Inverse Dynamics}
\label{sec:analytical_inverse_dynamics}

The dual optimal impulses of~\eqref{eq:dual_regularized} can be
constructed from the primal optimal velocities of~\eqref{eq:primal_regularized}
using a simple projection operation. Following~\cite{bib:todorov2014}, we call
this construction  \textit{analytical inverse dynamics}. Moreover, this
projection decomposes into a set of individual projections for each contact
impulse $\bgamma_i$ given the separable structure of the constraints. Letting
$\vf{y}_i(\vf{v}_{c,i}) = -\vf{R}_i^{-1}(\vf{v}_{c,i}-\hat{\vf{v}}_{c,i})$,
these projections take the form
\begin{equation}
  \begin{aligned}
	\bgamma_i(\vf{v}_{c,i})&= P_{\mathcal{F}_i}(\vf{y}_i(\vf{v}_{c,i}))\\
	&= \argmin_{\bgamma\in\mathcal{F}_i} \quad 
		\frac{1}{2}(\bgamma-\vf{y}_i)^T\vf{R}_i(\bgamma-\vf{y}_i),\\
	\end{aligned}
	\label{eq:y_projection}
\end{equation}
where $\vf{R}_i\in\mathbb{R}^{3\times3}$ is the $i\text{-th}$ diagonal block of
the regularization matrix $\mf{R}$. That is, $\bgamma_i$ is the projection
$P_{\mathcal{F}_i}$ of $\vf{y}_i(\vf{v}_{c,i})$ onto the friction cone
$\mathcal{F}_i$ using the norm defined by $\vf{R}_i$.
\reviewquestion{R1-Q7}{Remarkably, the projection map $P_{\mathcal{F}_i}$ can be
evaluated \emph{analytically}. We provide algebraic expressions for it in
Section~\ref{sec:physical_intuition} and derivations in
Appendix~\ref{app:analytical_inverse_dynamics_derivations}.} The projection
$P_{\mathcal{F}}(\mf{y})$ onto the full cone $\mathcal{F} := \mathcal{F}_1
\times \mathcal{F}_2 \times \cdots \times \mathcal{F}_{n_c}$ is obtained by
simply stacking together the individual projections
$P_{\mathcal{F}_i}(\vf{y}_i)$ from Eq. (\ref{eq:y_projection}), where we form
$\mf{y}$ by stacking together each $\vf{y}_i$ from all contact pairs.  In this
notation, the optimal impulse $\bgamma$ of~\eqref{eq:dual_regularized} and the
optimal velocities $\mf{v}$ of~\eqref{eq:primal_regularized} satisfy  $\bgamma =
P_{\mathcal{F}}(\mf{y}(\mf{v}))$.


\section{Derivation of the Gradients}
\label{app:gradients_derivation}
We first compute the regularization term as
\begin{eqnarray}
	\ell_R = \frac{1}{2}(R_t\Vert\bgamma_t\Vert^2+R_n\gamma_n^2)
\end{eqnarray}

During \textbf{stiction} the cost is written as
\begin{eqnarray}
	\ell_R(\vf{y}) = \frac{1}{2}(R_t y_r^2+R_n y_n^2)
\end{eqnarray}

During \textbf{sliding} Eq. (\ref{eq:inverse_dynamics_projection}) applies and
the cost can be written as
\begin{eqnarray}
	\ell_R(\vf{y}) =
	\frac{1}{2}\gamma_n^2(R_t\mu^2+R_n)=\frac{R_n}{2}(1+\tilde\mu^2)\gamma_n^2=\frac{R_n}{2(1+\tilde\mu^2)}\left(\mu\frac{R_t}{R_n}y_r+y_n\right)^2
\end{eqnarray}

and finally when there is \textbf{no contact}
\begin{eqnarray}
	\ell_R(\vf{y}) = 0
\end{eqnarray}

Therefore $\ell_R$ is a piecewise function that we can summarize as
\begin{equation}
	\ell_R(\vf{y}) = 
\begin{dcases}
	% Region I, stiction
	\frac{1}{2}(R_t y_r^2+R_n y_n^2) & \text{stiction, } y_r < \mu y_n\\
	% Region II, sliding.
	\frac{R_n}{2(1+\tilde\mu^2)}\left(\mu\frac{R_t}{R_n}y_r+y_n\right)^2 & \text{sliding, } -\mu \frac{R_t}{R_n} y_r < y_n \leq \frac{y_r}{\mu}\\
	% Region II, no contact.
    \vf{0} & \text{no contact, } y_n \leq -\mu \frac{R_t}{R_n} y_r
\end{dcases}	  
	\label{eq:ell_R_piecewise}
\end{equation}

\subsection{Gradients per Contact Point}
In this section we compute the gradients of $\ell_R(\mf{y})$ with respect to the
$i\text{-th}$ contact point variable $\vf{y}_i\in\mathbb{R}^3$. Unless otherwise
stated, in this section we'll drop the subindex $i$ for the $i\text{-th}$
contact point. Therore in this section $\nabla_\vf{y}\ell_R\in\mathbb{R}^3$ and
$\nabla_\vf{y}^2\ell_R\in\mathbb{R}^{3\times 3}$ (notice that as per our
notation, we consistently use a bold italic font for 3D vectors and only bold,
not italic, for $n\text{-dimensional}$ vectors.)

The full gradient $\nabla_\mf{y}\ell_R$ concatenates the three-dimensional
gradients $\nabla_\vf{y}\ell_R$ while the full Hessian matrix is block-diagonal
with each $3\times 3$ block containing the local $i\text{-th}$ point Hessian.
This is particularly useful to exploit sparsity in the computations.

We use the following identities to simplify the expressions
\begin{eqnarray}
	\frac{\partial y_r}{\partial\vf{y}_t} &=& \hat{\vf{t}}\nonumber\\
	\frac{\partial \hat{\vf{t}}}{\partial\vf{y}_t} &=&
	\frac{\vf{P}^\perp(\hat{\vf{t}})}{y_r}
	\label{eq:yt_derivatives}
\end{eqnarray}
where the $2\times 2$ projection matrices along and perpendicular to
$\hat{\vf{t}}$ are defined as
\begin{eqnarray}
	\vf{P}(\hat{\vf{t}}) &=& \hat{\vf{t}}\otimes\hat{\vf{t}}\nonumber\\
	\vf{P}^\perp(\hat{\vf{t}})&=&\vf{I}_2 - \vf{P}(\hat{\vf{t}})
	\label{eq:tangential_projections}
\end{eqnarray}

Taking the gradient of Eq. (\ref{eq:ell_R_piecewise}) results in
\begin{equation}
	\nabla_\vf{y}\ell_R(\vf{y}) = 
\begin{dcases}
	%%%%%%%%%%%%%%%%%%%%
	% Region I, stiction
	\vf{R}\,\vf{y} & 
	% when,
	\text{stiction, } y_r < \mu y_n\\
	%
	%%%%%%%%%%%%%%%%%%%%
	% Region II, sliding.
	\frac{1}{1+\tilde\mu^2}\hat{s}^\circ(\vf{y})\begin{bmatrix}
		\mu R_t\hat{\vf{t}}\\
		R_n\\
	\end{bmatrix} &
	% when,
	\text{sliding, } -\mu \frac{R_t}{R_n} y_r < y_n \leq \frac{y_r}{\mu}\\
	% Region II, no contact.
    \vf{0} & \text{no contact, } y_n \leq -\mu \frac{R_t}{R_n} y_r
\end{dcases}	  
	\label{eq:gradient_ell_R_piecewise}
\end{equation}
where we defined $\hat{\mu}=\mu R_t/R_n$ and $\hat{s}^\circ(\vf{y}) =
\hat{\mu}y_r+y_n$. $\hat{s}^\circ(\vf{y})$ is a measure how close the solution
is to the polar cone $\mathcal{F}^\circ$. $\hat{s}^\circ(\vf{y}) > 0$ in the
sliding region and $\hat{s}^\circ<0$ when there is no contact.

Similarly, we can compute the Hessian of $\ell_R$ by taking the gradient in Eq.
(\ref{eq:gradient_ell_R_piecewise})
\begin{equation}
	\nabla_\vf{y}^2\ell_R(\vf{y}) = 
\begin{dcases}
	%%%%%%%%%%%%%%%%%%%%
	% Region I, stiction
	\vf{R} & 
	% when,
	\text{stiction, } y_r < \mu y_n\\
	%
	%%%%%%%%%%%%%%%%%%%%
	% Region II, sliding.
	\frac{R_n}{1+\tilde\mu^2}
	\begin{bmatrix}
		% ∂²ℓ/∂yₜ²:
		\hat{\mu}\left(\hat{\mu}\vf{P}(\hat{\vf{t}})+\hat{s}^\circ(\vf{y})\vf{P}^\perp(\hat{\vf{t}})/y_r\right) & 
		% ∂²ℓ/∂yₙ∂yₜ:
		\hat{\mu}\vf{t}\\
		% ∂²ℓ/∂yₜ∂yₙ:
		\hat{\mu}\vf{t}^T & 
		% ∂²ℓ/∂yₙ²:
		1\\
	\end{bmatrix} &
	% when,
	\text{sliding, } -\mu \frac{R_t}{R_n} y_r < y_n \leq \frac{y_r}{\mu}\\
	% Region II, no contact.
    \vf{0} & \text{no contact, } y_n \leq -\mu \frac{R_t}{R_n} y_r
\end{dcases}	  
	\label{eq:hessian_ell_R_piecewise}
\end{equation}

Clearly in the stiction region we have $\nabla_\vf{y}^2\ell_R(\vf{y})\succ 0$.
Since in the stiction region we have $\hat{s}^\circ(\vf{y})>0$, the linear
combination of $\vf{P}(\hat{\vf{t}})$ and $\vf{P}(\hat{\vf{t}})^\perp$ in Eq.
(\ref{eq:hessian_ell_R_piecewise}) is PSD (since both projection matrices are
PSD). Therefore, in the sliding region we find out that
$\nabla_\vf{y}^2\ell_R(\vf{y})\succeq 0$.
\todo{Show that the jump condition across the cone's boundary is satisfied.}

\subsection{Gradients with Respect to Velocities}
These gradients are computed using the chain rule since we know that
\begin{equation}
	\mf{y}=\mf{Jv+b}
\end{equation}

The gradient is
\begin{equation}
	\nabla_\mf{v}\ell_R = -\mf{J}^T\mf{R}^{-1}\nabla_\mf{y}\ell_R
	\label{eq:ell_velocity_gradient}
\end{equation}
which, using Eq. (\ref{eq:gradient_ell_R_piecewise}), can be written as
\begin{equation}
	\nabla_\mf{v}\ell_R = -\mf{J}^T\bgamma
	\label{eq:ell_velocity_gradient_simplified}
\end{equation}

And the Hessian is
\begin{equation}
	\nabla_\mf{v}^2\ell_R = \mf{J}^T\mf{R}^{-1}\nabla_\mf{y}^2\ell_R\mf{R}^{-1}\mf{J}
	\label{eq:ell_velocity_hessian}
\end{equation}

Since we already showed $\nabla_\mf{y}^2\ell_R\succeq 0$, it follows that
$\nabla_\mf{v}^2\ell_R\succeq 0$. Therefore the cost $\ell_R(\mf{v})$ is convex.

Recall that $\nabla_\mf{y}^2\ell_R$ is block diagonal and $\mf{R}$ is diagonal.
Therefore the Hessian $\nabla_\mf{v}^2\ell_R$ inherits the same sparsity
structure as the product $\mf{J}^T\mf{J}$.

\subsection{Gradients of the Primal Cost}
With these results, we can now write the gradient and Hessian of the primal cost
$\ell_p(\mf{v})$ in Eq. (\ref{eq:primal_unconstrained}). For the gradient we
have
\begin{equation}
	\nabla_\mf{v}\ell_p(\mf{v}) = \mf{M}(\mf{v}-\mf{v}^*) + \nabla_\mf{v}\ell_R
\end{equation}

Notice that, using Eq. (\ref{eq:ell_velocity_gradient_simplified}), the gradient
can be written as
\begin{equation}
	\nabla_\mf{v}\ell_p(\mf{v}) = \mf{M}(\mf{v}-\mf{v}^*) - \mf{J}^T\bgamma
\end{equation}
and since the unconstrained minimization looks for $\nabla_\mf{v}\ell_p=\mf{0}$,
we essentially recover the balance of momentum, as expected.

Similarly, we can write the Hessian as
\begin{equation}
	\mf{H} = \nabla_\mf{v}^2\ell_p(\mf{v}) = \mf{M} + \nabla_\mf{v}^2\ell_R
\end{equation}
and since we already proved that $\nabla_\mf{v}^2\ell_R\succeq 0$, we find that
the $\nabla_\mf{v}^2\ell_p\succ 0$. Therefore $\ell_p(\mf{v})$ is strictly
convex and there is a unique solution to the minimization problems in the
velocities $\mf{v}$.


\section{Convergence Analysis of SAP}
\label{app:sap_converge}
\newcommand{\coneName}{\mathcal{K}}
\newcommand{\dist}{d}
\newcommand{\cond}{\text{cond}}
\newcommand{\vx}{\mf{v}}
\newcommand{\fx}{\ell_p(\mf{\vx})}

\newcommand{\vy}{\mf{u}}
\newcommand{\fy}{\ell_p(\mf{\vy})}
\newcommand{\vd}{\mf{d}}

% As required by the IEEE template.
\renewcommand\qedsymbol{$\IEEEQED$}

Convergence of SAP is established
by first showing that the objective function 
$\ell_p(\mf{v}) = \frac{1}{2}\Vert\mf{v}-\mf{v}^*\Vert_{A}^2 + P_\mathcal{F}(\mf{y}(\mf{v}))\Vert_R^2$
is \emph{strongly convex}
and differentiable with \emph{Lipschitz continuous} gradients.  The
former property is inherited from the positive-definite quadratic term 
provided by the positive definite matrix $\mf{A}$ in Eq. \eqref{eq:primal_unconstrained}.  The latter is shown using differentiability of the squared-distance function
 and the Lipschitz continuity of its gradient map (Theorems 5.3-i 6.1-i of~\cite{bib:delfour2011shapes})
 combined with the identity
\[
  \dist^2_{\coneName^\circ}(x) = \|P_{\coneName}(x)\|_{\mf{R}}^2,
\] 
for any closed, convex cone $\coneName$. Here 
the distance and projection functions are with respect to the
norm $\|\cdot\|_\mf{R}$, while $\coneName^\circ$ denotes the polar
cone with respect to the corresponding inner-product $\mf{x}^T \mf{R} \mf{y}$.
\begin{lemma}\label{lem:PropertiesOfObj}
  The following statements hold.
  \begin{itemize}
    \item The function $\fx$ is strongly convex, i.e., there exists $\mu >0$
      such that 
      \[
        \fy \ge \fx + \nabla \fx(\vy-\vx) + \frac{\mu}{2} \|\vy-\vx\|^2
      \]
    \item The function $\fx$ is differentiable and has Lipschitz continuous gradients, i.e.,
      $\nabla \fx$ exists for all $\mf{v}$ and there exists $L \ge 0$ satisfying
      \[
      \|\nabla \fx - \nabla \fy\| \le L \|\vy - \vx\|
      \]
  \end{itemize}
  \begin{proof}

The objective $\fx$ is a function $f : \mathbb{R}^n \rightarrow \mathbb{R}$
of the following form
\[
  f(\vx) = \frac{1}{2}\dist_{\coneName}^2(\mf{Z}\vx + \mf{b}) + \vx^{T}\mf{W}\vx + \mf{q}^T \vx,
\]
where $\mf{Z} \in \mathbb{R}^{m \times n}$, $\mf{W} \in \mathbb{R}^{n \times n}$ is
symmetric and positive definite, and $\dist_{\coneName} : \mathbb{R}^{m} \rightarrow \mathbb{R}$ 
denotes the distance function of a closed, convex set $\coneName \subseteq \mathbb{R}^m$
as measured by some quadratic norm $\|\mf{x}\|_Q$, i.e.,
\[
  \dist_{\coneName}(\vx) = \inf \{ \|\vx-\mf{z}\|_Q : \mf{z} \in \coneName\}.
\]
    The sum of a strongly convex function with a convex function is strongly
    convex.  Since the squared distance function is convex, and the quadratic
    term $\vx^T \mf{W}\vx$ is strongly convex (given that $\mf{W} \succ 0$), the  first
    statement holds.

    The second statement follows trivially if we can show it holds 
    for the squared distance function. 
    Differentiability follows from Chapter 4 (Theorems 5.3-i 6.1-i) of~\cite{bib:delfour2011shapes},
    which shows that
  \[
    \nabla \dist_{\coneName}^2(\vx) = 2 \mf{Q} (\vx - P_{\coneName}(\vx)).
  \]
    That the gradient of $\dist_{\coneName}^2(\vx)$ is Lipschitz follows
    from Lipschitz continuity of projection maps onto closed, convex sets.
  \end{proof}
\end{lemma}
\noindent We remark that strong convexity implies the reverse
Lipschitz inequality $\|\nabla f(\vx) - \nabla f(\vy)\| \ge \mu \|\vx - \vy\|$,
which in turn means that the parameter $\mu$ and the Lipschitz constant $L$
satisfy $\mu \le L$.




Recall that SAP (Algorithm~\ref{alg:sap}) is a special case
of the following iterative method
for minimizing a function $f : \mathbb{R}^n \rightarrow
\mathbb{R}$ given some initial
point $\vx_0 \in \mathbb{R}^n$:
\begin{align}\label{eq:quasiNewtonIter}
  \begin{aligned}
    \vd_m &= -\mf{H}^{-1}(\vx_m) \nabla f(\vx_m), \\
     t_m &= \argmin_{t} f(\vx_m + t \vd_m),\\
    \vx_{m+1} &= \vx_m + t_m \vd_m,      
  \end{aligned}
\end{align}
where $\mf{H} : \mathbb{R}^n \rightarrow \mathbb{R}^{n \times n}$ is a function 
into the set of symmetric positive definite matrices, i.e.,  $\mf{H}(\vx) = \mf{H}(\vx)^T$ 
and $\mf{H}(\vx) \succ 0$ for all $\vx\in\mathbb{R}^n$.  
It is well known that gradient descent exhibits linear convergence to the global minimum when applied to a strongly convex
function with Lipschitz continuous gradient. Incorporating
a condition number bound  $\sigma$ for $\mf{H}(\vx)$  into standard gradient-descent analysis
will prove that the iterations~\eqref{eq:quasiNewtonIter}
also have linear convergence.
To show this, we let
$\cond(\mf{H}(\vx))$ denote the condition number of $\mf{H}(\vx)$
and $S(\vx_0)$ denote the sub-level set $\{ \vx \in \mathbb{R}^n: f(\vx) \le f(\vx_0) \}$.
\begin{lemma}\label{lem:GlobalConv}
  Let $f : \mathbb{R}^n \rightarrow \mathbb{R}$ be strongly convex and differentiable
  with Lipschitz-continuous gradients.
  Fix $\vx_0 \in\mathbb{R}^n$. If there exists $\sigma > 0$
  such that $\cond(\mf{H}(\vx)) \le \sigma$ for all  $\vx \in S(\vx_0)$, then
  the iterations~\eqref{eq:quasiNewtonIter}
  converge to the global minimum $\vx_*$ of $f(\vx)$ when initialized at $\vx_0$.
  Moreover,
  \[
    f(\vx_m)  - f(\vx_*)  \le (1-\frac{\mu}{\sigma^2 L})^m (f(\vx_0) - f(\vx_*))
  \]
  for all iterations $m$, where $\mu$ is the strong-convexity parameter of $f(\vx)$ and $L$ is
  the Lipschitz constant of $\nabla f(\vx)$.
  \begin{proof}
    Dropping the subscript $m$ from $(\vx_m, t_m, \vd_m)$, we first observe that
\[
f(\vx+t\vd) \le f(\vx) + \langle \nabla f(\vx), \vd \rangle t + \frac{L}{2}\|\vd\|^2 t^2,
\]
by Lipschitz continuity.  Substituting $\vd =- \mf{H}^{-1}  \nabla  f(\vx)$ gives
\[
f(\vx+t\vd) = f(\vx) - \nabla f(\vx)^T  \mf{H}^{-1}  \nabla  f(\vx) t + \frac{L }{2}\|\mf{H}^{-1}\nabla f(\vx) \|^2 t^2.
\]
Letting $\lambda_{max}$ and $\lambda_{min}$ denote the
maximum and minimum eigenvalues of $\mf{H}$ evaluated at $\vx$, it also follows that
\[
  f(\vx+t\vd) \le f(\vx) - \frac{1}{\lambda_{\max}} \|\nabla f(\vx)\|^2 t  + 
    \frac{L }{2}  \frac{1}{\lambda^{2}_{\min}} \|\nabla f(\vx)\|^2 t^2.
\]
Letting $\bar t$ denote the  minimizer of the right-hand-side, we
conclude that
    \[
      f(\vx+t\vd) \le  f(\vx + \bar t\vd ) \le f(\vx) - \frac{1}{2}(\frac{\lambda^{2}_{\min}}{\lambda^2_{\max} L} 
      \|\nabla f(\vx)\|^2),
    \]
where the first inequality follows from the exact line search
used to select $t$.
Since  $\sigma^2 \ge \lambda^{2}_{\max}/\lambda^2_{\min}$,
we conclude that
\[
   f(\vx + t \vd) \le f(\vx) - \frac{1}{2\sigma^2 L}\| \nabla f(\vx) \|^2.
\]
    On the other hand, letting $f_* = f(\vx_*)$ 
    we have from strong convexity that the Polyak-Lojasiewicz
    inequality holds:
    \[
  \|\nabla f(\vx)\|^2 \ge 2\mu (f(\vx)-f_* ).
    \]
Hence,
\[
  f(\vx + t \vd)   \le f(\vx) - \frac{\mu}{\sigma^2 L} (f(\vx)-f_* ).
\]
Subtracting $f_*$ from both sides and factoring shows
\[
  f(\vx + t \vd)  - f_* \le  (1-\frac{\mu}{\sigma^2 L})(f(\vx) - f_*).
\]
It follows that each iteration $m$ satisfies
\[
  f(\vx_{m+1}) - f_* \le  (1-\frac{\mu}{\sigma^2 L})^{m}(f(\vx_{0}) - f_*).
\]
Since $\sigma \ge 1$ and $L \ge \mu$, the iterations converge, and the proof is completed.
  \end{proof}
\end{lemma}

Combining these lemmas shows that SAP globally convergences at (at least) a
linear rate.  By observing that SAP reduces to Newton's
method when the gradient is differentiable, we can also prove local quadratic
convergence assuming differentiability on a neighborhood of
the optimum $\vx_*$. 
\begin{theorem}
  The following statements hold.
  \begin{itemize}
    \item SAP globally converges from all initial conditions.
    \item If $\nabla f(\vx)$  is differentiable on the ball $B(\vx_*, r) := \{ \vx :  \|\vx-\vx_*\| \le r\}$
  for some $r > 0$, then SAP exhibits quadratic convergence, i.e., for some finite
  $M$ and $\zeta > 0$
  \[
    \|\vx_m - \vx_*\| \le \zeta \|\vx_{m+1} - \vx_*\|^2
  \]
  for all $m > M$.
  \end{itemize}

  \begin{proof}

    The first statement follows from Lemmas~\ref{lem:PropertiesOfObj}~and~\ref{lem:GlobalConv}.

  To prove the second, we show that $B(\vx_*, r)$
    contains a sublevel set $\Omega_{\beta} = \{ \vx : f(\vx) \le \beta\}$ for some
  $\beta > 0$,  implying that SAP 
  reduces to Newton's method with exact line search for some $m > M$,
 given that sublevel sets are invariant. 

  To begin, we have, by strong convexity, that
  \begin{equation}
    \beta \ge f(\vx) \ge f(\vx_*)  + \frac{\mu}{2} \|\vx-\vx_*\|^2,
    \label{eq:strong_convexity_at_differentiable_optimal}
  \end{equation}
    for all $\vx \in \Omega_{\beta}$.
    Rearranging shows that
    \[
      \|\vx-\vx_*\|^2 \le 2\frac{\beta- f(\vx_*)}{\mu}.
    \]
    Hence, $B(\vx_*, r)$ contains
    $\Omega_{\beta}$ for any $\beta$ satisfying $2\frac{\beta- f(\vx_*)}{\mu} < r$.
    For some finite $M$, we also have that $v_m \in \Omega_{\beta}$ for all $m > M$ 
    by Lemma~\ref{lem:GlobalConv}.

    Next, we prove that Newton iterations are quadratically convergent
  with exact line search. Indeed, using once more the strong convexity result in Eq.~\eqref{eq:strong_convexity_at_differentiable_optimal}
  \begin{align*}
    \|\vx_{m+1} - \vx_*\|^2&\le \frac{2}{\mu} ( f(\vx_{m+1}) - f(\vx_*)) \\
                    &= \frac{2}{\mu} ( f(\vx_m + t_m \vd_m   ) - f(\vx_*)) \\
                    &\le \frac{2}{\mu} ( f(\vx_m + \vd_m   ) - f(\vx_*)) \\
                    &\le \frac{2}{\mu} L \|\vx_m + \vd_m - \vx_*\|^2,
  \end{align*}
    where the first line uses strong convexity,
    the third exact line search, and the last
    Lipschitz continuity. But for some $\kappa > 0$,
    we have that $\|\vx_m + \vd_m - \vx_*\|^2 \le \kappa \|\vx_{m} - \vx_*\|^4$
    by quadratic convergence of Newton's method with unit step-size (\cite[Theorem 3.5]{bib:nocedal2006numerical}).
    Hence,
    \[
      \|\vx_{m+1} - \vx_*\|^2 \le \frac{2}{\mu}  L \kappa \|\vx_{m} - \vx_*\|^4,
    \]
    and the claim is proven.
  \end{proof}
\end{theorem}




% Acknowledgment: N.B. Use section*.
\section*{Acknowledgment}
The authors would like to thank especially to Michael Sherman for his trust on
this research from day one and to the Dynamics \& Simulation and Dexterous
Manipulation teams at TRI for their continuous patience and support.

% Can use something like this to put references on a page by themselves when
% using endfloat and the captionsoff option.
\ifCLASSOPTIONcaptionsoff
  \newpage
\fi

% References:
\bibliographystyle{./IEEEtran/IEEEtran}
\bibliography{./IEEEtran/IEEEabrv,sap_paper}

\end{document}


