\section{Compliant Contact, Principle of Maximum Dissipation and Artifacts}
\label{sec:physical_intuition}

Thus far $\vf{R}_k$ and $\hat{\vf{v}}_k$ have been treated as known problem
data. This section how to choose  quantities with
physical parameters to model compliant contact with regularized friction.\RedHighlight{Reword this 
sentence.
I think the point of this section is how to choose and interpret $R$. I would state that plainly. I 
would also consider moving this section to directly after VII or to after all the algorithmic sections, just before computational results.}  Notice
this approach is different from the one in \cite{bib:todorov2014}, where
regularization is not used to model physical compliance but rather to introduce
a user tunable Baumgarte-style stabilization.

We want to model compliant contact with stiffness $k$ (with units of N/m) and
dissipation $c = \tau k$ where $\tau$ (in seconds) is a \textit{dissipation time
scale}. Once again, dropping subindex $k$ for simplicity, our model is defined
by
\begin{eqnarray}
	\hat{\vf{v}}_c &=&
	\begin{bmatrix}
		0\\
		0\\
		\hat{v}_n \end{bmatrix}\nonumber\\
	\hat{v}_n &=& -\frac{\phi_0}{dt+\tau}
\end{eqnarray}
where $\phi_0$ is the previous step signed distance. The regularization
parameters in Eq. (\ref{eq:Rk}) are taken as $R_n^{-1} = dt k(dt+\tau)$ and
$R_t=\varepsilon_t R_n$, with the dimensionless parameter $\varepsilon_t \ll 1$
to control the amount of regularization in the tangential direction.

To gain physical insight into our model, we
substitute $\vf{y}=-\vf{R}^{-1}(\vf{v}_c - \hat{\vf{v}}_c)$ into Eq.
(\ref{eq:analytical_y_projection}) to obtain
\begin{eqnarray}
	&&\bgamma(\vf{v}_c) = P_\mathcal{F}(\vf{y}(\vf{v}_c))
    \label{eq:gamma_analytical_from_vc}\\
&&\begin{dcases}
	% Region I, stiction
	\begin{bmatrix}
		-\vf{v}_t/R_t\\
		-dt(k\phi + c v_n)
	\end{bmatrix}
	% When we  have: y_r < \mu y_n
	& \text{stiction, } \\
	%
	%
	% Region II, sliding.
	\begin{bmatrix}
		\mu\gamma_n\hat{\vf{t}}\\
		-\frac{dt}{1+\tilde\mu^2}\left(k(\phi-(dt+\tau)\mu\Vert\vf{v}_t\Vert) + cv_n \right)
	\end{bmatrix}
	% When we  have:  -\mu \frac{R_t}{R_n} y_r < y_n \leq \frac{y_r}{\mu}
	& \text{sliding, }\\
	%
	%
	% Region III, no contact.  y_n \leq -\mu \frac{R_t}{R_n} y_r
    \vf{0} & \text{no contact, }
\end{dcases}\nonumber	
\end{eqnarray}
where $\phi= \phi_0 + dt\,v_n$. Let us now analyze the resulting forces from
this model.

\subsection{Friction Forces}
From Eq. (\ref{eq:gamma_analytical_from_vc}) we see that friction forces behave exactly as a
model of regularized friction
\begin{equation}
	\bgamma_t = \min\left(\frac{\Vert\vf{v}_t\Vert}{R_t}, \mu\gamma_n\right)\hat{\vf{t}}
\end{equation}
with $\bgamma_t$ linear to the (very small) slip velocity during stiction and
with a maximum value given by $\mu\gamma_n$, effectively modeling Coulomb
friction. Notice that to better model stiction, we are interested in small
values of $\varepsilon_t$, or $\varepsilon_t\ll 1$.

Since $\hat{\vf{t}} = \vf{y}_t/\Vert\vf{y}_t\Vert =
-\vf{v}_t/\Vert\vf{v}_t\Vert$, friction forces oppose sliding therefore
satisfying the principle of maximum dissipation as desired.

\subsection{Normal Forces}
Firstly we observe that in stiction normal forces model compliance with linear
stiffness $k$ and linear dissipation $c = \tau k$. That is
\begin{equation}
    \gamma_n/dt = -(k\phi + c v_n)
\end{equation}

In the sliding region we see however that the convex approximation introduces
unphysical artifacts. Recall we are interested in the limit $R_t \ll R_n$ to
better model stiction and therefore $\tilde\mu \rightarrow 0$. This makes the
factor $1+\tilde{\mu}^2$ close to one and we can ignore it in our analysis.
While we'd like to recover $\gamma_n = -dt(k\phi + c v_n)$ as in stiction, we
instead see that the slip velocity $\Vert\vf{v}_t\Vert$ unphysically couples
into the normal forces as $\gamma_n=-dt(k(\phi-(dt+\tau)\mu\Vert\vf{v}_t\Vert) +
cv_n)$. We can write this as
\begin{equation}
  \gamma_n/dt=-(k\phi_\text{eff} + c v_n)
\end{equation}
with an \textit{effective} signed distance $\phi_\text{eff} =
\phi-(dt+\tau)\mu\Vert\vf{v}_t\Vert$. That is, we recover a model of compliant
contact but with a spurious drift of magnitude $(dt+\tau)\mu\Vert\vf{v}_t\Vert$.

This is consistent with the formulation in \cite{bib:anitescu2010} for rigid
contact when $k\rightarrow \infty$ leading to an unphysical \textit{gliding
effect} at a positive distance $\phi=dt\mu\Vert\vf{v}_t\Vert$. Notice that
\textit{gliding} goes away as $dt\rightarrow 0$ since the formulation converges
to the original contact problem \cite{bib:anitescu2006}. The effect of
compliance is to \textit{soften} this effect. 

In the limit to $dt\rightarrow 0$ we find out that the convex approximation
models, when sliding, the force law
$\gamma_n=-dt(k(\phi-\tau\mu\Vert\vf{v}_t\Vert)$. This tells us that, unlike the
rigid case, the \textit{gliding} effect unfortunately does not go away as
$dt\rightarrow 0$ but it persists with a finite value that now depends on the
dissipation rate, $\phi\approx\tau\mu\Vert\vf{v}_t\Vert$.

We close this discussion by making the following remarks relevant to robotics
applications:
\begin{enumerate}
	\item In robotics applications we are mostly interested in the stiction
	regime, typically for grasping, locomotion or rolling contact for mobile
	bases with wheels. This regime is precisely where the convex approximation
	does not introduce artifacts.
	\item Sliding is usually avoided or if importnat, it is usually at low
	velocities and therefore the term $dt\mu\Vert\vf{v}_t\Vert$ is negligible.
	\item Even though the effect of sliding cannot be neglected in the
	dissipation term during stiction, we point out that for robotics
	applications most often dissipation is high to model a zero restitution
	coefficient. Therefore the high dissipative nature of this term makes the
	effect of $\Vert\vf{v}_t\Vert$ negligible for most cases with low slip
	velocities.
	\item For robotics, we are definitely interested on the onset to sliding.
	This is captured by the approximation which properly models the Colulomb
	friction law.
\end{enumerate}
