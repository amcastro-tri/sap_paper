% Dummy commit so that this file shows in Reviewable.

\section{A Primal Formulation of Compliant Contact}
\label{sec:primal_formulation}

In this section we give a convex program that augments the balance of momentum
stated in Eq. (\ref{eq:momentum_linearized}) so that contact impulses model
Coulomb friction and satisfy the principle of maximum dissipation.
We call this convex program our \emph{primal formulation}, as we will see that
its dual is precisely the formulation stated in Eq.
(\ref{eq:dual_regularized}). We will see in Section
\ref{sec:unconstrained_convex_formulation} that with this primal formulation,
we can eliminate the constraints analytically to write an unconstrained convex
formulation of the same problem.

We introduce a new decision variable $\vf{\sigma}\in\mathbb{R}^{3n_c}$ and
set up our primal formulation of compliant contact as the following convex
optimization problem
\begin{equation}
	\begin{aligned}
	\min_{\mf{v},\bsigma} \quad & \ell_p(\mf{v},\bsigma) =
	\frac{1}{2}\Vert\mf{v}-\mf{v}^*\Vert_{A}^2 +
	\frac{1}{2} \Vert\bsigma\Vert_{R}^2\\
	\textrm{s.t.} \quad & \mf{g} = (\mf{J}\mf{v}-\hat{\mf{v}}_c + \mf{R}\bsigma) \in \mathcal{F}^*\\
	\end{aligned}
	\label{eq:primal_regularized}
\end{equation}
where $\Vert\mf{z}\Vert_X^2=\mf{z}^T\mf{X}\mf{z}$ with $\mf{X}\succ 0$ and
$\mathcal{F^*}= \mathcal{F}^*_1 \times \mathcal{F}^*_2 \times \cdots \times
\mathcal{F}^*_{n_c}$ is the \emph{dual cone} of the convex cone $\mathcal{F}$.
The positive diagonal matrix $\mf{R}\in\mathbb{R}^{3n_c\times 3n_c}$ and the
vector of stabilization velocities $\hat{\mf{v}}_c$ encode the problem data
needed to model compliant contact. We will establish a very clear physical
meaning for these terms when we provide analytical expressions for the impulses
in Section \ref{sec:physical_intuition}. In the limit of rigid contact
when the regularization $\mf{R}$ is zero, our primal formulation reduces to that
presented in \cite{bib:mazhar2014}.

\begin{theorem}\label{th:primal_dual}
The dual of \eqref{eq:primal_regularized} is given by 
\eqref{eq:dual_regularized}. Moreover, when $\{\mf{v},\bsigma\}$ is primal optimal and
$\bgamma$ is dual optimal, $\bsigma = \bgamma$.
\end{theorem}

\begin{IEEEproof}
The Lagrangian of the primal formulation in Eq. (\ref{eq:primal_regularized}) is
\begin{equation}
	\mathcal{L}(\mf{v},\bsigma,\vf{\gamma}) = 
	\frac{1}{2}\Vert\mf{v}-\mf{v}^*\Vert_A^2 + \frac{1}{2} \Vert\bsigma\Vert_{R}^2 - \vf{\gamma}^T\mf{g}
	\label{eq:primal_lagrangian}
\end{equation}
with $\vf{\gamma}\in\mathcal{F}$ the dual variable to enforce the constraint
$\vf{g}\in \mathcal{F}^*$. Minimizing the Lagrangian in
\eqref{eq:primal_lagrangian} jointly in variables $\mf{v}$ and $\bsigma$ leads
to the optimality conditions
\begin{subequations}\label{eq:primal_optimality_conditions}
\begin{align}
	\mf{A}(\mf{v}-\mf{v}^*) &= \mf{J}^T\vf{\gamma} \label{eq:momentum_optimality}\\
	\vf{\sigma} &= \vf{\gamma}.  \label{eq:sigma_equal_gamma}
\end{align}
\end{subequations}
The optimality condition \eqref{eq:momentum_optimality} reveals that the
multipliers $\bgamma$ are indeed the contact impulses, and we recover the balance
of momentum. The optimality condition \eqref{eq:sigma_equal_gamma} allows
us to eliminate $\vf{\sigma}$. We then substitute these results back into the
Lagrangian in \eqref{eq:primal_lagrangian} to recover the dual in
\eqref{eq:dual_regularized}
\begin{align*}
	\min_{\bgamma\in \mathcal{F}} \ell_d(\bgamma) =
	\frac{1}{2}\bgamma^T(\mathbf{W}+\mathbf{R})\bgamma + {\bm r}^T
	\bgamma
\end{align*}
where, in contrast to previous work, our Delassus operator
$\mf{W}=\mf{J}\mf{A}^{-1}\mf{J}^T$ now also contains the contribution of internal
force elements (through Eq. (\ref{eq:expression_for_A})) and
$\mf{r}=\mf{v}_c^*-\hat{\mf{v}}_c$ with $\mf{v}_c^*=\mf{J}\mf{v}^*$.
\end{IEEEproof}
