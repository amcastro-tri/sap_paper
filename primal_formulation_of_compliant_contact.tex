
\section{A Primal Formulation of Compliant Contact}
\label{sec:primal_formulation}

In this section we give a convex program that augments the balance of momentum
stated in Eq. (\ref{eq:momentum_linearized}) so that contact impulses model
Coulomb friction and satisfy the principle of maximum dissipation when sliding.
We call this convex program our \emph{primal formulation}, as we will see that
it's dual is precisely the formulation stated in Eq.
(\ref{eq:dual_regularized}). We will see in Section
\ref{sec:unconstrained_convex_formulation} that the reason to introduce this
primal formulation is so that we can eliminate constraints analytically to write
an unconstrained convex formulation of the same problem.

We write our primal formulation of compliant contact by introducing a new
decision variable $\vf{\sigma}\in\mathbb{R}^{3n_c}$ as
\begin{equation}
	\begin{aligned}
	\min_{\mf{v},\bsigma} \quad & \ell_p(\mf{v},\bsigma) =
	\frac{1}{2}\Vert\mf{v}-\mf{v}^*\Vert_{A}^2 +
	\frac{1}{2} \Vert\bsigma\Vert_{R}^2\\
	\textrm{s.t.} \quad & \mf{g} = (\mf{J}\mf{v}-\hat{\mf{v}}_c + \mf{R}\bsigma) \in \mathcal{F}^*\\
	\end{aligned}
	\label{eq:primal_regularized}
\end{equation}
where $\Vert\mf{z}\Vert_X^2=\mf{z}^T\mf{X}\mf{z}$ with $\mf{X}\succ 0$ and
$\mathcal{F^*}= \mathcal{F}^*_1 \times \mathcal{F}^*_2 \times \cdots \times
\mathcal{F}^*_{n_c}$ is the \emph{dual cone} of the convex cone $\mathcal{F}$.
The positive diagonal matrix $\mf{R}\in\mathbb{R}^{3n_c\times 3n_c}$ and the
vector of stabilization velocities $\hat{\mf{v}}_c$ encodes the problem data
needed to model compliant contact. We will establish a very clear physical
meaning for these terms when we provide analytical expressions for the impulses
in Section \ref{sec:physical_intuition}. In the limit to rigid contact
when regularization $\mf{R}$ is zero, our primal formulation reduces to that
presented in \cite{bib:mazhar2014}.

\begin{theorem}
The dual of Eq. (\ref{eq:primal_regularized}) is given by Eq.
(\ref{eq:dual_regularized}). The pair $\{\mf{v},\bsigma\}$ is primal optimal and
$\bgamma$ is dual optimal. Moreover, $\bsigma = \bgamma$.
\end{theorem}

\begin{IEEEproof}
The Lagrangian of the primal formulation in Eq. (\ref{eq:primal_regularized}) is
\begin{equation}
	\mathcal{L}(\mf{v},\bsigma,\vf{\gamma}) = 
	\frac{1}{2}\Vert\mf{v}-\mf{v}^*\Vert_A^2 + \frac{1}{2} \Vert\bsigma\Vert_{R}^2 - \vf{\gamma}^T\mf{g}
	\label{eq:primal_lagrangian}
\end{equation}
with $\vf{\gamma}\in\mathcal{F}$ the dual variable to enforce the constraint
$\vf{g}\in \mathcal{F}^*$. We obtain the dual of Eq.
(\ref{eq:primal_regularized}) by minimizing the Lagrangian jointly in the
variables $\mf{v}$ and $\bsigma$ and replacing the result back to obtain the
dual cost $\ell_d(\vf{\gamma})$. Minimizing jointly in the variables $\mf{v}$
and $\bsigma$ leads to the optimality conditions
\begin{eqnarray}
	\mf{A}(\mf{v}-\mf{v}^*) &=& \mf{J}^T\vf{\gamma}\nonumber\\
	\vf{\sigma} &=& \vf{\gamma}
	\label{eq:primal_optimality_conditions}
\end{eqnarray}
where with the first equation we find out that multipliers $\bgamma$ are indeed
impulses and we recover the balance of momentum, and the second equation allows
us to eliminate $\vf{\sigma}$. When we replace these results back into the
Lagrangian in Eq. (\ref{eq:primal_lagrangian}) we obtain the dual
\begin{eqnarray}
	\min_{\bgamma\in \mathcal{F}} \ell_d(\bgamma) =
	\frac{1}{2}\bgamma^T(\mathbf{W}+\mathbf{R})\bgamma + {\bm r}^T
	\bgamma
\end{eqnarray}
where, in contrast to previous work, our Delassus operator
$\mf{W}=\mf{J}\mf{A}^{-1}\mf{J}^T$ now also contains the contribution of internal
force elements (through Eq. (\ref{eq:expression_for_A})) and
$\mf{r}=\mf{v}_c^*-\hat{\mf{v}}_c$ with $\mf{v}_c^*=\mf{J}\mf{v}^*$.
\end{IEEEproof}

Todorov \cite{bib:todorov2014} shows that impulses solution to the dual
formulation, Eq. (\ref{eq:dual_regularized}), can be constructed analytically
from the optimal velocities of the primal-formulation, Eq.
(\ref{eq:primal_regularized}). This is referred to as the \textit{inverse
dynamics} solution. Moreover, given the separable structure of the constraints,
the impulse $\bgamma_i$ at the $i\text{-th}$ contact point is the solution to
the following convex optimization problem
\begin{eqnarray}
	\bgamma_i(\vf{v}_{c,i})&=& P_{\mathcal{F}_i}(\vf{y}_i(\vf{v}_{c,i})) \nonumber\\
	&=&\begin{aligned}
		\argmin_{\bgamma\in\mathcal{F}_i} \quad &
	\frac{1}{2}(\bgamma-\vf{y}_i)^T\vf{R}_i(\bgamma-\vf{y}_i) \end{aligned}
    \label{eq:y_projection}\\
	\vf{y}_i(\vf{v}_{c,i}) &=& -\vf{R}_i^{-1}(\vf{v}_{c,i}-\hat{\vf{v}}_i)    
\end{eqnarray}
where $\vf{R}_i\in\mathbb{R}^{3\times3}$ is the $k\text{-th}$ diagonal block of
the regularization matrix $\mf{R}$. In other words, $\bgamma_i$ is the projection $P_{\mathcal{F}_i}$ of $\vf{y}_i(\vf{v}_{c,i})$ onto
the friction cone $\mathcal{F}_i$ using the norm defined by $\vf{R}_i$. We give explicit algebraic expressions along with a clear physical interpretation of this analytical solution in Section \ref{sec:physical_intuition}.
