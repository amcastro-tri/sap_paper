\usepackage{amsmath} 
\usepackage{amsfonts}
\usepackage{mathtools} 
\usepackage{amssymb}
\usepackage{graphicx}
\usepackage{bm}  % Bold math
\usepackage[colorinlistoftodos, textsize=footnotesize]{todonotes}
\usepackage[colorlinks=true, allcolors=blue]{hyperref}
\usepackage{setspace}
\usepackage{algorithm}      % http://ctan.org/pkg/algorithm
\usepackage{algpseudocode}  % http://ctan.org/pkg/algorithmicx
\usepackage{tikz}
\usepackage{verbatim}
\usepackage{xcolor}
\usepackage{subcaption}

% Package adjustbox: Introduces \adjincludegraphics to include figures allowing
% trimming.
\usepackage{adjustbox}

% Package ulem: The package provides an \ul (underline) command which will break
% over line ends; this technique may be used to replace \em (both in that form
% and as the \emph command), so as to make output look as if it comes from a
% typewriter. The package also offers double and wavy underlining, and striking
% out (line through words) and crossing out (/// over words).
\usepackage[normalem]{ulem}

\usepackage[thinc]{esdiff}

% Package amsthm: Provides the proof environment. N.B. this MUST go before the
% \newtheorem below. Do not change the order.
\usepackage{amsthm}
% Theorems numbered on a per section basis.
%\newtheorem{theorem}{Theorem}[section]
%\newtheorem{corollary}{Corollary}[theorem] \newtheorem{lemma}{Lemma}[section]

\usepackage{mathrsfs}

% Theorems numbered as explained in the IEEEtran HOWTO guide.
\newtheorem{theorem}{Theorem}
\newtheorem{corollary}{Corollary}[theorem]
\newtheorem{lemma}{Lemma}
\newtheorem{prop}{Proposition}

\newif\ifcompile
%\compiletrue % uncomment out to compile

\newtheorem{remark}{Remark}

% Place this declaration AFTER amsmath is included.
\DeclareMathOperator*{\argmin}{arg\,min}

\newcommand{\RedHighlight}[1]{{\color{red}\textbf{#1}}}
\newcommand{\CyanHighlight}[1]{{\color{cyan}\textbf{#1}}}
\newcommand{\GreenHighlight}[1]{{\color{green}\textbf{#1}}}
\newcommand{\redcolor}[1]{{\color{red}#1}}

% Super short aliases to color indexes in matrices.
\newcommand{\rr}[1]{{\color{red}#1}} \newcommand{\cc}[1]{{\color{cyan}#1}}

% TikZ commands for drawing schematics.
\newcommand{\tikzmark}[1]{\tikz[overlay, remember picture] \coordinate (#1);}

%For making diagonal entries in a matrix.
\newcommand{\diagentry}[1]{\mathmakebox[1.8em]{#1}}

% Helper to create entries of the form Jₚₜ₁ᵀGₚJₚₜ₂ p = #1 t1 = #2 t2 = #3
\newcommand{\JTGJ}[3]{\mf{J}_{\rr{#1}\cc{#2}}^T\mf{G}_\rr{#1}\mf{J}_{\rr{#1}\cc{#3}}}

% Math Shortcuts: Vector Font: For 3D vectors.
\newcommand{\vf}[1]{{\bm{#1}}}
% Matrix Font: for matrices, arrays and concatenation of 3D vectors.
\newcommand{\mf}[1]{{\mathbf{#1}}}
%
% Bold greek symbols:
\newcommand{\bgamma}{{\bm\gamma}} \newcommand{\btgamma}{{\bm{\tilde\gamma}}}
\newcommand{\bsigma}{{\bm\sigma}}
% Matrices:
\newcommand{\sM}{\mf{M}} \newcommand{\sJ}{\mf{J}} \newcommand{\sJT}{\mf{J}^T}
\newcommand{\sJbar}{\bar{\mf{J}}} \newcommand{\sJbarT}{\bar{\mf{J}}^T}
\newcommand{\sB}{\mf{B}} \newcommand{\sBT}{\mf{B}^T} \newcommand{\sG}{\mf{G}}
\newcommand{\sGT}{\mf{G}^T} \newcommand{\sR}{\mf{R}}
\newcommand{\sRinv}{\mf{R}^{-1}} \newcommand{\ssqrtR}{\mf{R}^{1/2}}
\newcommand{\ssqrtRinv}{\mf{R}^{-1/2}}
% Vectors
\newcommand{\sthat}{\hat{\vf{t}}}
% Projections: gradient of gamma with respect to y.
\newcommand{\sP}{\mf{P}}
% gradient of gamma_tilde with respect to y_tilde.
\newcommand{\sGtilde}{\tilde{\mf{G}}} \newcommand{\sPt}{\mf{P}_t}

% Definition symbol. Section 3 (page 15) of "The Comprehensive LATEX Symbol
% List".
\newcommand\defeq{\stackrel{\text{\tiny def}}{=}}
