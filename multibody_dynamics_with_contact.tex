\section{Multibody Dynamics with Contact}
\label{sec:multibody_dynamics_with_contact}

We use generalized coordinates to describe our multibody system. Therefore the
state is fully described by $\mf{q}\in\mathbb{R}^{n_q}$ and
$\mf{v}\in\mathbb{R}^{n_v}$, where $n_q$ and $n_v$ denote the number of
generalized positions and velocities, respectively.

Given a configuration $\mf{q}$ of the system, our geometry engine reports a
discrete set of $n_c$ contacts between a pair of bodies. The $k\text{-th}$
\emph{contact pair} is described by the location $\vf{p}_k$ of the contact
point, a normal direction $\hat{\vf{n}}_k$ and the \emph{signed distance} or
\emph{gap function} $\phi_k(\mf{q})\in\mathbb{R}$. The kinematics of each
contact is further completed with the relative velocity
$\vf{v}_{c,k}\in\mathbb{R}^3$ between these two bodies at point $\vf{p}_k$,
expressed in a contact frame $C_k$ for which we arbitrarily choose the
$z\text{-axis}$ to coincide with the contact normal $\hat{\vf{n}}_k$.

We form the vector $\mf{v}_{c}\in\mathbb{R}^{3n_c}$ of contact velocities by
stacking velocities $\vf{v}_{c,k}$ of all contact pairs together. Similarly, we
denote by $\bgamma_k\in\mathbb{R}^3$ the contact impulse at a specific contact
point and with $\bgamma\in\mathbb{R}^{3n_c}$ the vector of all contact impulses.
In general, unless otherwise specified, we use bold italics for vectors in
$\mathbb{R}^3$ and non-italics bold for their stacked counterpart.

The generalized velocities $\mf{v}$ and contact velocities $\mf{v}_c$ satisfy
the equation $\mf{v}_c=\mf{J}\,\mf{v}$, where
$\mf{J}(\mf{q})\in\mathbb{R}^{3n_c\times n_v}$ denotes the contact Jacobian.

We use a discrete time stepping approach to advance the dynamics of the system
forward in time and we write the balance of momentum as
\begin{eqnarray}
	\mathbf{M}\mathbf{v} = \mathbf{M}\mathbf{v}^* + \mathbf{J}^T\mathbf{\gamma}
	\label{eq:momentum_balance}
\end{eqnarray}
where $\mf{M}(\mf{q})\in\mathbb{R}^{n_v\times n_v}$ is the mass matrix and
$\mf{v}^*\in\mathbb{R}^{n_v}$ are the \textit{free motion} generalized
velocities of the system if contact impulses are zero. In Section
\ref{sec:discrete_formulation} we augment Eq. (\ref{eq:momentum_balance}) to
implicitly include internal modeling elements such as springs and dampers.

Denoting with $\gamma_{n,k} \in \mathbb{R}$ and with $\bgamma_{t,k} \in
\mathbb{R}^2$ the normal and tangential components of the impulse
$\bgamma_k=[\bgamma_{t,k}\,\gamma_{n,k}]$ in frame $C_k$, the formulation is
most often completed with the following set of standard constraints to be
satisfied at each contact
\begin{enumerate}
	\item non-penetration constraint $0\le\phi_k\perp\gamma_{n,k}\ge0$,
	\item\label{it:cone_constraint} friction cone constraint
	$\Vert\bgamma_{t,k}\Vert\le \mu_k\gamma_{n,k}$ and,
	\item the \emph{maximum dissipation principle}, which states that given the
	normal impulse, the friction impulse maximizes the rate of energy
	dissipation. In other words, friction impulses oppose the sliding velocity
	direction.
\end{enumerate}
where $\mu_k \in \mathbb{R}$ is the coefficient of friction for the
$k\text{-th}$ contact. Condition \ref{it:cone_constraint} states that contact
impulses $\bgamma_k$ at point $k$ are constrained to belong to the friction cone
$\mathcal{F}_k=\{[\vf{x}_t, x_n] \in\mathbb{R}^3 \,|\, \Vert\vf{x}_t\Vert\le
\mu_k x_n\}$.

While these constraints model the widely used model of Colulomb friction, the
resulting formulation leads to a very difficult to solve, non-convex, Non-linear
Complementarity Problem (NCP). 

\RedHighlight{Alejandro: say something how other author's impose the principle
of maximum dissipation and more or less what it means.}

\RedHighlight{Alejandro: say something about
the existence of solutions for this formulation. Reference other people's work.}

Plausible good references on this are
\cite{bib:stewart1996implicit,bib:stewart2000implicit,bib:chakraborty2007implicit,bib:acary2018solving,bib:pang1999unified,bib:alart2018inconsistency}.
\RedHighlight{Do not use \cite{bib:stewart2000implicit} since it essentially is
the same paper as the one orginally written in \cite{bib:stewart1996implicit},
unless they mention something new. Ditto with \cite{bib:pang1999unified}}

\section{Convex Approximation of Contact Dynamics}
\label{sec:previous_work}

To ensure existence of a solution and improve computational tractability,
Anitescu introduces a \textit{convex approximation} of the contact
problem \cite{bib:anitescu2006}. In this approximation the contact impulses are
solution to the following convex optimization problem
\begin{eqnarray}
	\min_{\gamma\in \mathcal{F}} \ell(\bgamma) =
	\frac{1}{2}\bgamma^T\mathbf{W}\bgamma + {\bm r}^T
	\bgamma,
	\label{eq:dual_cost}
\end{eqnarray}
where $\mathbf{W} =
\mathbf{J}\mathbf{M}^{-1}\mathbf{J}^T\in\mathbb{R}^{3n_c\times 3n_c}$ is the
\emph{Delassus operator}, $\mathcal{F} := \mathcal{F}_1 \times F_2 \times \cdots
\times \mathcal{F}_{n_c}$ is a Cartesian product of friction cones, and ${\bm
r}$ defines a linear cost  that encodes free motion velocities $\mf{v}^*$ and
stabilization terms used to impose non-penetration at the position level.

Further, \cite{bib:anitescu2006} uses a polyhedral approximation to linearize
the friction cone constraint $\bgamma_k\in\mathcal{F}_k$. In this work we do not
linearize the cone constraints but we work with the second order cone
constraints directly. This approach is preferred given that the polyhedral
approximation is known to introduce non-physical anisotropy
\cite{bib:li2018implicit}.

Anitescu shows in \cite{bib:anitescu2006} that the optimality conditions for the
problem in Eq. (\ref{eq:dual_cost}) imply the conservation of momentum in Eq.
(\ref{eq:momentum_balance}), the maximum dissipation principle and, the
non-penetration condition
\begin{equation}
	0 < \phi_k - dt \Vert {\bm v}_{t,k} \Vert \perp \gamma_{n,k} > 0
	\label{eq:convex_approximation_complementarity_condition}
\end{equation}
where $dt$ is the discrete time step and $\vf{v}_{t,k}$ is the tangential
component of the contact velocity $\vf{v}_{c,k} = [\vf{v}_{t,k}\,v_{n,k}]$.
Notice that Eq. (\ref{eq:convex_approximation_complementarity_condition}) is a
modified non-penetration condition that introduces coupling with the sliding
velocity, a result of the convex approximation. We provide a detailed discussion
on the physical validity of this approximation in Section
\ref{sec:physical_intuition}, along with guidelines for determining its
applicability to robotic simulation.

In \cite{bib:todorov2011, bib:todorov2014} Todorov introduces regularization to
the formulation in Eq. (\ref{eq:dual_cost}). Though not strictly applicable to
contact problems, Todorov uses the \emph{Gauss's principle of least
constraint} to obtain the following regularized form of Anitescu's formulation
\begin{eqnarray}
	\min_{\gamma\in \mathcal{F}} \ell(\bgamma) =
	\frac{1}{2}\bgamma^T(\mathbf{W}+\mathbf{R})\bgamma + {\bm r}^T
	\bgamma
	\label{eq:dual_regularized}
\end{eqnarray}
where $\mathbf{R}$ is a diagonal positive definite matrix introduced to make
$\mathbf{W}+\mathbf{R}\succ 0$ since in general we only have $\mathbf{W} \succeq
0$. This makes the problem strictly convex, with a unique solution.
\cite{bib:todorov2014} uses regularization as a means to introduce constraint
stabilization with a set of global parameters to control the amount of numerical
compliance introduced by the formulation. This is different from our approach
described in Section \ref{sec:physical_intuition}, where we show how to use
regularization to model physical compliance, with well defined physical
parameters.

Given the lack of software at the time, both Anitescu in \cite{bib:anitescu2010}
and Todorov in \cite{bib:todorov2014} use projected Gauss-Seidel (PGS) based
solvers to demonstrate the efficacy of their formulations. Though very popular
in the computer graphics community due to their straightforward implementation
even in parallel architectures, PGS methods are known for their slow convergence
rates (not even proven to converge if a convex approximation is not used), low
accuracy and lack of robustness.

Using Lagrangian duality, and building on Todorov's analysis, we convert Eq.
(\ref{eq:dual_regularized}) into an unconstrained problem in velocities in
Section \ref{sec:unconstrained_convex_formulation} and provide efficient
algorithms and software for its solution in Sections \ref{sec:sap_solver} and
\ref{sec:geodesic_solver}. Further, our methods warm start effectively when
using the solution from previous time steps, which remains challenging for the
formulation in impulses. We demonstrate the accuracy, robustness and efficacy
our methods with a variety of simulation problems in Section
\ref{sec:test_cases}.
