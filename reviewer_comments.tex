\documentclass[12pt]{article}

\usepackage{amsmath} 
\usepackage{amsfonts}
\usepackage{mathtools} 
\usepackage{amssymb}
\usepackage{graphicx}
\usepackage{bm}  % Bold math
\usepackage[colorinlistoftodos, textsize=footnotesize]{todonotes}
\usepackage[colorlinks=true, allcolors=blue]{hyperref}
\usepackage{setspace}
\usepackage{algorithm}      % http://ctan.org/pkg/algorithm
\usepackage{algpseudocode}  % http://ctan.org/pkg/algorithmicx
\usepackage{tikz}
\usepackage{verbatim}
\usepackage{xcolor}
\usepackage{subcaption}

% Package adjustbox: Introduces \adjincludegraphics to include figures allowing
% trimming.
\usepackage{adjustbox}

% Package ulem: The package provides an \ul (underline) command which will break
% over line ends; this technique may be used to replace \em (both in that form
% and as the \emph command), so as to make output look as if it comes from a
% typewriter. The package also offers double and wavy underlining, and striking
% out (line through words) and crossing out (/// over words).
\usepackage[normalem]{ulem}

\usepackage[thinc]{esdiff}

% Package amsthm: Provides the proof environment. N.B. this MUST go before the
% \newtheorem below. Do not change the order.
\usepackage{amsthm}
% Theorems numbered on a per section basis.
%\newtheorem{theorem}{Theorem}[section]
%\newtheorem{corollary}{Corollary}[theorem] \newtheorem{lemma}{Lemma}[section]

\usepackage{mathrsfs}

% Theorems numbered as explained in the IEEEtran HOWTO guide.
\newtheorem{theorem}{Theorem}
\newtheorem{corollary}{Corollary}[theorem]
\newtheorem{lemma}{Lemma}
\newtheorem{prop}{Proposition}

\newif\ifcompile
%\compiletrue % uncomment out to compile

\newtheorem{remark}{Remark}

% Place this declaration AFTER amsmath is included.
\DeclareMathOperator*{\argmin}{arg\,min}

\newcommand{\RedHighlight}[1]{{\color{red}\textbf{#1}}}
\newcommand{\CyanHighlight}[1]{{\color{cyan}\textbf{#1}}}
\newcommand{\GreenHighlight}[1]{{\color{green}\textbf{#1}}}
\newcommand{\redcolor}[1]{{\color{red}#1}}

% We use this command to mark text in reply to reviewers's comments/questions.
% Use the no-op definition below to get rid of all the marks.
%\newcommand{\reviewquestion}[2]{{\color{red}\textbf{#1: }#2}}
\newcommand{\reviewquestion}[2]{#2}

% Super short aliases to color indexes in matrices.
\newcommand{\rr}[1]{{\color{red}#1}} \newcommand{\cc}[1]{{\color{cyan}#1}}

% TikZ commands for drawing schematics.
\newcommand{\tikzmark}[1]{\tikz[overlay, remember picture] \coordinate (#1);}

%For making diagonal entries in a matrix.
\newcommand{\diagentry}[1]{\mathmakebox[1.8em]{#1}}

% Helper to create entries of the form Jₚₜ₁ᵀGₚJₚₜ₂ p = #1 t1 = #2 t2 = #3
\newcommand{\JTGJ}[3]{\mf{J}_{\rr{#1}\cc{#2}}^T\mf{G}_\rr{#1}\mf{J}_{\rr{#1}\cc{#3}}}

% Math Shortcuts: Vector Font: For 3D vectors.
\newcommand{\vf}[1]{{\bm{#1}}}
% Matrix Font: for matrices, arrays and concatenation of 3D vectors.
\newcommand{\mf}[1]{{\mathbf{#1}}}
%
% Bold greek symbols:
\newcommand{\bgamma}{{\bm\gamma}} \newcommand{\btgamma}{{\bm{\tilde\gamma}}}
\newcommand{\bsigma}{{\bm\sigma}}
\newcommand{\bxi}{{\bm\xi}}
% Matrices:
\newcommand{\sM}{\mf{M}} \newcommand{\sJ}{\mf{J}} \newcommand{\sJT}{\mf{J}^T}
\newcommand{\sJbar}{\bar{\mf{J}}} \newcommand{\sJbarT}{\bar{\mf{J}}^T}
\newcommand{\sB}{\mf{B}} \newcommand{\sBT}{\mf{B}^T} \newcommand{\sG}{\mf{G}}
\newcommand{\sGT}{\mf{G}^T} \newcommand{\sR}{\mf{R}}
\newcommand{\sRinv}{\mf{R}^{-1}} \newcommand{\ssqrtR}{\mf{R}^{1/2}}
\newcommand{\ssqrtRinv}{\mf{R}^{-1/2}}
% Vectors
\newcommand{\sthat}{\hat{\vf{t}}}
% Projections: gradient of gamma with respect to y.
\newcommand{\sP}{\mf{P}}
% gradient of gamma_tilde with respect to y_tilde.
\newcommand{\sGtilde}{\tilde{\mf{G}}} \newcommand{\sPt}{\mf{P}_t}

% Definition symbol. Section 3 (page 15) of "The Comprehensive LATEX Symbol
% List".
\newcommand\defeq{\stackrel{\text{\tiny def}}{=}}


\usepackage[margin = 1in]{geometry}
\geometry{letterpaper}                   		% ... or a4paper or a5paper or ...
\usepackage{amsmath}
\usepackage{multirow}
%\setlength{\parindent}{0pt}

\begin{document}

\author{Alejandro Castro, Frank Permenter, Xuchen Han}
\title{Response to the Reviewers \\
  \large An Unconstrained Convex Formulation of Compliant Contact}
\maketitle

We thank the reviewers for their valuable time, comments and overall positive
assessment of our work. As suggested by Reviewer 1 and the Associate Editor, we have
significantly reorganized the contents. In particular, now sections II
and III of the paper are organized with the attempt to make the reading process
more linear. We also identified text repetition and removed it. A specific
Section I.C discusses \emph{novel contributions} and a new Section VIII
discusses \emph{limitations}. On top of this
newly organized text, we addressed the comments from the reviewers. In the
following, the comments of the reviewers are reproduced here in
\textcolor{blue}{blue} for the reviewers' convenience, followed by our
responses. We numbered the comments/questions from the reviewers in this
document and marked edits in the revised manuscript with
\textcolor{red}{R?-Q\#}, where \textcolor{red}{?} corresponds to the reviewer
number and \textcolor{red}{\#} corresponds to the question number. Questions
from the associate editor are marked with \textcolor{red}{AE} instead of
\textcolor{red}{R}. All revised text and responses are marked in
\textcolor{red}{red}. We hope that the reviewers find our edits acceptable and
the revised manuscript suitable for publication.

\section{Associate Editor}
\label{sec:associate_editor}

\textcolor{blue}{Three Reviewers evaluated the submitted paper. Their comments
are reported and should be carefully followed. In particular, the Reviewers
highlighted several criticisms.}
\textcolor{blue}{\begin{itemize}
    \item Linearization effects should be better highlighted (Reviewer 1).
\end{itemize}}
Linearization of the equations of motion is a very common approximation in many
other time stepping schemes for multibody dynamics in the literature, specially
so if contact is involved. However, the details might be elusive if not properly
mentioned. Therefore we do agree with this review, these approximations must be
mentioned explicitly.
\RedHighlight{TODO: consider adding this into a Limitations section.}

Reviewer 1 comments on linearization effects in questions R1-Q1, R1-Q2 and
R1-Q3. We provided specific answers to each of these questions in this document
and updated the manuscript accordingly. 

\textcolor{blue}{\begin{itemize}
    \item It seems that the approach has an implicitly defined constraint,
\end{itemize}}
Constraints in this formulation are eliminated using analytic formulae that we
provide in this work. We provided more details in the answer to R1-Q7 in this
document.

\textcolor{blue}{which contradicts the claim of an unconstrained problem. If
this impression is confirmed, then the global convergence of the method becomes
unclear (Reviewer 1).}

We address this point in answer to R1-Q4, with mention specific theorems and
results in our work. We highlighted updates in the manuscript accordingly.

\textcolor{blue}{
\begin{itemize}
    \item A more in-depth analysis of the matrices K, D, and H should be
    carried out. Can they be non-symmetric? (Reviewer 4)
\end{itemize}}

This is an important aspect of our work. While we do not impose any restriction
on the gradient $\partial \mf{m}/\partial \mf{v}$ (with the momentum residual
$\mf{m}$ defined in Eq. 14), we do require its approximation $\mf{A}$ to be SPD.
We address this question in detail in R4-Q3 and in the updated manuscript.


\textcolor{blue}{
\begin{itemize}
    \item Authors should clearly state the limitation of the approach. This
regards scalability (Reviewer 4), objects convexity (Reviewer 8), change in
contact modes during the task( Reviewer 8), and possible others. \end{itemize}}

Scalability is addressed in R4-Q7, objects convexity in R8-Q1, and changes in
contact modes in R8-Q3, along specific updates to the manuscript.

In addition, we added \RedHighlight{Section XXX} discussing these and other
limitations of our approach.

\RedHighlight{TODO: ok, Done deal. Add a Limitations section that includes all
the topics here and the TODOs I sprinkled all over.}

\vspace{5mm}
\textcolor{blue}{After a careful reading of the paper, this Associate Editor
agrees with all the points raised by the Reviewers.}

We agree with the points raised as well. We believe the updated manuscript is an
improvement over the original version thanks to the reviewers' comments.

\vspace{5mm}
\textcolor{blue}{In addition, this AE has the following further comments.}
\textcolor{blue}{
\begin{itemize}
    \item[AE-Q1] As also highlighted by Reviewer 1, the current paper's outline
confuses the reader and makes the reading flow difficult. This AE suggests
deeply revising the organization of the contents. One solution is to avoid
fragmenting the content in so many different sections. Authors should organize
the same focus in one or two bigger sections and exploit subsections to split
the related content. For instance, this AE believes that Sections V, VI, VII,
and IX may be unified. Instead, it is not clear why Sections II, III, and IV
have been split in such a way, also because the same concepts are often repeated
(as the fact that "the proposed approach does not approximate the friction cone
but [...]").\end{itemize}}

We do realize the original organization of the contents was far from optimal. We
are glad the reviewers agreed on the need for this change. We reorganized the
updated manuscript in order to make the contents flow more linearly, group
similar concepts together and avoid repetition. In the new manuscript Section II
introduces the mathematical model of multibody dynamics with frictional contact,
introduces notation and establishes the non-linear complementarity problem (NCP)
that must be solved in the most general case. Section III now is entirely
focused to the convex approximation of this NCP formulation, effectively
grouping together concepts previously scattered across different sections.
Section IV now follows naturally, describing our proposed SAP strategy to solve
the resulting unconstrained optimization problem. Finally, everything related to
the physical interpretation and modeling parameters of the convex approximation
is condensed in Section V.

\vspace{5mm}
\textcolor{blue}{AE-Q2: On the other hand, the state of the art can be removed from
Section I, and a dedicated (sub)section may be created. Novelties (and
limitations of the approach as said above) should be addressed in separated
subsections. Therefore, a considerable re-organization of the contents is
suggested.}

In the updated manuscript Section I.B summarizes available software that uses a
convex approximation of contact, Section I.C states our contributions and
Section VIII explicitly states limitations of our approach.

\textcolor{blue}{
\begin{itemize}
    \item[AE-Q3] Authors performed a good state of the art within the Introduction,
    mentioning much software such as ODE, Dart, Vortex, Mujoco, and Chrono. The
    authors also provide differences between the cited software and the one
    proposed in the reviewed paper. However, RaiSim [AE1,AE2] is
    missing.\end{itemize}}

Thank you for the reference, we added it to the updated manuscript.

\textcolor{blue}{
\begin{itemize}
    \item[AE-Q4] What is the "gap function" in section II?\end{itemize}}
    
In the context of multibody dynamics with contact \emph{gap function} often
refers to the \emph{signed distance function}, see for instance
\cite{bib:negrut2018posing}. We do realize however that \emph{signed distance}
is the more common term in the robotics community and we adopted this
terminology instead. Section II.A has been updated and references are provided.

\textcolor{blue}{\begin{itemize}
\item[AE-Q5] Maybe, a $\delta t$ term is missing in the last term of Eq. (2).
\end{itemize}} 

\textbf{Please note:} in the updated manuscript the balance of momentum in the
previous Eq. (2) is now stated in Eq. (7).

The reason $\delta t$ does not appear in the last term of Eq. (7) is because
$\bgamma$ denote impulses, the integral (in this case over one time step) of the
forces. Mean forces over a time step are recovered dividing impulses by $\delta
t$. This shows in Eq. (2) for the modeling of compliance. We updated the
manuscript to explicitly state this.

\textcolor{blue}{\begin{itemize}
\item[AE-Q6] In Section II, the non-penetration constraint equation has a non-explained
symbolism. Please, clarify the expressions with the $\perp$ symbol (see also Eq.
(4)).\end{itemize}}

Thank you, we missed this in the original manuscript. We updated the text in
Section II.B to define the $\perp$ symbol.
    
\textcolor{blue}{\begin{itemize}
\item[AE-Q7] At the beginning of Section VII, it is written that "the approach is
different from the one in [16], where regularization is not used to model
physical compliance but rather to introduce a user tunable Baumgarte-style"
stabilization to avoid constant drift." It is unclear to this AE whether this
difference is due to a different interpretation of the regularization term or it
is substantial, affecting the theoretical part.\end{itemize}} 

\vspace{5mm}
\textcolor{blue}{\begin{itemize}
    \item[AE1] https://raisim.com
    \item[AE2] J. Hwangbo, J. Lee, M. Hutter, "Pre-contact iteration method for
solving contact dynamics," IEEE Robotics and Automation Letters, vol. 3, n. 2,
pp. 895-902, 2018.\end{itemize}} 


\section{Reviewer 1}
\label{sec:reviewer_1}
\textcolor{blue}{
The authors provide a very comprehensive paper. The authors use impulse based
contact dynamics models to create a simulator that can handle contacts by
introducing an unconstrained optimization problem instead. The main
contributions of the paper are:
\begin{enumerate}
    \item an unconstrained convex scheme at the points of contact to discover
    the contact velocities and impulses that is equivalent to the constrained
    problem under certain ``tunable" assumptions.
    \item a custom solver that uses exact line search to find the optimal
    step length.
    \item An explanation of regularisation proposed by Todorov in [16], and
    a link with the friction and dissipation parameters proposed by the authors.
    \item Test cases which compare various aspects of the model with
    state-of-the-art implementations. e.g. solvers comparison with Gurobi,
    energy dissipation in various implicit and explicit discretization schemes
    etc.
\end{enumerate}
Firstly, the authors use Anitescu's convex formulation for finding the contact
impulses by maximum dissipation principle (Eq. (3)), and convert it to equivalent
primal form with velocity and impulse parameters instead of impulses (Eq. (15)).
Secondly, the authors provide a compliance based physical explanation for the
regularization used by Todorov in [16] (Section IX, Eq. (18) etc), and use this
explanation to create a mapping between the optimal impulses and velocities.
Thanks to this relation, the authors remove the cone constraints from Eq. (15),
and solve Eq. (17) instead.
}
\vspace{5mm}

\textcolor{blue}{
R1-Q1: The authors use a first order approximation of the impulse dynamics (Eq. (2))
in order to recursively solve the linearized problem around the unconstrained
position $\vf{v^*}$ (Eq. (14)). This is an interesting assumption, since the linearized
problem might not be easy to converge for a high DoF system. While the authors
provide test cases with manipulators and griper contact, it still doesn't
guarantee that the problem would still converge to a solution.}

\textbf{Please note:} previous Eqs. (2) and (14) are now Eqs. (7) and (18) in the
revised manuscript, respectively.

We hope that the reviewer finds the new organization of the contents more
pleasant to read. We believe that this new organization would hopefully help
make the overall strategy more clear. To address this particular question
directly however, we added text at the end of Section III.C. In particular, we
emphasize that our algorithm has guaranteed convergence to Eq. (18), accurate to
second order (with the time step size) to the possibly non-linear balance of
momentum in Eq. (7). The accuracy statement is made in Proposition 1 and proven
in Appendix A. Guaranteed convergence of our SAP solver to the linear
approximation in Eq. (18) is proven in Appendix E.
\vspace{5mm}

\textcolor{blue}{
R1-Q2: Could the authors discuss the effects of this linearization, and what the
worst-case situation in this setup would be. }

There is a significant amount of work on the validity of the approximation in
Eq. (18) for multibody systems with or without contact. See, for instance,
\cite{bib:potra2006linearly}. As mentioned above, we comment on the validity of
these linearizations in Section III.C. A worst-case scenario would involve large
Coriolis terms, since they are quadratic functions of the generalized
velocities. We note however that these are included in the computation of the
free motion velocities $\mf{v}^*$ when using implicit schemes in Eq. (13) (when
$\theta>0$ and $\theta_{vq}>0$).
\vspace{5mm}

\textcolor{blue}{
R1-Q3: Also, the linearization would introduce joint constraint errors the
multi-body linkages. Do the authors have a measure of their convergence? If so,
could they discuss it here please?}

We suspect there is confusion about our problem setup. In this work, we do not
use maximal coordinates and we do not model joints using constraints. We instead
describe our robot models using generalized joint coordinates. Therefore our
formulation does not introduce joint drift and no constraint
stabilization is needed. To address this directly, we explicitly stated our
choice at the beginning of Section II.
\vspace{5mm}

\textcolor{blue}{
R1-Q4: The above point is linked to the the claim of ``proven global convergence"
that the authors use in the abstract. The authors do create an unconstrained
optimization problem, but the underlined dynamics seems to be implicitly
constrained, and non-convex, for which the authors use a linearized
assumption. Thus I would be interested to understand better the claim of ``proven global
convergence".}

The claim of global convergence applies to the SAP solver: on all inputs, it
converges to the unique solution of the primal formulation (19). A precise
statement of this claim appears in Appendix E. Additionally, we highlighted
relevant text in the main body of the manuscript at the beginning of Section IV.

As we clarify in our revision (at the end of Section III.C) and in the answers
to questions Q1 and Q2 above, the relation to the original possibly non-linear
balance of momentum in Eq. (7) is given in Proposition 1. Proposition 1 states
that the linearized momentum balance in Eq. (18) is a second order approximation
of the original momentum balance in Eq. (7). The text we added at the end of
Section III.C in the revised manuscript now explicitly states these
observations.
\vspace{5mm}

\textcolor{blue}{R1-Q5: ``In addition, the linearization of the friction cone
results in a far larger problem due to the additional constraints needed to
represent the polyhedral cone." Please elaborate ``larger problems" here. Could
the authors compare this to the assumptions that they need to take for the
dynamics to work?}

We removed this statement from the revised manuscript since we felt it was a
distraction from the contributions in this work. We still discuss the
role of LCPs in the introduction and mention the artificial anisotropy they
introduce.

However, we would still like to provide an answer to the reviewer on the
question about what we meant by ``larger problems''.

Our formulation does not linearize the friction cone but it works with the
second order cone constraints directly, avoiding spurious non-physical
anisotropy \cite{bib:li2018implicit}. We eliminate constraints analytically and
solve an unconstrained convex problem. In comparison, an LCP formulation would
introduce $(2u+1)$ constraints per contact \cite{bib:anitescu1997}, where $u$ is
the number of edges in the polygonal approximation. This strongly affects the
performance of solvers in practice given their worst case exponential complexity
in the number of constraints
\cite{bib:baraff1994fast},\cite{bib:erleben2007velocity}.

\vspace{5mm}

\textcolor{blue}{R1-Q6: $\theta_q$, $\theta_v$, $\theta_{vq}$ could the authors
please provide explanations for these variables in Eq. (6).}

We thank the reviewer for this question. While reviewing the manuscript we
realized that we could perform an additional simplification thanks to this
question. While in our original version we had three theta parameters, the
revised version now only contains two parameters. We highlighted the relevant changes
in Section II.C. In addition, we added text at the end of Section II.C
explaining that the reason for the additional parameter ($\theta_{vq}$) is so
that we can also incorporate the popular symplectic Euler scheme. In total, our
scheme is parameterized by the ``regular" $\theta$ parameter as in the
traditional $\theta$-method and the additional $\theta_{vq}$ parameter to
support the symplectic Euler scheme. These scalar parameters are the weights
that define the time stepping scheme. We refer the reviewer to the excellent
book by
\cite{bib:hairer2008solving}
(Section II.7) for further details on this method. We highlighted the relevant text
explaining these parameters in the revised manuscript.
\vspace{5mm}

\textcolor{blue}{R1-Q7: Could the authors please explain the significance of
``$\vf{y}$" in equations 16 and above? Since the Projection operation ($Pf()$)
is constrained, the authors still don't have an unconstrained equation in Eq.
(17), and have an implicitly defined constraint instead. Could the authors
please justify the claim of ``unconstrained convex problem" here and comment on
the implicit nature of the constraints?}

\textbf{Please note:} previous Eqs. (16) and (17) are now Eqs. (23) and (24)
in the revised manuscript, respectively.

The primal-formulation (24) is, by definition, an unconstrained optimization
problem since the variables can assume any value. Hence, it can be solved using
gradient descent (or any other descent method). Unconstrained problems, however,
are not necessarily easy. As the reviewer points out, evaluating the objective
function of (24) ostensibly requires the implicit solution of another
optimization problem. Fortunately, analytic formulae for evaluating this
objective (and its gradient and Hessian) exist and are efficiently computable.
As in our original submission, they are provided  in Section V.A and derived in
Appendix C.
%We believe that the large reorganization of the manuscript we performed
%following the excellent suggestion by this reviewer helps to make the flow of
%the paper more clear, in particular in this regard. But let us answer the
%question directly here to help in this regard better. The reason we make the
%claim ``unconstrained convex problem" is because the projection operation
%$\vf{\gamma}=P(\vf{y})$ can be computed analytically. Let us stress this again:
%$\vf{\gamma}=P(\vf{y})$ is an algebraic function for which we provide the
%explicit functional form in Section V.A and derive in Appendix C. Yes, the
%reviewer is correct that $\vf{\gamma}=P(\vf{y})$ is the result of a constrained
%convex optimization problem (Eq. 23, Section III.B), but we use the remarkable
%observation from Todorov that this can be computed analytically, see
%introduction in Section III and details in III.B. Going back to the original
%question, the formulation in now Eq. 24 (Section III.C) is ``unconstrained"
%because we do provide algebraic expressions for each of the terms in that
%equation. Yes, now the cost is a more complex nonlinear function of the
%velocities, while in Eq. 19 the cost is simply quadratic. However, we prove the
%remarkable property that this now more complex cost still is (strongly) convex
%(Appendix B) and that it is equivalent to the constrained formulation in Eq.
%19. We find this result simply beautiful and it is definitely a cornerstone of
%our work. We are pleased that the reviewer asked about this because it's a
%point that should not be missed. We updated the text to make these observations
%clear in Section III.B and III.C. We also made a major rearrangement of now
%Section III so that the arrival of the unconstrained formulation in Eq. 24
%(Section C) is not unexpected but flows naturally from the previous sections.
\vspace{5mm}

\textcolor{blue}{R1-Q8: ``fifteen years after the introduction of these convex
approximations, robust and performant algorithms for their solution in practice
are lacking." There are already multiple simulators in practice that provide
solutions with various levels of accuracy and robustness and different
assumptions. I would suggest that authors remove this claim here, it is broad
and not supported.}

Thank you for the suggestion. We do realize that such a claim could be
understood as being applicable in a broader sense and therefore we decided to
remove it to avoid the confusion of future readers. We want, however, to share
our experience with these algorithms and decided to cite the very extensive
study performed by \cite{bib:acary2018solving} (text added in Section I.B). This
study shows that even though there are good solution alternatives, most
numerical methods aimed at solving the convex approximation of contact suffer
from robustness and accuracy issues. We tried many of these algorithms ourselves
and had the same experience. This is one of the reasons that inspired our
research in this direction and that finally led to this paper. We hope this new
reference helps the reviewer understand the purpose of our research efforts.
\vspace{5mm}

\textcolor{blue}{R1-Q9: ``It is not yet clear if these formulations present a real
advantage when compared to more traditional approaches and whether the artifacts
introduced by the approximation are acceptable in robotics applications". Could
the authors please discuss the approximations that they mean here? Is this
limited to the author's explanation of MuJoCo regularization, or do the authors
claim to explain other artififacts as well?}

The particular approximation we refer to is the convex approximation
to the contact problem. The modeling of contact requires the solution of
challenging Non-linear Complementarity Problems (NCP), which are non-convex.
Therefore, any convex formulation of contact is an approximation of reality.
Therefore, we do not refer in particular to MuJoCo, but any solver that uses
this approximation (we mention Chrono, MuJoCo and Siconos in Section I.B).
To address this question, we updated the text to say ``convex approximation"
explicitly and ``non-convex NCP" also explicitly, so that discussions of the approximation
and the original problem are concentrated. We highlighted the change in the
text.

\section{Reviewer 4}
\label{sec:reviewer_4}

\textcolor{blue}{
The Authors present an interesting and original research on a numerical
method for simulating contact problems. The method can resolve corner
case scenarios that are known to be difficult in conventional multibody
solvers, hence it is very promising in the framework of robotics, that
always strive for maximum robustness and stability. The transformation
dual-vs-primal is an interesting strategy that allows the simulation of
near-rigid FEA structures in NSCD, something that was always
challenging.
Some comments that the Authors can address:
}
\vspace{5mm}

\textcolor{blue}{R4-Q1: page 2: "the convex approximation introduces a gliding
artifact at a distance proportional to the time step size and to the sliding
velocity" ... I would recall: *and to the friction coefficient $\mu$ *. Later
(ex. pag.6) this is mentioned. The proportionality with the friction coefficient
means that such artifact can be irrelevant for problems with lubricated contacts
as in many mechanisms, but still relevant for sticky materials, ex. rubber etc.
like in robot grippers etc. (maybe worth mentioning this, as the paper is
involving robotic \& manipulation)}

Thank you for this very useful remark. We added text in the revised manuscript
highlighting this observation.

\vspace{5mm}

\textcolor{blue}{R4-Q2: pag.2: "The multi-physics simulation package Chrono implements a
variant of the PGS method [...] the solver exhibits low convergence
rates.". True, but that PGS is not the only solver available in that
simulation library: additional methods based on Nesterov,
Barzilai-Borwein accelerations and ADMM have been introduced with the
aim of better convergence.}

\RedHighlight{TODO: Find out more about these methods. Basically figure out: are
they also first order? do they also require many iterations. See how to update
this text in the intro to make are more accurate assessment.}

\vspace{5mm}

\textcolor{blue}{R4-Q3: pag.4: "K and D are constant, diagonal, and positive
definite matrices.". Are there assumptions on K and D being symmetric in your
implementation? In many FEA nonlinear problems these are not symmetric, for
example. Can this have an impact on numerical performances? Comment on this,
please.}

Yes. Actually, we split the forces term $\mf{k}$ into $\mf{k}_1$ and $\mf{k}_2$
precisely to ensure that the stiffness and damping matrices are symmetric
positive definite. This then allows us to obtain an SPD approximation $\mf{A}$ of
the gradient $\partial \mf{m}/\partial \mf{v}$. We highlighted edited text in
Section II.D making specific mention to the important case of joint spring
dampers, for which $\mf{K}$ and $\mf{D}$ are positive diagonal matrices.
Regarding finite elements modeling, we are currently working on a new
publication that incorporates our new solver SAP within a FEM framework. In this
new work we essentially restrict to materials whose internal stress comes from
an internal elastic energy. Significantly more details will be provided in that
piece of work. We updated the text in Section VII to briefly mention this
discussion.

\vspace{5mm}

\textcolor{blue}{R4-Q4: pag.7: on line search: have you considered to use an
inexact "buffered" line search like Grippo-Lampariello-Lucidi (GLL)? This could
speed up line search where Armijo or other strategies can be too conservative
during zones of frequent constraint transitions.}

\RedHighlight{TODO: I doubt the peformance could be better. It is already pretty
good. However, it'd seem like giving this method a try would not be hard.
Consider implementing and making at least a quick comparison that can be briefly
discussed in a one liner.}

\vspace{5mm}

\textcolor{blue}{R4-Q5: pag.7: "supernodal Cholesky factorization" ... this
implies that H is Hermitian, but in some FEA problems the D and K matrices might
be non-symmetric, thus H as well, correct? How do you deal with this, if so? Can
you comment on this? }

Supernodal ideas generalize to LU factorization. However, it is a requirement of
the formulation for the stiffness and damping matrices that are included in the
linear dynamics matrix $\mf{A}$ to be symmetric positive definite. We discussed
this in the answer to R4-Q3. We provided two examples: joint spring and dampers,
which we include in term $\mf{k}_1$, and simulation of deformable solids for
which stresses stem from an internal elastic energy. However, modeling
of deformable objects with frictional contact will be discussed in depth in our
next publication. Please see highlighted text labeled as R4-Q3 in the revised
manuscript.

\vspace{5mm}

\textcolor{blue}{R4-Q6: pag.8, bottom: "...we form the approximation [....].
Using this approximation, we estimate the frequency of the contact dynamics...".
This is a smart idea, although I have a counterexample: suppose you have two
very light particles in contact, but both of them are welded to two very heavy
bodies via two joints: the simplified evaluation of the Delassus operator would
underestimate the contact masses in the constraint metric. If this is the case,
please add a comment on the practical limitation of this simplified Delassus
formula, mentioning a real example like the one I wrote here.}

This is an excellent question, thank you. This particular case you mention is
not problematic because we are using joint coordinates. Lets call the two heavy
bodies H1 and H2 and for simplicity, consider they are free. Lets refer to the
light particle welded to H1 as L1 and to the light particle welded to H2 as L2.
In joint coordinates, there will be two trees: t1 formed by L1 welded to H1 and
t2 formed by L2 welded to H2. Therefore the mass matrix for each tree will be
dominated by the mass of the heavy particle and the Delassus approximation works
as desired in this case.

However, as the reviewer pointed out, we can still come up with a
counterexample. Consider a stack of books on a table, in particular the one book
at the bottom of the stack in contact with the table. In the \emph{near-rigid}
regime, our approximation will estimate a stiffness based solely on the mass of
this one book. However, the normal force on this book will be the result of the
accumulated weight of all other books on top of it. In this case, we will
underestimate the required compliance and we might need the users's
intervention. If we encounter such a case, we simply set $\beta=0$ (which means
the stiffness is not bounded) and set the stiffness to a value that allows the
book at the bottom to support the entire stack.

The updated manuscript mentions this case in the \emph{Limitations} section.

\vspace{5mm}
\textcolor{blue}{R4-Q7: pag. 12: for scalability, you tested up to hundreds of
bodies. Do you think this approach (with the SAP solver, I mean) can scale up to
hundreds of thousands of bodies or millions? Are there limitations? Is it more
practical for "low contact density" scenarios like robotics, or also for "high
contact density" scenarios like granular flows, or both?}

The scalability of SAP, like any second-order optimization method, depends on
the complexity of solving the Newton system. For fully dense problems,
direct methods have $\mathcal{O}(n^3)$ complexity, where $n$
denotes the number of variables.  For sparse problems, improved bounds can be
stated in terms of \emph{tree-width} $d$, a complexity measure defined by the
\cite{chordal-extensions} of the linear system matrix.

We see no reason that SAP could not be applied to millions of bodies if
tree-width $d$ is small.  Further, when $d$ is large, SAP can be modified to improve
scalability.  Specifically, we can approximately solve the Newton system using
the conjugate-gradient (CG) method, which is widely used in large-scale
optimization.  Note that if we terminate CG after one-iteration, SAP reduces to
gradient descent, i.e., a first-order optimization algorithm. 

Note that the  per-iteration complexity of SAP is the same
for interior-point methods (IPMs).  Our computational results show the per-iteration
complexity is comparable to Gurobi, a commercial implementation of an IPM.

Recently, we tested the scalability of SAP for high contact density problems,
though with a small $d$. That is, we performed a grid study with our most recent
contact model that leads to thousands of contact pairs. Still, not all objects
are in contact with each other and therefore SAP achieves almost linear
convergence with the number of contact constraints (an exponent about $n\sim1.3$
to be precise). Details of this scalability study are provided in
\cite{bib:masterjohn2021discrete}.

We added text at the end of Section IV.D and in Section VIII on Limitations.


\section{Reviewer 8}
\label{sec:reviewer_8}

\textcolor{blue}{In this paper, the authors have presented a convex formulation
of compliant frictional contact and a method to solve it. The proposed
formulation can be used for robotic simulation and planning purposes. The
formulation is unconstrained in the sense that it has been described in terms of
the contact impulses and corresponding velocities rather than the contact
constraints directly. The mathematical analysis provided in this paper and the
appendix is rigorous and sufficient for understanding the convex nature of the
optimization formulation. To solve the proposed formulation, the authors have
developed SAP, the Semi-Analytic Primal Solver. They have studied its
initialization, stopping criteria and convergence properties. They have also
studied the solver's performance compared with commercially available solvers
like Gurobi and Geodesic IPM. The video attachment provided along with the paper
is very well made and helps in understanding the key contributions, experiments
and their results.}

\textcolor{blue}{\textbf{Comments to the authors:}}

\textcolor{blue}{R8-Q1: 1) It would be useful to understand whether this method
only works for convex objects. Will the convergence properties and results of
the proposed method be affected if we use non-convex objects with non-convex
contact surface patches?}

This method is not limited to convex objects only. Support for non-convex
geometries will depend on whether the geometry engine provides support for
non-convex geometries or not. The mathematical formulation presented here only
requires the geometry engine to report \emph{contact pairs}, regardless of
whether those pairs correspond to convex or non-convex geometries. We describe
the characterization of contact pairs in in Section II.A. At the moment of this
writing, Drake's support for non-convex geometry is limited, though we are
actively working on it. However, coming back to the topic of this paper, the
solver itself is not limited to convex geometry.

\vspace{5mm}
\textcolor{blue}{R8-Q2: 2) Is it possible to incorporate maximum contact force
constraints directly in the optimization formulation for the dual-arm
manipulation? It would be interesting to be able to simulate tasks where the
contact forces in the normal direction are not large enough to firmly grasp and
manipulate the object against the effect of external wrenches.}

If we understand the question correctly, the reviewer is asking to simulate
actuation limits on the Allegro hands. While we are not actively applying
actuation limits on the hands, the parameters of the simulation are well within
real specifications. To be more specific to the contribution of this work, our
method does allow to specify these limits. 

We believe it would be very interesting to investigate the limits of different
parameters for this particular platform and task. However it was not our
intention to investigate this particular task to inform for instance, how to
build real platform. This robotic platform is only hypothetical and for
demonstration purposes. The objective of this simulation demo is mostly to
showcase the new capability.

\vspace{5mm}
\textcolor{blue}{R8-Q3: 3) It is not evident to me whether the proposed
formulation can be used to study the change in the contact modes (stiction,
sliding, no contact) for prehensile manipulation tasks such as pivoting a cubic
object about an edge using a manipulator. For examples, impulses applied at the
object-manipulator contact may cause the object to slip and slide instead of
pivoting on the edge. The change in contact modes can be used as feedback for
adjusting the contact impulses provided by the manipulator to the object.}

Our formulation does handle contact modes implicitly. As an example, consider
the \emph{Slip Control} example in Section IV.D. In this example the controller
commands a variable grip force to the gripper. In this case, the gripper always
makes contact with the spatula. However, given the change in grip force, the
grip transitions periodically between a loose grip where the spatula rotates
within grasp (sliding) and a secure grasp where the spatula stops moving
(stiction). This transition between stiction and sliding is made very evident in
the accompanying video (starting at 0:42) where the visualization shows a
contact patch at all times, though its size changes due to the changes in grasp
force.

Consider now the dual arm manipulation task (better shown in the video at 3:23).
Initially the jar is closed, with the lid held in place only by stiction. As the
robot pulls the lid out, the lid comes off given that contact between the jar
and lid transitions to sliding contact. The model fully predicts these
transitions.

Similarly for the example case the reviewer provided. If there was a transition
between sliding and stiction at the edge of this cubic object, our model is able
to resolve it. 

Finally, the particular contact mode can be recovered by inspecting the value of
the impulses before projection $\mf{y}$, see Eq. (27). The region in which
$\mf{y}$ lies determines the contact mode, see Fig. (22) in Appendix C for a
graphical schematic.

To make this point clear, we added text for the \emph{Slip Control} case in
Section VI.D and for the \emph{Dual Arm Manipulation} case in Section VI.E.


\bibliographystyle{./IEEEtran/IEEEtran}
\bibliography{sap_paper}

\end{document}
