% The very first letter is a 2 line initial drop letter followed
% by the rest of the first word in caps.
% 
% form to use if the first word consists of a single letter:
% \IEEEPARstart{A}{demo} file is ....
% 
% form to use if you need the single drop letter followed by
% normal text (unknown if ever used by the IEEE):
% \IEEEPARstart{A}{}demo file is ....
% 
% Some journals put the first two words in caps:
% \IEEEPARstart{T}{his demo} file is ....

\section{Introduction}
\IEEEPARstart{A}{Anitescu}'s convex relaxation of contact
\cite{bib:anitescu2006} has been in existence for 14 years. However to my
knowledge there are no simulation/trajectory optimization codes out there that
can exploit this formulation. In other words, the formulation is theoretically
beautiful, but there are no practical solvers out there for it. The best
reported solver for this formulation, later on presented by Anitescu in
\cite{bib:anitescu2010, bib:tasora2011}, consists of a Projected Gauss-Seidel
(PGS) iteration which is very well known not to converge in practice and it is
usually stoped at a fixed number of iterations with an unknown amount of
numerical error. Therefore, it is unclear even today whether Anitescu's
formulation presents any advantage over other non-convex solution strategies. In
\cite{bib:todorov2014} Todorov mentions a so called \textit{generalized}
projected Gauss-Seidel (though the details were never published), which
generalizes PGS to deal with the true quadratic friction cone, anisotropy,
torsional and rolling friction.

The objectives of this reasearch are:
\begin{enumerate}
	\item To develop a \textit{practical} solution strategy that is robust and
	has good scalability to number of bodies and contacts.
	\item To extend the formulation to enable the simulation of soft deformable
	bodies with contact.
	\item To properly quantify the \textit{unphysical artifacts} introduced by
	the convex relaxation of the original physics.
	\item To answer the questions ``does the Convex formulation provide any
	significant advantage over other methods?'', ``are the unphysical artifacts
	introduced worth the price?''
	\item To provide an open source implementation of this research that
	outperforms existing methods.
\end{enumerate}
