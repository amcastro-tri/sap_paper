% Dummy commit so that this file shows in Reviewable.

\section{Introduction}
\label{sec:introduction}

\IEEEPARstart{S}{imulation} of multibody systems with frictional contact has
proven indispensable in robotics, aiding at multiple stages during the
mechanical and control design, testing, and training of robotic systems.
Robotic applications often require robust simulation tools that can perform at
interactive rates without sacrificing accuracy, a critical prerequisite for
meaningful sim-to-real transfer. However, reliable modeling and simulation
through contact remains somewhat elusive.

Rigid body dynamics with frictional contact is complicated by the non-smooth
nature of the solutions. It is well known \cite{bib:baraff1993issues} that rigid
contact when combined with the Coulomb model of friction can lead to paradoxical
configurations for which solutions in terms of accelerations and forces do not
exist. These phenomena are known as Painlev\'e paradoxes
\cite{bib:hogan2017regularization}. In theory, these paradoxes can be resolved by allowing
discrete velocity jumps and impulsive forces and formally casting the problem as a
differential variational inequality \cite{bib:pang2008differential}. In practice,
event based approaches can resolve impulsive transitions \cite{bib:haug1986},
though it is not clear how to detect these events even for simple one degree of
freedom systems \cite{bib:hogan2017regularization}.

Nevertheless, the problem can be solved in a weaker formulation at the velocity
level using a time-stepping scheme where the next step velocities and impulses
are the unknowns at each time
step \cite{bib:stewart1996implicit, bib:anitescu1997}. These formulations lead
in general to a non-linear complementarity problem (NCP) or a linear
complementarity problem (LCP) when polyhedral approximations of the friction
cone are used. Even though LCP formulations guarantee solution existence
\cite{bib:anitescu1997, bib:stewart1998convergence}, solving them accurately and
efficiently has remained difficult in practice. This is 
partly due to the fact that these formulations are equivalent to nonconvex problems
in global optimization, which are generally NP-hard \cite{bib:Kaufman2008}.
Indeed, popular direct methods based on Lemke's pivoting algorithm to solve
LCPs may exhibit exponential worst-case complexity \cite{bib:baraff1994fast}. Similar
to direct methods, popular iterative methods based on projected
Gauss-Seidel (PGS) \cite{bib:duriez2006_realistic_haptic_rendering, bib:bullet}
have also shown exponentially slow convergence \cite{bib:erleben2007velocity}.
These observations are not just of theoretical value, but in practice,
these methods are numerically brittle and lack robustness when tasked with computing contact forces.
This inherent lack of stability and robustness is tackled in software with a
large amount of non-physical constraint softening and stabilization.
The resulting simulation requires a significant amount of bespoke parameter tuning.

To improve computational tractability, Anitescu introduces a \textit{convex
relaxation} of the contact problem \cite{bib:anitescu2006}. This relaxation is a
convex approximation with proven convergence to the solution of a measure
differential inclusion as the time step goes to zero. For sliding contacts,
the convex approximation introduces a \emph{gliding} artifact at a distance $\phi$
proportional to the time step size $\delta t$ and to the sliding velocity
$\Vert\vf{v}_t\Vert$ \cite{bib:mazhar2014}, i.e. $\phi\sim \delta
t\Vert\vf{v}_t\Vert$. The approximation is exact for sticking contacts,
and it can be adequate for robotics applications for which the product $\delta
t\Vert\vf{v}_t\Vert$ are usually sufficiently small. For trajectory optimization,
Todorov \cite{bib:todorov2011} introduces regularization into Anitescu's
formulation in order to write a strictly convex formulation with a unique,
smooth and invertible solution. For simulation, Todorov \cite{bib:todorov2014}
uses regularization to introduce \emph{numerical compliance} that provides
Baumgarte-like stabilization and avoids constraints drift. As a side effect, the
regularized formulation can lead to a noticeable non-zero slip velocity even
during stiction \cite{bib:simbenchmark}.

Even though these formulations introduce a tractable approximation of frictional
contact, they haven't become a popular approach in practice. We
believe this is not because the approximation is not suitable for robotics
applications, but rather because of the lack of robust solution methods with a
computational cost suitable for real-time simulation. Software such as ODE
\cite{bib:ode}, Dart \cite{bib:dart} and Vortex \cite{bib:vortex} use a
polyhedral approximation of the friction cone leading to an LCP formulation.
Algoryx \cite{bib:algoryx} uses a \emph{split solver}, reminiscent of one
iteration in the staggered projections method \cite{bib:Kaufman2008}. Drake
\cite{bib:drake} solves compliant contact with regularized friction with its
transition aware solver TAMSI \cite{bib:castro2020}. 

To the knowledge of the authors, Chrono \cite{bib:hrono2016} and Mujoco
\cite{bib:mujoco} are the only software that implement the convex approximation
of contact. The multi-physics simulation package Chrono implements a variant of
the PGS method for solving Anitescu's convex formulation \cite{bib:tasora2011}.
This method is specially targeted to the simulation of very large scale systems
as those found in granular flow. Even though some convergence guarantees are
provided in \cite{bib:anitescu2010}, the solver exhibits low convergence rates.
This is expected for such first order method. Mujoco is a software targeted to
robotics that implements both a PGS solver variant \cite{bib:todorov2014} and a
second order Newton solver. However, the underlying technology is proprietary and
implementation details are not known in the community.

Summarizing, 15 years after the introduction of these convex
approximations, robust and performant algorithms for their solution in practice
are lacking. It is not even clear in the community if these formulations present
a real advantage when compared to more traditional approaches and whether the
artifacts introduced by the approximation are appropriate for robotics
applications. In this work, we aim to provide answers to these questions. 

We make a number of novel contributions in this work. To answer the question of
physical accuracy and validity of the approximation, in Section
\ref{sec:physical_intuition} we build on Todorov's work \cite{bib:todorov2014}
to write compact analytical expressions of the impulses that provide a clear
physical interpretation of the convex formulation. This allows us to succinctly
describe the model in physical terms rather than in optimization specific terms
that might be elusive to non-experts of optimization. Moreover, the artifacts
introduced by the approximation become apparent and can be characterized
precisely. This contribution is novel to our work and different from
\cite{bib:todorov2014} where regularization is used as a Baumgarte-like
stabilization. This allow us to incorporate not only compliant point contact,
but also complex models of compliant surface patches
\cite{bib:elandt2019pressure}, as we demonstrate in Section
\ref{sec:slip_control}. We introduce in Section
\ref{sec:discrete_time_formulation} a time stepping approach based on the
$\theta\text{-method}$. This includes the popular symplectic Euler method and
the implicit midpoint rule. In Section \ref{sec:spring_cylinder} we demonstrate
that the midpoint rule can achieve second order accuracy even in problems with
frictional contact.

Unlike previous work \cite{bib:anitescu2010,bib:todorov2014} that formulate the
problem in terms of impulses, we write a primal formulation of compliant contact
in terms of velocities in Section \ref{sec:primal_formulation}. In Section
\ref{sec:unconstrained_convex_formulation}, we eliminate constraints from the
primal formulation analytically to write an unconstrained convex formulation of
the same problem. This leads us to the development of our semi-analytical primal
solver (SAP) in Section \ref{sec:sap_solver} to solve the unconstrained
formulation in practice. SAP has proven convergence and warm-starts effectively
using only the velocities from the previous time step. We provide thorough
details for its implementation including analytic computation of the Hessian,
our fast exact line search, and sparsity analysis. Section
\ref{sec:understanding_model_parameters} presents a judicious parameterization
of regularization that allows for a tight resolution of stiction and simulation
of \emph{near-rigid} objects while keeping the conditioning of the Hessian under
control. This is critical for the simulation of demanding robotics tasks such as
manipulation. We evaluate the accuracy, robustness and performance of SAP
against available commercial and open-source optimization solvers. In Section
\ref{sec:test_cases}, we demonstrate the effectiveness of our approach in a
number of simulation cases, including the simulation of a challenging dual arm
manipulation task. We conclude our work with a discussion on extensions and
variations in Section \ref{sec:variations_and_extensions} and provide
conclusions and future research directions in Section
\ref{sec:future_directions}.
