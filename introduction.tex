% The very first letter is a 2 line initial drop letter followed
% by the rest of the first word in caps.
% 
% form to use if the first word consists of a single letter:
% \IEEEPARstart{A}{demo} file is ....
% 
% form to use if you need the single drop letter followed by
% normal text (unknown if ever used by the IEEE):
% \IEEEPARstart{A}{}demo file is ....
% 
% Some journals put the first two words in caps:
% \IEEEPARstart{T}{his demo} file is ....

\section{Introduction}
\label{sec:introduction}

\CyanHighlight{\IEEEPARstart{I}{ntro} paragraph stating why sim and modeling
with contact is important for robotics planning and control plus some
sim-to-real bullshit.}

Rigid body dynamics with frictional contact is complicated by the non-smooth
nature of the solutions. It is well known \cite{bib:baraff1993issues} that rigid
contact when combined with the Coulomb model of friction can lead to paradoxical
configurations for which solutions in terms of accelerations and reaction forces
do not exist. These phenomena are known as Painlev\'e paradoxes
\cite{bib:hogan2017regularization}. Theory resolves these paradoxes by allowing
discrete velocity jumps and impulsive forces, formally casting the problem as a
differential variational inequality \cite{bib:pang2008differential}. In practice
event based approaches can resolve impulsive transitions \cite{bib:haug1986},
though it is not clear how to detect these events even for simple one degree of
free systems \cite{bib:hogan2017regularization}.

Nevertheless, the problem can be solved in a weaker formulation at the velocity
level using a time-stepping scheme where impulses are the unknowns at each time
step \cite{bib:stewart1998convergence, bib:anitescu1997}. 

\CyanHighlight{Brief overview of time stepping methods, how we go from NCP to LCP
approximations. Anisotropy of LCP methods. How Anitescu Potra (or was it
Stewaret Trinkle?) have guaranteed solution but the LCP still might be an
NP-hard problem, cite Kaufman's statement. Essentially this paragraph states how
difficult is to solve the original NCP, whether with NCP or other approximation.
Methods like the staggered projection I believe have not convergence guarantees,
see what Kaufman himself says.
\begin{itemize}
	\item Stewart and Trinkle 1996 \cite{bib:chakraborty2007implicit} are probably the first ones to give an LCP formulation, though Anitescu below seems to claim it it was the first time existence was proved. I'd just put them at the same level.
	\item Anitescu 2007\cite{bib:anitescu1997}: ``is the first consistent discretization of the
	dynamic multi-body contact problem with friction that is guaranteed to have
	a solution for the general case.''
\end{itemize}
}

Brief paragraph on available engineering solvers. Essentially copy what you have
in the folder above. State how Mujoco is possibly the only software that uses the convex approximation of contact, though the inner workings are not common knowledge in the community. Also state here or below, how after 15 years after Anitescu's work, there is no freely available knowledge on how to solve this problem effectively.

\RedHighlight{Careful not to say twice what we talk about in Section
\ref{sec:previous_work}. See if worth merging or how to consolidate these two.}
Quick paragraph introducing convex approximations (probably state we explicitly
say \emph{approximation} instead of \emph{relaxation} as Anitescu calls it to
cleary covey the fact that it is an approximation to contact.). Cite work by
Anitescu, state convergence to original problem with step size. Cite work from
Acary (2011) in whatev context is applied. Quickly introduce todorov's saying he
came up with regularization so that the problem has unique solution and that he
came up with \emph{inverse dynamics}. Say how he uses regularization as a means
of constraint stabilization, which is different from our work.

\RedHighlight{Probably a small paragraph here on convex optimization methods and how for simulation there is the need for a methods that warms starts effectively? Ask Frank for help.}

Paragraph clearly stating our contributions. Probably bullets to make it very
clear?

Paragraph on outline of this work.


\section{Introduction (Old)}
\RedHighlight{TODO: Rewrite for paper.}

\IEEEPARstart{A}{Anitescu}'s convex relaxation of contact
\cite{bib:anitescu2006} has been in existence for 14 years. However to my
knowledge there are no simulation/trajectory optimization codes out there that
can exploit this formulation. In other words, the formulation is theoretically
beautiful, but there are no practical solvers out there for it. The best
reported solver for this formulation, later on presented by Anitescu in
\cite{bib:anitescu2010, bib:tasora2011}, consists of a Projected Gauss-Seidel
(PGS) iteration which is very well known not to converge in practice and it is
usually stoped at a fixed number of iterations with an unknown amount of
numerical error. Therefore, it is unclear even today whether Anitescu's
formulation presents any advantage over other non-convex solution strategies. In
\cite{bib:todorov2014} Todorov mentions a so called \textit{generalized}
projected Gauss-Seidel (though the details were never published), which
generalizes PGS to deal with the true quadratic friction cone, anisotropy,
torsional and rolling friction.

The objectives of this reasearch are:
\begin{enumerate}
	\item To develop a \textit{practical} solution strategy that is robust and
	has good scalability to number of bodies and contacts.
	\item To extend the formulation to enable the simulation of soft deformable
	bodies with contact.
	\item To properly quantify the \textit{unphysical artifacts} introduced by
	the convex relaxation of the original physics.
	\item To answer the questions ``does the Convex formulation provide any
	significant advantage over other methods?'', ``are the unphysical artifacts
	introduced worth the price?''
	\item To provide an open source implementation of this research that
	outperforms existing methods.
\end{enumerate}

\section{Other Papers Worth Citing}
\RedHighlight{Notes to help writing the intro and suplement other sections
better.}

\begin{enumerate}
	\item Acary \cite{bib:acary2011formulation} also uses the same convex
	formulation. It might be worth mentioning differences, contributions and
	specific field of engineering.
	\item Kaufman's thesis on staggered projections
	\cite{bib:kaufman2009coupled} states this interesting observation on LCP's:
	``It has been recently noted, however, that direct LCP solvers do not, in
	practice, scale. They are not, in fact, currently able to return solutions
	for non-trivial frictionally contacting problems beyond relatively
	small-scale examples [Anitescu and Hart 2004; Erleben 2007]. In Chapter 5 we
	will discuss these difficulties further and investigate the reasons why
	these methods often fail.'' This is backed up on the works of Anitescu and
	Hart 2004; Erleben 2007 \cite{bib:anitescu2004fixed,bib:erleben2007velocity}.
	\item Anitescu \cite{bib:anitescu2010} also talks about how LCPs do not work
	well in practice: ``However, these methods may exhibit an exponential
	worst-case complexity [9]. Our experience shows that, in spite of deep opti-
	mizations [40], simplex methods still cannot practically handle multibody
	systems with more than one hundred colliding bodies.''
\end{enumerate}

\section{Novel Contributions}
\RedHighlight{TODO: Rewrite for paper.}

Since we build on previous work \cite{bib:anitescu2006,
bib:anitescu2010,bib:todorov2014}, it is worth summarizing our novel
contributions:
\begin{itemize}
	\item Clear algebraic expressions for the \textit{inverse dynamics} of
	contact are provided.
	\item We show how to use these expressions do model true compliant contact.
	This is different from \cite{bib:todorov2014} where regularization is used
	as a Baumgarte-like stabilization.
	\item The analasis of the analytical inverse dynamics expressions here
	provided allows to make a clear analogy with compliant contact.
	\item Our clear analogy with compliant contact allow us to succinctly
	describe the artifacts introduced by the convex approximation of contact.
	\item We write a primal formulation of compliant contact written in terms of
	velocities, in contrast to its dual in terms of impulses.
	\item We show how the primal formulation when combined with the analytical
	 inverse dynamics leads to an unconstrained convex optimization problem.
	\item We develop a custom solver for this unconstrained formulation and
	provide thorough details for a practical implementation.
	\item We conclude with a number of demonstrations illustrating the
	effectiveness of our methodology and provide future research directions.
\end{itemize}
