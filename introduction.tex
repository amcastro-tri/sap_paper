% The very first letter is a 2 line initial drop letter followed
% by the rest of the first word in caps.
% 
% form to use if the first word consists of a single letter:
% \IEEEPARstart{A}{demo} file is ....
% 
% form to use if you need the single drop letter followed by
% normal text (unknown if ever used by the IEEE):
% \IEEEPARstart{A}{}demo file is ....
% 
% Some journals put the first two words in caps:
% \IEEEPARstart{T}{his demo} file is ....

\section{Introduction}
\RedHighlight{TODO: Rewrite for paper.}

\IEEEPARstart{A}{Anitescu}'s convex relaxation of contact
\cite{bib:anitescu2006} has been in existence for 14 years. However to my
knowledge there are no simulation/trajectory optimization codes out there that
can exploit this formulation. In other words, the formulation is theoretically
beautiful, but there are no practical solvers out there for it. The best
reported solver for this formulation, later on presented by Anitescu in
\cite{bib:anitescu2010, bib:tasora2011}, consists of a Projected Gauss-Seidel
(PGS) iteration which is very well known not to converge in practice and it is
usually stoped at a fixed number of iterations with an unknown amount of
numerical error. Therefore, it is unclear even today whether Anitescu's
formulation presents any advantage over other non-convex solution strategies. In
\cite{bib:todorov2014} Todorov mentions a so called \textit{generalized}
projected Gauss-Seidel (though the details were never published), which
generalizes PGS to deal with the true quadratic friction cone, anisotropy,
torsional and rolling friction.

The objectives of this reasearch are:
\begin{enumerate}
	\item To develop a \textit{practical} solution strategy that is robust and
	has good scalability to number of bodies and contacts.
	\item To extend the formulation to enable the simulation of soft deformable
	bodies with contact.
	\item To properly quantify the \textit{unphysical artifacts} introduced by
	the convex relaxation of the original physics.
	\item To answer the questions ``does the Convex formulation provide any
	significant advantage over other methods?'', ``are the unphysical artifacts
	introduced worth the price?''
	\item To provide an open source implementation of this research that
	outperforms existing methods.
\end{enumerate}

\section{Other Papers Worth Citing}
\RedHighlight{Notes to help writing the intro and suplement other sections
better.}

\begin{enumerate}
	\item Acary \cite{bib:acary2011formulation} also uses the same convex
	formulation. It might be worth mentioning differences, contributions and
	specific field of engineering.
	\item Kaufman's thesis on staggered projections
	\cite{bib:kaufman2009coupled} states this interesting observation on LCP's:
	``It has been recently noted, however, that direct LCP solvers do not, in
	practice, scale. They are not, in fact, currently able to return solutions
	for non-trivial frictionally contacting problems beyond relatively
	small-scale examples [Anitescu and Hart 2004; Erleben 2007]. In Chapter 5 we
	will discuss these difficulties further and investigate the reasons why
	these methods often fail.'' This is backed up on the works of Anitescu and
	Hart 2004; Erleben 2007 \cite{bib:anitescu2004fixed,bib:erleben2007velocity}.
\end{enumerate}

\section{Novel Contributions}
\RedHighlight{TODO: Rewrite for paper.}

Since we build on previous work \cite{bib:anitescu2006,
bib:anitescu2010,bib:todorov2014}, it is worth summarizing our novel
contributions:
\begin{itemize}
	\item Clear algebraic expressions for the \textit{inverse dynamics} of
	contact are provided.
	\item We show how to use these expressions do model true compliant contact.
	This is different from \cite{bib:todorov2014} where regularization is used
	as a Baumgarte-like stabilization.
	\item The analasis of the analytical inverse dynamics expressions here
	provided allows to make a clear analogy with compliant contact.
	\item Our clear analogy with compliant contact allow us to succinctly
	describe the artifacts introduced by the convex approximation of contact.
	\item We write a primal formulation of compliant contact written in terms of
	velocities, in contrast to its dual in terms of impulses.
	\item We show how the primal formulation when combined with the analytical
	 inverse dynamics leads to an unconstrained convex optimization problem.
	\item We develop a custom solver for this unconstrained formulation and
	provide thorough details for a practical implementation.
	\item We conclude with a number of demonstrations illustrating the
	effectiveness of our methodology and provide future research directions.
\end{itemize}
