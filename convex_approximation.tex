% Dummy comment for Reviewable.

\section{Convex Approximation of Contact Dynamics}\label{sec:convex_approximation}

Previous work from Anitescu~\cite{bib:anitescu2006} introduces an approximation
of the nonpenetration constraint that together with
Eqs.~\eqref{eq:scheme_mdp_multiplier}-\eqref{eq:scheme_mdp_cone} leads to a
convex formulation in impulses. Later on Todorov~\cite{bib:todorov2011,
bib:todorov2014} uses the \emph{Gauss's principle of least constraint} to obtain
a regularized form of Anitescu's formulation that admits a unique solution.
Moreover, Todorov shows that impulses can be constructed from the computed
velocities, an operation referred to as \textit{analytical inverse dynamics}. It
should be pointed out that the Gauss's principle of least constraint can only be
applied to problems with Tresca type dry friction, for which normal forces are
known in advance \cite{bib:yunt2014}. Therefore the Gauss's principle of least
constraint is only an approximation to the model of Coulomb friction for which
normal forces are unknown and the formulation is non-convex.

In this section we build from this previous work to write a new convex formulation
of compliant contact in terms of \emph{velocities}.  We then show that (with
symplectic Euler) its dual corresponds to Todorov's~\cite{bib:todorov2014}
formulation in impulses. Finally, we use
Todorov's \textit{analytical inverse dynamics} to eliminate constraints
analytically, yielding an unconstrained formulation.


\section{A Primal Formulation of Compliant Contact}

% I took a stab at rewriting this intro to address some 
% of the comments I had (see below). 

% START REWRITE
%In this section we give a quadratic program that augments the balance of momentum stated in Eq.
%(\ref{eq:momentum_linearized}) so that contact impulses model Coulomb friction
%and satisfy the principle of maximum dissipation when sliding. 
%We call this QP our \emph{primal formulation}, as we will see that it's dual
%is precisely the QP~(\ref{eq:dual_regularized}).


%Recall that this QP~(\ref{eq:dual_regularized}) has cost matrix $W =J^T M^{-1} J^{-1} + R$
%where $R$ is a regularization term. 
%Our primal formulation will avoid construction of $J^T M^{-1} J^{-1}$, 
%which improves practical robustness. In addition, it always has a unique solution, even for $R = 0$.
%Finally, we note that for $R =0$, our formulation reduces to \cite{bib:mazhar2014}.
%\ref{sec:unconstrained_convex_formulation} and \ref{sec:solver_details} we
%describe a methodology to solve it in practice.

% END REWRITE


%START ORIGINAL
In this section we augment the balance of momentum stated in Eq.
(\ref{eq:momentum_linearized}) so that contact impulses model Colomb friction
and satisfy the principle of maximum dissipation when sliding. 

\RedHighlight{Havent we already justified regularization when we introduced
\ref{eq:dual_regularized}?
This seems repetitive.  I would focus only on how our formulation differs-compares with \ref{eq:dual_regularized}.}
An alternative is to use the convex approximation as stated in Eq.
\RedHighlight{It is unclear why this is an alternative, e.g., how does the principle
of max disp. appear in this equation?}
(\ref{eq:dual_cost}). However, this formulation is severely ill conditioned due
to the fact that contact forces for rigid body dynamics problems are most often
underdetermined. Even if compliance is added in the normal direction, friction
\RedHighlight{Unclear what you mean by compliance. Compliance is added to the primal constraints, right?}
forces for the simplest problem configurations will be underdetermined.
Regularization in Eq. (\ref{eq:dual_regularized}) helps to solve this problem in
theory but it leads to very ill conditioned systems of equations in practice. 

We make the following observation for the convex approximation in Eq.
(\ref{eq:dual_cost}); even if the set of contact forces is not unique (when no
regularization is added), velocities are. This fact inspired the search
for an equivalent formulation but in velocities instead of impulses. Such a
\RedHighlight{The name "primal formulation" what make much sense unless you such duality first.}
\textit{primal} formulation is presented in \cite{bib:mazhar2014} for rigid
contact, though to the knowledge of the authors a practical solver based on this
formulation has never been presented.

In this section we extend the formulation in \cite{bib:mazhar2014} to include
the modeling of compliance and in Sections
\ref{sec:unconstrained_convex_formulation} and \ref{sec:solver_details} we
describe a methodology to solve it in practice.
% END ORIGINAL 



We write our primal formulation of compliant contact by introducing a new
decision variable $\vf{\sigma}\in\mathbb{R}^{3n_c}$ as
\begin{equation}
	\begin{aligned}
	\min_{\mf{v},\bsigma} \quad & \ell_p(\mf{v},\bsigma) = \frac{1}{2}(\mf{v}-\mf{v}^*)^T\mf{A}(\mf{v}-\mf{v}^*) + \frac{1}{2} \Vert\bsigma\Vert_{R}^2\\
	\textrm{s.t.} \quad & \mf{g} = (\mf{J}\mf{v}-\hat{\mf{v}} + \mf{R}\bsigma) \in \mathcal{F}^*\\
	\end{aligned}
	\label{eq:primal_regularized}
\end{equation}
where $\mathcal{F^*}= \mathcal{F}^*_1 \times \mathcal{F}^*_2 \times \cdots \times \mathcal{F}^*_{n_k}$ is the \emph{dual cone} of the convex
cone $\mathcal{F}$. The positive diagonal matrix $\mf{R}\in\mathbb{R}^{3n_c\times
3n_c}$ and the vector of stabilization velocities $\hat{\mf{v}}$ encode the
problem data needed to model compliant contact. We will establish a clear
physical meaning for these terms when we provide analytical expressions for the
impulses in Section \ref{sec:analytical_inverse_dynamics}. Finally,
$\Vert\bsigma\Vert_R^2=\bsigma^T\mf{R}\bsigma$.

\begin{theorem}
The dual of Eq. (\ref{eq:primal_regularized}) is given by Eq.
(\ref{eq:dual_regularized}). The pair $\{\mf{v},\bsigma\}$ is primal optimal and
$\bgamma$ is dual optimal. Moreover, $\bsigma = \bgamma$.
\end{theorem}

\begin{proof}
The Lagrangian of the primal formulation in Eq. (\ref{eq:primal_regularized}) is
\begin{equation}
	\mathcal{L}(\mf{v},\bsigma,\vf{\gamma}) = \frac{1}{2}(\mf{v}-\mf{v}^*)^T\mf{A}(\mf{v}-\mf{v}^*) + \frac{1}{2} \Vert\bsigma\Vert_{R}^2 - \vf{\gamma}^T\mf{g}
	\label{eq:primal_lagrangian}
\end{equation}
with $\vf{\gamma}\in\mathcal{F}$ the dual variable to enforce the constraint
$\vf{g}\in \mathcal{F}^*$. We obtain the dual of Eq.
(\ref{eq:primal_regularized}) by minimizing the Lagrangian jointly in the
variables $\mf{v}$ and $\bsigma$ and replacing the result back to obtain the
dual cost $\ell_d(\vf{\gamma})$. Minimizing jointly in the variables $\mf{v}$
and $\bsigma$ leads to the conditions
\begin{eqnarray}
	\mf{A}(\mf{v}-\mf{v}^*) &=& \mf{J}^T\vf{\gamma}\\
	\vf{\sigma} &=& \vf{\gamma}
\end{eqnarray}
where with the first equation we find out that multipliers $\bgamma$ are indeed
impulses and we recover the balance of momentum, and the second equation allows
us to eliminate $\vf{\sigma}$. When we replace these results back into the
Lagrangian in Eq. (\ref{eq:primal_lagrangian}) we obtain the dual
\begin{eqnarray}
	\min_{\bgamma\in \mathcal{F}} \ell_d(\bgamma) =
	\frac{1}{2}\bgamma^T(\mathbf{W}+\mathbf{R})\bgamma + {\bm r}^T
	\bgamma
\end{eqnarray}
where, in contrast to previous work, our Delassus operator
$\mf{W}=\mf{J}\mf{A}\mf{J}^T$ now also contains the contribution of internal
force elements (through Eq. (\ref{eq:expression_for_A})) and
$\mf{r}=\mf{v}_c^*-\hat{\mf{v}}$ with $\mf{v}_c^*=\mf{J}\mf{v}^*$.
\end{proof}

\RedHighlight{This seems repetitive if we also mention mazhar in the intro of this section}
We note that our formulation is an extension of~\cite{bib:mazhar2014} that
incorporates  compliance through the term $R$. Sections
\ref{sec:unconstrained_convex_formulation} and \ref{sec:solver_details} 
describe a methodology to solve our formulation efficiently practice, whereas,
to our knowledge, no efficient algorithm for~\cite{bib:mazhar2014} has been given. 



% formulation in \cite{bib:mazhar2014} to include
%the modeling of compliance and in Sections
%\ref{sec:unconstrained_convex_formulation} and \ref{sec:solver_details} we



% Dummy comment for Reviewable.

\subsection{Analytical Inverse Dynamics}
\label{sec:analytical_inverse_dynamics}

The dual optimal impulses of~\eqref{eq:dual_regularized} can be
constructed from the primal optimal velocities of~\eqref{eq:primal_regularized}
using a simple projection operation. Following~\cite{bib:todorov2014}, we call
this construction  \textit{analytical inverse dynamics}. Moreover, this
projection decomposes into a set of individual projections for each contact
impulse $\bgamma_i$ given the separable structure of the constraints. Letting
$\vf{y}_i(\vf{v}_{c,i}) = -\vf{R}_i^{-1}(\vf{v}_{c,i}-\hat{\vf{v}}_{c,i})$,
these projections take the form
\begin{equation}
  \begin{aligned}
	\bgamma_i(\vf{v}_{c,i})&= P_{\mathcal{F}_i}(\vf{y}_i(\vf{v}_{c,i}))\\
	&= \argmin_{\bgamma\in\mathcal{F}_i} \quad 
		\frac{1}{2}(\bgamma-\vf{y}_i)^T\vf{R}_i(\bgamma-\vf{y}_i),\\
	\end{aligned}
	\label{eq:y_projection}
\end{equation}
where $\vf{R}_i\in\mathbb{R}^{3\times3}$ is the $i\text{-th}$ diagonal block of
the regularization matrix $\mf{R}$. That is, $\bgamma_i$ is the projection
$P_{\mathcal{F}_i}$ of $\vf{y}_i(\vf{v}_{c,i})$ onto the friction cone
$\mathcal{F}_i$ using the norm defined by $\vf{R}_i$.
\reviewquestion{R1-Q7}{Remarkably, the projection map $P_{\mathcal{F}_i}$ can be
evaluated \emph{analytically}. We provide algebraic expressions for it in
Section~\ref{sec:physical_intuition} and derivations in
Appendix~\ref{app:analytical_inverse_dynamics_derivations}.} The projection
$P_{\mathcal{F}}(\mf{y})$ onto the full cone $\mathcal{F} := \mathcal{F}_1
\times \mathcal{F}_2 \times \cdots \times \mathcal{F}_{n_c}$ is obtained by
simply stacking together the individual projections
$P_{\mathcal{F}_i}(\vf{y}_i)$ from Eq. (\ref{eq:y_projection}), where we form
$\mf{y}$ by stacking together each $\vf{y}_i$ from all contact pairs.  In this
notation, the optimal impulse $\bgamma$ of~\eqref{eq:dual_regularized} and the
optimal velocities $\mf{v}$ of~\eqref{eq:primal_regularized} satisfy  $\bgamma =
P_{\mathcal{F}}(\mf{y}(\mf{v}))$.

\section{An Unconstrained Convex Formulation}
\label{sec:unconstrained_convex_formulation}

From Theorem \ref{th:primal_dual_equivalence} we know that $\bsigma = \bgamma$.
Therefore we can use the analytical inverse dynamics from Section
\ref{sec:analytical_inverse_dynamics} to eliminate $\bsigma$ from the primal
formulation and obtain an unconstrained convex formulation in terms of
generalized velocities $\mf{v}$ only.

We then substitute $\bsigma=\bgamma=P_\mathcal{F}(\mf{y})$ from Eq.
\ref{eq:analytical_y_projection} into Eq. (\ref{eq:primal_regularized}) to
obtain our unconstrained convex formulation in velocities only
\begin{eqnarray}
	\min_{\mf{v}} \ell_p(\mf{v}) =
	\frac{1}{2}(\mf{v}-\mf{v}^*)^T\mf{A}(\mf{v}-\mf{v}^*) +
	\ell_R(\mf{v})
	\label{eq:primal_unconstrained}
\end{eqnarray}
with the regularizer
\begin{equation}
	\ell_R(\mf{v}) = \frac{1}{2}\Vert P_\mathcal{F}(\mf{y}(\mf{v}))\Vert_R^2
\end{equation}

In order for the velocities solution of Eq. (\ref{eq:primal_unconstrained}) to
be a solution of the original problem stated in Eq.
(\ref{eq:primal_regularized}), they must also satisfy the conic constraint
$\mf{g}\in\mathcal{F}^*$. We stablish this result in the following
\begin{lemma}
    velocites solution to Eq. (\ref{eq:primal_unconstrained}) satisfy the conic
    constraint $\mf{g}\in\mathcal{F}^*$ when the impulses are given by
    $\bsigma=\bgamma=P_\mathcal{F}(\mf{y})$.
    \label{lemma:conic_constraints_are_satisfied_analytically}
\end{lemma}
\begin{IEEEproof}
    The proof follows immediately from Moreau's decomposition theorem. That is,
    with $\mf{v}_c - \hat{\mf{v}}\in\mathbb{R}^{3n_c}$ and $\bsigma$ the result
    from the projection into $\mathcal{F}$ using the $\mf{R}$ norm, the linear
    combination $\mf{v}_c - \hat{\mf{v}} + \mf{R}\bsigma$ must be in
    $\mathcal{F}^*$.
\end{IEEEproof}

Therefore we can state the following
\begin{theorem}
    Velocities $\mf{v}$ solution to Eq. (\ref{eq:primal_unconstrained}) and
    $\bsigma=P_\mathcal{F}(\mf{y})$ from the analytical solution in Eq.
    (\ref{eq:analytical_y_projection}) are solution to the original primal
    formulation stated in Eq. (\ref{eq:primal_regularized}).
\end{theorem}
\begin{IEEEproof}
    Theorem \ref{th:primal_dual_equivalence} states that $\bsigma=\bgamma$.
    Given known primal-optimal velocities, $\bgamma$ is given by the same
    analytical inverse dynamics used in the unconstrained formulation in Eq.
    (\ref{eq:primal_unconstrained}).
    In addition, these velocities and the analytical impulses satisfy the cone
    constraints by Lemma \ref{lemma:conic_constraints_are_satisfied_analytically}.
\end{IEEEproof}

This new formulation forms the foundation of the novel SAP solver developed in
this work, Section \ref{sec:sap_solver}.


