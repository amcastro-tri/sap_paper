\section{Multibody Dynamics with Contact}
\label{sec:multibody_dynamics_with_contact}

We use generalized coordinates to describe our multibody system. Therefore, the
state is fully described by the generalized positions
$\mf{q}\in\mathbb{R}^{n_q}$ and the generalized velocities
$\mf{v}\in\mathbb{R}^{n_v}$, where $n_q$ and $n_v$ denote the number of
generalized positions and velocities, respectively. Time derivatives of the
configurations relate to generalized velocities by the kinematic map $\mf{N}(\mf{q})\in\mathbb{R}^{n_q\times
n_v}$ as
\begin{equation}
    \dot{\mf{q}}=\mf{N}(\mf{q})\mf{v}.
    \label{eq:v_to_qdot}
\end{equation}

\subsection{Contact Kinematics}
\label{sec:contact_modeling}

Given a configuration $\mf{q}$ of the system, we assume our geometry engine
reports a set $\mathscr{C}(\mf{q})$ of $n_c$ potential contacts between
pairs of bodies. We characterize the $i\text{-th}$ \emph{contact pair} in
$\mathscr{C}(\mf{q})$ by the location $\vf{p}_i$ of the contact point, a normal
direction $\hat{\vf{n}}_i$ and \emph{gap function} $\phi_i(\mf{q})\in\mathbb{R}$
\cite{bib:flores2021contact}. The kinematics of each contact is further
completed with the relative velocity $\vf{v}_{c,i}\in\mathbb{R}^3$ between these
two bodies at point $\vf{p}_i$, expressed in a contact frame $C_i$ for which we
arbitrarily choose the $z\text{-axis}$ to coincide with the contact normal
$\hat{\vf{n}}_i$. In this frame the normal and tangential components of
$\vf{v}_{c,i}$ are given by $v_{n,i} = \hat{\vf{n}}_i\cdot\vf{v}_{c,i}$ and
$\vf{v}_{t,i} = \vf{v}_{c,i}-v_{n,i}\hat{\vf{n}}_i$ respectively, so that
$\vf{v}_{c,i}=[\vf{v}_{t,i}\,v_{n,i}]$.

We form the vector $\mf{v}_{c}\in\mathbb{R}^{3n_c}$ of contact velocities by
stacking velocities $\vf{v}_{c,i}$ of all contact pairs together. In general,
unless otherwise specified, we use bold italics for vectors in $\mathbb{R}^3$
and non-italics bold for their stacked counterpart. The generalized velocities
$\mf{v}$ and contact velocities $\mf{v}_c$ satisfy the equation
$\mf{v}_c=\mf{J}\,\mf{v}$, where $\mf{J}(\mf{q})\in\mathbb{R}^{3n_c\times n_v}$
denotes the contact Jacobian.

\subsection{Contact Modeling}
We model the normal component of the impulse $\gamma_{n,i}$ during a time
interval of size $\delta t$ with the compliant force law
\begin{equation}
    \gamma_{n,i}/\delta t = (-k_i\phi_i - \tau_{d,i}\,k_i\,v_{n,i})_+
    \label{eq:compliant_model}
\end{equation}
where $k_i$ is the stiffness parameter, $\tau_{d,i}$ is a \textit{dissipation time
scale} and $(x)_+=\max(0, x)$ is the \textit{positive part} operator. This model
of compliance can be written as the equivalent complementarity condition
\begin{equation}
    0 \le \phi_i + \tau_{d,i}\,v_{n,i} + c_i \gamma_{n,i}\perp \gamma_{n,i} \ge 0
\end{equation}
where $c_i=k_i^{-1}$ is the compliance parameter and $0 \le a\perp b \ge 0$ denotes
complementarity, i.e. $a \ge 0$, $b \ge 0$ and $a\,b=0$.

The tangential component $\bgamma_{t,i}\in\mathbb{R}^2$ of the contact force is
modeled with Coulomb's law of dry friction as
\begin{equation}
    \bgamma_{t,i}=\argmin_{\Vert\bgamma_{t}\Vert\leq\mu_i\gamma_{n,i}}\vf{v}_{t,i}\cdot\bgamma_{t}
    \label{eq:maximum_dissipation_principle}
\end{equation}
where $\mu_i > 0$ is the coefficient of friction for the $i\text{-th}$ contact.
Equation (\ref{eq:maximum_dissipation_principle}) describes the \emph{maximum
dissipation principle}, which states that friction impulses maximize the rate of
energy dissipation. In other words, friction impulses oppose the sliding
velocity direction. Moreover, Eq. (\ref{eq:maximum_dissipation_principle})
states that contact impulses $\bgamma_i$ at point $i$ are constrained to belong
to the friction cone $\mathcal{F}_i=\{[\vf{x}_t, x_n] \in\mathbb{R}^3 \,|\,
\Vert\vf{x}_t\Vert\le \mu_i x_n\}$.

The optimality conditions for Eq. (\ref{eq:maximum_dissipation_principle}) are
\cite{bib:stewart2000rigid, bib:tasora2011}
\begin{flalign}
    &\mu_i\gamma_{n,i}\vf{v}_{t,i} + \lambda \bgamma_{t,i} = \vf{0}\nonumber\\
    &0\le \lambda \perp \mu_i\gamma_{n,i}-\Vert\bgamma_{t,i}\Vert \ge 0
    \label{eq:mpd_optimality_conditions}
\end{flalign}
where $\lambda$ is the multiplier needed to enforce Coulomb's law condition
$\Vert\bgamma_{t,i}\Vert \le \mu_i\gamma_{n,i}$. Notice that in the form we
wrote Eq. (\ref{eq:mpd_optimality_conditions}), multiplier $\lambda_i$
has units of velocity and it is zero during stiction and takes the value
$\lambda_i=\Vert\vf{v}_{t,i}\Vert$ during sliding.

Finally, the total contact
impulse $\bgamma_i\in\mathbb{R}^3$ expressed in the contact frame $C_i$ is given
by $\bgamma_i=[\bgamma_{t,i}\,\gamma_{n,i}]$.

\subsection{Discrete Model}

We base our time-stepping scheme on the $\theta\text{-method}$ \cite[\S
II.7]{bib:hairer2008solving}. We discretize time into intervals of fixed size
$\delta t$ and seek to advance the state of the system from time $t^n$ to the
next step at $t^{n+1} = t^n + \delta t$. In the $\theta\text{-method}$,
variables are evaluated at intermediate time steps $t^\theta = \theta
t^{n+1}+(1-\theta)t^{n}$, with $\theta \in [0, 1]$. We define \emph{mid-step
quantities} $\mf{q}^{\theta_{q}}$, $\mf{v}^{\theta_{v}}$, and
$\mf{v}^{\theta_{vq}}$ in accordance with the standard $\theta\text{-method}$
using scalar parameters $\theta_q$, $\theta_v$, and $\theta_{vq}$
\begin{align}
	\mf{q}^{\theta_q} &\defeq \theta_q\mf{q} + (1-\theta_{q})\mf{q}_0,\nonumber\\
	\mf{v}^{\theta_v} &\defeq \theta_v\mf{v} + (1-\theta_v)\mf{v}_0,\nonumber\\
	\mf{v}^{\theta_{vq}} &\defeq \theta_{vq}\mf{v} + (1-\theta_{vq})\mf{v}_0,
	\label{eq:theta_method}
\end{align}
where, to simplify notation, we use the naught subscript to denote quantities
evaluated at the previous time step $t^n$ while no additional subscript is used
for quantities at the next time step $t^{n+1}$. Using these definitions we write
the following time stepping scheme where the unknowns are the next time step
generalized velocities $\mf{v}\in\mathbb{R}^{n_v}$, impulses
$\bgamma\in\mathbb{R}^{3n_c}$ and multipliers ${\bm\lambda}\in\mathbb{R}^{n_c}$
\begin{flalign}
    % Momentum equation.
	&\mf{M}(\mf{q}^{\theta_{q}}(\mf{v}))(\mf{v}-\mf{v}_0) =\nonumber\\
	&\qquad\delta
	t\,\mf{k}(\mf{q}^{\theta_{q}}(\mf{v}),\mf{v}^{\theta_v}(\mf{v})) +
	\mf{J}(\mf{q}_0)^T\mf{\bgamma}, \label{eq:scheme_momentum}\\
    % Non-penetration condition.
    &0 \le \phi_i(\mf{q}) + \tau_{d,i}\,v_{n,i}(\mf{q}, \mf{v}) + c_i\gamma_{n,i}\nonumber\\
    &\qquad\perp \gamma_{n,i} \ge 0, \quad\qquad\qquad\qquad i\in\mathscr{C}(\mf{q}_0)
    \label{eq:scheme_nonpenetration}\\
    % Maximum dissipation principle.
    &\mu_i\gamma_{n,i}\vf{v}_{t,i} + \lambda \bgamma_{t,i} = \vf{0},
    \!\!\quad\qquad\qquad i\in\mathscr{C}(\mf{q}_0)
    \label{eq:scheme_mdp_multiplier}\\
    &0\le \lambda \perp \mu_i\gamma_{n,i}-\Vert\bgamma_{t,i}\Vert \ge 0
    , \qquad i\in\mathscr{C}(\mf{q}_0)
    \label{eq:scheme_mdp_cone}\\
    % Positions update.
    &\dot{\mf{q}}^{\theta_{vq}} = \mf{N}(\mf{q}^{\theta_{q}})\mf{v}^{\theta_{vq}},\\    
    &\mf{q} = \mf{q}_0 + \delta t \dot{\mf{q}}^{\theta_{vq}},
    \label{eq:scheme_q_update}
\end{flalign}
where $\mf{M}(\mf{q})\in\mathbb{R}^{n_v\times n_v}$ is the mass matrix and
$\mf{k}(\mf{q},\mf{v})\in\mathbb{R}^{n_v}$ models external forces such as
gravity, gyroscopic terms and other smooth generalized forces such as those
arising from springs and dampers.

This scheme based on the $\theta\text{-method}$ includes some of the most
popular schemes for forward dynamics:
\begin{itemize}
	\item Explicit Euler with $\theta_q=\theta_{v}=\theta_{vq} = 0$,
	\item Symplectic Euler with $\theta_{q} = \theta_v = 0$ and $\theta_{vq}=1$,
	\item Implicit Euler with $\theta_{q} = \theta_v = \theta_{vq}= 1$, and
	\item Symplectic midpoint rule, which is second order, with $\theta_{q} =
	\theta_v = \theta_{vq}= 1/2$,
\end{itemize}

In particular, when only conservative forces are considered in
$\mf{k}(\mf{q},\mf{v})$, the symplectic Euler scheme keeps the total mechanical
energy bounded while exact energy conservations can be attained with the second
order midpoint rule, see results in Section \ref{sec:spring_cylinder}. In
addition, stability analysis in \cite{bib:anitescu2002,bib:potra2006linearly},
shows that these implicit schemes are appropriate for the integration of stiff
forces arising in multibody applications such as springs and dampers.

Introducing the Coulomb model of dry friction however, turns the problem into a
non-convex nonlinear complementarity problem (NCP). This kind of problems have
remained difficult to solve in practice, specially so in engineering
applications for which robustness and accuracy are a requirement. In the next
sections we introduce a number of approximations to this original NCP that allow
us to make the problem tractable and solve it efficiently in practice.

\subsection{Two-Stage Scheme}

Similar to the work in \cite{bib:duriez2005realistic} for the simulation of
deformable objects and to projection methods used in fluid mechanics
\cite{bib::bell1991efficient}, we solve Eqs.
(\ref{eq:scheme_momentum})-(\ref{eq:scheme_q_update}) in two stages. In the
first stage, we solve for the \emph{free motion velocities} $\mf{v}^*$ the
system would have in the absence of contact constraints, according to
\begin{align}
	\mf{m}(\mf{v}^*) &= \mf{0},
	\label{eq:vstar_definition}
\end{align}
where we defined the momentum residual $\mf{m}(\mf{v})$ from Eq.
(\ref{eq:scheme_momentum}) as
\begin{multline}
	\mf{m}(\mf{v}) =
	\mf{M}(\mf{q}^{\theta_{q}}(\mf{v}))(\mf{v}-\mf{v}_0) -
	\delta t\,\mf{k}(\mf{q}^{\theta_{q}}(\mf{v}),\mf{v}^{\theta_v}(\mf{v})).
	\label{eq:m_definition}
\end{multline}

For integration schemes that are implicit in $\mf{v}^*$ (e.g. the implicit Euler
scheme and the midpoint rule), we solve Eq. (\ref{eq:vstar_definition}) with
Newton's method. For schemes explicit in $\mf{v}^*$, only the mass matrix
$\mf{M}$ needs to be inverted, which can be accomplished efficiently using the
$\mathcal{O}(n)$ \emph{Articulated Body Algorithm}
\cite{bib:featherstone2008_rigid_body_dynamics_algorithms}.

The second stage solves a linear approximation of the balance of momentum in Eq.
(\ref{eq:scheme_momentum}) about $\mf{v}^*$ that satisfies the contact
constraints, Eqs. (\ref{eq:scheme_nonpenetration})-(\ref{eq:scheme_mdp_cone}). To write
a convex formulation of contact in Section \ref{sec:previous_work}, our
linearization uses a symmetric positive definite (SPD) approximation $\mf{A}$ of
the Jacobian $\partial \mf{m}/\partial \mf{v}$. To achieve this we split the
non-contact forces $\mf{k}$ in Eq. (\ref{eq:scheme_momentum}) as
\begin{align*}
	&\mf{k}(\mf{q}^{\theta_{q}}(\mf{v}), \mf{v}^{\theta_v}(\mf{v})) = \\
    &\qquad\qquad \mf{k}_1(\mf{q}^{\theta_{q}}(\mf{v}), \mf{v}^{\theta_v}(\mf{v}))+
	\mf{k}_2(\mf{q}^{\theta_{q}}(\mf{v}), \mf{v}^{\theta_v}(\mf{v})),
\end{align*}
such that the Jacobians $\partial \mf{k}_1/\partial\mf{q}$ and $\partial
\mf{k}_1/\partial\mf{v}$ are negative definite matrices while the same is
generally not true for the Jacobians of $\mf{k}_2$. The term $\mf{k}_1(\mf{q},
\mf{v})$ can include forces from modeling elements such as spring and dampers
and even internal forces for the modeling of soft-body deformation. The term
$\mf{k}_2(\mf{q}, \mf{v})$ includes all other contributions that cannot
guarantee negative definiteness of their Jacobians, such as Coriolis and
gyroscopic forces arising in multibody dynamics with generalized coordinates. We
can now define the SPD approximation of $\partial \mf{m}/\partial \mf{v}$ as
\begin{align}
	\mf{A}&=\mf{M}+\delta t^2\,\theta_q\theta_{qv}\mf{K}+\delta t\,\theta_v\mf{D},
	\label{eq:expression_for_A}\\
	\mf{K}(\mf{q}, \mf{v})&=-\frac{\partial \mf{k}_1(\mf{q}, \mf{v})}{\partial
	\mf{q}}\frac{\partial\dot{\mf{q}}^{\theta_{vq}}}{\partial\mf{v}},
	\label{eq:stiffness_matrix}\\
	\mf{D}(\mf{q}, \mf{v})&=-\frac{\partial \mf{k}_1(\mf{q}, \mf{v})}{\partial
	\mf{v}},
	\label{eq:dissipation_matrix}
\end{align}
where $\mf{K} \succ 0$ and $\mf{D}\succ 0$ are the stiffness and damping
matrices of the system, respectively. As an example, for joint level
spring-dampers models, $\mf{K}$ and $\mf{D}$ are constant, diagonal, and
positive definite matrices. A more complex example arises in the Finite Element
Model (FEM) of soft body deformations. In this case, $\mf{K}$ and $\mf{D}$ are
sparse positive definite matrices.

Using this linear approximation, our approximation to the original contact
problem given by Eqs. (\ref{eq:scheme_momentum})-(\ref{eq:scheme_q_update})
reads
\begin{flalign}
    % Momentum equation.
	&\mf{A}(\mf{v}-\mf{v}^*) = \mf{J}^T\mf{\bgamma},
	\label{eq:momentum_linearized}\\
    % Non-penetration condition.
    &0 \le \phi_i(\mf{q}) + \tau_{d,i}\,v_{n,i}(\mf{q}, \mf{v}) + c_i\gamma_{n,i}\nonumber\\
    &\qquad\perp \gamma_{n,i} \ge 0, \quad\qquad\qquad\qquad i\in\mathscr{C}(\mf{q}_0)
    \tag{\ref{eq:scheme_nonpenetration}}\\
    % Maximum dissipation principle.
    &\mu_i\gamma_{n,i}\vf{v}_{t,i} + \lambda \bgamma_{t,i} = \vf{0},
    \!\!\quad\qquad\qquad i\in\mathscr{C}(\mf{q}_0)
    \tag{\ref{eq:scheme_mdp_multiplier}}\\
    &0\le \lambda \perp \mu_i\gamma_{n,i}-\Vert\bgamma_{t,i}\Vert \ge 0
    , \qquad i\in\mathscr{C}(\mf{q}_0)
    \tag{\ref{eq:scheme_mdp_cone}}\\
    % Positions update.
    &\mf{q} = \mf{q}_0 + \delta t \dot{\mf{q}}^{\theta_{vq}} = \mf{q}_0 + \delta
    t\mf{N}(\mf{q}^{\theta_{q}})\mf{v}^{\theta_{vq}},
    \tag{\ref{eq:scheme_q_update}}
\end{flalign}
where for convenience we use $\mf{J}$ as a shorthand to denote
$\mf{J}(\mf{q}_0)$. The approximation in Eq. (\ref{eq:momentum_linearized}) and
the original discrete momentum update in Eq. (\ref{eq:scheme_momentum}) agree to
second order as shown by the following result, proved in Appendix
\ref{app:gradient_of_m_approximation}.
\begin{prop}	
Matrix $\mf{A}$ is a first order approximation to the Jacobian of $\mf{m}$,
i.e.,
\begin{align*}
	\left. \frac{\partial \mf{m}}{\partial \mf{v}} \right|_{\mf{v}=\mf{v}^*} = \mf{A} + \mathcal{O}(\delta t).
\end{align*}
Therefore, Eq. (\ref{eq:momentum_linearized}) is a second order approximation of
the discrete balance of momentum in Eq. (\ref{eq:v_update}). Moreover, $\mf{A}
\succ 0$.
\label{prop:gradient_of_m_approximation}
\end{prop}

Notice that, in the absence of constraint impulses, the velocities at the next
time step are equal to the free motion velocities, i.e., $\mf{v}=\mf{v}^*$, and
they are computed with the order of accuracy of the $\theta\text{-method}$.
Furthermore, we also expect to recover the properties of the
$\theta\text{-method}$ when contact constraints are not active. As an example,
for bodies in contact that are under rolling friction, the contact constraints
behave as bi-lateral constraints that impose zero slip velocity. In this case,
our two stage method using the midpoint rule to compute $\mf{v}^*$ exhibits
considerably less numerical dissipation than first order methods. We demonstrate
this in Section \ref{sec:spring_cylinder} with an example of a mechanical system
with rolling friction.
