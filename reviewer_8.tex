\section{Reviewer 8}
\label{sec:reviewer_8}

\textcolor{blue}{In this paper, the authors have presented a convex formulation
of compliant frictional contact and a method to solve it. The proposed
formulation can be used for robotic simulation and planning purposes. The
formulation is unconstrained in the sense that it has been described in terms of
the contact impulses and corresponding velocities rather than the contact
constraints directly. The mathematical analysis provided in this paper and the
appendix is rigorous and sufficient for understanding the convex nature of the
optimization formulation. To solve the proposed formulation, the authors have
developed SAP, the Semi-Analytic Primal Solver. They have studied its
initialization, stopping criteria and convergence properties. They have also
studied the solver's performance compared with commercially available solvers
like Gurobi and Geodesic IPM. The video attachment provided along with the paper
is very well made and helps in understanding the key contributions, experiments
and their results.}

\textcolor{blue}{\textbf{Comments to the authors:}}

\textcolor{blue}{R8-Q1: 1) It would be useful to understand whether this method
only works for convex objects. Will the convergence properties and results of
the proposed method be affected if we use non-convex objects with non-convex
contact surface patches?}

This method is not limited to convex objects only. Support for non-convex
geometries will depend on whether the geometry engine provides support for
non-convex geometries or not. The mathematical formulation presented here only
requires the geometry engine to report \emph{contact pairs}, regardless of
whether those pairs correspond to convex or non-convex geometries. We describe
the characterization of contact pairs in Section II.A. At the moment of this
writing, Drake's support for non-convex geometries is limited, though we are
actively working on it. However, this is merely an area of improvement for the
geometry engine, and the contact solver itself is not limited to handling convex
geometries.

\vspace{5mm}
\textcolor{blue}{R8-Q2: 2) Is it possible to incorporate maximum contact force
constraints directly in the optimization formulation for the dual-arm
manipulation? It would be interesting to be able to simulate tasks where the
contact forces in the normal direction are not large enough to firmly grasp and
manipulate the object against the effect of external wrenches.}

If we understand the question correctly, the reviewer is asking to simulate
actuation limits on the Allegro hands. While we are not actively applying
actuation limits on the hands, the parameters of the simulation are well within
real specifications. To be more specific to the contribution of this work, our
method does allow to specify these limits. 

We believe it would be very interesting to investigate the effect of limits of different
parameters for this particular platform and task. However, it is not our
intention to investigate this particular task to inform how to
build real platforms. This robotic platform is only hypothetical and for
demonstration purposes. The objective of this simulation demo is mostly to
showcase the capability of the contact formuation and the solver.

\vspace{5mm}
\textcolor{blue}{R8-Q3: 3) It is not evident to me whether the proposed
formulation can be used to study the change in the contact modes (stiction,
sliding, no contact) for prehensile manipulation tasks such as pivoting a cubic
object about an edge using a manipulator. For examples, impulses applied at the
object-manipulator contact may cause the object to slip and slide instead of
pivoting on the edge. The change in contact modes can be used as feedback for
adjusting the contact impulses provided by the manipulator to the object.}

Our formulation does handle contact mode transitions implicitly. As an example, consider
the \emph{Slip Control} example in Section IV.D. In this example the controller
commands a variable grip force to the gripper. In this case, the gripper always
makes contact with the spatula. However, given the change in grip force, the
grip transitions periodically between a loose grip where the spatula rotates
within grasp (sliding) and a secure grasp where the spatula stops moving
(stiction). This transition between stiction and sliding is made very evident in
the accompanying video (starting at 0:42) where the visualization shows a
contact patch at all times, though its size changes due to the changes in grasp
force.

Consider now the dual arm manipulation task (better shown in the video at 3:23).
Initially the jar is closed, with the lid held in place only by stiction. As the
robot pulls the lid out, the lid comes off given that contact between the jar
and lid transitions to sliding contact. The model fully predicts these
transitions.

Similarly for the example case the reviewer provided. If there was a transition
between sliding and stiction at the edge of this cubic object, our model is able
to resolve it. 

Finally, the particular contact mode can be recovered by inspecting the value of
the impulses before projection $\mf{y}$, see Eq. (27). The region in which
$\mf{y}$ lies determines the contact mode, see Fig. (22) in Appendix C for a
graphical schematic.

To make this point clear, we added text for the \emph{Slip Control} case in
Section VI.D and for the \emph{Dual Arm Manipulation} case in Section VI.E.
